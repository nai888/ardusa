\documentclass[12pt,letterpaper,openany,twoside]{memoir}

% Set new Part to always start on odd page
\let\originalpart=\part
\def\part{\cleardoublepage\originalpart}

% Set the page margins
\setlrmarginsandblock{1in}{1in}{*}
\setulmarginsandblock{1in}{1in}{*}
\checkandfixthelayout

% Layouts
\usepackage{afterpage}
\usepackage{multicol}
\usepackage{rotating}
\usepackage{longtable}
\usepackage{tabu}
\usepackage{multirow}
\usepackage{etoolbox}

% Graphics
\usepackage{graphicx}
\usepackage{tkz-graph}

% Calculations
\usepackage{calc}

% Unicode
\usepackage{xunicode}
\usepackage{xltxtra}

% Handle language and quotation marks
\usepackage{polyglossia}
\setdefaultlanguage[variant=usmax]{english}
\usepackage[style=american,english=american,autopunct]{csquotes} % Put quotations in \enquote{} or \textquote{}

% Date and time
\usepackage[useregional]{datetime2}

% Improve typography
\usepackage[final]{microtype}

% Set fonts
\usepackage{fontspec}
\setmainfont{Junicode}[Ligatures=TeX,Numbers=Lowercase]
\setsansfont{Fira Sans}[Ligatures=TeX,Numbers=Lowercase]
\setmonofont{Noto Sans Mono}[Ligatures=TeX,Numbers=Lowercase]
\newfontface{\scrff}{Ardusan Script}[Path = ./]
\DeclareTextFontCommand{\scr}{\scrff}

% FontAwesome icons
\usepackage{fontawesome}

% Colors customization
\usepackage{xcolor}
\definecolor{sapphire}{HTML}{1A527C}
\definecolor{lightgray}{HTML}{777777}

% Reformat page headers
\pagestyle{Ruled}
\nouppercaseheads

% Reformat chapter titles
\chapterstyle{ell}

% Reformat section headings
\hangsecnum % Put section numbers in the margins
\setsecheadstyle{\LARGE\sffamily}
\setsubsecheadstyle{\Large\sffamily}
\setsubsubsecheadstyle{\large\sffamily}
\setsubsubsecindent{1em}
\setparaheadstyle{\sffamily\bfseries}
\setsubparaheadstyle{\sffamily\bfseries}

% Set Table of Contents to include subsections
% \setcounter{secnumdepth}{2}% Doesn't work for some reason
\setsecnumdepth{subsection}
\settocdepth{subsection}

% Increase amount of space between the number and entry in the Table of Contents
\setlength\cftpartnumwidth{2.5em}
\setlength\cftchapternumwidth{2.5em}

% Increase indent of section and subsection in Table of Contents
\setlength\cftsectionindent{2.5em}
\setlength\cftsubsectionindent{5em}

% Bibliography
\usepackage[backend=biber,natbib]{biblatex-chicago}
\addbibresource{bibliography.bib}

% Indexes
\usepackage{makeidx}
\makeindex

% Links
\usepackage{hyperref}
\hypersetup{
	% bookmarks=true,% show bookmarks bar?
	unicode=true,% non-Latin characters in Acrobat’s bookmarks
	pdftitle={Ardusa},% title
	pdfauthor={Ian A.~Cook},% author
	pdfsubject={A grammar of the Ardusan languages},% subject of the document
	pdfcreator={Ian A.~Cook},% creator of the document
	pdfproducer={Ian A.~Cook},% producer of the document
	pdfkeywords={Ardusa, language, linguistics, grammar},% list of keywords
	linktoc=all,% defines which part of the table of contents is made into a link
	colorlinks=true,% false: boxed links; true: colored links
	linkcolor=.,% color of internal links
	citecolor=.,% color of links to bibliography
	filecolor=.,% color of file links
	urlcolor=sapphire% color of external links
}
\urlstyle{same}

% Calculate remaining space in line for ex and pex
% see https://tex.stackexchange.com/a/376534
\newlength{\remaining}
\newcommand{\remainpex}{\setlength{\remaining}{\linewidth-\lingtextoffset-\linglabelwidth-\lingnumoffset-\linglabeloffset-\widthof{\exnoprint}}}
\newcommand{\remainex}{\setlength{\remaining}{\linewidth-\lingnumoffset-\lingtextoffset-\widthof{\exnoprint}}}
\pretocmd{\pex}{\remainpex}{}{}% Not working for some reason
\pretocmd{\ex}{\remainex}{}{}% Not working for some reason

% Glossaries
\usepackage[mcolblock,glosses,nomain,toc]{leipzig}
\makeglossaries
\newleipzig{aff}{aff}{affirmative, confirmation}
%\newleipzig{neg}{neg}{negative, negation}% Already defined
\newleipzig{act}{act}{active}
%\newleipzig{pass}{pass}{passive}% Already defined
%\newleipzig{antip}{antip}{antipassive}% Already defined
%\newleipzig{caus}{caus}{causative}% Already defined
%\newleipzig{refl}{refl}{reflexive}% Already defined
%\newleipzig{recp}{recp}{reciprocal}% Already defined
%\newleipzig{ind}{ind}{indicative}% Already defined
\newleipzig{sbjv}{sbjv}{subjunctive}% Already defined as subj
%\newleipzig{cond}{cond}{conditional}% Already defined
\newleipzig{opt}{opt}{optative}
\newleipzig{des}{des}{desiderative}
\newleipzig{tent}{tent}{tentative}
\newleipzig{pot}{pot}{potential}
\newleipzig{perm}{perm}{permissive}
\newleipzig{nec}{nec}{necessitative}
\newleipzig{int}{int}{interrogative}
%\newleipzig{imp}{imp}{imperative}% Already defined
%\newleipzig{pst}{pst}{past}% Already defined
\newleipzig{npst}{npst}{nonpast}
%\newleipzig{fut}{fut}{future}% Already defined
%\renewleipzig{prf}{prf}{perfect}% Already defined
\newleipzig{mom}{mom}{momentane}
\newleipzig{smlf}{smlf}{semelfactive}
\newleipzig{iter}{iter}{iterative}
\newleipzig{hab}{hab}{habitual}
\newleipzig{inc}{inc}{inceptive}
\newleipzig{prg}{prg}{progressive}
\newleipzig{cnt}{cnt}{continuative}
\newleipzig{rsm}{rsm}{resumptive}
%\newleipzig{dur}{dur}{durative}% Already defined
\newleipzig{pstv}{pstv}{pausative}
\newleipzig{term}{term}{terminative}
\newleipzig{rtsp}{rtsp}{retrospective}
\newleipzig{prsp}{prsp}{prospective}
\newleipzig{dsc}{dsc}{discontinuous}
\newleipzig{gno}{gno}{gnomic}
%\newleipzig{inf}{inf}{infinitive}% Already defined
%\newleipzig{ptcp}{ptcp}{participle}% Already defined
\newleipzig{grv}{grv}{gerundive}
\newleipzig{serg}{serg}{same ergative referent as previous clause}
\newleipzig{sabs}{sabs}{same absolutive referent as previous clause}
\newleipzig{sptv}{sptv}{same partitive referent as previous clause}
\newleipzig{psv}{psv}{positive}
\newleipzig{cmp}{cmp}{comparative}
\newleipzig{sup}{sup}{superlative}
\newleipzig{an}{an}{animate}
\newleipzig{in}{in}{inanimate}
\newleipzig{ab}{ab}{abstract}
\newleipzig{fp}{1}{first person}
\newleipzig[short={1\glstextup{s}}]{fps}{1s}{first person singular}
\newleipzig[short={1\glstextup{pc}}]{fpc}{1pc}{first person paucal}
\newleipzig[short={1\glstextup{p}}]{fpp}{1p}{first person plural}
\newleipzig[short={1\glstextup{p}i}]{fpi}{1pi}{first person plural inclusive}
\newleipzig[short={1\glstextup{p}e}]{fpe}{1pe}{first person plural exclusive}
\newleipzig{sp}{2}{second person}
\newleipzig[short={2\glstextup{s}}]{sps}{2s}{second person singular}
\newleipzig[short={2\glstextup{pc}}]{spc}{2pc}{second person paucal}
\newleipzig[short={2\glstextup{p}}]{spp}{2p}{second person plural}
\newleipzig{tp}{3}{third person}
\newleipzig[short={3\glstextup{s}}]{tps}{3s}{third person singular}
\newleipzig[short={3\glstextup{pc}}]{tpc}{3pc}{third person paucal}
\newleipzig[short={3\glstextup{p}}]{tpp}{3p}{third person plural}
\renewleipzig{prox}{prox}{proximate}% Already defined as proximal
\newleipzig{obv}{obv}{obviative}
%\newleipzig{indf}{indf}{indefinite}% Already defined
%\newleipzig{incl}{incl}{inclusive}% Already defined
%\newleipzig{excl}{excl}{exclusive}% Already defined
%\newleipzig{poss}{poss}{possessive}% Already defined
%\newleipzig{poss}{poss}{possessive}% Already defined
\newleipzig{ali}{ali}{alienable}
\newleipzig{inal}{inal}{inalienable}
%\newleipzig{sg}{sg}{singular}% Already defined
\newleipzig{pc}{pc}{paucal}
%\newleipzig{pl}{pl}{plural}% Already defined
%\newleipzig{def}{def}{definite}% Already defined
%\newleipzig{indf}{indf}{indefinite}% Already defined
\newleipzig{at}{at}{agent trigger}
\newleipzig{pt}{pt}{patient trigger}
\newleipzig{dir}{dir}{direct}
\newleipzig{idr}{idr}{indirect}
%\newleipzig{top}{top}{topic}% Already defined
%\newleipzig{nom}{nom}{nominative}% Already defined
%\newleipzig{acc}{acc}{accusative}% Already defined
%\newleipzig{erg}{erg}{ergative}% Already defined
%\newleipzig{abs}{abs}{absolutive}% Already defined
\newleipzig{ptv}{ptv}{partitive}
%\newleipzig{voc}{voc}{vocative}% Already defined
%\newleipzig{gen}{gen}{genitive}% Already defined
%\newleipzig{dat}{dat}{dative}% Already defined
\newleipzig{lat}{lat}{lative}
%\newleipzig{abl}{abl}{ablative}% Already defined
\newleipzig{pro}{pro}{prolative}
%\newleipzig{ins}{ins}{instrumental}% Already defined
%\newleipzig{ben}{ben}{benefactive}% Already defined
\newleipzig{cau}{cau}{causal}
%\newleipzig{com}{com}{comitative}% Already defined
\newleipzig{prv}{prv}{privative}
%\newleipzig{dist}{dist}{distal}% Already defined
\newleipzig{med}{med}{medial}
%\newleipzig{prox}{prox}{proximal}% Already defined
\newleipzig{crd}{crd}{cardinal}% Numbers
\newleipzig{ord}{ord}{ordinal}% Numbers
\newleipzig{part}{part}{partitive (number)}% Numbers
\newleipzig{mult}{mult}{multiplicative}% Numbers
\newleipzig{coll}{coll}{collective}% Numbers
%\newleipzig{distr}{distr}{distributive}% Numbers % Already defined
\newleipzig{sbst}{sbst}{substantive possessive}
%\newleipzig{rel}{rel}{relative}% Already defined
\newleipzig{rtrv}{rtrv}{restrictive}
\newleipzig{nrtrv}{nrtrv}{non-restrictive}
%\newleipzig{q}{q}{interrogative, question}% Already defined
\newleipzig{dim}{dim}{diminutive}
\newleipzig{aug}{aug}{augmentative}
\newleipzig{lau}{lau}{laudative}
\newleipzig{pej}{pej}{pejorative}
\newleipzig{nmz}{nmz}{nominalizer}

% Linguistics packages
\usepackage{vowel} % Draw vowel charts
\usepackage[linguistics]{forest} % Syntax trees
\usepackage{expex} % Examples and glosses
\usepackage{phonrule} % Phonological rules

% Define tree styles
\forestset{
	dotted tier/.style={
		where n children=0{tier=word,edge=dotted,calign with current edge}{}
	}
}

% Define macro to make ellipsis words in syntax trees light gray
\newcommand{\elps}{\color{lightgray}}

% Implement 3 levels of embedding in ExPex
\def\beginsubsub{%
	\par
	\begingroup
	\advance\leftskip by 2em
	\def\b##1{\par\leavevmode\llap{\hbox to 2em{##1\hfil}}\ignorespaces}}
\def\endsubsub{\par\endgroup}

% Linguistic conventions
% Square brackets for exact phonetic pronunciations
\newcommand{\phnt}[1]{[#1]}
% Slashes for approximate phonemic representations
\newcommand{\phnm}[1]{/#1/}
% Angle brackets for orthographic representations
\newcommand{\orth}[1]{⟨#1⟩}
% Quotes for translations
\newcommand{\defn}[1]{\enquote*{#1}}
% Parentheses for inline gloss
\newcommand{\gloss}[1]{(#1)}
% Astrisk for ungrammatical/incorrect
\newcommand{\ungr}{*}
% Question mark for questionable grammar
\newcommand{\ques}{\fakesuperscript{?}}
% Exclamation point for semantic oddity
\newcommand{\excl}{\fakesuperscript{!}}

% Commands for certain IPA symbols
% Primary stress
\newcommand{\pstrs}{ˈ}
% Secondary stress
\newcommand{\sstrs}{ˌ}
% Geminates
\newcommand{\gem}[1]{#1ː}
% Ejectives
\newcommand{\eje}[1]{#1'}
% Dentalized
\newcommand{\dent}[1]{#1̪}
% Non-syllabic
\newcommand{\nsyl}[1]{#1̯}
% Affricates
\newcommand{\affr}[2]{#1͡#2}

% Define macro so foreign words are italicized
\newcommand{\fw}[1]{\textit{#1}}

\newcommand{\acrnm}[1]{\textsc{#1}}

% Custom name commands
\newcommand{\landn}{Ardusa}
\newcommand{\landadj}{Ardusan}
\newcommand{\langtvk}{Tavonic}
\newcommand{\nlangtvk}{Tavonak}
\newcommand{\peoptvk}{Tavotath}
\newcommand{\npeoptvk}{Tavotaþ}
\newcommand{\langank}{Alnuric}
\newcommand{\nlangank}{Alnurek}
\newcommand{\peopank}{Alnureth}
\newcommand{\npeopank}{Alnureþ}
\newcommand{\langrdk}{Redodhic}
\newcommand{\nlangrdk}{Redoðik}
\newcommand{\peoprdk}{Redodhith}
\newcommand{\npeoprdk}{Redoðiþ}

% Title page data
\title{\landn}
\newcommand{\subtitle}{A Grammar of the \landadj{} Languages}
\author{Ian A.~Cook}
\date{\today}

% Define title page style
\makeatletter
\newcommand{\Titlep}{%
	\begingroup
	\centering
	{\Huge \@title}\\[\baselineskip]
	{\Huge \scr{\@title}}\\[\baselineskip]
	{\LARGE\textsc \subtitle{}}\\[\baselineskip]
	{\Large\textit{by \@author}}\\
	\vfill
	\textit{last edited}\\
	{\large \@date}\par
	\endgroup
}
\makeatother

% Write ref tags to external tag file
\gathertags

%%%%%%%%%%%%%%%%%%%%%%%%%%%%%%%%%%%%
% Document

\begin{document}

% Title leaf

\begin{titlingpage}
	% Title page
	\Titlep{}
	\clearpage
	% Title verso
	~\vfill
\makeatletter
{\setlength\parindent{0in}
{\large\scr{\@title}}\\
{\large\textbf{\fw{\@title}: \subtitle{}}}\\
by \@author{}\\[\baselineskip]

Copyright ©~2018--\the\year{} by \@author{}.\\
Last edited \today.\\[\baselineskip]

Typeset in Junicode and {\sffamily Fira Sans} with \XeLaTeX{}.\\[\baselineskip]

\landn{} is a fictional landmass set in a fictional constructed world. All of the languages spoken on \landn{}, such as \langtvk, \langank, \langrdk, and others, are themselves fictional, spoken by fictional groups of people, and as such are not related to any naturally existing languages. These languages' vocabularies are entirely \fw{a priori}, which means that no words are derived from the vocabularies of real-world languages. That being said, these languages are intended to be naturalistic, so similarities will occur. Nonetheless, any actual duplication is accidental.\\[\baselineskip]

\begin{tabu} to \linewidth {c X[l]}
	\faLink           & No website yet.\\
	\faCode            & \href{https://github.com/nai888/ardusa}{https://github.com/nai888/ardusa}\\
	\faCreativeCommons & This document is copyrighted ©~2018--\the\year{} by \@author{} under the \href{https://creativecommons.org/licenses/by-nc-sa/4.0/}{Creative Commons Attribution-NonCommercial-ShareAlike 4.0 license, \faCreativeCommons{} BY-NC-SA 4.0}.\\
	\faCopyright[regular] & The Ardusan Script and all languages described within this document are copyrighted ©~2018--\the\year{} by \@author{}, all rights reserved.
\end{tabu}
}
\makeatother
\end{titlingpage}

% Front matter

\frontmatter
\pdfbookmark[0]{\contentsname}{toc}\label{cha:toc}
\tableofcontents*
\clearpage
\listoffigures\label{cha:figures}
\clearpage
\listoftables\label{cha:tables}
\clearpage
\printglosses\label{cha:glossary}
\bigskip
\noindent\begin{tabular}{@{} l l}
\ungr & ungrammatical\\
\ques & grammatically questionable\\
\excl & semantically odd or ill-formed\\
\end{tabular}
\clearpage

% Acknowledgments

\chapter{Acknowledgments}
\label{cha:acknowledgments}

Given that I have not taken any official linguistics coursework, this work would not be possible without several sources of linguistic education. Mark Rosenfelder's \textit{The Language Construction Kit}\nocite{mrlck} and \textit{Advanced Language Construction Kit}\nocite{mralck} were important to my first starting out in the world of language construction, with further knowledge gained from David J.~Peterson's \textit{The Art of Language Invention}\nocite{djpali}. Of course, I received an unmeasurable amount of education via several online sources, especially the articles available on Wikipedia. Yet more education, as well as inspiration and motivation, have come from the \textit{Conlangery} podcast and all its hosts and guests. Lexicon generation received guidance from Mark Rosenfelder's \textit{The Conlanger's Lexipedia}\nocite{mrcl} and William S.~Annis' \textit{A Conlanger's Thesaurus}\nocite{wsact}.

Finally, this document's format, layout, and organization have been influenced by several sources, particularly Thomas E.~Payne's \textit{Describing Morphosyntax}\nocite{descms}, Carsten Becker's \textit{A Grammar of Ayeri}\nocite{ayeri}, and Matt Pearson's \textit{The Okuna Reference Grammar}\nocite{okuna}.

% Preface

\chapter{Preface}
\label{cha:preface}

This document provides a detailed grammatical description of the languages of \landn, a fictional landmass set in a fictional constructed world. This project serves as a method for linguistic research, as an intellectual exercise, as an outlet for creative and artistic expression, and as a setting for potential future works of fiction. It is intended primarily for my own personal use and entertainment, though others with similar linguistic interests will hopefully find it interesting and entertaining as well. I have chosen to use \LaTeX{} to typeset this grammar because it provides a way to be clear, consistent, and organized. Further, since \LaTeX{} uses plain text files, it allows me to use Git for version control so I can keep track of changes over time.

My goal is to build a series of languages with naturalistic grammars that are linguistically plausible and consistent, yet also original in their content and details. This project consists of three distinct and unrelated language families, each of which contains one or more related languages. Some elements of these languages are influenced by existing languages such as Japanese, Finnish, Navajo, Nahuatl, and Arabic, but they are not meant to simply mimic these, instead drawing this inspiration into new forms along with entirely \fw{a priori} lexicons. \landn{} and the \landadj{} languages is an ongoing project with no fixed endpoint or goal.

This concise grammar is my attempt to document the \landadj{} languages in an official and systematic way, and as comprehensively as possible. It is intended to be the official description of the languages. This is a concise grammar because, admittedly, I am not a professional linguist, nor have I taken any linguistics coursework. My education in linguistics consists solely of self-guided research, which means invariably my knowledge will be limited. It is a concise grammar because, frankly, I don't know enough to go into greater detail. That being said, I'm always eager to learn, and will always accept feedback. Again, learning is one of the reasons for this endeavor.

Since the purpose of writing this grammar is to provide a comprehensive description of the \landadj{} languages, not to teach them to others, it is not intended to serve as a textbook or as a way to learn the languages. I have organized topics thematically, rather than curricularly, and I employ technical terms when they are precise, accurate, and appropriate. I have not conducted a formal analysis of the languages, but I have worked to make it as descriptive as possible.

The discussion is ordered from the smallest elements of the languages to the largest. It begins with a description of each language's place in \landn{} followed by their phonologies, it addresses morphology and the combining of words, it discusses vocabulary and derivation, and it explains syntax and discourse. The final chapter serves as a reference grammar, summarizing all of the previous chapters. There are also several appendices describing the conceptual metaphors that organize much of the lexicons, the naming practices of the fictional speakers of these languages, several translation examples, and lexicons. Other resources include a glossary of linguistic glossing abbreviations, a bibliography, and an index.

This document uses several linguistics conventions to clarify meaning. Any reference to specific orthographic spelling is marked with angled brackets, such as \orth{hin}. Pronunciations are usually given phonemically, in which case they are marked with slashes, such as \phnm{hin}. Phonetic pronunciations are used only when conveying specific details like the difference between allophones, and are marked with square brackets, such as \phnt{çin}. Both phonemic and phonetic pronunciations are given using the International Phonetic Alphabet. Foreign words are always written in italics, such as \fw{lu}. English glosses are surrounded by single quotes, such as \defn{and}. If a morphological gloss is provided in-line, it is surrounded by parentheses, such as \gloss{\Inf}.

Many short examples are provided in one single line.

\ex<ex:prfshortgloss>
	\langtvk: \scr{šek} \fw{šek} \phnm{ʃek} \defn{ran} \gloss{run-\Ind.\Pst.\Pfv}
\xe

Longer examples are usually provided with a multi-line, or interlinear, gloss. In these examples, the optional first line will indicate which language the example is in, if it is not clear from context. The next two lines present the text in that language, one in the Ardusan Script and one using the romanization, followed by the pronunciation. After this, the text is broken into its component morphemes, and the following line provides a morpheme-by-morpheme gloss. The final line provides an English translation of the example phrase or sentence.

\ex<ex:prf-fullgloss>
	\begingl
		\glpreamble \langtvk\\
		\scr{Nan oko šeðo.}\\
		\fw{Nan oko šeðo.}\\
		\phnm{nan o\pstrs ko \pstrs ʃe.ðo}//
		\gla nan= oko š-eðo//
		\glb \Pl.\An.\Top= dog run-\Ind.\Pst.\Prg//
		\glft \defn{The dogs were running.}//
	\endgl
\xe

As shown in example \getfullref{ex:prf-fullgloss}, morpheme glosses are labeled with abbreviations in \textsc{small caps}. A full list of all glossing abbreviations is given on page \pageref{cha:glossary}. A hyphen marks a morpheme boundary within a word that is shared between the text and its gloss, while a period marks a boundary present in only one or the other, including when a single word in the text corresponds to multiple words in its gloss. Clitics are marked with an equals sign, reduplication with a tilde, discontinuous affixes (e.g., infixes, circumfixes) with angle brackets, and morphemes that cannot be easily separated out with backslashes.

The \LaTeX{} source code for this grammar and a copy of this PDF are available in a public \href{https://github.com/nai888/ardusa}{\faGithub~GitHub} repository. Undoubtedly, there will be errors in this document. If you notice any, please feel free to open an issue in the GitHub repository with a description and the location of the error.

\begin{flushright}
	\makeatletter
	\textit{\@author}\\
	\textit{Minneapolis, \DTMdisplaydate{2018}{9}{8}}
	\makeatother
\end{flushright}

% Main matter

\mainmatter

\part{Tavonic Family: Tavonic}

\excnt=1% Reset the numbering of examples to 1
\chapter{History and Ethnography}
\label{cha:tvk-ethnography}

This chapter will present a brief history of the \langtvk{} language family, followed by a short description of its ethnolinguistic context.

\section{Brief History}
\label{sec:tvk-history}

The \peoptvk{} (the \langtvk{} people) migrated to \landn{} hundreds of years ago in what they termed Year 1 of the \landadj{} Era (\acrnm{ae}). \landn{} is far from any other landmasses and is isolated from the influence of other lands and other peoples. The \peoptvk{} landed in the warm southeastern regions of \landn{} where they first established their new home, naming this new realm \fw{Urdeso}, a compound word meaning \defn{Safe Land}. Over the following centuries, the \peoptvk{} spread westward and northward throughout the whole of \landn.

As the \peoptvk{} spread, they formed several individual territories, each of which eventually developed into small kingdoms. These kingdoms constantly battled one another for power, and borders were continually shifting. Those who fled the fighting fled northward, furthering the \langtvk{} expansion throughout \landn. As the \peoptvk{} spread farther apart and splintered, their language diverged. Two main dialects emerged, one in the north and one in the south.

After a few hundred years, one kingdom in the south emerged as dominant, conquering or allying with more and more kingdoms until, by 327~\acrnm{ae}, the entire south of \landn{} was united under one empire. This empire enforced the usage of the language that had emerged in the south, thus forming the \langank{} language. The empire continued to push northward until it spread too thin and reached a stalemate with the allied kingdoms in the north around 371~\acrnm{ae}. Finally, in 582~\acrnm{ae} after a couple hundred years of relatively stable rule, the empire declined and divided again into individual territories, leaving behind six sovereign kingdoms.

While the empire was emerging in the south, the kingdoms in the north formed a loose alliance to resist its spread. The alliance managed to reach a stalemate with the empire, stopping its spread northward. The allied kingdoms together maintained the language that emerged in the north, thus forming the \langrdk{} language. Eventually, as the empire split in 582~\acrnm{ae} and the northern alliance was no longer needed, the north also split into individual territories, leaving behind four sovereign kingdoms.

\section{Ethnography}
\label{sec:tvk-ethnography}

This section will attempt to place the \langtvk{} languages within their ethnolinguistic context.

\subsection{Demonyms and Language Names}
\label{subsec:tvk-demonyms}

\paragraph{\langtvk}

The \peoptvk{} were a tribe that migrated to \landn{} together, fleeing their previous home. The \langtvk{} word \fw{tavo} \phnm{ta\pstrs vo} means \defn{person}, and so the derived word \fw{\npeoptvk} \phnm{ta.vo\pstrs taθ} means \defn{people} or \defn{tribe}. In other words, the \peoptvk{} referred to themselves as the People, with \fw{\nlangtvk} being the Language of the People. The \langank- and \langrdk-derived words, \fw{Tevodeþ} \phnm{te.vo\pstrs deθ} and \fw{Tovujiþ} \phnm{to.vu\pstrs \affr{d}{ʒ}iθ} respectively, refer to all people who descended from the original \peoptvk{} tribe. Both \langank{} and \langrdk{} are \peoptvk{} languages and part of the \langtvk{} language family.

\paragraph{\langank}

For hundreds of years, the empire ruled in the southern region of \landn. The \langtvk{} word \fw{unner} \phnm{un\pstrs ner} \defn{empire} evolved into the \langank{} word \fw{alnur} \phnm{al\pstrs nur}. \fw{\nlangank} \phnm{al.nu\pstrs rek} \defn{\langank} takes its name from this word. Meanwhile, the \langrdk{} name for the empire is \fw{nonar} \phnm{no\pstrs nar}, and its name for the \langank{} language is \fw{Nonrik} \phnm{non\pstrs rik}. Similarly, the \langank{} and \langrdk{} names for the \langank{} people are \fw{Alnureþ} \phnm{al.nu\pstrs reþ} and \fw{Nonriþ} \phnm{non\pstrs riþ} respectively.

\paragraph{\langrdk}

In the north, the alliance resisted the empire's expansion. The \langtvk{} word \fw{aroltutaþ} \phnm{a\sstrs rol.tu\pstrs taθ} signifies \defn{alliance}, however the alliance instead used the simpler form \fw{arutaþ} \phnm{a.ru\pstrs taθ} \defn{standers} to signify the alliance of those kingdoms standing against the empire. \fw{Arutaþ} evolved into the \langrdk{} word \fw{rejiþ} \phnm{re\pstrs\affr{d}{ʒ}iθ}, and \fw{\nlangrdk} \phnm{re.do\pstrs ðik} \defn{\langrdk} takes its name from this word. The \langank{} name for the alliance is \fw{eradeþ} \phnm{e.ra\pstrs deþ}, and its name for the \langrdk{} language is \fw{Eratþek} \phnm{e.rat\pstrs θek}. Similarly, the \langrdk{} and \langank{} names for the \langrdk{} people are \fw{Redoðiþ} \phnm{re.do\pstrs ðiþ} and \fw{Eratþeþ} \phnm{e.rat\pstrs þeþ} respectively.

\excnt=1% Reset the numbering of examples to 1
\chapter{Phonology}
\label{cha:tvk-phonology}

This chapter will present the inventory of consonants and vowels. An observational analysis of the \langtvk{} languages' syllable structures and phonotactics will follow. The chapter will close with notes on syllable stress within words and a brief exploration of intonation.

\section{\langtvk{} Phoneme Inventory}
\label{sec:tvk-phone-inventory}

\subsection{Consonants}
\label{subsec:tvk-consonants}

\afterpage{\clearpage
	\begin{sidewaystable}
		\scriptsize
		\index{consonants!inventory}\index{allophony}\index{consonants!allophones|see {allophony}}
		\caption[\langtvk{} Consonant Inventory]{\langtvk{} Phonetic Consonant Inventory (allophones in parentheses)}
		\label{tab:tvk-consonants}
		\begin{tabu} to \textheight {| r | X[c] X[c] X[c] X[c] X[c] X[c] X[c] X[c] X[c] X[c] X[c] X[c] X[c] X[c] X[c]}
			\toprule
			Consonants
			& \multicolumn{2}{c}{Bilabial}
			& \multicolumn{2}{c}{Labio-dental}
			& \multicolumn{2}{c}{Dental}
			& \multicolumn{2}{c}{Alveolar}
			& \multicolumn{2}{c}{Post-alveolar}
			& \multicolumn{2}{c}{Velar}
			\\
			\midrule
			Nasal
			&      & m    % Bilabial
			&      &      % Labiodental
			&      &      % Dental
			&      & n    % Alveolar
			&      &      % Post-alveolar
			&      & (ŋ)  % Velar
			\\
			\midrule
			Plosive
			&      &      % Bilabial
			& p    & b    % Labiodental
			& t    & d    % Dental
			&      &      % Alveolar
			&      &      % Post-alveolar
			& k    & g    % Velar
			\\
			\midrule
			Fricative
			&      &      % Bilabial
			& f    & v    % Labiodental
			& θ    & ð    % Dental
			& s    & z    % Alveolar
			& ʃ    & ʒ    % Post-alveolar
			& x    & ɣ    % Velar
			\\
			\midrule
			Flap/Tap
			&      &      % Bilabial
			&      &      % Labiodental
			&      &      % Dental
			&      & ɾ    % Alveolar
			&      &      % Post-alveolar
			&      &      % Velar
			\\
			\midrule
			Trill
			&      &      % Bilabial
			&      &      % Labiodental
			&      &      % Dental
			&      & (r)  % Alveolar
			&      &      % Post-alveolar
			&      &      % Velar
			\\
			\midrule
			Approximant
			&      &      % Bilabial
			&      &      % Labiodental
			&      &      % Dental
			&      & (ɹ)  % Alveolar
			&      &      % Post-alveolar
			&      &      % Velar
			\\
			\midrule
			Lateral
			&      &      % Bilabial
			&      &      % Labiodental
			&      &      % Dental
			&      & l    % Alveolar
			&      &      % Post-alveolar
			&      &      % Velar
			\\
			\bottomrule
		\end{tabu}
	\end{sidewaystable}
	\clearpage
	\index{consonants!romanization}
	\begin{longtabu} to \textwidth {c c c c X[l]}
		\caption{\langtvk{} Consonant Romanization}\label{tab:tvk-consromanization}\\
		\toprule
		Phone & Phoneme & Romanization & English & Notes\\
		\midrule
		\endhead
		\multicolumn{4}{r}{\textit{continued on the next page\ldots}}\\
		\endfoot
		\bottomrule
		\endlastfoot
		\phnt{m} & \phnm{m} & \orth{m} & \orth{m} & \\
		\midrule
		\phnt{n} & \phnm{n} & \orth{n} & \orth{n} & \\
		\midrule
		\phnt{ŋ} & \phnm{n} & \orth{n} & \orth{n} & \phnm{n} becomes velarized before a velar consonant\\
		\midrule
		\phnt{p} & \phnm{p} & \orth{p} & \orth{p} & \\
		\midrule
		\phnt{b} & \phnm{b} & \orth{b} & \orth{b} & \\
		\midrule
		\phnt{t} & \phnm{t} & \orth{t} & \orth{t} & \\
		\midrule
		\phnt{d} & \phnm{d} & \orth{d} & \orth{d} & \\
		\midrule
		\phnt{k} & \phnm{k} & \orth{k} & \orth{k} & \\
		\midrule
		\phnt{g} & \phnm{g} & \orth{g} & \orth{g} & \\
		\midrule
		\phnt{f} & \phnm{f} & \orth{f} & \orth{f} & \\
		\midrule
		\phnt{v} & \phnm{v} & \orth{v} & \orth{v} & \\
		\midrule
		\phnt{θ} & \phnm{θ} & \orth{þ} & \orth{th} & \\
		\midrule
		\phnt{ð} & \phnm{ð} & \orth{ð} & \orth{dh} & \\
		\midrule
		\phnt{s} & \phnm{s} & \orth{s} & \orth{s} & \\
		\midrule
		\phnt{z} & \phnm{z} & \orth{z} & \orth{z} & \\
		\midrule
		\phnt{ʃ} & \phnm{ʃ} & \orth{š} & \orth{sh} & \\
		\midrule
		\phnt{ʒ} & \phnm{ʒ} & \orth{ž} & \orth{zh} & \\
		\midrule
		\phnt{x} & \phnm{x} & \orth{ǩ} & \orth{kh} & \\
		\midrule
		\phnt{ɣ} & \phnm{ɣ} & \orth{ǧ} & \orth{gh} & \\
		\midrule
		\phnt{ɾ} & \phnm{r} & \orth{r} & \orth{r} & \\
		\midrule
		\phnt{r} & \phnm{r} & \orth{rr} & \orth{rr} & \orth{r} is trilled when doubled \\
		\midrule
		\phnt{ɹ} & \phnm{r} & \orth{r} & \orth{r} & \orth{r} is occasionally pronounced as an approximant when a part of a consonant cluster \\
		\midrule
		\phnt{l} & \phnm{l} & \orth{l} & \orth{l} & \\
	\end{longtabu}
	\clearpage
}

With approximately 20 consonants, \langtvk{} has an \enquote{average} inventory.\autocite{wals-1} \autoref{tab:tvk-consonants} shows the full chart of consonant phonemes, along with several allophones enclosed in parentheses. \autoref{tab:tvk-consromanization} shows how each consonant in \langtvk{} is romanized.

Despite its \enquote{average} inventory of consonants, there are many more allophones\index{allophony} that occur in the language. First, any doubled consonant is realized as a geminated\index{consonants!gemination} (elongated) consonant.

\pex<gemcons>
	\fw{unner} \phnm{u\gem{n}er} \defn{empire}
\xe

Thus, example~\getfullref{gemcons} above is realized with a lengthened \phnt{n}. A doubled \orth{r} is similarly geminated, but the pronunciation changes from a flap/tap to a trill.

The remaining allophones\index{allophony} occur due to various sound change processes, mostly by assimilation. For example, \phnm{n} becomes velarized when it appears immediately before a velar consonant.

\ex<velarn>
	\fw{tavonga} \phnt{ta.voŋ\pstrs ga} \defn{humanlike}
\xe

As discussed above, \orth{r} can be pronounced as both a tap/flap \phnt{ɾ} and as a trill \phnt{r}. Additionally, when part of certain consonant clusters, it can be pronounced as an approximant \phnt{ɹ}. This primarily occurs when the \orth{r} leads into a cluster or immediately follows a nasal.

\ex<velarn>
	\fw{frorgali} \phnt{fɾoɹ.\pstrs ga.li} \defn{to un-see}
\xe

\subsection{Vowels}
\label{subsec:tvk-vowels}

\afterpage{\clearpage
	\begin{table}\centering
		\index{vowels!inventory}
		\caption{\langtvk{} Vowel Inventory}
		\label{tab:tvk-vowels}
		{\large
			\begin{vowel}
				\putcvowel{i}{1}
				\putcvowel{e}{2}
				\putcvowel{a}{4}
				\putcvowel{o}{7}
				\putcvowel{u}{8}
			\end{vowel}
		}
	\end{table}
}

\langtvk{} distinguishes five vowel qualities, as shown in \autoref{tab:tvk-vowels}, giving it an \enquote{average} inventory.\autocite{wals-2} This means the consonant--vowel ratio is 20:5 or 4.0, which is \enquote{average}.\autocite{wals-3} \langtvk{} does not distinguish long and short vowels and does not allow any diphthongs.

Note that all \langtvk{} vowels have a very rigid acceptable pronunciation with very little variance.

\pex<gemvowels>
	\a<i> \fw{akrinsali} \defn{to rewrite} is pronounced \phnm{ak.rin\pstrs sa.li}. \orth{i} is not pronounced with a lax \phnt{ɪ} in closed syllables (i.e., \phnm{ak.rɪn\pstrs sa.li})
	\a<e> \fw{tloþevem} \defn{permission} is pronounced \phnm{tlo.θe\pstrs vem}. \orth{e} is not pronounced with a central \phnt{ə} in unaccented syllables or an open \phnt{ɛ} in closed syllables (i.e., \phnm{tlo.θə\pstrs vɛm}), nor is it diphthongized to \phnt{e\nsyl{ɪ}} (i.e., \phnm{tlo.θe\pstrs ve\nsyl{ɪ}m})
	\a<a> \fw{ǩalo} \defn{man} is pronounced \phnm{xa\pstrs lo}. \orth{a} is not pronounced with a raised \phnt{æ} (i.e., \phnm{xæ\pstrs lo}), a backed \phnt{ɑ} (i.e., \phnm{xɑ\pstrs lo}), or a centralized \phnt{ɜ} (i.e., \phnm{xɜ\pstrs lo})
	\a<o> \fw{esondi} \defn{arable} is pronounced \phnm{e.son\pstrs di}. \orth{o} is not pronounced with an open \phnt{ɔ} (i.e., \phnt{e.sɔn\pstrs di}), nor is it diphthongized to \phnt{o\nsyl{u}} (i.e., \phnm{e.so\nsyl{u}n\pstrs di})
	\a<u> \fw{frumbali} \defn{to misunderstand} is pronounced \phnm{frum\pstrs ba.li}. \orth{u} is not pronounced with an open \phnt{ʌ} (i.e., \phnm{frʌm\pstrs ba.li}) or a centralized \phnt{ʊ} (i.e., \phnm{frʊm\pstrs ba.li})
\xe

\section{\langtvk{} Phonotactics}
\label{sec:tvk-phonotactics}

At the time of writing, there does not yet exist a sufficient corpus for a meaningful statistical analysis of \langtvk's phonotactics. Therefore, this section will present only a cursory observational analysis.

\subsection{Syllable Structures}
\label{subsec:tvk-syll-struc}

Placeholder

\subsection{Phonological Changes}
\label{subsec:tvk-phone-changes}

Placeholder

\subsection{Syllable Parsing}
\label{subsec:tvk-syll-parse}

Placeholder

\subsection{Number of Syllables per Word}
\label{subsec:tvk-num-syll}

Placeholder

\section{\langtvk{} Prosody}
\label{sec:tvk-prosody}

Placeholder

\subsection{Syllable Weight}
\label{subsec:tvk-syll-weight}

Placeholder

\subsection{Word Stress}
\label{subsec:tvk-word-stress}

Placeholder

\subsection{Intonation}
\label{subsec:tvk-intonation}

Placeholder

\section{\langank{} Phoneme Inventory}
\label{sec:ank-phone-inventory}

Placeholder

\subsection{Consonants}
\label{subsec:ank-consonants}

Placeholder

\subsection{Vowels}
\label{subsec:ank-vowels}

Placeholder

\section{\langank{} Phonotactics}
\label{sec:ank-phonotactics}

Placeholder

\subsection{Syllable Structures}
\label{subsec:ank-syll-struc}

Placeholder

\subsection{Phonological Changes}
\label{subsec:ank-phone-changes}

Placeholder

\subsection{Syllable Parsing}
\label{subsec:ank-syll-parse}

Placeholder

\subsection{Number of Syllables per Word}
\label{subsec:ank-num-syll}

Placeholder

\section{\langank{} Prosody}
\label{sec:ank-prosody}

Placeholder

\subsection{Syllable Weight}
\label{subsec:ank-syll-weight}

Placeholder

\subsection{Word Stress}
\label{subsec:ank-word-stress}

Placeholder

\subsection{Intonation}
\label{subsec:ank-intonation}

Placeholder

\section{\langrdk{} Phoneme Inventory}
\label{sec:rdk-phone-inventory}

Placeholder

\subsection{Consonants}
\label{subsec:rdk-consonants}

Placeholder

\subsection{Vowels}
\label{subsec:rdk-vowels}

Placeholder

\section{\langrdk{} Phonotactics}
\label{sec:rdk-phonotactics}

Placeholder

\subsection{Syllable Structures}
\label{subsec:rdk-syll-struc}

Placeholder

\subsection{Phonological Changes}
\label{subsec:rdk-phone-changes}

Placeholder

\subsection{Syllable Parsing}
\label{subsec:rdk-syll-parse}

Placeholder

\subsection{Number of Syllables per Word}
\label{subsec:rdk-num-syll}

Placeholder

\section{\langrdk{} Prosody}
\label{sec:rdk-prosody}

Placeholder

\subsection{Syllable Weight}
\label{subsec:rdk-syll-weight}

Placeholder

\subsection{Word Stress}
\label{subsec:rdk-word-stress}

Placeholder

\subsection{Intonation}
\label{subsec:rdk-intonation}

Placeholder

\excnt=1% Reset the numbering of examples to 1
\chapter{Morphological Typology}
\label{cha:tvk-morphological-typology}

\section{Typology}
\label{sec:tvk-typology}

\section{Morphological Processes}
\label{sec:tvk-morphological-processes}

\subsection{Prefixation}
\label{sec:tvk-prefixation}

\subsection{Suffixation}
\label{sec:tvk-suffixation}

\subsection{Clitics}
\label{sec:tvk-clitics}

\section{Marking Strategies}
\label{sec:tvk-marking-strategies}


\excnt=1% Reset the numbering of examples to 1
\chapter{Grammatical Categories}
\label{cha:tvk-grammatical-categories}

\langtvk{} words can be divided into several different categories, or parts of speech. While the previous chapter dealt with the general mechanisms of marking words, this chapter will examine each of the various parts of speech in order to define their morphology more closely. The discussion will begin with an examination of nouns, pronouns, and verbs. Following this will be a discussion of the remaining parts of speech, including adverbs, numerals, and conjunctions.

\section{Nouns}
\label{sec:tvk-nouns}

Nouns in \langtvk{} decline to express number and gender (animacy) and are marked for case to indicate their grammatical role within the clause. As discussed in \autoref{cha:tvk-morphological-typology}, this inflection takes place not directly on the noun itself but on prepositional clitics that convey this grammatical meaning. For a full illustration of the declension paradigms, compare \autoref{tab:tvk-an-vowel-decl} and \autoref{tab:tvk-in-vowel-decl}. As shown in these tables, \langtvk{} noun inflections are never syncretic\autocite{wals-28}.

\afterpage{\clearpage
	\begin{table}\centering
		\caption[\langtvk{} Animate Noun Declension Paradigm]{\langtvk{} Animate Noun Declension Paradigm for the word \fw{bruþa} \defn{hand, tool}}
		\label{tab:tvk-an-vowel-decl}
		\begin{tabu}{| l | l l l |}
			\toprule
			\rowfont[c]\bfseries & \Sg & \Pc & \Pl\\
			\midrule
			\textbf{\Abs} & \fw{bruþa} & \fw{ri bruþa} & \fw{ran bruþa}\\
			\textbf{\Erg} & \fw{do bruþa} & \fw{das bruþa} & \fw{din bruþa}\\
			\textbf{\Acc} & \fw{tu bruþa} & \fw{tos bruþa} & \fw{ton bruþa}\\
			\textbf{\Dat} & \fw{ke bruþa} & \fw{kas bruþa} & \fw{ken bruþa}\\
			\textbf{\Gen} & \fw{su bruþa} & \fw{sar bruþa} & \fw{san bruþa}\\
			\midrule
			\textbf{\Top} & \fw{no bruþa} & \fw{nas bruþa} & \fw{nan bruþa}\\
			\textbf{\Top.\Acc} & \fw{nut bruþa} & \fw{nutos bruþa} & \fw{nuton bruþa}\\
			\textbf{\Top.\Dat} & \fw{nek bruþa} & \fw{nekas bruþa} & \fw{naken bruþa}\\
			\textbf{\Top.\Gen} & \fw{nus bruþa} & \fw{nosar bruþa} & \fw{nosan bruþa}\\
			\bottomrule
		\end{tabu}
	\end{table}
	\begin{table}\centering
		\caption[\langtvk{} Inanimate Noun Declension Paradigm]{\langtvk{} Inanimate Noun Declension Paradigm for the word \fw{šem} \defn{busyness}}
		\label{tab:tvk-in-vowel-decl}
		\begin{tabu}{| l | l l l |}
			\toprule
			\rowfont[c]\bfseries & \Sg & \Pc & \Pl\\
			\midrule
			\textbf{\Abs} & \fw{šem} & \fw{le šem} & \fw{ren šem}\\
			\textbf{\Erg} & \fw{ða šem} & \fw{ðes šem} & \fw{dun šem}\\
			\textbf{\Acc} & \fw{ti šem} & \fw{þis šem} & \fw{ten šem}\\
			\textbf{\Dat} & \fw{ǩo šem} & \fw{kos šem} & \fw{ǩun šem}\\
			\textbf{\Gen} & \fw{šo šem} & \fw{se šem} & \fw{šen šem}\\
			\midrule
			\textbf{\Top} & \fw{mi šem} & \fw{mes šem} & \fw{nun šem}\\
			\textbf{\Top.\Acc} & \fw{mati šem} & \fw{moþes šem} & \fw{noten šem}\\
			\textbf{\Top.\Dat} & \fw{moǩ šem} & \fw{mekos šem} & \fw{nikun šem}\\
			\textbf{\Top.\Gen} & \fw{miš šem} & \fw{mise šem} & \fw{nušen šem}\\
			\bottomrule
		\end{tabu}
	\end{table}
}

\subsection{Gender}
\label{subsec:tvk-nouns-gender}

Grammatical gender in \langtvk{} consists of two\autocite{wals-30} non-sex-based\autocite{wals-31} classes based primarily on semantic ontological properties\autocite{wals-32}. The animate gender refers primarily to entities that are considered alive or are associated with life, movement, change, or dynamism. The inanimate gender refers primarily to entities that are not alive and are generally stationary or abstract. Grammatical gender in \langtvk{} can also be referred to as \enquote{animacy} since that is what the genders denote. Examples of nouns in each gender can be seen in example~\getref{ex:tvk-noun-genders}.

\pex<ex:tvk-noun-genders>
	\a<an>Animate nouns:\\
		\fw{botra} \defn{woman}, \fw{ǩalo} \defn{man}, \fw{eson} \defn{farmer}, \fw{okotik} \defn{puppy}, \fw{urdatil} \defn{ward}, \fw{bilt} \defn{breath}
	\a<in>Inanimate nouns:\\
		\fw{esotik} \defn{country}, \fw{dedu} \defn{sky}, \fw{elbi} \defn{egg}, \fw{usudir} \defn{basket}, \fw{akrapis} \defn{letter}, \fw{fradir} \defn{glasses}
\xe

Since the nouns themselves are not directly inflected, with grammatical information instead shown on prepositional particles, it is impossible to tell what gender a noun is based solely on its word form.

Some nouns are able to change category in certain circumstances. For example, plants and animals switch from the animate gender to the inanimate gender when they serve as food. Further, there exist some duplicates with otherwise identical words declining to opposite genders.


\excnt=1% Reset the numbering of examples to 1
\chapter{Syntax}
\label{cha:tvk-syntax}

How do words go together?

\excnt=1% Reset the numbering of examples to 1
\chapter{Lexical Operations}
\label{cha:tvk-lex-operations}

\section{Compounding}
\label{sec:tvk-lex-compounding}

How does compounding work?

\section{Derivation}
\label{sec:tvk-lex-derivation}

How do you make new words?

\excnt=1% Reset the numbering of examples to 1
\chapter{Discourse}
\label{cha:tvk-discourse}

How does conversation work?

\section{Topic}
\label{sec:tvk-discourse-topic}


\excnt=1% Reset the numbering of examples to 1
\chapter{Sociolinguistic Context}
\label{app:tvk-sociolinguistic-context}

\section{Conceptual Metaphors}
\label{sec:tvk-conceptual-metaphors}

What metaphors do the vocabulary convey?

For example: language is a tool. I speak \textit{with} or \textit{using} \langtvk, rather than just speaking \langtvk.

\section{Kinship Terms}
\label{sec:tvk-kinship-terms}
\index{kinship|(}

\textbf{This section needs to be overhauled.}

The \langtvk{} kinship system is large descriptive, with only a few classificatory terms. Siblings are distinguished from cousins, and parallel cousins are distinguished from cross cousins. Siblings and parallel cousins are identified by gender, while cross cousins are not. Parallel aunts and uncles are distinguished from cross aunts and uncles. Grandparents are identified by gender, but are otherwise undistinguished. Children and grandchildren are similarly identified by gender but otherwise undistinguished. See \autoref{fig:kinship} for a full kinship tree.

All of the kinship terms within a nuclear family have distinct names distinguishing gender and generation.

\exdisplay\noexno
\begin{tabu} {l l}
	mother & \fw{júói}\\
	father & \fw{uutói}\\
	sister & \fw{náatói}\\
	brother & \fw{toóbói}\\
	wife & \fw{ólói}\\
	husband & \fw{udói}\\
	daughter & \fw{lárrói}\\
	son & \fw{zohiói}\\
\end{tabu}
\xe

Relation by marriage is expressed with a prefix \fw{wen-}. This prefix can be added to several terms, such as \defn{sister}, \defn{brother}, \defn{daughter}, and \defn{son}.

\exdisplay\noexno
\begin{tabu} {l l}
	mother-in-law & \fw{wenjúói}\\
	father-in-law & \fw{wenuutói}\\
	sister-in-law & \fw{wennáatói}\\
	brother-in-law & \fw{wentoóbói}\\
	daughter-in-law & \fw{wenlárrói}\\
	son-in-law & \fw{wenzohiói}\\
\end{tabu}
\xe

Terms for one's nieces and nephews are derived from a combination of the terms for \defn{sister} or \defn{brother} and the terms for \defn{daughter} or \defn{son}.

\exdisplay\noexno
\begin{tabu} {l l l}
	niece & \fw{náalár} & sister's daughter\\
	niece & \fw{toólár} & brother's daughter\\
	niece-in-law & \fw{wennáalár} & sister's daughter-in-law\\
	niece-in-law & \fw{wentoólár} & brother's daughter-in-law\\
	nephew & \fw{náazo} & sister's son\\
	nephew & \fw{toózo} & brother's son\\
	nephew-in-law & \fw{wennáazo} & sister's son-in-law\\
	nephew-in-law & \fw{wentoózo} & brother's son-in-law\\
	niefling & \fw{tírtrabói} & gender-neutral term for niece or nephew\\
\end{tabu}
\xe

One's grandchildren are distinguished by gender, but not by their parents. In other words, one's daughter's daughter is called the same term as one's son's daughter. As discussed in \autoref{cha:tvk-grammatical-categories}, these terms are derived using reduplication\index{reduplication}.

\exdisplay\noexno
\begin{tabu} {l l}
	granddaughter & \fw{lálárrói}\\
	grandson & \fw{zozohiói}\\
\end{tabu}
\xe

The children of one's nieces and nephews are all called \fw{tírtrabói}, regardless of their gender. This term is identical to the gender-neutral term for one's nieces and nephews.

\langtvk{} distinguishes between parallel and cross aunts and uncles. In other words, one's mother's sister is called differently than one's father's sister. In fact, the term for cross aunts is identical to the term for parallel aunt-in-laws, and the term for cross uncles is identical to the term for parallel uncle-in-laws. Similar to the terms for grandchildren, the terms for parallel aunts and uncles are derived using reduplication.

\exdisplay\noexno
\begin{tabu} {l l l}
	aunt & \fw{nánáatói} & mother's sister\\
	aunt & \fw{náugámói} & father's sister\\
	aunt & \fw{náugámói} & aunt by marriage\\
	uncle & \fw{totoóbói} & father's brother\\
	uncle & \fw{tougámói} & mother's brother\\
	uncle & \fw{tougámói} & uncle by marriage\\
\end{tabu}
\xe

\langtvk{} distinguishes between parallel and cross cousins, but does not distinguish them by gender. Within parallel cousins, different terms are used to distinguish maternal vs. paternal cousins. Cousins' spouses are treated the same as in-laws by adding the \fw{wen-} prefix.

\exdisplay\noexno
\begin{tabu} {l l l}
	cousin & \fw{tírnánáatói} & mother's sister's child\\
	cousin-in-law & \fw{wentírnánáatói} & mother's sister's child's spouse\\
	cousin & \fw{tírtotoóbói} & father's brother's child\\
	cousin-in-law & \fw{wentírtotoóbói} & father's brother's child's spouse\\
	cousin & \fw{tratrabói} & cross cousin\\
	cousin-in-law & \fw{wentratrabói} & cross cousin's spouse\\
\end{tabu}
\xe

The children and grandchildren of one's cousins are not distinguished in any way, even between parallel and cross cousins. In fact, they are all called by the same term as one's cross cousins, \fw{tratrabói}.

Grandparents are distinguished by gender, but there is no distinction made between maternal and paternal grandparents. Similar to the terms for grandchildren, the terms for grandparents are derived using reduplication.

\exdisplay\noexno
\begin{tabu} {l l}
	grandmother & \fw{jújúói}\\
	grandfather & \fw{utuutói}\\
\end{tabu}
\xe

One's grandparents' siblings are called by the same terms as for one's aunts and uncles. In other words, one would call one's maternal grandmother's brother the same term as one's mother would call that person.

\exdisplay\noexno
\begin{tabu} {l l l}
	grand-aunt & \fw{nánáatói} & grandmother's sister\\
	grand-aunt & \fw{náugámói} & grandfather's sister\\
	grand-aunt & \fw{náugámói} & grandparent's sister-in-law\\
	grand-uncle & \fw{totoóbói} & grandfather's brother\\
	grand-uncle & \fw{tougámói} & grandmother's brother\\
	grand-uncle & \fw{tougámói} & grandparent's brother-in-law\\
\end{tabu}
\xe

\begin{sidewaysfigure}[h]\centering
	\caption{Kinship Tree}
	\label{fig:kinship}
	\index{kinship}
	\tiny
	\begin{tikzpicture}[scale=0.5]
	\GraphInit[vstyle=Normal]
	\SetVertexLabelOut
	% Female
	\tikzset{VertexStyle/.append style={shape=circle,minimum size=1em}}
	\Vertex[x=-14,y=8,Lpos=180,L=nánáatói]{MGMS}
	\Vertex[x=-10,y=8,Lpos=270,L=jújúói]{MGM}
	\Vertex[x=-6,y=8,Lpos=275,L=náugámói]{MGFS}
	\Vertex[x=4,y=8,Lpos=180,L=nánáatói]{PGMS}
	\Vertex[x=8,y=8,Lpos=270,L=jújúói]{PGM}
	\Vertex[x=12,y=8,Lpos=275,L=náugámói]{PGFS}
	
	\Vertex[x=-18,y=4,Lpos=90,L=náugámói]{MBW}
	\Vertex[x=-12,y=4,Lpos=0,L=nánáatói]{MS}
	\Vertex[x=-1,y=4,Lpos=270,L=júói]{M}
	\Vertex[x=14,y=4,Lpos=90,L=náugámói]{FBW}
	\Vertex[x=16,y=4,Lpos=265,L=náugámói]{FS}
	
	\Vertex[x=-9,y=0,Lpos=265,L=náatói]{S}
	\Vertex[x=-2,y=0,Lpos=270,L=ólói]{W}
	\Vertex[x=7,y=0,Lpos=265,L=wennáatói]{BW}
	
	\Vertex[x=-11,y=-4,Lpos=270,L=náalár]{SD}
	\Vertex[x=-7,y=-4,Lpos=90,L=wennáalár]{SNW}
	\Vertex[x=-3,y=-4,Lpos=270,L=lárrói]{D}
	\Vertex[x=1,y=-4,Lpos=265,L=wenlárrói]{NW}
	\Vertex[x=5,y=-4,Lpos=270,L=toólár]{BD}
	\Vertex[x=9,y=-4,Lpos=90,L=wentoólár]{BNW}
	
	\Vertex[x=-3,y=-8,Lpos=270,L=lálárrói]{DD}
	\Vertex[x=1,y=-8,Lpos=270,L=lálárrói]{ND}
	
	\Vertex[x=-17,y=-11,L=female]{femalekey}
	% Male
	\tikzset{VertexStyle/.append style={shape=rectangle,minimum size=1em}}
	\Vertex[x=-12,y=8,Lpos=265,L=tougámói]{MGMB}
	\Vertex[x=-8,y=8,Lpos=270,L=utuutói]{MGF}
	\Vertex[x=-4,y=8,Lpos=0,L=totoóbói]{MGFB}
	\Vertex[x=6,y=8,Lpos=265,L=tougámói]{PGMB}
	\Vertex[x=10,y=8,Lpos=270,L=utuutói]{PGF}
	\Vertex[x=14,y=8,Lpos=0,L=totoóbói]{PGFB}
	
	\Vertex[x=-16,y=4,Lpos=275,L=tougámói]{MB}
	\Vertex[x=-14,y=4,Lpos=90,L=tougámói]{MSH}
	\Vertex[x=1,y=4,Lpos=270,L=uutói]{F}
	\Vertex[x=12,y=4,Lpos=180,L=totoóbói]{FB}
	\Vertex[x=18,y=4,Lpos=90,L=tougámói]{FSH}
	
	\Vertex[x=-7,y=0,Lpos=275,L=wentoóbói]{SH}
	\Vertex[x=2,y=0,Lpos=270,L=udói]{H}
	\Vertex[x=9,y=0,Lpos=0,L=toóbói]{B}
	
	\Vertex[x=-9,y=-4,Lpos=275,L=wennáazo]{SDH}
	\Vertex[x=-5,y=-4,Lpos=270,L=náazo]{SN}
	\Vertex[x=-1,y=-4,Lpos=90,L=wenzohiói]{DH}
	\Vertex[x=3,y=-4,Lpos=270,L=zohiói]{N}
	\Vertex[x=7,y=-4,Lpos=275,L=wentoózo]{BDH}
	\Vertex[x=11,y=-4,Lpos=270,L=toózo]{BN}
	
	\Vertex[x=-1,y=-8,Lpos=270,L=zozohiói]{DN}
	\Vertex[x=3,y=-8,Lpos=270,L=zozohiói]{NN}
	
	\Vertex[x=-17,y=-12,L=male]{malekey}
	% Either male or female
	\tikzset{VertexStyle/.append style={shape=diamond,minimum size=0.75em}}
	\Vertex[x=-18,y=0,Lpos=265,L=tratrabói]{MBC}
	\Vertex[x=-16,y=0,Lpos=90,L=wentratrabói]{MBCS}
	\Vertex[x=-14,y=0,Lpos=265,L=tírnánáatói]{MSC}
	\Vertex[x=-12,y=0,Lpos=90,L=wentratrabói]{MSCS}
	\Vertex[x=12,y=0,Lpos=265,L=tírtotoóbói]{FBC}
	\Vertex[x=14,y=0,Lpos=90,L=wentratrabói]{FBCS}
	\Vertex[x=16,y=0,Lpos=265,L=tratrabói]{FSC}
	\Vertex[x=18,y=0,Lpos=90,L=wentratrabói]{FSCS}
	
	\Vertex[x=-17,y=-4,Lpos=265,L=tratrabói]{MBCC}
	\Vertex[x=-13,y=-4,Lpos=265,L=tratrabói]{MSCC}
	\Vertex[x=13,y=-4,Lpos=275,L=tratrabói]{FBCC}
	\Vertex[x=17,y=-4,Lpos=275,L=tratrabói]{FSCC}
	
	\Vertex[x=-17,y=-8,Lpos=270,L=tratrabói]{MBCCC}
	\Vertex[x=-13,y=-8,Lpos=270,L=tratrabói]{MSCCC}
	\Vertex[x=-10,y=-8,Lpos=270,L=tírtrabói]{SDC}
	\Vertex[x=-6,y=-8,Lpos=270,L=tírtrabói]{SNC}
	\Vertex[x=6,y=-8,Lpos=270,L=tírtrabói]{BDC}
	\Vertex[x=10,y=-8,Lpos=270,L=tírtrabói]{BNC}
	\Vertex[x=13,y=-8,Lpos=270,L=tratrabói]{FBCCC}
	\Vertex[x=17,y=-8,Lpos=270,L=tratrabói]{FSCCC}
	
	\Vertex[x=-17,y=-13,L=either female or male]{eitherkey}
	% Ego
	\SetVertexLabelIn
	\tikzset{VertexStyle/.append style={shape=diamond,minimum size=3em}}
	\Vertex[x=0,y=0,Lpos=270,L=mé]{E}
	
	\Edge(MGM)(MGF)
	\Edge(PGM)(PGF)
	\Edge(MS)(MSH)
	\Edge(MBW)(MB)
	\Edge(M)(F)
	\Edge(FBW)(FB)
	\Edge(FS)(FSH)
	\Edge(MBC)(MBCS)
	\Edge(MSC)(MSCS)
	\Edge(S)(SH)
	\Edges(W,E,H)
	\Edge(BW)(B)
	\Edge(FBC)(FBCS)
	\Edge(FSC)(FSCS)
	\Edge(SD)(SDH)
	\Edge(SNW)(SN)
	\Edge(D)(DH)
	\Edge(NW)(N)
	\Edge(BD)(BDH)
	\Edge(BNW)(BN)
	
	\draw (0,4) -- (E);
	\draw (S) -- (-9,2) -- (9,2) -- (B);
	\draw (E) -- (0,-2);
	\draw (D) -- (-3,-2) -- (3,-2) -- (N);
	\draw (-2,-4) -- (-2,-6);
	\draw (DD) -- (-3,-6) -- (-1,-6) -- (DN);
	\draw (2,-4) -- (2,-6);
	\draw (ND) -- (1,-6) -- (3,-6) -- (NN);
	\draw (-8,0) -- (-8,-2);
	\draw (SD) -- (-11,-2) -- (-5,-2) -- (SN);
	\draw (8,0) -- (8,-2);
	\draw (BD) -- (5,-2) -- (11,-2) -- (BN);
	\draw (-10,-4) -- (SDC);
	\draw (-6,-4) -- (SNC);
	\draw (6,-4) -- (BDC);
	\draw (10,-4) -- (BNC);
	\draw (-9,8) -- (-9,6);
	\draw (MB) -- (-16,6) -- (-1,6) -- (M);
	\draw (-12,6) -- (MS);
	\draw (9,8) -- (9,6);
	\draw (F) -- (1,6) -- (16,6) -- (FS);
	\draw (12,6) -- (FB);
	\draw (-17,4) -- (-17,2) -- (-18,2) -- (MBC);
	\draw (-13,4) -- (-13,2) -- (-14,2) -- (MSC);
	\draw (-17,0) -- (MBCC) -- (MBCCC);
	\draw (-13,0) -- (MSCC) -- (MSCCC);
	\draw (13,4) -- (13,2) -- (12,2) -- (FBC);
	\draw (17,4) -- (17,2) -- (16,2) -- (FSC);
	\draw (13,0) -- (FBCC) -- (FBCCC);
	\draw (17,0) -- (FSCC) -- (FSCCC);
	\draw (MGMS) -- (-14,9) -- (-10,9) -- (MGM);
	\draw (-12,9) -- (MGMB);
	\draw (MGF) -- (-8,9) -- (-4,9) -- (MGFB);
	\draw (-6,9) -- (MGFS);
	\draw (PGMS) -- (4,9) -- (8,9) -- (PGM);
	\draw (6,9) -- (PGMB);
	\draw (PGF) -- (10,9) -- (14,9) -- (PGFB);
	\draw (12,9) -- (PGFS);
	\end{tikzpicture}
\end{sidewaysfigure}

\index{kinship|)}

\section{Names}
\label{sec:tvk-names}
\index{names|(}

\subsection{Masculine Names}
\label{subsec:tvk-names-masc}

\begin{itemize}
	\item \scr{Bol} Bol \phnm{\pstrs bol}
	\item \scr{Lerk} Lerk \phnm{\pstrs lerk}
	\item \scr{Mollur} Mollur \phnm{mo\pstrs\gem{l}ur}
	\item \scr{Ote} Ote \phnm{o\pstrs te}
\end{itemize}

\subsection{Feminine Names}
\label{subsec:tvk-names-femi}

\begin{itemize}
	\item \scr{Blimva} Blimva \phnm{blim\pstrs va}
	\item \scr{Goltu} Goltu \phnm{gol\pstrs tu}
	\item \scr{Tlunda} Tlunda \phnm{tlun\pstrs da}
	\item \scr{Zarsa} Zarsa \phnm{zar\pstrs sa}
\end{itemize}

\subsection{Gender-Neutral Names}
\label{subsec:tvk-names-neut}

\begin{itemize}
	\item \scr{Erme} Erme \phnm{er\pstrs me}
	\item \scr{Inki} Inki \phnm{in\pstrs ki}
	\item \scr{Ronne} Ronne \phnm{ron\pstrs ne}
\end{itemize}



\index{names|)}

\excnt=1% Reset the numbering of examples to 1
\chapter{Reference Grammar}
\label{cha:tvk-reference}

Here is a reference grammar.

\part{Tavonic Family: Alnuric}

\excnt=1% Reset the numbering of examples to 1
\chapter{History and Ethnography}
\label{cha:ank-ethnography}

This chapter will present a brief history of the \langank{} language, followed by a short description of its ethnolinguistic context.

\section{Brief History}
\label{sec:ank-history}

Here will be a brief historical description of the \peopank.

\section{Ethnography}
\label{sec:ank-ethnography}

\subsection{Demonyms and Language Names}
\label{subsec:ank-demonyms}

For hundreds of years, the empire ruled in the southern region of \landn. The \langtvk{} word \fw{unner} \phnm{un\pstrs ner} \defn{empire} evolved into the \langank{} word \fw{alnur} \phnm{al\pstrs nur}. \fw{\nlangank} \phnm{al.nu\pstrs rek} \defn{\langank} takes its name from this word. Meanwhile, the \langrdk{} name for the empire is \fw{nonar} \phnm{no\pstrs nar}, and its name for the \langank{} language is \fw{Nonrik} \phnm{non\pstrs rik}. Similarly, the \langank{} and \langrdk{} names for the \langank{} people are \fw{\npeopank} \phnm{al.nu\pstrs reθ} and \fw{Nonriþ} \phnm{non\pstrs riθ} respectively.

\subsection{Ethnology}
\label{subsec:ank-ethnology}

Here will be a brief ethnological description of the \peopank.

\subsection{Demography}
\label{subsec:ank-demography}

Here will be a brief demographical description of the \peopank.

\excnt=1% Reset the numbering of examples to 1
\chapter{Phonology}

\excnt=1% Reset the numbering of examples to 1
\chapter{Morphological Typology}

\excnt=1% Reset the numbering of examples to 1
\chapter{Grammatical Categories}

\excnt=1% Reset the numbering of examples to 1
\chapter{Syntax}

\excnt=1% Reset the numbering of examples to 1
\chapter{Lexical Operations}

\excnt=1% Reset the numbering of examples to 1
\chapter{Discourse}

\excnt=1% Reset the numbering of examples to 1
\chapter{Sociolinguistic Context}

\excnt=1% Reset the numbering of examples to 1
\chapter{\langank{} Reference Grammar}
\label{cha:ank-reference}

Here is a reference grammar for \langank.

\part{Tavonic Family: Redodhic}

\excnt=1% Reset the numbering of examples to 1
\chapter{History and Ethnography}
\label{cha:rdk-ethnography}

This chapter will present a brief history of the \langrdk{} language, followed by a short description of its ethnolinguistic context.

\section{Brief History}
\label{sec:rdk-history}

Here will be a brief historical description of the \peoprdk.

\section{Ethnography}
\label{sec:rdk-ethnography}

\subsection{Demonyms and Language Names}
\label{subsec:rdk-demonyms}

In the north, the alliance resisted the empire's expansion. The \langtvk{} word \fw{aroltutaþ} \phnm{a\sstrs rol.tu\pstrs taθ} signifies \defn{alliance}, however the alliance instead used the simpler form \fw{arutaþ} \phnm{a.ru\pstrs taθ} \defn{standers} to signify the alliance of those kingdoms standing against the empire. \fw{Arutaþ} evolved into the \langrdk{} word \fw{rejiþ} \phnm{re\pstrs\affr{d}{ʒ}iθ}, and \fw{\nlangrdk} \phnm{re.do\pstrs ðik} \defn{\langrdk} takes its name from this word. The \langank{} name for the alliance is \fw{eradeþ} \phnm{e.ra\pstrs deθ}, and its name for the \langrdk{} language is \fw{Eratþek} \phnm{e.rat\pstrs θek}. Similarly, the \langrdk{} and \langank{} names for the \langrdk{} people are \fw{\npeoprdk} \phnm{re.do\pstrs ðiθ} and \fw{Eratþeþ} \phnm{e.rat\pstrs θeθ} respectively.

\subsection{Ethnology}
\label{subsec:rdk-ethnology}

Here will be a brief ethnological description of the \peoprdk.

\subsection{Demography}
\label{subsec:rdk-demography}

Here will be a brief demographical description of the \peoprdk.

\excnt=1% Reset the numbering of examples to 1
\chapter{Phonology}

\excnt=1% Reset the numbering of examples to 1
\chapter{Morphological Typology}

\excnt=1% Reset the numbering of examples to 1
\chapter{Grammatical Categories}

\excnt=1% Reset the numbering of examples to 1
\chapter{Syntax}

\excnt=1% Reset the numbering of examples to 1
\chapter{Lexical Operations}

\excnt=1% Reset the numbering of examples to 1
\chapter{Discourse}

\excnt=1% Reset the numbering of examples to 1
\chapter{Sociolinguistic Context}

\excnt=1% Reset the numbering of examples to 1
\chapter{\langrdk{} Reference Grammar}
\label{cha:rdk-reference}

Here is a reference grammar for \langrdk.

\part{Kalaakan Family: Kalaakan}

\excnt=1% Reset the numbering of examples to 1
\chapter{History and Ethnography}

\excnt=1% Reset the numbering of examples to 1
\chapter{Phonology}

\excnt=1% Reset the numbering of examples to 1
\chapter{Morphological Typology}

\excnt=1% Reset the numbering of examples to 1
\chapter{Grammatical Categories}

\excnt=1% Reset the numbering of examples to 1
\chapter{Syntax}

\excnt=1% Reset the numbering of examples to 1
\chapter{Lexical Operations}

\excnt=1% Reset the numbering of examples to 1
\chapter{Discourse}

\excnt=1% Reset the numbering of examples to 1
\chapter{Sociolinguistic Context}

\excnt=1% Reset the numbering of examples to 1
\chapter{Kalaakan Reference Grammar}

\part{Kalaakan Family: Elvish}

\excnt=1% Reset the numbering of examples to 1
\chapter{History and Ethnography}

\excnt=1% Reset the numbering of examples to 1
\chapter{Phonology}

\excnt=1% Reset the numbering of examples to 1
\chapter{Morphological Typology}

\excnt=1% Reset the numbering of examples to 1
\chapter{Grammatical Categories}

\excnt=1% Reset the numbering of examples to 1
\chapter{Syntax}

\excnt=1% Reset the numbering of examples to 1
\chapter{Lexical Operations}

\excnt=1% Reset the numbering of examples to 1
\chapter{Discourse}

\excnt=1% Reset the numbering of examples to 1
\chapter{Sociolinguistic Context}

\excnt=1% Reset the numbering of examples to 1
\chapter{Elvish Reference Grammar}

\part{Kalaakan Family: Dwarvish}

\excnt=1% Reset the numbering of examples to 1
\chapter{History and Ethnography}

\excnt=1% Reset the numbering of examples to 1
\chapter{Phonology}

\excnt=1% Reset the numbering of examples to 1
\chapter{Morphological Typology}

\excnt=1% Reset the numbering of examples to 1
\chapter{Grammatical Categories}

\excnt=1% Reset the numbering of examples to 1
\chapter{Syntax}

\excnt=1% Reset the numbering of examples to 1
\chapter{Lexical Operations}

\excnt=1% Reset the numbering of examples to 1
\chapter{Discourse}

\excnt=1% Reset the numbering of examples to 1
\chapter{Sociolinguistic Context}

\excnt=1% Reset the numbering of examples to 1
\chapter{Dwarvish Reference Grammar}

\part{Kalaakan Family: Orcish}

\excnt=1% Reset the numbering of examples to 1
\chapter{History and Ethnography}

\excnt=1% Reset the numbering of examples to 1
\chapter{Phonology}

\excnt=1% Reset the numbering of examples to 1
\chapter{Morphological Typology}

\excnt=1% Reset the numbering of examples to 1
\chapter{Grammatical Categories}

\excnt=1% Reset the numbering of examples to 1
\chapter{Syntax}

\excnt=1% Reset the numbering of examples to 1
\chapter{Lexical Operations}

\excnt=1% Reset the numbering of examples to 1
\chapter{Discourse}

\excnt=1% Reset the numbering of examples to 1
\chapter{Sociolinguistic Context}

\excnt=1% Reset the numbering of examples to 1
\chapter{Orcish Reference Grammar}

\part{Kunmian Family: Kunmian}

\excnt=1% Reset the numbering of examples to 1
\chapter{History and Ethnography}

\excnt=1% Reset the numbering of examples to 1
\chapter{Phonology}

\excnt=1% Reset the numbering of examples to 1
\chapter{Morphological Typology}

\excnt=1% Reset the numbering of examples to 1
\chapter{Grammatical Categories}

\excnt=1% Reset the numbering of examples to 1
\chapter{Syntax}

\excnt=1% Reset the numbering of examples to 1
\chapter{Lexical Operations}

\excnt=1% Reset the numbering of examples to 1
\chapter{Discourse}

\excnt=1% Reset the numbering of examples to 1
\chapter{Sociolinguistic Context}

\excnt=1% Reset the numbering of examples to 1
\chapter{Kunmian Reference Grammar}

\part{Kunmian Family: Gnomish}

\excnt=1% Reset the numbering of examples to 1
\chapter{History and Ethnography}

\excnt=1% Reset the numbering of examples to 1
\chapter{Phonology}

\excnt=1% Reset the numbering of examples to 1
\chapter{Morphological Typology}

\excnt=1% Reset the numbering of examples to 1
\chapter{Grammatical Categories}

\excnt=1% Reset the numbering of examples to 1
\chapter{Syntax}

\excnt=1% Reset the numbering of examples to 1
\chapter{Lexical Operations}

\excnt=1% Reset the numbering of examples to 1
\chapter{Discourse}

\excnt=1% Reset the numbering of examples to 1
\chapter{Sociolinguistic Context}

\excnt=1% Reset the numbering of examples to 1
\chapter{Gnomish Reference Grammar}

% Appendices

\appendix

\part{Appendices}

\excnt=1% Reset the numbering of examples to 1
\chapter{Example Texts}
\label{app:example-texts}

Here are some longer example translations.

% Back matter

\backmatter

% Bibliography

\printbibliography

% Index

\printindex

\end{document}