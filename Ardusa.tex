\documentclass[12pt,letterpaper,openany,twoside]{memoir}

% Set new Part to always start on odd page
\let\originalpart=\part
\def\part{\cleardoublepage\originalpart}

% Set the page margins
\setlrmarginsandblock{1in}{1in}{*}
\setulmarginsandblock{1in}{1in}{*}
\checkandfixthelayout

% Layouts
\usepackage{afterpage}
\usepackage{multicol}
\usepackage{rotating}
\usepackage{longtable}
\usepackage{tabu}
\usepackage{multirow}
\usepackage{etoolbox}

% Graphics
\usepackage{graphicx}
\usepackage{tkz-graph}

% Calculations
\usepackage{calc}

% Unicode
\usepackage{xunicode}
\usepackage{xltxtra}

% Handle language and quotation marks
\usepackage{polyglossia}
\setdefaultlanguage[variant=usmax]{english}
\usepackage[style=american,english=american,autopunct]{csquotes} % Put quotations in \enquote{} or \textquote{}

% Date and time
\usepackage[useregional]{datetime2}

% Improve typography
\usepackage[final]{microtype}

% Set fonts
\usepackage{fontspec}
\setmainfont{Junicode}[Ligatures=TeX,Numbers=Lowercase]
\setsansfont{Fira Sans}[Ligatures=TeX,Numbers=Lowercase]
\setmonofont{Noto Sans Mono}[Ligatures=TeX,Numbers=Lowercase]

% FontAwesome icons
\usepackage{fontawesome}

% Colors customization
\usepackage{xcolor}
\definecolor{sapphire}{HTML}{1A527C}
\definecolor{lightgray}{HTML}{777777}

% Reformat page headers
\pagestyle{Ruled}
\nouppercaseheads

% Reformat chapter titles
\chapterstyle{ell}

% Reformat section headings
\hangsecnum % Put section numbers in the margins
\setsecheadstyle{\LARGE\sffamily}
\setsubsecheadstyle{\Large\sffamily}
\setsubsubsecheadstyle{\large\sffamily}
\setsubsubsecindent{1em}
\setparaheadstyle{\sffamily\bfseries}
\setsubparaheadstyle{\sffamily\bfseries}

% Set Table of Contents to include subsections
% \setcounter{secnumdepth}{2}% Doesn't work for some reason
\setsecnumdepth{subsection}
\settocdepth{subsection}

% Increase amount of space between the number and entry in the Table of Contents
\setlength\cftpartnumwidth{2.5em}
\setlength\cftchapternumwidth{2.5em}

% Increase indent of section and subsection in Table of Contents
\setlength\cftsectionindent{2.5em}
\setlength\cftsubsectionindent{5em}

% Bibliography
\usepackage[backend=biber,natbib]{biblatex-chicago}
\addbibresource{bibliography.bib}

% Indexes
\usepackage{makeidx}
\makeindex

% Links
\usepackage{hyperref}
\hypersetup{
	% bookmarks=true,% show bookmarks bar?
	unicode=true,% non-Latin characters in Acrobat’s bookmarks
	pdftitle={Ardusa},% title
	pdfauthor={Ian A.~Cook},% author
	pdfsubject={A grammar of the Ardusan languages},% subject of the document
	pdfcreator={Ian A.~Cook},% creator of the document
	pdfproducer={Ian A.~Cook},% producer of the document
	pdfkeywords={Ardusa, language, linguistics, grammar},% list of keywords
	linktoc=all,% defines which part of the table of contents is made into a link
	colorlinks=true,% false: boxed links; true: colored links
	linkcolor=.,% color of internal links
	citecolor=.,% color of links to bibliography
	filecolor=.,% color of file links
	urlcolor=sapphire% color of external links
}
\urlstyle{same}

% Calculate remaining space in line for ex and pex
% see https://tex.stackexchange.com/a/376534
\newlength{\remaining}
\newcommand{\remainpex}{\setlength{\remaining}{\linewidth-\lingtextoffset-\linglabelwidth-\lingnumoffset-\linglabeloffset-\widthof{\exnoprint}}}
\newcommand{\remainex}{\setlength{\remaining}{\linewidth-\lingnumoffset-\lingtextoffset-\widthof{\exnoprint}}}
\pretocmd{\pex}{\remainpex}{}{}% Not working for some reason
\pretocmd{\ex}{\remainex}{}{}% Not working for some reason

% Glossaries
\usepackage[mcolblock,glosses,nomain,toc]{leipzig}
\makeglossaries
\newleipzig{aff}{aff}{affirmative, confirmation}
%\newleipzig{neg}{neg}{negative, negation}% Already defined
\newleipzig{act}{act}{active}
%\newleipzig{pass}{pass}{passive}% Already defined
%\newleipzig{antip}{antip}{antipassive}% Already defined
%\newleipzig{caus}{caus}{causative}% Already defined
%\newleipzig{refl}{refl}{reflexive}% Already defined
%\newleipzig{recp}{recp}{reciprocal}% Already defined
%\newleipzig{ind}{ind}{indicative}% Already defined
\newleipzig{sbjv}{sbjv}{subjunctive}% Already defined as subj
%\newleipzig{cond}{cond}{conditional}% Already defined
\newleipzig{opt}{opt}{optative}
\newleipzig{des}{des}{desiderative}
\newleipzig{tent}{tent}{tentative}
\newleipzig{pot}{pot}{potential}
\newleipzig{perm}{perm}{permissive}
\newleipzig{nec}{nec}{necessitative}
\newleipzig{int}{int}{interrogative}
%\newleipzig{imp}{imp}{imperative}% Already defined
%\newleipzig{pst}{pst}{past}% Already defined
\newleipzig{npst}{npst}{nonpast}
%\newleipzig{fut}{fut}{future}% Already defined
%\renewleipzig{prf}{prf}{perfect}% Already defined
\newleipzig{mom}{mom}{momentane}
\newleipzig{smlf}{smlf}{semelfactive}
\newleipzig{iter}{iter}{iterative}
\newleipzig{hab}{hab}{habitual}
\newleipzig{inc}{inc}{inceptive}
\newleipzig{prg}{prg}{progressive}
\newleipzig{cnt}{cnt}{continuative}
\newleipzig{rsm}{rsm}{resumptive}
%\newleipzig{dur}{dur}{durative}% Already defined
\newleipzig{pstv}{pstv}{pausative}
\newleipzig{term}{term}{terminative}
\newleipzig{rtsp}{rtsp}{retrospective}
\newleipzig{prsp}{prsp}{prospective}
\newleipzig{dsc}{dsc}{discontinuous}
\newleipzig{gno}{gno}{gnomic}
%\newleipzig{inf}{inf}{infinitive}% Already defined
%\newleipzig{ptcp}{ptcp}{participle}% Already defined
\newleipzig{grv}{grv}{gerundive}
\newleipzig{serg}{serg}{same ergative referent as previous clause}
\newleipzig{sabs}{sabs}{same absolutive referent as previous clause}
\newleipzig{sptv}{sptv}{same partitive referent as previous clause}
\newleipzig{psv}{psv}{positive}
\newleipzig{cmp}{cmp}{comparative}
\newleipzig{sup}{sup}{superlative}
\newleipzig{an}{an}{animate}
\newleipzig{in}{in}{inanimate}
\newleipzig{ab}{ab}{abstract}
\newleipzig{fp}{1}{first person}
\newleipzig[short={1\glstextup{s}}]{fps}{1s}{first person singular}
\newleipzig[short={1\glstextup{pc}}]{fpc}{1pc}{first person paucal}
\newleipzig[short={1\glstextup{p}}]{fpp}{1p}{first person plural}
\newleipzig[short={1\glstextup{p}i}]{fpi}{1pi}{first person plural inclusive}
\newleipzig[short={1\glstextup{p}e}]{fpe}{1pe}{first person plural exclusive}
\newleipzig{sp}{2}{second person}
\newleipzig[short={2\glstextup{s}}]{sps}{2s}{second person singular}
\newleipzig[short={2\glstextup{pc}}]{spc}{2pc}{second person paucal}
\newleipzig[short={2\glstextup{p}}]{spp}{2p}{second person plural}
\newleipzig{tp}{3}{third person}
\newleipzig[short={3\glstextup{s}}]{tps}{3s}{third person singular}
\newleipzig[short={3\glstextup{pc}}]{tpc}{3pc}{third person paucal}
\newleipzig[short={3\glstextup{p}}]{tpp}{3p}{third person plural}
\renewleipzig{prox}{prox}{proximate}% Already defined as proximal
\newleipzig{obv}{obv}{obviative}
%\newleipzig{indf}{indf}{indefinite}% Already defined
%\newleipzig{incl}{incl}{inclusive}% Already defined
%\newleipzig{excl}{excl}{exclusive}% Already defined
%\newleipzig{poss}{poss}{possessive}% Already defined
%\newleipzig{poss}{poss}{possessive}% Already defined
\newleipzig{ali}{ali}{alienable}
\newleipzig{inal}{inal}{inalienable}
%\newleipzig{sg}{sg}{singular}% Already defined
\newleipzig{pc}{pc}{paucal}
%\newleipzig{pl}{pl}{plural}% Already defined
%\newleipzig{def}{def}{definite}% Already defined
%\newleipzig{indf}{indf}{indefinite}% Already defined
\newleipzig{at}{at}{agent trigger}
\newleipzig{pt}{pt}{patient trigger}
\newleipzig{dir}{dir}{direct}
\newleipzig{idr}{idr}{indirect}
%\newleipzig{top}{top}{topic}% Already defined
%\newleipzig{nom}{nom}{nominative}% Already defined
%\newleipzig{acc}{acc}{accusative}% Already defined
%\newleipzig{erg}{erg}{ergative}% Already defined
%\newleipzig{abs}{abs}{absolutive}% Already defined
\newleipzig{ptv}{ptv}{partitive}
%\newleipzig{voc}{voc}{vocative}% Already defined
%\newleipzig{gen}{gen}{genitive}% Already defined
%\newleipzig{dat}{dat}{dative}% Already defined
\newleipzig{lat}{lat}{lative}
%\newleipzig{abl}{abl}{ablative}% Already defined
\newleipzig{pro}{pro}{prolative}
%\newleipzig{ins}{ins}{instrumental}% Already defined
%\newleipzig{ben}{ben}{benefactive}% Already defined
\newleipzig{cau}{cau}{causal}
%\newleipzig{com}{com}{comitative}% Already defined
\newleipzig{prv}{prv}{privative}
%\newleipzig{dist}{dist}{distal}% Already defined
\newleipzig{med}{med}{medial}
%\newleipzig{prox}{prox}{proximal}% Already defined
\newleipzig{crd}{crd}{cardinal}% Numbers
\newleipzig{ord}{ord}{ordinal}% Numbers
\newleipzig{part}{part}{partitive (number)}% Numbers
\newleipzig{mult}{mult}{multiplicative}% Numbers
\newleipzig{coll}{coll}{collective}% Numbers
%\newleipzig{distr}{distr}{distributive}% Numbers % Already defined
\newleipzig{sbst}{sbst}{substantive possessive}
%\newleipzig{rel}{rel}{relative}% Already defined
\newleipzig{rrel}{rrel}{restrictive relative}
\newleipzig{nrrel}{nrrel}{non-restrictive relative}
%\newleipzig{q}{q}{interrogative, question}% Already defined
\newleipzig{dim}{dim}{diminutive}
\newleipzig{aug}{aug}{augmentative}
\newleipzig{lau}{lau}{laudative}
\newleipzig{pej}{pej}{pejorative}
\newleipzig{nmz}{nmz}{nominalizer}

% Linguistics packages
\usepackage{vowel} % Draw vowel charts
\usepackage[linguistics]{forest} % Syntax trees
\usepackage{expex} % Examples and glosses
\usepackage{phonrule} % Phonological rules

% Define tree styles
\forestset{
	dotted tier/.style={
		where n children=0{tier=word,edge=dotted,calign with current edge}{}
	}
}

% Define macro to make ellipsis words in syntax trees light gray
\newcommand{\elps}{\color{lightgray}}

% Implement 3 levels of embedding in ExPex
\def\beginsubsub{%
	\par
	\begingroup
	\advance\leftskip by 2em
	\def\b##1{\par\leavevmode\llap{\hbox to 2em{##1\hfil}}\ignorespaces}}
\def\endsubsub{\par\endgroup}

% Linguistic conventions
% Square brackets for exact phonetic pronunciations
\newcommand{\phnt}[1]{[#1]}
% Slashes for approximate phonemic representations
\newcommand{\phnm}[1]{/#1/}
% Angle brackets for orthographic representations
\newcommand{\orth}[1]{⟨#1⟩}
% Quotes for translations
\newcommand{\defn}[1]{\enquote*{#1}}
% Parentheses for inline gloss
\newcommand{\gloss}[1]{(#1)}
% Astrisk for ungrammatical/incorrect
\newcommand{\ungr}{*}
% Question mark for questionable grammar
\newcommand{\ques}{\fakesuperscript{?}}
% Exclamation point for semantic oddity
\newcommand{\excl}{\fakesuperscript{!}}

% Commands for certain IPA symbols
% Primary stress
\newcommand{\pstrs}{ˈ}
% Secondary stress
\newcommand{\sstrs}{ˌ}
% Geminates
\newcommand{\gem}[1]{#1ː}
% Ejectives
\newcommand{\eje}[1]{#1'}
% Dentalized
\newcommand{\dent}[1]{#1̪}
% Non-syllabic
\newcommand{\nsyl}[1]{#1̯}
% Affricates
\newcommand{\affr}[2]{#1͡#2}

% Define macro so foreign words are italicized
\newcommand{\fw}[1]{\textit{#1}}

\newcommand{\acrnm}[1]{\textsc{#1}}

% Custom name commands
\newcommand{\landn}{Ardusa}
\newcommand{\landadj}{Ardusan}
\newcommand{\langtvk}{Tavonic}
\newcommand{\nlangtvk}{Tavonak}
\newcommand{\peoptvk}{Tavotath}
\newcommand{\npeoptvk}{Tavotaþ}
\newcommand{\langank}{Alnuric}
\newcommand{\nlangank}{Alnurek}
\newcommand{\peopank}{Alnureth}
\newcommand{\npeopank}{Alnureþ}
\newcommand{\langrdk}{Redodhic}
\newcommand{\nlangrdk}{Redoðik}
\newcommand{\peoprdk}{Redodhith}
\newcommand{\npeoprdk}{Redoðiþ}

% Title page data
\title{\landn}
\newcommand{\subtitle}{A Grammar of the \landadj{} Languages}
\author{Ian A.~Cook}
\date{\today}

% Define title page style
\makeatletter
\newcommand{\Titlep}{%
	\begingroup
	\centering
	{\Huge \@title}\\[\baselineskip]
	{\LARGE\textsc \subtitle{}}\\[\baselineskip]
	{\Large\textit{by \@author}}\\
	\vfill
	\textit{last edited}\\
	{\large \@date}\par
	\endgroup
}
\makeatother

% Write ref tags to external tag file
\gathertags

%%%%%%%%%%%%%%%%%%%%%%%%%%%%%%%%%%%%
% Document

\begin{document}

% Title leaf

\begin{titlingpage}
	% Title page
	\Titlep{}
	\clearpage
	% Title verso
	~\vfill
\makeatletter
{\setlength\parindent{0in}
{\large\scr{\@title}}\\
{\large\textbf{\fw{\@title}: \subtitle{}}}\\
by \@author{}\\[\baselineskip]

Copyright ©~2018--\the\year{} by \@author{}.\\
Last edited \today.\\[\baselineskip]

Typeset in Junicode and {\sffamily Fira Sans} with \XeLaTeX{}.\\[\baselineskip]

\landn{} is a fictional landmass set in a fictional constructed world. All of the languages spoken on \landn{}, such as \langtvk, \langank, \langrdk, and others, are themselves fictional, spoken by fictional groups of people, and as such are not related to any naturally existing languages. These languages' vocabularies are entirely \fw{a priori}, which means that no words are derived from the vocabularies of real-world languages. That being said, these languages are intended to be naturalistic, so similarities will occur. Nonetheless, any actual duplication is accidental.\\[\baselineskip]

\begin{tabu} to \linewidth {c X[l]}
	\faLink           & No website yet.\\
	\faCode            & \href{https://github.com/nai888/ardusa}{https://github.com/nai888/ardusa}\\
	\faCreativeCommons & This document is copyrighted ©~2018--\the\year{} by \@author{} under the \href{https://creativecommons.org/licenses/by-nc-sa/4.0/}{Creative Commons Attribution-NonCommercial-ShareAlike 4.0 license, \faCreativeCommons{} BY-NC-SA 4.0}.\\
	\faCopyright[regular] & The Ardusan Script and all languages described within this document are copyrighted ©~2018--\the\year{} by \@author{}, all rights reserved.
\end{tabu}
}
\makeatother
\end{titlingpage}

% Front matter

\frontmatter
\pdfbookmark[0]{\contentsname}{toc}\label{cha:toc}
\tableofcontents*
\clearpage
\listoffigures\label{cha:figures}
\clearpage
\listoftables\label{cha:tables}
\clearpage
\printglosses\label{cha:glossary}
\bigskip
\noindent\begin{tabular}{@{} l l}
\ungr & ungrammatical\\
\ques & grammatically questionable\\
\excl & semantically odd or ill-formed\\
\end{tabular}
\clearpage

% Acknowledgments

\chapter{Acknowledgments}
\label{cha:acknowledgments}

Given that I have not taken any official linguistics coursework, this work would not be possible without several sources of linguistic knowledge. Mark Rosenfelder's \textit{The Language Construction Kit} and \textit{Advanced Language Construction Kit} were important to my first starting out in the world of language construction, with further knowledge gained from David J.~Peterson's \textit{The Art of Language Invention}. Of course, I received an unmeasurable amount of education via several online sources, especially the articles available on Wikipedia. Yet more education, as well as inspiration and motivation, have come from the \textit{Conlangery} podcast and all its hosts and guests.

Finally, this document's format, layout, and organization have been influenced by a few sources, including:

\begin{itemize}
	\item \cite{descms}
	\item \cite{ayeri}
	\item \cite{okuna}\\[\baselineskip]
\end{itemize}

% Preface

\chapter{Preface}
\label{cha:preface}

This document provides a detailed grammatical description of the languages of \landn, a fictional landmass set in a fictional constructed world. This project serves as a method for linguistic research, as an intellectual exercise, as an outlet for creative and artistic expression, and as a setting for potential future works of fiction. It is intended primarily for my own personal use and entertainment, though others with similar linguistic interests will hopefully find it interesting and entertaining as well. I have chosen to use \LaTeX{} to typeset this grammar because it provides a way to be clear, consistent, and organized. Further, since \LaTeX{} uses plain text files, it allows me to use Git for version control so I can keep track of changes over time.

My goal is to build a series of languages with naturalistic grammars that are linguistically plausible and consistent, yet also original in their content and details. This project consists of three distinct and unrelated language families, each of which contains one or more related languages. Some elements of these languages are influenced by existing languages such as Japanese, Finnish, Navajo, Nahuatl, and Arabic, but they are not meant to simply mimic these, instead drawing this inspiration into new forms along with entirely \fw{a priori} lexicons. \landn{} and the \landadj{} languages is an ongoing project with no fixed endpoint or goal.

This concise grammar is my attempt to document the \landadj{} languages in an official and systematic way, and as comprehensively as possible. It is intended to be the official description of the languages. This is a concise grammar because, admittedly, I am not a professional linguist, nor have I taken any linguistics coursework. My education in linguistics consists solely of self-guided research, which means invariably my knowledge will be limited. It is a concise grammar because, frankly, I don't know enough to go into greater detail. That being said, I'm always eager to learn, and will always accept feedback. Again, learning is one of the reasons for this endeavor.

Since the purpose of writing this grammar is to provide a comprehensive description of the \landadj{} languages, not to teach them to others, it is not intended to serve as a textbook or as a way to learn the languages. I have organized topics thematically, rather than curricularly, and I employ technical terms when they are precise, accurate, and appropriate. I have not conducted a formal analysis of the languages, but I have worked to make it as descriptive as possible.

The discussion is ordered from the smallest elements of the languages to the largest. It begins with a description of each language's place in \landn{} followed by their phonologies, it addresses morphology and the combining of words, it discusses vocabulary and derivation, and it explains syntax and discourse. The final chapter serves as a reference grammar, summarizing all of the previous chapters. There are also several appendices describing the conceptual metaphors that organize much of the lexicons, the naming practices of the fictional speakers of these languages, several translation examples, and lexicons. Other resources include a glossary of linguistic glossing abbreviations, a bibliography, and an index.

This document uses several linguistics conventions to clarify meaning. Any reference to specific orthographic spelling is marked with angled brackets, such as \orth{hin}. Pronunciations are usually given phonemically, in which case they are marked with slashes, such as \phnm{hin}. Phonetic pronunciations are used only when conveying specific details like the difference between allophones, and are marked with square brackets, such as \phnt{çin}. Both phonemic and phonetic pronunciations are given using the International Phonetic Alphabet. Foreign words are always written in italics, such as \fw{lu}. English glosses are surrounded by single quotes, such as \defn{and}. If a morphological gloss is provided in-line, it is surrounded by parentheses, such as \gloss{\Inf}.

Many short examples are provided in one single line.

\ex<ex:prfshortgloss>
	\langtvk: \fw{šek} \phnm{ʃek} \defn{ran} \gloss{run-\Ind.\Pst.\Pfv}
\xe

Longer examples are usually provided with a multi-line, or interlinear, gloss. In these examples, the optional first line will indicate which language the example is in, if it is not clear from context. The next line presents the text in that language, followed by the pronunciation. After this, the text is broken into its component morphemes, and the following line provides a morpheme-by-morpheme gloss. The final line provides an English translation of the example phrase or sentence.

\ex<ex:prf-fullgloss>
	\begingl
		\glpreamble \langtvk\\
		\fw{Nan oko šeðo.}\\
		\phnm{nan o\pstrs ko \pstrs ʃe.ðo}//
		\gla nan= oko š-eðo//
		\glb \Pl.\An.\Top= dog run-\Ind.\Pst.\Prg//
		\glft \defn{The dogs were running.}//
	\endgl
\xe

As shown in example \getfullref{ex:prf-fullgloss}, morpheme glosses are labeled with abbreviations in \textsc{small caps}. A full list of all glossing abbreviations is given on page \pageref{cha:glossary}. A hyphen marks a morpheme boundary within a word that is shared between the text and its gloss, while a period marks a boundary present in only one or the other, including when a single word in the text corresponds to multiple words in its gloss. Clitics are marked with an equals sign, reduplication with a tilde, discontinuous affixes (e.g., infixes, circumfixes) with angle brackets, and morphemes that cannot be easily separated out with backslashes.

The \LaTeX{} source code for this grammar and a copy of this PDF are available in a public \href{https://github.com/nai888/ardusa}{\faGithub~GitHub} repository. Undoubtedly, there will be errors in this document. If you notice any, please feel free to open an issue in the GitHub repository with a description and the location of the error.

\begin{flushright}
	\makeatletter
	\textit{\@author}\\
	\textit{Minneapolis, \DTMdisplaydate{2018}{9}{8}}
	\makeatother
\end{flushright}

% Main matter

\mainmatter

\part{Tavonic Family: Tavonic}

\excnt=1% Reset the numbering of examples to 1
\chapter{History and Ethnography}
\label{cha:tvk-ethnography}

This chapter will present a brief history of the \langtvk{} language family, followed by a short description of its ethnolinguistic context.

\section{Brief History}
\label{sec:tvk-history}

The \peoptvk{} (the \langtvk{} people) migrated to \landn{} hundreds of years ago in what they termed Year 1 of the \landadj{} Era (\acrnm{ae}). \landn{} is far from any other landmasses and is isolated from the influence of other lands and other peoples. The \peoptvk{} landed in the warm southeastern regions of \landn{} where they first established their new home, naming this new realm \fw{Urdeso}, a compound word meaning \defn{Safe Land}. Over the following centuries, the \peoptvk{} spread westward and northward throughout the whole of \landn.

As the \peoptvk{} spread, they formed several individual territories, each of which eventually developed into small kingdoms. These kingdoms constantly battled one another for power, and borders were continually shifting. Those who fled the fighting fled northward, furthering the \langtvk{} expansion throughout \landn. As the \peoptvk{} spread farther apart and splintered, their language diverged. Two main dialects emerged, one in the north and one in the south.

After a few hundred years, one kingdom in the south emerged as dominant, conquering or allying with more and more kingdoms until, by 327~\acrnm{ae}, the entire south of \landn{} was united under one empire. This empire enforced the usage of the language that had emerged in the south, thus forming the \langank{} language. The empire continued to push northward until it spread too thin and reached a stalemate with the allied kingdoms in the north around 371~\acrnm{ae}. Finally, in 582~\acrnm{ae} after a couple hundred years of relatively stable rule, the empire declined and divided again into individual territories, leaving behind six sovereign kingdoms.

While the empire was emerging in the south, the kingdoms in the north formed a loose alliance to resist its spread. The alliance managed to reach a stalemate with the empire, stopping its spread northward. The allied kingdoms together maintained the language that emerged in the north, thus forming the \langrdk{} language. Eventually, as the empire split in 582~\acrnm{ae} and the northern alliance was no longer needed, the north also split into individual territories, leaving behind four sovereign kingdoms.

\section{Ethnography}
\label{sec:tvk-ethnography}

\subsection{Demonyms and Language Names}
\label{subsec:tvk-demonyms}

\subsubsection{\langtvk}

The \peoptvk{} were a tribe that migrated to \landn{} together, fleeing their previous home. The \langtvk{} word \fw{tavo} \phnm{ta\pstrs vo} means \defn{person}, and so the derived word \fw{\npeoptvk} \phnm{ta.vo\pstrs taθ} means \defn{people} or \defn{tribe}. In other words, the \peoptvk{} referred to themselves as the People, with \fw{\nlangtvk} being the Language of the People. The \langank- and \langrdk-derived words, \fw{Tevodeþ} \phnm{te.vo\pstrs deθ} and \fw{Tovujiþ} \phnm{to.vu\pstrs \affr{d}{ʒ}iθ} respectively, refer to all people who descended from the original \peoptvk{} tribe. Both \langank{} and \langrdk{} are \peoptvk{} languages and part of the \langtvk{} language family.

\subsubsection{\langank}

For hundreds of years, the empire ruled in the southern region of \landn. The \langtvk{} word \fw{unner} \phnm{un\pstrs ner} \defn{empire} evolved into the \langank{} word \fw{alnur} \phnm{al\pstrs nur}. \fw{\nlangank} \phnm{al.nu\pstrs rek} \defn{\langank} takes its name from this word. Meanwhile, the \langrdk{} name for the empire is \fw{nonar} \phnm{no\pstrs nar}, and its name for the \langank{} language is \fw{Nonrik} \phnm{non\pstrs rik}. Similarly, the \langank{} and \langrdk{} names for the \langank{} people are \fw{\npeopank} \phnm{al.nu\pstrs reθ} and \fw{Nonriþ} \phnm{non\pstrs riθ} respectively.

\subsubsection{\langrdk}

In the north, the alliance resisted the empire's expansion. The \langtvk{} word \fw{aroltutaþ} \phnm{a\sstrs rol.tu\pstrs taθ} signifies \defn{alliance}, however the alliance instead used the simpler form \fw{arutaþ} \phnm{a.ru\pstrs taθ} \defn{standers} to signify the alliance of those kingdoms standing against the empire. \fw{Arutaþ} evolved into the \langrdk{} word \fw{rejiþ} \phnm{re\pstrs\affr{d}{ʒ}iθ}, and \fw{\nlangrdk} \phnm{re.do\pstrs ðik} \defn{\langrdk} takes its name from this word. The \langank{} name for the alliance is \fw{eradeþ} \phnm{e.ra\pstrs deθ}, and its name for the \langrdk{} language is \fw{Eratþek} \phnm{e.rat\pstrs θek}. Similarly, the \langrdk{} and \langank{} names for the \langrdk{} people are \fw{\npeoprdk} \phnm{re.do\pstrs ðiθ} and \fw{Eratþeþ} \phnm{e.rat\pstrs θeθ} respectively.

\subsection{Ethnology}
\label{subsec:tvk-ethnology}

Here will be a brief ethnological description of the \peoptvk.

\subsection{Demography}
\label{subsec:tvk-demography}

Here will be a brief demographical description of the \peoptvk.

\excnt=1% Reset the numbering of examples to 1
\chapter{Phonology}
\label{cha:tvk-phonology}

This chapter will present the inventory of consonants and vowels. An observational analysis of the \langtvk{} languages' syllable structures and phonotactics will follow. The chapter will close with notes on syllable stress within words and a brief exploration of intonation.

\section{\langtvk{} Phoneme Inventory}
\label{sec:tvk-phone-inventory}

\subsection{Consonants}
\label{subsec:tvk-consonants}

\afterpage{\clearpage
	\begin{sidewaystable}
		\scriptsize
		\index{consonants!inventory}\index{allophony}\index{consonants!allophones|see {allophony}}
		\caption[\langtvk{} Consonant Inventory]{\langtvk{} Phonetic Consonant Inventory (allophones in parentheses)}
		\label{tab:tvk-consonants}
		\begin{tabu} to \textheight {| r | X[c] X[c] X[c] X[c] X[c] X[c] X[c] X[c] X[c] X[c] X[c] X[c] X[c] X[c] X[c]}
			\toprule
			Consonants
			& \multicolumn{2}{c}{Bilabial}
			& \multicolumn{2}{c}{Labio-dental}
			& \multicolumn{2}{c}{Dental}
			& \multicolumn{2}{c}{Alveolar}
			& \multicolumn{2}{c}{Post-alveolar}
			& \multicolumn{2}{c}{Velar}
			\\
			\midrule
			Nasal
			&      & m    % Bilabial
			&      &      % Labiodental
			&      &      % Dental
			&      & n    % Alveolar
			&      &      % Post-alveolar
			&      & (ŋ)  % Velar
			\\
			\midrule
			Plosive
			&      &      % Bilabial
			& p    & b    % Labiodental
			& t    & d    % Dental
			&      &      % Alveolar
			&      &      % Post-alveolar
			& k    & g    % Velar
			\\
			\midrule
			Fricative
			&      &      % Bilabial
			& f    & v    % Labiodental
			& θ    & ð    % Dental
			& s    & z    % Alveolar
			& ʃ    & ʒ    % Post-alveolar
			& x    & ɣ    % Velar
			\\
			\midrule
			Flap/Tap
			&      &      % Bilabial
			&      &      % Labiodental
			&      &      % Dental
			&      & ɾ    % Alveolar
			&      &      % Post-alveolar
			&      &      % Velar
			\\
			\midrule
			Trill
			&      &      % Bilabial
			&      &      % Labiodental
			&      &      % Dental
			&      & (r)  % Alveolar
			&      &      % Post-alveolar
			&      &      % Velar
			\\
			\midrule
			Approximant
			&      &      % Bilabial
			&      &      % Labiodental
			&      &      % Dental
			&      & (ɹ)  % Alveolar
			&      &      % Post-alveolar
			&      &      % Velar
			\\
			\midrule
			Lateral
			&      &      % Bilabial
			&      &      % Labiodental
			&      &      % Dental
			&      & l    % Alveolar
			&      &      % Post-alveolar
			&      &      % Velar
			\\
			\bottomrule
		\end{tabu}
	\end{sidewaystable}
	\clearpage
	\index{consonants!romanization}
	\begin{longtabu} to \textwidth {c c c c X[l]}
		\caption{\langtvk{} Consonant Romanization}\label{tab:tvk-consromanization}\\
		\toprule
		Phone & Phoneme & Romanization & English & Notes\\
		\midrule
		\endhead
		\multicolumn{4}{r}{\textit{continued on the next page\ldots}}\\
		\endfoot
		\bottomrule
		\endlastfoot
		\phnt{m} & \phnm{m} & \orth{m} & \orth{m} & \\
		\midrule
		\phnt{n} & \phnm{n} & \orth{n} & \orth{n} & \\
		\midrule
		\phnt{ŋ} & \phnm{n} & \orth{\underline{n}g}, \orth{\underline{n}k}, \orth{\underline{n}ǩ}, or \orth{\underline{n}ǧ} & \orth{n} & \phnm{n} becomes velarized before a velar consonant\\
		\midrule
		\phnt{p} & \phnm{p} & \orth{p} & \orth{p} & \\
		\midrule
		\phnt{b} & \phnm{b} & \orth{b} & \orth{b} & \\
		\midrule
		\phnt{t} & \phnm{t} & \orth{t} & \orth{t} & \\
		\midrule
		\phnt{d} & \phnm{d} & \orth{d} & \orth{d} & \\
		\midrule
		\phnt{k} & \phnm{k} & \orth{k} & \orth{k} & \\
		\midrule
		\phnt{g} & \phnm{g} & \orth{g} & \orth{g} & \\
		\midrule
		\phnt{f} & \phnm{f} & \orth{f} & \orth{f} & \\
		\midrule
		\phnt{v} & \phnm{v} & \orth{v} & \orth{v} & \\
		\midrule
		\phnt{θ} & \phnm{θ} & \orth{þ} & \orth{th} & \\
		\midrule
		\phnt{ð} & \phnm{ð} & \orth{ð} & \orth{dh} & \\
		\midrule
		\phnt{s} & \phnm{s} & \orth{s} & \orth{s} & \\
		\midrule
		\phnt{z} & \phnm{z} & \orth{z} & \orth{z} & \\
		\midrule
		\phnt{ʃ} & \phnm{ʃ} & \orth{š} & \orth{sh} & \\
		\midrule
		\phnt{ʒ} & \phnm{ʒ} & \orth{ž} & \orth{zh} & \\
		\midrule
		\phnt{x} & \phnm{x} & \orth{ǩ} & \orth{kh} & \\
		\midrule
		\phnt{ɣ} & \phnm{ɣ} & \orth{ǧ} & \orth{gh} & \\
		\midrule
		\phnt{ɾ} & \phnm{r} & \orth{r} & \orth{r} & \\
		\midrule
		\phnt{r} & \phnm{r} & \orth{rr} & \orth{rr} & \orth{r} is trilled when doubled \\
		\midrule
		\phnt{ɹ} & \phnm{r} & \orth{r} & \orth{r} & \orth{r} is occasionally pronounced as an approximant when a part of a consonant cluster \\
		\midrule
		\phnt{l} & \phnm{l} & \orth{l} & \orth{l} & \\
	\end{longtabu}
	\clearpage
}

With 20 consonants, \langtvk{} has an \enquote{average} inventory.\autocite{wals-1} \autoref{tab:tvk-consonants} shows the full chart of consonant phonemes, along with several allophones enclosed in parentheses. \autoref{tab:tvk-consromanization} shows how each consonant in \langtvk{} is romanized.

Despite its \enquote{average} inventory of consonants, there are many more allophones\index{allophony} that occur in the language. First, any doubled consonant is realized as a geminated\index{consonants!gemination} (elongated) consonant.

\pex<gemcons>
	\fw{unner} \phnm{u\gem{n}er} \defn{empire}
\xe

Thus, example~\getfullref{gemcons} above is realized with a lengthened \phnt{n}. A doubled \orth{r} is similarly geminated, but the pronunciation changes from a flap/tap to a trill.

The remaining allophones\index{allophony} occur due to various sound change processes, mostly by assimilation. For example, \phnm{n} becomes velarized when it appears immediately before a velar consonant.

\ex<velarn>
	\fw{tavonga} \phnt{ta.voŋ\pstrs ga} \defn{humanlike}
\xe

As discussed above, \orth{r} can be pronounced as both a tap/flap \phnt{ɾ} and as a trill \phnt{r}. Additionally, when part of certain consonant clusters, it can be as an approximant \phnt{ɹ}. This primarily occurs when the \orth{r} leads into a cluster or immediately follows a nasal.

\ex<velarn>
	\fw{frorgali} \phnt{fɾoɹ.\pstrs ga.li} \defn{to un-see}
\xe

\subsection{Vowels}
\label{subsec:tvk-vowels}

Placeholder

\section{\langtvk{} Phonotactics}
\label{sec:tvk-phonotactics}

Placeholder

\subsection{Syllable Structures}
\label{subsec:tvk-syll-struc}

Placeholder

\subsection{Phonological Changes}
\label{subsec:tvk-phone-changes}

Placeholder

\subsection{Syllable Parsing}
\label{subsec:tvk-syll-parse}

Placeholder

\subsection{Number of Syllables per Word}
\label{subsec:tvk-num-syll}

Placeholder

\section{\langtvk{} Prosody}
\label{sec:tvk-prosody}

Placeholder

\subsection{Syllable Weight}
\label{subsec:tvk-syll-weight}

Placeholder

\subsection{Word Stress}
\label{subsec:tvk-word-stress}

Placeholder

\subsection{Intonation}
\label{subsec:tvk-intonation}

Placeholder

\section{\langank{} Phoneme Inventory}
\label{sec:ank-phone-inventory}

Placeholder

\subsection{Consonants}
\label{subsec:ank-consonants}

Placeholder

\subsection{Vowels}
\label{subsec:ank-vowels}

Placeholder

\section{\langank{} Phonotactics}
\label{sec:ank-phonotactics}

Placeholder

\subsection{Syllable Structures}
\label{subsec:ank-syll-struc}

Placeholder

\subsection{Phonological Changes}
\label{subsec:ank-phone-changes}

Placeholder

\subsection{Syllable Parsing}
\label{subsec:ank-syll-parse}

Placeholder

\subsection{Number of Syllables per Word}
\label{subsec:ank-num-syll}

Placeholder

\section{\langank{} Prosody}
\label{sec:ank-prosody}

Placeholder

\subsection{Syllable Weight}
\label{subsec:ank-syll-weight}

Placeholder

\subsection{Word Stress}
\label{subsec:ank-word-stress}

Placeholder

\subsection{Intonation}
\label{subsec:ank-intonation}

Placeholder

\section{\langrdk{} Phoneme Inventory}
\label{sec:rdk-phone-inventory}

Placeholder

\subsection{Consonants}
\label{subsec:rdk-consonants}

Placeholder

\subsection{Vowels}
\label{subsec:rdk-vowels}

Placeholder

\section{\langrdk{} Phonotactics}
\label{sec:rdk-phonotactics}

Placeholder

\subsection{Syllable Structures}
\label{subsec:rdk-syll-struc}

Placeholder

\subsection{Phonological Changes}
\label{subsec:rdk-phone-changes}

Placeholder

\subsection{Syllable Parsing}
\label{subsec:rdk-syll-parse}

Placeholder

\subsection{Number of Syllables per Word}
\label{subsec:rdk-num-syll}

Placeholder

\section{\langrdk{} Prosody}
\label{sec:rdk-prosody}

Placeholder

\subsection{Syllable Weight}
\label{subsec:rdk-syll-weight}

Placeholder

\subsection{Word Stress}
\label{subsec:rdk-word-stress}

Placeholder

\subsection{Intonation}
\label{subsec:rdk-intonation}

Placeholder

\excnt=1% Reset the numbering of examples to 1
\chapter{Morphological Typology}
\label{cha:tvk-morphological-typology}

Now that \langtvk, \langank, and \langrdk's phonologies have been defined in \autoref{cha:tvk-phonology}, this chapter will discuss the next larger unit of language: morphemes. A morpheme is the smallest meaningful unit in a language. A morpheme can be a root, or it can be another element that affects or modifies the meaning of a root. Further, a morpheme may be freestanding, or it may be bound to other morphemes to form a larger word.

The discussion will begin with a general explanation of the \langtvk{} family's morphological typology. Following this will be a brief summary of the various morphological processes that occur in the languages, ending with an explanation of the locus of marking.

\section{Morphological Typology}
\label{sec:tvk-typology}
\index{morphological typology|(}

Traditional research would show that \langtvk{} is typologically partially isolating and partially fusional, meaning that morphemes are often either separated into distinct words or fused together such that a single phonological unit represents several morphemes. However, according to Bickel and Nichols, \blockquote{Recent research has shown that such a scale [ranging from isolating to agglutinative to fusional to introflexive] conflates many different typological variables and incorrectly assumes that these parameters covary universally\autocite{Plank-1999,Bickel-and-Nichols-2005}. Three prominent variables involved in this are phonological fusion, formative exponence, and flexivity (i.e. allomorphy, inflectional classes).\autocite{wals-20}} Therefore, we will examine each of these areas---phonological fusion, formative exponence, and flexivity, as well as the degree of synthesis---separately.

\subsection{Phonological Fusion}
\label{subsec:tvk-fusion}
\index{morphological typology!fusion|(}

\langtvk's phonological formatives are partially fusional, being partially \enquote{isolating} and partially \enquote{concatenative}\autocite{wals-20}. The concatenative morphemes are phonologically bound, requiring a \enquote{host word} with which they form one single phonological word, while the isolating morphemes are \enquote{full-fledged phonological words of their own}.

Verbs are almost exclusively concatenative, with tense, aspect, and mood morphemes attached directly to the verb's stem.

\pex<ex:tvk-concat-verbs>
	\a<inf>\begingl
		\glpreamble\fw{ufuli}\\
		\phnm{u\pstrs fu.li}//
		\gla uf-uli//
		\glb sing-\Inf//
		\glft \defn{to sing}//
	\endgl
	\a<imp>\begingl
		\glpreamble\fw{Ufunte!}\\
		\phnm{u\pstrs fun.te}//
		\gla uf-unte//
		\glb sing-\Imp//
		\glft \defn{Sing!}//
	\endgl
	\a<phrase>\begingl
		\glpreamble\fw{Mon ufuk.}\\
		\phnm{\pstrs mon u\pstrs fuk}//
		\gla mon uf-uk//
		\glb \Fps.\Top{} sing-\Ind.\Pst.\Pfv//
	\glft \defn{I sang.}//
	\endgl
\xe

Example~\getref{ex:tvk-concat-verbs} shows how morphemes are attached to the stem of a verb through suffixes, rather than with separate (isolating) modifying words or nonlinear ablaut or tone modifications.

Example~\getfullref{ex:tvk-concat-verbs.phrase} similarly shows how personal pronouns are fusional. Example~\getref{ex:tvk-concat-prons} demonstrates further how each personal pronoun simultaneously indicates the person, number, animacy in the third person, case, and whether it is the topic.

\pex<ex:tvk-concat-prons>
	\a<fpsa> \fw{mor} \phnm{mor} \defn{I} \gloss{\Fps.\Abs}
	\a<sppa> \fw{þeton} \phnm{θe\pstrs ton} \defn{you} \gloss{\Spp.\Acc}
	\a<tpcitd> \fw{ginsek} \phnm{gin\pstrs sek} \defn{to it} \gloss{\Tpc.\In.\Top.\Dat}
\xe

This concatenation appears not only in inflectional morphology, but also in derivational morphology. For example, the word \fw{ablutik} \phnm{a.blu\pstrs tik} \defn{kitten} is formed from the root noun \fw{ablu} \phnm{a\pstrs blu} \defn{cat} with a diminutive suffix attached \gloss{\fw{ablu}-\Dim}. Similarly, the word \fw{akradir} \phnm{ak.ra\pstrs dir} \defn{pen} is formed from the root verb \fw{akrali} \phnm{ak\pstrs ra.li} \defn{to write} with a nominalizing suffix \gloss{\fw{akra}-\Nmz}.

Nouns, on the other hand, are exclusively isolating. All grammatical markings, including number, gender, case, and topicality, are indicated using phonologically separate prepositions.

\pex<ex:tvk-iso-nouns>
	\a<asta>\begingl
		\glpreamble\fw{No akrakon aruþ.}\\
		\phnm{no ak.ra\pstrs kon a\pstrs ruθ}//
		\gla no= akrakon ar-uþ//
		\glb \An.\Sg.\Top.\Abs= writer stand-\Ind.\Npst.\Prg//
		\glft \defn{The writer is standing.}//
	\endgl
	\a<phrase>\begingl
		\glpreamble\fw{Esokon moþes elbi šus ken botra draš.}\\
		\phnm{e.so\pstrs kon mo\sstrs θes el\pstrs bi \pstrs ʃus ken bot\pstrs ra \pstrs draʃ}//
		\gla ∅= esokon moþes= elbi šus ken= botra dr-aš//
		\glb \An.\Sg.\Abs= farmer \In.\Pc.\Top.\Acc= egg \Tps.\An.\Gen{} \An.\Pl.\Dat= wife give-\Ind.\Npst.\Rtsp//
		\glft \defn{The farmer has given the eggs to his wife.}//
	\endgl
\xe

Notice in example~\getref{ex:tvk-iso-nouns} how every noun is preceded by a preposition that identifies that noun's grammatical role within the sentence.

\index{morphological typology!fusion|)}

\subsection{Formative Exponence}
\label{subsec:tvk-exponence}
\index{morphological typology!exponence|(}

\langtvk{} has mostly polyexponential formatives, meaning that, in almost all cases, single morphemes express multiple grammatical categories each\autocite{wals-21}. Derivational morphemes are all monoexponential while inflectional morphemes are almost exclusively polyexponential.

\ex<ex:tvk-exponence>
	\begingl
		\glpreamble\fw{Nan tavotik one vi?}\\
		\phnm{nan ta.vo\pstrs tik o\pstrs ne vi}//
		\gla nan= tavo-tik on-e =vi//
		\glb \An.\Pl.\Top= person-\Dim{} play-\Ind.\Npst.\Ipfv{} =\Int//
		\glft \defn{Do children play?}//
	\endgl
\xe

Example~\getref{ex:tvk-exponence} includes one derivational morpheme and three inflectional morphemes attached to the roots \fw{tavo} and \fw{oneli}, two of which are polyexponential. The preposition \fw{nan} is a polyexponential morpheme that identifies the preceding noun's gender (animate), number (plural), and topicality. The affix \fw{-tik}, a diminutive that derives the word \defn{child} from the root \defn{person}, is a monoexponential derivational suffix. The single-letter suffix \fw{-e} attaches to the verb to express the mood (indicative), tense (nonpast), and aspect (imperfective). Finally, the word \fw{vi} is a monoexponential interrogative clitic that turns the sentence into a question.

Noun prepositions can additionally encode case. In example~\getref{ex:tvk-exponence}, the noun \fw{tavotik} is inferred to be in the absolutive case despite being unmarked for it. In many other situations, this grammatical case is additionally encoded within the same polyexponential preposition. In example~\getfullref{ex:tvk-iso-nouns.phrase}, the word \fw{moþes} indicates that the noun \defn{egg} is inanimate, paucal, the topic, and in the accusative case.

One noun preposition, \fw{nut} has not fully cumulated, with the noun's number being still separated into a distinct segment.

\pex<ex:tvk-noncumulated>
	\a<sg>\fw{nut-∅} \phnm{nut} \gloss{\An.\Top.\Acc-\Sg}
	\a<pc>\fw{nut-os} \phnm{nu\pstrs tos} \gloss{\An.\Top.\Acc-\Pc}
	\a<pl>\fw{nut-on} \phnm{nu\pstrs ton} \gloss{\An.\Top.\Acc-\Pl}
\xe

All other noun prepositions are fully cumulated and cannot be separated into their component morphemes.

\pex<ex:tvk-cumulated>
	\a<ie>Inanimate Ergative
	\beginsubsub
		\b{i.}\fw{ða} \phnm{ða} \gloss{\In.\Sg.\Erg}
		\b{ii.}\fw{ðes} \phnm{ðes} \gloss{\In.\Pc.\Erg}
		\b{iii.}\fw{dun} \phnm{dun} \gloss{\In.\Pl.\Erg}
	\endsubsub
	\a<itd>Inanimate Topic Dative
	\beginsubsub
		\b{i.}\fw{moǩ} \phnm{mox} \gloss{\In.\Sg.\Top.\Dat}
		\b{ii.}\fw{mekos} \phnm{me\pstrs kos} \gloss{\In.\Pc.\Top.\Dat}
		\b{iii.}\fw{nikun} \phnm{ni \pstrs kun} \gloss{\In.\Pl.\Top.\Dat}
	\endsubsub
\xe

\index{morphological typology!exponence|)}

\subsection{Flexivity}
\label{subsec:tvk-flexivity}
\index{morphological typology!flexivity|(}

\langtvk{} nouns, adjectives, and verbs display flexivity, which means that these words are divided into separate classes that receive distinct inflectional allomorphs. On such allomorphs, otherwise identical morphemes take distinct phonological shapes.

Nouns are divided into animate and inanimate genders. These two genders determine which prepositions are used to provide the grammatical context of the noun.

\pex<ex:tvk-flex-nouns>
	\a<animate>\begingl
		\glpreamble\fw{ri bilt}\\
		\phnm{ri \pstrs bilt}//
		\gla ri= bilt//
		\glb \An.\Pc.\Abs= breath//
		\glft \defn{breaths}//
	\endgl
	\a<inanimate>\begingl
		\glpreamble\fw{l'eðer}\\
		\phnm{le\pstrs ðer}//
		\gla le=eðer//
		\glb \In.\Pc.\Abs=pen//
		\glft \defn{pens}//
	\endgl
\xe

In example~\getref{ex:tvk-flex-nouns}, both \fw{bilt} and \fw{eðer} are marked for the paucal number and the absolutive case, but because \fw{bilt} is animate and \fw{eðer} is inanimate, the shape of the prepositions are entirely different.

Although they are distinct, the shapes are often more closely related than in example~\getref{ex:tvk-flex-nouns}. Example~\getref{ex:tvk-flex-nouns-related} shows the animate and inanimate forms of the plural ergative preposition; the relation between the two forms is much clearer, as only the vowel changes.

\pex<ex:tvk-flex-nouns-related>
	\a<animate>\begingl
		\glpreamble\fw{din bilt}\\
		\phnm{din \pstrs bilt}//
		\gla din= bilt//
		\glb \An.\Pl.\Erg= breath//
		\glft \defn{breaths}//
	\endgl
	\a<inanimate>\begingl
		\glpreamble\fw{dun eðer}\\
		\phnm{dun e\pstrs ðer}//
		\gla dun= eðer//
		\glb \In.\Pl.\Erg= pen//
		\glft \defn{pens}//
	\endgl
\xe

Nouns do not show possessive flexivity, as there is no possessive classification\autocite{wals-59}. There is only one method of forming a possessive relationship: using the genitive case.

Adjectives also show flexivity since they decline to match the gender of the noun they modify. Each adjective has a distinct animate and inanimate form, with animate adjectives ending in \fw{-a}, \fw{-i}, or \fw{-u} and inanimate adjectives ending in \fw{-e} or \fw{-o}.

\pex<ex:tvk-flex-adjectives>
	\a<animate>\begingl
		\glpreamble\fw{su frandi bilt}\\
		\phnm{su fran\pstrs di \pstrs bilt}//
		\gla su= frandi bilt//
		\glb \An.\Sg.\Gen= visible.\An{} breath//
		\glft \defn{of the visible breath}//
	\endgl
	\a<inanimate>\begingl
		\glpreamble\fw{šo frando eðer}\\
		\phnm{ʃo fran\pstrs do e\pstrs ðer}//
		\gla šo= frando eðer//
		\glb \In.\Sg.\Gen= visible.\In{} pen//
		\glft \defn{of the visible pen}//
	\endgl
\xe

In example~\getref{ex:tvk-flex-adjectives}, the form of \fw{frandi} changes depending on whether it is modifying an animate noun like \fw{bilt} or an inanimate noun like \fw{eðer}.

Verbs are divided into three distinct conjugation classes, each identified by the infinitive form. Class I verb infinitives end in \fw{-ali}, class II verb infinitives end in \fw{-eli}, and class III verb infinitives end in \fw{-uli}.

\pex<ex:tvk-flex-verbs>
	\a<cl1>Class I: \fw{bruþat-ali} \phnm{bru.θa\pstrs ta.li} \defn{to handle} \gloss{handle-\Inf}
	\a<cl2>Class II: \fw{š-eli} \phnm{\pstrs ʃe.li} \defn{to run} \gloss{run-\Inf}
	\a<cl3>Class III: \fw{teg-uli} \phnm{te\pstrs gu.li} \defn{to worry} \gloss{worry-\Inf}
\xe

Beyond just the form of the infinitive, the verb's class determines the entire conjugation paradigm for that verb.

\pex<ex:tvk-flex-verbs>
	\a<cl1>Class I: \fw{bruþat-abe} \phnm{bru.θa\pstrs ta.be} \defn{handling} \gloss{handle-\Act.\Ptcp}
	\a<cl2>Class II: \fw{š-iba} \phnm{\pstrs ʃi.ba} \defn{running} \gloss{run-\Act.\Ptcp}
	\a<cl3>Class III: \fw{teg-ube} \phnm{te\pstrs gu.be} \defn{worrying} \gloss{worry-\Act.\Ptcp}
\xe

As shown in example~\getref{ex:tvk-flex-verbs}, the same inflection takes a different form when attached to a verb of a different class. To form the active participle, \fw{bruþatali} becomes \fw{bruþatabe} and \fw{teguli} becomes \fw{tegube}. Following this pattern, one might expect \fw{šeli} to become \ungr\fw{šebe}, but instead it becomes \fw{šiba}.

\index{morphological typology!flexivity|)}

\subsection{Synthesis}
\label{subsec:tvk-synthesis}
\index{morphological typology!synthesis|(}

As discussed in \autoref{subsec:tvk-fusion}, derivation and verb inflection occurs by attaching affixes to a stem or root, forming singular phonological words. Meanwhile, noun declension occurs using prepositions that mark the grammatical information for the noun. These prepositions are separate phonological words from the nouns themselves.

In all cases, however, inflected forms constitute singular \emph{syntactic} words because the inflections cannot be separated or reordered at all. This means that \langtvk{} morphology is synthetic\autocite{wals-22}.

\langtvk{} verbs normally inflect to show mood, tense, and aspect, a total of three morpheme categories per word. The maximally inflected form adds negation, a particle that is a separate phonological word but remains a part of the syntactic word of the verb, bringing \langtvk's category-per-word ratio up to 4\autocite{wals-22}.

\ex<ex:tvk-verb-cpw>
	\begingl
		\glpreamble\fw{Šun onek bo.}\\
		\phnm{\pstrs ʃun o\pstrs nek bo}//
		\gla šun on-ek -bo//
		\glb \Tps.\An.\Top{} play-\Ind.\Pst.\Pfv{} -\Neg//
		\glft \defn{S/he did not play.}//
	\endgl
\xe

\index{morphological typology!synthesis|)}

\index{morphological typology|)}

\section{Morphological Processes}
\label{sec:tvk-processes}
\index{morphological typology!processes|(}

\langtvk{} is \enquote{predominantly suffixing}\autocite{wals-26} and primarily makes use of suffixes and clitics to derive and inflect words. The language does not employ infixation, stem modification, or suprafixation, no prefixation has yet been identified, and reduplication only appears in wordplay and child-directed speech.

\subsection{Suffixation}
\label{subsec:tvk-suffixation}
\index{morphological typology!processes!suffixation|(}

Suffixes in \langtvk{} apply mainly to verbs. All verbal inflections occur via the addition of suffixes, whether phonologically bound or not. This is illustrated in example~\getref{ex:tvk-verbs-sfxs}.

\pex<ex:tvk-verbs-sfxs>
	\a<i:ipr>\begingl
		\glpreamble\fw{Šona git akraǧ.}\\
		\phnm{ʃo\pstrs na git ak\pstrs raɣ}//
		\gla šona git akr-aǧ//
		\glb \Tpp.\An.\Top{} \Tps.\In.\Acc{} write-\Ind.\Pst.\Rtsp//
		\glft \defn{They had written it.}//
	\endgl
	\a<sbjv>\begingl
		\glpreamble\fw{Monsa ufut oþ nikis.}\\
		\phnm{mon\pstrs sa u\pstrs fut oθ ni\pstrs kis}//
		\gla monsa uf-ut oþ nik-is//
		\glb \Fpc.\Top{} sing-\Ind.\Npst.\Pfv{} if be.able-\Sbjv.\Npst.\Ipfv//
		\glft \defn{We will sing if we are able.}//
	\endgl
	\a<iii:pptcp>\begingl
		\glpreamble\fw{usombe akrapis}\\
		\phnm{u\pstrs som.be ak.ra\pstrs pis}//
		\gla us-ombe akrapis//
		\glb hold-\Pass.\Ptcp.\In{} letter//
		\glft \defn{held letter}//
	\endgl
	\a<i:imp>\begingl
		\glpreamble\fw{Mi þro akrorganta.}\\
		\phnm{mi \pstrs θro ak.ror\pstrs gan.ta}//
		\gla mi þro akrorg-anta//
		\glb \In.\Sg.\Top{} that.\Med{} erase-\Imp//
		\glft \defn{Erase that.}//
	\endgl
	\a<neg>\begingl
		\glpreamble\fw{Mana kantenta bo.}\\
		\phnm{ma\pstrs na kan\pstrs ten.ta bo}//
		\gla mana kant-enta -bo//
		\glb \Fpp.\Top{} thank-\Imp{} -\Neg//
		\glft \defn{Don't thank us.}//
	\endgl
\xe

As discussed in \autoref{subsec:tvk-synthesis}, although the particle \fw{bo} is a separate phonological word, it functions syntactically as a suffix. This is shown in example~\getfullref{ex:tvk-verbs-sfxs.neg} where it attaches to the verb \fw{kantenta} to negate it.

Suffixes are also present on adjectives, though only minimally. Adjectives take one of two vowel endings to mark the gender of its referent, with animate adjectives ending in \fw{-i}, \fw{-a}, or \fw{u} and inanimate adjectives ending in \fw{-e} or \fw{-o}.

\pex<ex:tvk-adjective-sfxs>
	\a<ae>\fw{ablunga} \phnm{ab.lun\pstrs ga} \gloss{\An} vs. \fw{ablunge} \phnm{ab.lun\pstrs ge} \gloss{\In} \defn{catlike}
	\a<io>\fw{akrandi} \phnm{ak.ran\pstrs di} \gloss{\An} vs. \fw{akrando} \phnm{ak.ran\pstrs do} \gloss{\In} \defn{writable}
	\a<ao>\fw{bruþatla} \phnm{bru.θat\pstrs la} \gloss{\An} vs. \fw{bruþatlo} \phnm{bru.θat\pstrs lo} \gloss{\In} \defn{manual}
	\a<uo>\fw{fraþru} \phnm{fraθ\pstrs ru} \gloss{\An} vs. \fw{fraþro} \phnm{fraθ\pstrs ro} \gloss{\In} \defn{observant}
\xe

Suffixation also occurs regularly in derivational inflection. In fact, several derivational suffixes can be strung together to derive yet more words. Example~\getref{ex:tvk-derivation-sfxs} shows this process.

\pex<ex:tvk-derivation-sfxs>
	\a<root1>\fw{frali} \phnm{\pstrs fra.li} \defn{to see}
	\a<der11>\fw{fravem} \phnm{fra\pstrs vem} \defn{sight}
	\a<der12>\fw{fravemitla} \fw{-o} \phnm{fra.vem.it\pstrs la} \defn{visual}
	\a<root2>\fw{onaš} \phnm{o\pstrs naʃ} \defn{rug}
	\a<der21>\fw{onašuli} \phnm{o.na\pstrs ʃu.li} \defn{to place}
	\a<der22>\fw{onašinsuli} \phnm{o.na.ʃin\pstrs su.li} \defn{to re-place}
\xe

In example~\getfullref{ex:tvk-derivation-sfxs.der22}, the \fw{-ins} affix may not immediately appear to be a suffix, however it should be noted that it is being attached to the end of the \emph{stem} of the word, which is \fw{onaš-}, prior to the verb's infinitive ending \fw{-uli}, which is an \emph{inflectional} suffix.

\index{morphological typology!processes!suffixation|)}

\subsection{Cliticization}
\label{subsec:tvk-cliticization}
\index{morphological typology!processes!cliticization|(}

Clitics can be difficult to define in a formal way, and it is therefore worthwhile to explain how certain morphemes in \langtvk{} can be classified as such.

A \enquote*{clitic} is often characterized as \enquote{a \enquote{small}, prosodically weak, or non-prominent word which fails to respect normal principles of syntactic distribution because it requires a host to which it can attach phonologically}\autocite{spencer-luis-2012}. Clitics are different from affixes in that they will typically \enquote{cliticize \enquote{promiscuously} to a word of any old category, including uninflectable words which otherwise fail to take any affixes whatever}\autocite{spencer-luis-2012}, whereas affixes are limited to only specific parts of speech to which they can connect\autocite{Zwicky-and-pullum-1983}. Yet, they are different from function words in that they are bound, that is they do not have the free ordering afforded to words\autocite{Zwicky-1985}.

The primary example of clitics in \langtvk{} is the noun prepositions. These particles cannot appear alone, conveying solely grammatical, not lexical, information. They are not affixes because they attach to the beginning of the entire noun phrase, no matter what word comes after, rather than attaching directly to the head noun.

\pex<ex:tvk-noun-clitics>
	\a<simple>\begingl
		\glpreamble\fw{Mod nas oko fra.}\\
		\phnm{\pstrs mod nas o\pstrs ko \pstrs fra}//
		\gla mod nas= oko fr-a//
		\glb \Fps.\Erg{} \An.\Pc.\Top{} dog see-\Ind.\Npst.\Ipfv//
		\glft \defn{I see the dogs.}//
	\endgl
	\a<adjective>\begingl
		\glpreamble\fw{Mod nas urda oko fra.}\\
		\phnm{\pstrs mod nas ur\pstrs da o\pstrs ko \pstrs fra}//
		\gla mod nas= urd-a oko fr-a//
		\glb \Fps.\Erg{} \An.\Pc.\Top{} protected-\An{} dog see-\Ind.\Npst.\Ipfv//
		\glft \defn{I see the protected dogs.}//
	\endgl
	\a<possessedpro>\begingl
		\glpreamble\fw{Mod nas tesar urda oko fra.}\\
		\phnm{\pstrs mod nas te\pstrs sar ur\pstrs da o\pstrs ko \pstrs fra}//
		\gla mod nas= tesar urd-a oko fr-a//
		\glb \Fps.\Erg{} \An.\Pc.\Top{} \Spc.\Gen{} protected-\An{} dog see-\Ind.\Npst.\Ipfv//
		\glft \defn{I see your protected dogs.}//
	\endgl
	\a<possessednoun>\begingl
		\glpreamble\fw{Mod nas su esokon urda oko fra.}\\
		\phnm{\pstrs mod nas su e.so\pstrs kon ur\pstrs da o\pstrs ko \pstrs fra}//
		\gla mod nas= su= esokon urd-a oko fr-a//
		\glb \Fps.\Erg{} \An.\Pc.\Top{} \An.\Sg.\Gen{} farmer protected-\An{} dog see-\Ind.\Npst.\Ipfv//
		\glft \defn{I see the farmer's protected dogs.}//
	\endgl
\xe

Notice in example~\getref{ex:tvk-noun-clitics} how the particle \fw{nas} directly precedes the entire noun phrase, even when separated from the head noun by an adjective (\getfullref{ex:tvk-noun-clitics.adjective}), a pronoun (\getfullref{ex:tvk-noun-clitics.possessedpro}), and even another modifying noun and its preposition (\getfullref{ex:tvk-noun-clitics.possessednoun}).

In some cases, the noun prepositions reduce phonologically and attach to the following word. Any time a noun preposition ends with the same vowel with which the following word begins, that vowel is dropped and the preposition is attached orthographically to the following word with an apostrophe.

\pex<ex:tvk-clitic-reduction>
	\a<le>\fw{le eðer} → \fw{l'eðer} \phnm{le\pstrs ðer} \defn{pens} \gloss{\In.\Pc.\Abs-pen}
	\a<mati>\fw{mati inam} → \fw{mat'inam} \phnm{ma.ti\pstrs nam} \defn{location} \gloss{\In.\Sg.\Top.\Acc-location}
	\a<no>\fw{no oko} → \fw{n'oko} \phnm{no\pstrs ko} \defn{dog} \gloss{\An.\Sg.\Top-pen}
	\a<su>\fw{su urda ablu} → \fw{s'urda ablu} \phnm{sur\pstrs da ab\pstrs lu} \defn{of the protected cat} \gloss{\An.\Sg.\Gen-protected-\An{} cat}
\xe

The other main example of cliticization is the particle \fw{vi}. It is used to ask questions and is most often added at the end of a sentence after the verb, as shown in example~\getref{ex:tvk-clitic-question}.

\ex<ex:tvk-clitic-question>
	\begingl
		\glpreamble\fw{No šekon tu fraþru oko usu vi?}\\
		\phnm{no ʃe\pstrs kon tu fraθ\pstrs ru o\pstrs ko u\pstrs su vi}//
		\gla no= šekon tu= fraþr-u oko us-u =vi//
		\glb \An.\Sg.\Top{}= runner \An.\Sg.\Acc{}= observant-\An{} dog have-\Ind.\Npst.\Ipfv{} =\Int//
		\glft \defn{Does the runner have an observant dog?}//
	\endgl
\xe

A speaker can, however, move the interrogative particle earlier in the sentence to focus the question on some specific element.

\pex<ex:tvk-clitic-question-focus>
	\a<erg>\begingl
		\glpreamble\fw{No šekon vi tu fraþru oko usu?}\\
		\phnm{no ʃe\pstrs kon vi tu fraθ\pstrs ru o\pstrs ko u\pstrs su}//
		\gla no= šekon =vi tu= fraþr-u oko us-u//
		\glb \An.\Sg.\Top{}= runner =\Int{} \An.\Sg.\Acc{}= observant-\An{} dog have-\Ind.\Npst.\Ipfv//
		\glft \defn{Is it the runner who has an observant dog?}//
	\endgl
	\a<acc>\begingl
		\glpreamble\fw{No šekon tu fraþru vi oko usu?}\\
		\phnm{no ʃe\pstrs kon tu fraθ\pstrs ru vi o\pstrs ko u\pstrs su}//
		\gla no= šekon tu= fraþr-u =vi oko us-u//
		\glb \An.\Sg.\Top{}= runner \An.\Sg.\Acc{}= observant-\An{} =\Int{} dog have-\Ind.\Npst.\Ipfv//
		\glft \defn{Is it an \emph{observant} dog the runner has?}//
	\endgl
	\a<adj>\begingl
		\glpreamble\fw{No šekon tu fraþru oko vi usu?}\\
		\phnm{no ʃe\pstrs kon tu fraθ\pstrs ru o\pstrs ko vi u\pstrs su}//
		\gla no= šekon tu= fraþr-u oko =vi us-u//
		\glb \An.\Sg.\Top{}= runner \An.\Sg.\Acc{}= observant-\An{} dog =\Int{} have-\Ind.\Npst.\Ipfv//
		\glft \defn{Is it an observant \emph{dog} the runner has?}//
	\endgl
\xe

\index{morphological typology!processes!cliticization|)}
\index{morphological typology!processes|)}

\section{Locus of Marking}
\label{sec:tvk-locus}
\index{morphological typology!locus of marking|(}

\langtvk{} is almost exclusively dependent-marking\autocite{wals-25}. This can readily be seen in the expression of possessive relationships, where the dependent is marked with the genitive case.

\pex<ex:tvk-depmarking-possessive>
	\a<pn>\remainpex\begin{minipage}[t]{0.5\remaining}
		\begingl
			\glpreamble\fw{tes botra}\\
			\phnm{\pstrs tes bot\pstrs ra}//
			\gla tes botra//
			\glb \Sps.\Gen{} wife//
			\glft \defn{your wife}//
		\endgl
	\end{minipage}
	\begin{minipage}[t]{0.5\remaining}
		\begin{forest} dotted tier
			[\fw{botra}
				[\fw{tes}
					[\fw{tes}, name=dependent]
				]
				[\fw{botra}, name=head]]
				\node at (head.south) {\textsc{\tiny head}};
				\node at (dependent.south) {\textsc{\tiny dependent}};
		\end{forest}
	\end{minipage}
	\a<n>\remainpex\begin{minipage}[t]{0.5\remaining}
		\begingl
			\glpreamble\fw{su esobiš botra}\\
			\phnm{su e.so\pstrs biʃ bot\pstrs ra}//
			\gla su= esobiš botra//
			\glb \An.\Sg.\Gen= patriot wife//
			\glft \defn{the patriot's wife}//
		\endgl
	\end{minipage}
	\begin{minipage}[t]{0.5\remaining}
		\begin{forest} dotted tier
			[\fw{botra}
				[\fw{esobiš}
					[\fw{su}
						[\fw{su}]]
					[\fw{esobiš}, name=dependent]
				]
				[\fw{botra}, name=head]]
			\node at (head.south) {\textsc{\tiny head}};
			\node at (dependent.south) {\textsc{\tiny dependent}};
		\end{forest}
	\end{minipage}
\xe

In example~\getfullref{ex:tvk-depmarking-possessive.pn}, \defn{you} are grammatically in possession of \fw{botra} \defn{wife}; the possessee forms the head of the clause while it is modified by the possessor, which receives the genitive inflection. In example~\getfullref{ex:tvk-depmarking-possessive.n}, \fw{botra} is still the possessee and thus the head of the clause while the genitive is marked on the dependent, \fw{esobiš} \defn{patriot}, using a noun preposition.

\index{morphological typology!locus of marking|)}


\excnt=1% Reset the numbering of examples to 1
\chapter{Grammatical Categories}
\label{cha:tvk-grammatical-categories}

\langtvk{} words can be divided into several different categories, or parts of speech. While the previous chapter dealt with the general mechanisms of marking words, this chapter will examine each of the various parts of speech in order to define their morphology more closely. The discussion will begin with an examination of nouns, pronouns, and verbs. Following this will be a discussion of the remaining parts of speech, including adverbs, numerals, and conjunctions.

\section{Nouns}
\label{sec:tvk-nouns}

Nouns in \langtvk{} decline to express number and gender (animacy) and are marked for case to indicate their grammatical role within the clause. As discussed in \autoref{cha:tvk-morphological-typology}, this inflection takes place not directly on the noun itself but on prepositional clitics that convey this grammatical meaning\autocite{wals-51}. For a full illustration of the declension paradigms, compare \autoref{tab:tvk-an-noun-decl} and \autoref{tab:tvk-in-noun-decl}. As shown in these tables, \langtvk{} noun inflections are never syncretic\autocite{wals-28}.

\afterpage{\clearpage
	\begin{table}\centering
		\caption[\langtvk{} Animate Noun Declension Paradigm]{\langtvk{} Animate Noun Declension Paradigm for the word \fw{bruþa} \defn{hand} or \defn{tool}}
		\label{tab:tvk-an-noun-decl}
		\begin{tabu}{| l | l l l |}
			\toprule
			\rowfont[c]\bfseries Case & \Sg & \Pc & \Pl\\
			\midrule
			\textbf{\Abs} & \fw{bruþa} & \fw{ri bruþa} & \fw{ran bruþa}\\
			\textbf{\Erg} & \fw{do bruþa} & \fw{das bruþa} & \fw{din bruþa}\\
			\textbf{\Acc} & \fw{tu bruþa} & \fw{tos bruþa} & \fw{ton bruþa}\\
			\textbf{\Dat} & \fw{ke bruþa} & \fw{kas bruþa} & \fw{ken bruþa}\\
			\textbf{\Gen} & \fw{su bruþa} & \fw{sar bruþa} & \fw{san bruþa}\\
			\midrule
			\textbf{\Top} & \fw{no bruþa} & \fw{nas bruþa} & \fw{nan bruþa}\\
			\textbf{\Top.\Acc} & \fw{nut bruþa} & \fw{nutos bruþa} & \fw{nuton bruþa}\\
			\textbf{\Top.\Dat} & \fw{nek bruþa} & \fw{nekas bruþa} & \fw{naken bruþa}\\
			\textbf{\Top.\Gen} & \fw{nus bruþa} & \fw{nosar bruþa} & \fw{nosan bruþa}\\
			\bottomrule
		\end{tabu}
	\end{table}
	\begin{table}\centering
		\caption[\langtvk{} Inanimate Noun Declension Paradigm]{\langtvk{} Inanimate Noun Declension Paradigm for the word \fw{šem} \defn{busyness}}
		\label{tab:tvk-in-noun-decl}
		\begin{tabu}{| l | l l l |}
			\toprule
			\rowfont[c]\bfseries Case & \Sg & \Pc & \Pl\\
			\midrule
			\textbf{\Abs} & \fw{šem} & \fw{le šem} & \fw{ren šem}\\
			\textbf{\Erg} & \fw{ða šem} & \fw{ðes šem} & \fw{dun šem}\\
			\textbf{\Acc} & \fw{ti šem} & \fw{þis šem} & \fw{ten šem}\\
			\textbf{\Dat} & \fw{ǩo šem} & \fw{kos šem} & \fw{ǩun šem}\\
			\textbf{\Gen} & \fw{šo šem} & \fw{se šem} & \fw{šen šem}\\
			\midrule
			\textbf{\Top} & \fw{mi šem} & \fw{mes šem} & \fw{nun šem}\\
			\textbf{\Top.\Acc} & \fw{mati šem} & \fw{moþes šem} & \fw{noten šem}\\
			\textbf{\Top.\Dat} & \fw{moǩ šem} & \fw{mekos šem} & \fw{nikun šem}\\
			\textbf{\Top.\Gen} & \fw{miš šem} & \fw{mise šem} & \fw{nušen šem}\\
			\bottomrule
		\end{tabu}
	\end{table}
}

\subsection{Gender}
\label{subsec:tvk-nouns-gender}

Grammatical gender in \langtvk{} consists of two\autocite{wals-30} non-sex-based\autocite{wals-31} classes based primarily on semantic ontological properties\autocite{wals-32}. The animate gender refers primarily to entities that are considered alive or are associated with life, movement, change, or dynamism. The inanimate gender refers primarily to entities that are not alive and are generally stationary or abstract. Grammatical gender in \langtvk{} can also be referred to as \enquote{animacy} since that is what the genders denote. Examples of nouns in each gender can be seen in example~\getref{ex:tvk-noun-genders}.

\pex<ex:tvk-noun-genders>
	\a<an>Animate nouns:\\
		\fw{botra} \defn{woman}, \fw{ǩalo} \defn{man}, \fw{eson} \defn{farmer}, \fw{okotik} \defn{puppy}, \fw{urdatil} \defn{ward}, \fw{bilt} \defn{breath}
	\a<in>Inanimate nouns:\\
		\fw{esotik} \defn{country}, \fw{dedu} \defn{sky}, \fw{elbi} \defn{egg}, \fw{usudir} \defn{basket}, \fw{akrapis} \defn{letter}, \fw{fradir} \defn{glasses}
\xe

Since the nouns themselves are not directly inflected, with grammatical information instead shown on prepositional particles, it is impossible to tell what gender a noun is based solely on its word form.

Some nouns are able to change category in certain circumstances. For example, plants and animals switch from the animate gender to the inanimate gender when they serve as food. Further, there exist some duplicates with otherwise identical words declining to opposite genders.

\subsection{Number}
\label{subsec:tvk-nouns-number}

Grammatical number in \langtvk{} consists of three numbers, all of which are coded on the noun prepositions\autocite{wals-33}. The singular is always used when there is only one of the referent noun, the paucal is used when there are two to five of the referent noun, and the plural is used when there are more than five of the referent noun.

\pex<ex:tvk-noun-numbers>
	\a<sg> \fw{su ima} \phnm{su i\pstrs ma} \defn{of mother} \gloss{\Sg.\An.\Gen= mother}
	\a<pc> \fw{sar ima} \phnm{sar i\pstrs ma} \defn{of (some) mothers} \gloss{\Pc.\An.\Gen= mother}
	\a<pl> \fw{san ima} \phnm{san i\pstrs ma} \defn{of (several) mothers} \gloss{\Pl.\An.\Gen= mother}
\xe

When a numeral is used to identify the number of a referent noun, the singular is used instead of the paucal or plural, even if without the numeral the other forms would be used\autocite{wals-34}.

\pex<ex:tvk-noun-numbers-numerals>
	\a<sg> \fw{k'eþ ima} \phnm{keθ i\pstrs ma} \defn{to one mother} \gloss{\Sg.\An.\Dat=one mother}
	\a<pc> \fw{ke arsi ima} \phnm{ke ar\pstrs si i\pstrs ma} \defn{to three mothers} \gloss{\Sg.\An.\Dat= three mother}, not \ungr{\fw{kas arsi ima}}
	\a<pl> \fw{ke bruð abom ima} \phnm{ke bruð a\pstrs bom i\pstrs ma} \defn{to seven mothers} \gloss{\Sg.\An.\Dat= five two mother}, not \ungr{\fw{ken bruð abom ima}}
\xe

Most nouns that represent concrete entities are countable, including some words that in English are uncountable like corn, and by default they are used in the singular form unlike English words like pants or glasses. However, many entities that are not easily split into discreet parts like liquids, grains, and certain abstract concepts are uncountable, such as \fw{elto} \phnm{el\pstrs to} \defn{water}. Occasionally, when a word's semantics cover multiple concepts, a word can be variably countable or uncountable; when \fw{dedu} \phnm{de\pstrs du} is used to mean \defn{sky} or \defn{heaven}, it is uncountable, but when it is used to mean \defn{ceiling}, it is countable and can be made paucal or plural.

People's names can also be declined to the paucal or plural number to indicate the associative plural\autocite{wals-36}. This form is used to refer to a person and the other people associated with that person. For example, \fw{ri Bol} \phnm{ri bol} \gloss{\Pc.\An.\Abs{} Bol} refers to Bol and two to five other people associated with him. Similarly, \fw{ran Ote} \phnm{ran o\pstrs te} \gloss{\Pl.\An.\Abs{} Ote} refers to Ote and the group he is with.

\subsection{Case}
\label{subsec:tvk-nouns-case}

As shown in Tables \ref{tab:tvk-an-vowel-decl} and \ref{tab:tvk-in-vowel-decl}, \langtvk{} noun phrases decline to five different grammatical cases\autocite{wals-49} in order to show their role in the sentence. These cases are governed by the phrase's verb or assigned to adjuncts depending on their purpose or meaning. As shown in the same declension tables, any of these grammatical cases can be replaced by or combined with topic markers. See \autoref{subsec:tvk-nouns-topicality} for more information on topicality.

\subsubsection{Absolutive and Intransitive}
\label{subsubsec:tvk-nouns-absolutive}

The intransitive case marks a noun or noun phrase that serves as the subject of an intransitive verb like \fw{šeli} \defn{to run} or a transitive verb used intransitively like \fw{ufuli} \defn{to sing} (without naming the object, what is being sung). This means that when a verb has only a single argument, that argument will by default be in the intransitive case. That is true whether the subject is serving like an agent as in words like \fw{šeli} \defn{to run} or \fw{ufuli} \defn{to sing} or when the subject is serving more like a patient as in words like \fw{orðali} \defn{to fall}.

\pex<ex:tvk-noun-abs>
	\a<a1>\begingl
		\glpreamble\fw{Mollur šeþ.}\\
		\phnm{mo\pstrs\gem{l}ur \pstrs ʃeθ}//
		\gla ∅= Mollur š-eþ//
		\glb \An.\Sg.\Intr= Mollur run-\Ind.\Npst.\Prg//
		\glft\defn{Mollur is running.}//
	\endgl
	\a<a2>\begingl
		\glpreamble\fw{R'ima ufu.}\\
		\phnm{ri\pstrs ma u\pstrs fu}//
		\gla ri=ima uf-u//
		\glb \An.\Pc.\Intr=mother sing-\Ind.\Npst.\Ipfv//
		\glft\defn{The mothers sing.}//
	\endgl
	\a<p>\begingl
		\glpreamble\fw{Ren fild orðak.}\\
		\phnm{ren \pstrs fild or\pstrs ðak}//
		\gla ren= fild orð-ak//
		\glb \In.\Pl.\Intr= doll fall-\Ind.\Pst.\Pfv//
		\glft\defn{The dolls fell.}//
	\endgl
\xe

Note that the singular intransitive case is entirely unmarked by any preposition. This is true whether the noun is animate or inanimate.

\pex<ex:tvk-noun-abs-sg>
	\a<an>\begingl
		\glpreamble\fw{Alum uldeteš.}\\
		\phnm{a\pstrs lum ul.de\pstrs teʃ}//
		\gla ∅= alum uldet-eš//
		\glb \An.\Sg.\Intr= cloud change-\Ind.\Npst.\Rtsp//
		\glft\defn{The cloud has changed.}//
	\endgl
	\a<in>\begingl
		\glpreamble\fw{Almaþ uldeteš.}\\
		\phnm{al\pstrs maθ ul.de\pstrs teʃ}//
		\gla ∅= almaþ uldet-eš//
		\glb \In.\Sg.\Intr= village change-\Ind.\Npst.\Rtsp//
		\glft\defn{The village has changed.}//
	\endgl
\xe

However, the subject of certain transitive verbs will also take the intransitive case if the semantic meaning of the verb is stative. See \autoref{subsubsec:tvk-nouns-dative} Dative for more information on this. Since it is used in these situations, and since the intransitive is the citation form, the case is normally referred to as the absolutive case, even when used intransitively. These terms are interchangeable.

\ex<ex:tvk-noun-abs-trans>
	\begingl
		\glpreamble\fw{Ter ke arb fra vi?}\\
		\phnm{ter ke arb \pstrs fra vi}//
		\gla ter ke= arb fr-a =vi//
		\glb \Sps.\Abs{} \An.\Sg.\Dat= bird see-\Ind.\Npst.\Ipfv{} =\Int//
		\glft\defn{Do you see a bird?}//
	\endgl
\xe

The absolutive case is frequently used with postpositions to indicate a location where or through which an action is taken, for example being placed at, on, or in something.

\pex<ex:tvk-noun-abs-pp>
	\a<on>\begingl
		\glpreamble\fw{Ablu onaš e onek.}\\
		\phnm{ab\pstrs lu o\pstrs naʃ e o\pstrs nek}//
		\gla ∅= ablu ∅= onaš e on-ek//
		\glb \An.\Sg.\Abs= cat \In.\Sg.\Abs= rug on play-\Ind.\Pst.\Pfv//
		\glft\defn{The cat played on the rug.}//
	\endgl
	\a<over>\begingl
		\glpreamble\fw{Mod ti ennis l'elbi arku ǧirak.}\\
		\phnm{mod ti e\pstrs\gem{n}is lel\pstrs bi ar\pstrs ku ɣi\pstrs rak}//
		\gla mod ti= ennis le=elbi arku ǧir-ak//
		\glb \Fps.\Erg{} \In.\Sg.\Acc= ball \In.\Pc.\Abs=egg above throw-\Ind.\Pst.\Pfv//
		\glft\defn{I threw the ball over the eggs.}//
	\endgl
\xe

When an action is done \defn{with} or \defn{without} a noun, the absolutive case will be used.

\ex<ex:tvk-noun-abs-with>
	\begingl
		\glpreamble\fw{Oko ablu mo oneþ.}\\
		\phnm{o\pstrs ko ab\pstrs lu mo o\pstrs neθ}//
		\gla ∅= oko ∅= ablu mo on-eþ//
		\glb \An.\Sg.\Abs= dog \An.\Sg.\Abs= cat with play-\Ind.\Npst.\Prg//
		\glft\defn{The dog is playing with the cat.}//
	\endgl
\xe

The absolutive case is also used when directly addressing someone in a vocative function. The noun functioning in this way is often placed at the beginning or end of the sentence separated by a pause in speech or a comma in writing.

\pex<ex:tvk-noun-abs-voc>
	\a<imp>\begingl
		\glpreamble\fw{Lerk, šebanta.}\\
		\phnm{\pstrs lerk, ʃe\pstrs ban.ta}//
		\gla ∅= Lerk šeb-anta//
		\glb \An.\Sg.\Abs= Lerk run-\Imp//
		\glft\defn{Run, Lerk.}//
	\endgl
	\a<nimp>\begingl
		\glpreamble\fw{Sud tu tavotik urdateþ, Erme.}\\
		\phnm{sud tu ta.vo\pstrs tik ur.da\pstrs teθ er\pstrs me}//
		\gla sud tu= tavotik urdat-eþ ∅= Erme//
		\glb \Tps.\An.\Erg{} \An.\Sg.\Acc= child guard-\Ind.\Npst.\Prg{} \An.\Sg.\Abs= Erme//
		\glft\defn{He is guarding the child, Erme.}//
	\endgl
\xe

\subsubsection{Ergative}
\label{subsubsec:tvk-nouns-ergative}

The ergative case marks a noun or noun phrase that serves as the subject of an active transitive verb or any ditransitive verb. This means that when a verb has multiple arguments and the semantic meaning of the verb is active, the subject argument will by default by in the ergative case.

\pex<ex:tvk-noun-erg>
	\a<eðerali>\begingl
		\glpreamble\fw{Do Tlunda ti akrapis eðeraš.}\\
		\phnm{do tlun\pstrs da ti ak.ra\pstrs pis e.ðe\pstrs raʃ}//
		\gla do= Tlunda ti= akrapis eðer-aš//
		\glb \An.\Sg.\Erg= Tlunda \In.\Sg.\Acc= letter pen-\Ind.\Npst.\Rtsp//
		\glft\defn{Tlunda has penned a letter.}//
	\endgl
	\a<okotali>\begingl
		\glpreamble\fw{Das oko tu ablu okotam.}\\
		\phnm{das o\pstrs ko tu ab\pstrs lu o.ko\pstrs tam}//
		\gla das= oko tu= ablu okot-am//
		\glb \An.\Pc.\Erg= dog \An.\Sg.\Acc= cat chase-\Ind.\Pst.\Ipfv//
		\glft\defn{The dogs chased the cat.}//
	\endgl
	\a<visali>\begingl
		\glpreamble\fw{Din avo ten usudir visaǧ.}\\
		\phnm{din a\pstrs vo ten u.su\pstrs dir vi\pstrs saɣ}//
		\gla din= avo ten= usudir vis-aǧ//
		\glb \An.\Pl.\Erg= father \In.\Pl.\Acc= basket take.away-\Ind.\Pst.\Rtsp//
		\glft\defn{The father and his associates had taken away the baskets.}//
	\endgl
\xe

\subsubsection{Accusative}
\label{subsubsec:tvk-nouns-accusative}

The accusative case marks a noun or noun phrase that serves as the direct object of an active transitive verb or any ditransitive verb.

\pex<ex:tvk-noun-acc>
	\a<onašuli>\begingl
		\glpreamble\fw{Do akrakon þis eðerik alma e onašuk.}\\
		\phnm{do ak.ra\pstrs kon θis e.ðe\pstrs rik al\pstrs ma e o.na\pstrs ʃuk}//
		\gla do= akrakon þis= eðerik alma e onaš-uk//
		\glb \An.\Sg.\Erg= writer \In.\Pc.\Acc= pencil house in place-\Ind.\Pst.\Pfv//
		\glft\defn{The writer placed the pencils in the house.}//
	\endgl
	\a<uldetuli>\begingl
		\glpreamble\fw{Do šus botra ti šus akrapis uldetuk.}\\
		\phnm{do ʃus bot\pstrs ra ti ʃus ak.ra\pstrs pis ul.de\pstrs tuk}//
		\gla do= šus botra ti= šus akrapis uldet-uk//
		\glb \An.\Sg.\Erg= \Tpp.\An.\Gen{} wife \In.\Sg.\Acc= \Tpp.\An.\Gen{} letter change-\Ind.\Pst.\Pfv//
		\glft\defn{His wife changed his letter.}//
	\endgl
\xe

\subsubsection{Dative}
\label{subsubsec:tvk-nouns-dative}

The dative case marks a noun or noun phrase that serves as the indirect object of a ditransitive verb, a recipient of an action, or the entity for whose benefit or detriment the action is taken.

\ex<ex:tvk-noun-dat>
	\begingl
		\glpreamble\fw{Do eson ti ennis ke oko draš.}\\
		\phnm{do e\pstrs son ti e\pstrs\gem{n}is ke o\pstrs ko \pstrs draʃ}//
		\gla do= eson ti= ennis ke= oko dr-aš//
		\glb \An.\Sg.\Erg= farmer \In.\Sg.\Acc= ball \An.\Sg.\Dat= dog give-\Ind.\Npst.\Rtsp//
		\glft\defn{The farmer has given the dog a ball.}//
	\endgl
\xe

Certain monotransitive verbs are used with the absolutive and dative cases instead of the ergative and accusative cases. These tend to be stative verbs in which the object of the verb is unaffected by the action or there is little volition on the part of the subject.

\pex<ex:tvk-noun-dat-trans>
	\a<teguli>\begingl
		\glpreamble\fw{Mor tek tegu.}\\
		\phnm{mor tek te\pstrs gu}//
		\gla mor tek teg-u//
		\glb \Fps.\Abs{} \Sps.\Dat{} worry-\Ind.\Npst.\Ipfv//
		\glft\defn{I worry for you.}//
	\endgl
	\a<keðali>\begingl
		\glpreamble\fw{Ran urdaton ken ufukon keðam.}\\
		\phnm{ran ur.da\pstrs ton ken u.fu\pstrs kon ke\pstrs ðam}//
		\gla ran= urdaton ken= ufukon keð-am//
		\glb \An.\Pl.\Abs= guard \An.\Pl.\Dat= singer admire-\Ind.\Pst.\Ipfv//
		\glft\defn{The guards admired the singers.}//
	\endgl
\xe

When a verb is done on behalf of or for someone or something, the beneficiary of that action will be declined to the dative and followed by the postposition \fw{li} \phnm{li} \defn{for}.

\pex<ex:tvk-noun-dat-beneficiary>
	\a<oveli>\begingl
		\glpreamble\fw{Sur kas šus botrašut li ove.}\\
		\phnm{sur kas ʃus bot.ra\pstrs ʃut li o\pstrs ve}//
		\gla sur kas= šus botrašut li ov-e//
		\glb \Tps.\An.\Abs{} \An.\Pc.\Dat= \Tps.\An.\Gen{} fiancée for cook-\Ind.\Npst.\Ipfv//
		\glft\defn{He cooks for his fiancée and her friends.}//
	\endgl
	\a<urdateli>\begingl
		\glpreamble\fw{Do Blimva tu okotik ke šus avo li urdateþ.}\\
		\phnm{do blim\pstrs va tu o.ko\pstrs tik ke ʃus a\pstrs vo li ur.da\pstrs teθ}//
		\gla do= Blimva tu= okotik ke= šus avo li urdat-eþ//
		\glb \An.\Sg.\Erg= Blimva \An.\Sg.\Acc= puppy \An.\Sg.\Dat= \Tps.\An.\Gen{} father for protect-\Ind.\Npst.\Prg//
		\glft\defn{Blimva is protecting the puppy for her father.}//
	\endgl
\xe

The dative case can also be used in an allative sense to express movement to or toward.

\ex<ex:tvk-noun-dat-movement>
	\begingl
		\glpreamble\fw{Mor ǩo alma bi šeba.}\\
		\phnm{mor xo al\pstrs ma bi ʃe\pstrs ba}//
		\gla mor ǩo= alma to šeb-a//
		\glb \Fps.\An{} \In.\Sg.\Dat= house to run-\Ind.\Npst.\Ipfv//
		\glft\defn{I run to the house.}//
	\endgl
\xe

This can result in subtle changes in meaning when used with ditransitive verbs.

\pex<ex:tvk-noun-dat-movement-ditr>
	\a<to>\begingl
		\glpreamble\fw{Mod þis ennis tek ǧira.}\\
		\phnm{mod θis e\pstrs\gem{n}is tek ɣi\pstrs ra}//
		\gla mod þis= ennis tek ǧir-a//
		\glb \Fps.\Erg{} \In.\Pc.\Acc= ball \Sps.\Dat{}  throw-\Ind.\Npst.\Ipfv//
		\glft\defn{I throw the balls to you.}//
	\endgl
	\a<at>\begingl
		\glpreamble\fw{Mod þis ennis tek bi ǧira.}\\
		\phnm{mod θis e\pstrs\gem{n}is tek bi ɣi\pstrs ra}//
		\gla mod þis= ennis tek bi ǧir-a//
		\glb \Fps.\Erg{} \In.\Pc.\Acc= ball \Sps.\Dat{} at  throw-\Ind.\Npst.\Ipfv//
		\glft\defn{I throw the balls at you.}//
	\endgl
\xe

Notice in example~\getfullref{ex:tvk-noun-dat-movement-ditr.to} that \fw{tek} is the recipient of the action while in example~\getfullref{ex:tvk-noun-dat-movement-ditr.at} \fw{tek} is the target of the action.

\subsubsection{Genitive}
\label{subsubsec:tvk-nouns-genitive}

The genitive case is used to mark the possessor of a noun or noun phrase.

\ex<ex:tvk-noun-gen-roles>
	\begingl
		\glpreamble\fw{Su Goltu botra mok fra.}\\
		\phnm{su gol\pstrs tu bot\pstrs ra mok \pstrs fra}//
		\gla ∅= su= Goltu botra mok fr-a//
		\glb \An.\Sg.\Abs= \An.\Sg.\Gen= Goltu wife \Fps.\Dat{} see-\Ind.\Npst.\Ipfv//
		\glft\defn{Goltu's wife sees me.}//
	\endgl
\xe

Just like other attributives, the genitive phrase will occur between the possessee and its declension clitic.

\pex<ex:tvk-noun-gen>
	\a<ergacc>\begingl
		\glpreamble\fw{Do su Zarsa oko tu mos ablu okotaða!}\\
		\phnm{do su zar\pstrs sa o\pstrs ko tu mos ab\pstrs lu o.ko\pstrs ta.ða}//
		\gla do= su= Zarsa oko tu= mos ablu okot-aða//
		\glb \An.\Sg.\Erg= \An.\Sg.\Gen= Zarsa dog \An.\Sg.\Acc= \Fps.\Gen{} cat chase-\Ind.\Pst.\Prg//
		\glft\defn{Zarsa's dog was chasing my cat!}//
	\endgl
	\a<dat>\begingl
		\glpreamble\fw{Mod ti ennis ke su Inki oko ǧira.}\\
		\phnm{mod ti e\pstrs\gem{n}is ke su in\pstrs ki o\pstrs ko ɣi\pstrs ra}//
		\gla mod þis= ennis ke= su= Inki oko ǧir-a//
		\glb \Fps.\Erg{} \In.\Sg.\Acc= ball \An.\Sg.\Dat= \An.\Sg.\Gen= Inki dog throw-\Ind.\Npst.\Ipfv//
		\glft\defn{I throw the ball to Inki's dog.}//
	\endgl
\xe

When a verb is done because of or due to someone or something, the cause of that action will be declined to the genitive and followed by the postposition \fw{li} \phnm{li} \defn{because of}.

\pex<ex:tvk-noun-gen-cause>
	\a<oveli>\begingl
		\glpreamble\fw{Sur su šus botrašut li puzaða arke ovek.}\\
		\phnm{sur su ʃus bot.ra\pstrs ʃut li pu\pstrs za.ða ar\pstrs ke o\pstrs vek}//
		\gla sur su= šus botrašut li puz-aða arke ov-ek//
		\glb \Tps.\An.\Abs{} \An.\Sg.\Gen= \Tps.\An.\Gen{} fiancée because.of cry-\Ind.\Pst.\Prg{} who cook-\Ind.\Pst.\Pfv//
		\glft\defn{He cooked because his fiancée was crying.}//
	\endgl
	\a<urdateli>\begingl
		\glpreamble\fw{Do Blimva tu okotik su šus avo li urdateþ.}\\
		\phnm{do blim\pstrs va tu o.ko\pstrs tik su ʃus a\pstrs vo li ur.da\pstrs teθ}//
		\gla do= Blimva tu= okotik su= šus avo li urdat-eþ//
		\glb \An.\Sg.\Erg= Blimva \An.\Sg.\Acc= puppy \An.\Sg.\Gen= \Tps.\An.\Gen{} father because.of protect-\Ind.\Npst.\Prg//
		\glft\defn{Blimva is protecting the puppy from her father.}//
	\endgl
\xe

The genitive can also be used in an ablative sense to express movement from or away.

\ex<ex:tvk-noun-gen-movement>
	\begingl
		\glpreamble\fw{Mor šo alma gu šeba.}\\
		\phnm{mor ʃo al\pstrs ma gu ʃe\pstrs ba}//
		\gla mor šo= alma to šeb-a//
		\glb \Fps.\An{} \In.\Sg.\Gen= house from run-\Ind.\Npst.\Ipfv//
		\glft\defn{I run from the house.}//
	\endgl
\xe

\subsection{Topicality}
\label{subsec:tvk-nouns-topicality}

Several noun cases have variants that mark a noun as the topic of a discourse. The topic is the entity most closely associated with the higher-level theme of the paragraph.

The case preposition that encodes \textit{only} topicality completely replaces the case marking for a noun that is in the absolutive or the ergative.

\pex<ex:tvk-noun-top-subj>
	\a<abs>\begingl
		\glpreamble\fw{No Mollur šeþ.}\\
		\phnm{no mo\pstrs\gem{l}ur \pstrs ʃeθ}//
		\gla no= Mollur š-eþ//
		\glb \An.\Sg.\Top= Mollur run-\Ind.\Npst.\Prg//
		\glft\defn{Mollur is running.}//
	\endgl
	\a<abs-tr>\begingl
		\glpreamble\fw{Þan ke arb fra vi?}\\
		\phnm{θan ke arb \pstrs fra vi}//
		\gla þan ke= arb fr-a =vi//
		\glb \Sps.\Top{} \An.\Sg.\Dat= bird see-\Ind.\Npst.\Ipfv{} =\Int//
		\glft\defn{Do you see a bird?}//
	\endgl
	\a<erg>\begingl
		\glpreamble\fw{Nas oko tu ablu okotam.}\\
		\phnm{nas o\pstrs ko tu ab\pstrs lu o.ko\pstrs tam}//
		\gla nas= oko tu= ablu okot-am//
		\glb \An.\Pc.\Top= dog \An.\Sg.\Acc= cat chase-\Ind.\Pst.\Ipfv//
		\glft\defn{The dogs chased the cat.}//
	\endgl
\xe

This case preposition also completely replaces the accusative and dative cases, but only in certain situations when the intended case is inferable. In other words, it replaces the accusative case only when the ergative is present within the sentence, it replaces the dative in a monotransitive sentence only when the absolutive case is present, and it replaces the dative in a ditransitive sentence only when both the ergative and the accusative are present.

\pex<ex:tvk-noun-top-nonsubj>
	\a<acc>\begingl
		\glpreamble\fw{Do šus botra mi šus akrapis uldetuk.}\\
		\phnm{do ʃus bot\pstrs ra mi ʃus ak.ra\pstrs pis ul.de\pstrs tuk}//
		\gla do= šus botra mi= šus akrapis uldet-uk//
		\glb \An.\Sg.\Erg= \Tpp.\An.\Gen{} wife \In.\Sg.\Top= \Tpp.\An.\Gen{} letter change-\Ind.\Pst.\Pfv//
		\glft\defn{His wife changed his letter.}//
	\endgl
	\a<dat-st>\begingl
		\glpreamble\fw{Ran urdaton nan ufukon keðam.}\\
		\phnm{ran ur.da\pstrs ton nan u.fu\pstrs kon ke\pstrs ðam}//
		\gla ran= urdaton nan= ufukon keð-am//
		\glb \An.\Pl.\Abs= guard \An.\Pl.\Top= singer admire-\Ind.\Pst.\Ipfv//
		\glft\defn{The guards admired the singers.}//
	\endgl
	\a<dat-ac>\begingl
		\glpreamble\fw{Do eson ti ennis no oko draš.}\\
		\phnm{do e\pstrs son ti e\pstrs\gem{n}is no o\pstrs ko \pstrs draʃ}//
		\gla do= eson ti= ennis no= oko dr-aš//
		\glb \An.\Sg.\Erg= farmer \In.\Sg.\Acc= ball \An.\Sg.\Top= dog give-\Ind.\Npst.\Rtsp//
		\glft\defn{The farmer has given the dog a ball.}//
	\endgl
\xe

For other situations, there exist combined forms to mark a noun as the topic when it is in the accusative, dative, or genitive case.

\pex<ex:tvk-noun-top-combined>
	\a<acc>\begingl
		\glpreamble\fw{Nut ablu okotam.}\\
		\phnm{nut ab\pstrs lu o.ko\pstrs tam}//
		\gla nut= ablu okot-am//
		\glb \An.\Sg.\Acc.\Top= cat chase-\Ind.\Pst.\Ipfv//
		\glft\defn{The cats were chased.}//
	\endgl
	\a<dat>\begingl
		\glpreamble\fw{Naken ufukon keðam.}\\
		\phnm{na\pstrs ken u.fu\pstrs kon ke\pstrs ðam}//
		\gla naken= ufukon keð-am//
		\glb \An.\Pl.\Dat.\Top= singer admire-\Ind.\Pst.\Ipfv//
		\glft\defn{The singers were admired.}//
	\endgl
	\a<gen>\begingl
		\glpreamble\fw{Mod ti ennis ke þansu oko ǧira.}\\
		\phnm{mod ti e\pstrs\gem{n}is ke θan\pstrs su o\pstrs ko ɣi\pstrs ra}//
		\gla mod þis= ennis ke= þansu oko ǧir-a//
		\glb \Fps.\Erg{} \In.\Sg.\Acc= ball \An.\Sg.\Dat= \Sps.\Gen.\Top{} dog throw-\Ind.\Npst.\Ipfv//
		\glft\defn{I throw the ball to your dog.}//
	\endgl
\xe

See \autoref{sec:tvk-discourse-topic} for a greater explanation of how the topic is used within discourse.

\section{Pronouns and Determiners}
\label{sec:tvk-pronouns-determiners}
\langtvk{} has several types of pronouns and determiners that serve as anaphora, including personal pronouns, demonstrative pronouns, interrogative pronouns, relative pronouns, and other indefinite pronouns.

\subsection{Personal Pronouns}
\label{subsec:tvk-personal-pronouns}

\afterpage{\clearpage
	\begin{table}\centering
		\caption[\langtvk{} Personal Pronouns]{\langtvk{} Personal Pronouns}
		\label{tab:tvk-prs-pn}
		\begin{tabu}{| l | l l l l l l l l l |}
			\toprule
			\rowfont[c]\bfseries Person & \Abs{} & \Erg{} & \Acc{} & \Dat{} & \Gen{} & \Top{} & \Top.\Acc{} & \Top.\Dat{} & \Top.\Gen{}\\
			\midrule
			\Fps{} & \fw{mor} & \fw{mod} & \fw{mot} & \fw{mok} & \fw{mos} & \fw{mon} & \fw{montu} & \fw{monke} & \fw{monsu}\\
			\Fpc{} & \fw{morsa} & \fw{modas} & \fw{motos} & \fw{mokas} & \fw{mosar} & \fw{monsa} & \fw{monsut} & \fw{monsek} & \fw{monsus}\\
			\Fpp{} & \fw{morna} & \fw{modin} & \fw{moton} & \fw{moken} & \fw{mosan} & \fw{mana} & \fw{manut} & \fw{manek} & \fw{manus}\\
			\midrule
			\Sps{} & \fw{ter} & \fw{ted} & \fw{þet} & \fw{tek} & \fw{tes} & \fw{þan} & \fw{þantu} & \fw{þanke} & \fw{þansu}\\
			\Spc{} & \fw{tersa} & \fw{tedas} & \fw{þetos} & \fw{tekas} & \fw{tesar} & \fw{tensa} & \fw{tensut} & \fw{tensek} & \fw{tensus}\\
			\Spp{} & \fw{terna} & \fw{tedin} & \fw{þeton} & \fw{token} & \fw{tesan} & \fw{tana} & \fw{tanut} & \fw{tanek} & \fw{tanus}\\
			\midrule
			\Tps.\An{} & \fw{sur} & \fw{sud} & \fw{sut} & \fw{suk} & \fw{šus} & \fw{šun} & \fw{šuntu} & \fw{šunke} & \fw{šunsu}\\
			\Tpc.\An{} & \fw{suša} & \fw{sudas} & \fw{sutos} & \fw{sukas} & \fw{šusar} & \fw{sunas} & \fw{šunsut} & \fw{šunsek} & \fw{šunsus}\\
			\Tpp.\An{} & \fw{surna} & \fw{sudin} & \fw{suton} & \fw{suken} & \fw{šusan} & \fw{šona} & \fw{šonut} & \fw{šonek} & \fw{šonus}\\
			\midrule
			\Tps.\In{} & \fw{gir} & \fw{gid} & \fw{git} & \fw{gake} & \fw{gis} & \fw{gin} & \fw{gintu} & \fw{ginke} & \fw{ginsu}\\
			\Tpc.\In{} & \fw{girsa} & \fw{gidas} & \fw{gitos} & \fw{gokas} & \fw{gisar} & \fw{ginsa} & \fw{ginsut} & \fw{ginsek} & \fw{ginsus}\\
			\Tpp.\In{} & \fw{girna} & \fw{gidun} & \fw{giton} & \fw{goken} & \fw{gisan} & \fw{gana} & \fw{ganut} & \fw{ganek} & \fw{ganus}\\
			\bottomrule
		\end{tabu}
	\end{table}
}

As shown in \autoref{tab:tvk-prs-pn}, \langtvk{} contains several personal pronouns. These pronouns are symmetrical to other nouns and noun phrases\autocite{wals-50}, declining to show gender, number, case, and topicality just like nouns while adding person.

Historically, all pronouns were regular formations with the case-marking preposition and a person-marking pronoun, but over time, these words combined and fused as grammaticalization progressed. The forms are now completely fused.

\pex<ex:tvk-pers-pronoun>
	\a<none>\begingl
		\glpreamble\fw{Do Tlunda ti ennis ke su Lerk oko ǧirak.}\\
		\phnm{do tlun\pstrs da ti e\pstrs\gem{n}is ke su \pstrs lerk o\pstrs ko ɣi\pstrs rak}//
		\gla do= Tlunda ti= ennis ke= su= Lerk oko ǧir-ak//
		\glb \An.\Sg.\Erg= Tlunda \In.\Sg.\Acc= ball \An.\Sg.\Dat= \An.\Sg.\Gen= Lerk dog throw-\Ind.\Pst.\Pfv//
		\glft\defn{Tlunda threw the ball to Lerk's dog.}//
	\endgl
	\a<erg>\begingl
		\glpreamble\fw{Sud ti ennis ke su Lerk oko ǧirak.}\\
		\phnm{\pstrs sud ti e\pstrs\gem{n}is ke su \pstrs lerk o\pstrs ko ɣi\pstrs rak}//
		\gla Sud ti= ennis ke= su= Lerk oko ǧir-ak//
		\glb \Tps.\An.\Erg{} \In.\Sg.\Acc= ball \An.\Sg.\Dat= \An.\Sg.\Gen= Lerk dog throw-\Ind.\Pst.\Pfv//
		\glft\defn{She threw the ball to Lerk's dog.}//
	\endgl
	\a<acc>\begingl
		\glpreamble\fw{Do Tlunda git ke su Lerk oko ǧirak.}\\
		\phnm{do tlun\pstrs da \pstrs git ke su \pstrs lerk o\pstrs ko ɣi\pstrs rak}//
		\gla do= Tlunda git ke= su= Lerk oko ǧir-ak//
		\glb \An.\Sg.\Erg= Tlunda \Tps.\In.\Acc{} \An.\Sg.\Dat= \An.\Sg.\Gen= Lerk dog throw-\Ind.\Pst.\Pfv//
		\glft\defn{Tlunda threw it to Lerk's dog.}//
	\endgl
	\a<gen>\begingl
		\glpreamble\fw{Do Tlunda ti ennis ke šus oko ǧirak.}\\
		\phnm{do tlun\pstrs da ti e\pstrs\gem{n}is ke \pstrs ʃus o\pstrs ko ɣi\pstrs rak}//
		\gla do= Tlunda ti= ennis ke= šus oko ǧir-ak//
		\glb \An.\Sg.\Erg= Tlunda \In.\Sg.\Acc= ball \An.\Sg.\Dat= \Tps.\An.\Gen{} dog throw-\Ind.\Pst.\Pfv//
		\glft\defn{Tlunda threw the ball to his dog.}//
	\endgl
	\a<dat>\begingl
		\glpreamble\fw{Do Tlunda ti ennis suk ǧirak.}\\
		\phnm{do tlun\pstrs da ti e\pstrs\gem{n}is \pstrs suk ɣi\pstrs rak}//
		\gla do= Tlunda ti= ennis suk ǧir-ak//
		\glb \An.\Sg.\Erg= Tlunda \In.\Sg.\Acc= ball \An.\Sg.\Dat= \An.\Sg.\Gen= Lerk dog throw-\Ind.\Pst.\Pfv//
		\glft\defn{Tlunda threw the ball to him.}//
	\endgl
\xe

Personal pronouns are used the same way their full noun phrase counterparts are, in both core and non-core cases, and replace the full noun phrase for which they are serving as anaphor. Example~\getfullref{ex:tvk-pers-pronoun.none} shows a full sentence without any pronouns; examples~\getfullref{ex:tvk-pers-pronoun.erg}–\getref{ex:tvk-pers-pronoun.dat} then show variations on this sentence with different noun phrases replaced with pronouns. Notice that the pronoun replaces the full noun phrase, for example in \getfullref{ex:tvk-pers-pronoun.gen} where \fw{šus} replaces only \fw{su Lerk}, the noun in the genitive, whereas in \getfullref{ex:tvk-pers-pronoun.dat}, \fw{suk} replaces \fw{ke su Lerk oko}, the full dative noun phrase. Similarly, when a noun phrase contains an adjective, the whole noun phrase is replaced, including the adjective, as in example~\getref{ex:tvk-pers-pronoun-adj}.

\pex<ex:tvk-pers-pronoun-adj>
	\a<none>\begingl
		\glpreamble\fw{Bol no fraþru botra kantek.}\\
		\phnm{\pstrs bol no fraθ\pstrs ru bot\pstrs ra kan\pstrs tek}//
		\gla ∅= Bol no= fraþru botra kant-ek//
		\glb \An.\Sg.\Abs= Bol \An.\Sg.\Top= observant woman thank-\Ind.\Pst.\Ipfv//
		\glft\defn{Bol thanked the observant woman.}//
	\endgl
	\a<bad>\begingl
		\glpreamble\ljudge{\ungr}\fw{Bol fraþru šun kantek.}\\
		\phnm{\pstrs bol fraθ\pstrs ru \pstrs ʃun kan\pstrs tek}//
		\gla ∅= Bol fraþru šun kant-ek//
		\glb \An.\Sg.\Abs= Bol observant \Tps.\An.\Top{} thank-\Ind.\Pst.\Ipfv//
		\glft\ljudge{\ungr}\defn{Bol thanked the observant her.}//
	\endgl
	\a<good>\begingl
		\glpreamble\fw{Bol šun kantek.}\\
		\phnm{\pstrs bol \pstrs ʃun kan\pstrs tek}//
		\gla ∅= Bol šun kant-ek//
		\glb \An.\Sg.\Abs= Bol \Tps.\An.\Top{} thank-\Ind.\Pst.\Ipfv//
		\glft\defn{Bol thanked her.}//
	\endgl
\xe

\subsection{Demonstrative Pronouns and Determiners}
\label{subsec:tvk-demonstrative-pronouns-determiners}

Demonstrative pronouns and determiners

\subsection{Interrogative Pronouns and Determiners}
\label{subsec:tvk-interrogative-pronouns-determiners}

Interrogative pronouns and determiners

\subsection{Relative Pronouns}
\label{subsec:tvk-relative-pronouns}

Relative pronouns

\subsection{Indefinite Pronouns and Determiners}
\label{subsec:tvk-indefinite-pronouns-determiners}

Indefinite pronouns and determiners


\excnt=1% Reset the numbering of examples to 1
\chapter{Syntax}
\label{cha:tvk-syntax}

How do words go together?

\excnt=1% Reset the numbering of examples to 1
\chapter{Lexical Operations}
\label{cha:tvk-lex-operations}

\section{Compounding}
\label{sec:tvk-lex-compounding}

How does compounding work?

\section{Derivation}
\label{sec:tvk-lex-derivation}

How do you make new words?

\excnt=1% Reset the numbering of examples to 1
\chapter{Discourse}
\label{cha:tvk-discourse}

How does conversation work?

\excnt=1% Reset the numbering of examples to 1
\chapter{Sociolinguistic Context}
\label{app:tvk-sociolinguistic-context}

\section{Conceptual Metaphors}
\label{sec:tvk-conceptual-metaphors}

What metaphors do the vocabulary convey?

For example: language is a tool. I speak \textit{with} or \textit{using} \langtvk, rather than just speaking \langtvk.

\section{Kinship Terms}
\label{sec:tvk-kinship-terms}
\index{kinship|(}

The \langtvk{} kinship system is similar to Lewis Henry Morgan's Sudanese kinship pattern, being largely descriptive with only a few classificatory terms. Siblings are distinguished from cousins, and parallel cousins are distinguished from cross cousins. Siblings and parallel cousins are identified by gender, while cross cousins are not. Parallel aunts and uncles are distinguished from cross aunts and uncles. Grandparents are identified by gender, but are otherwise undistinguished. Children and grandchildren are similarly identified by gender but otherwise undistinguished. See \autoref{fig:tvk-kinship} for a full kinship tree.

All of the kinship terms within a nuclear family have distinct names distinguishing gender and generation.

\begin{description}[leftmargin=!,labelwidth=\widthof{\bfseries daughter}]
	\item[mother] \scr{ima} \fw{ima} \phnm{i\pstrs ma}
	\item[father] \scr{avo} \fw{avo} \phnm{a\pstrs vo}
	\item[parent] \scr{vota} \fw{vota} \phnm{vo\pstrs ta}
	\item[sister] \scr{persa} \fw{persa} \phnm{per\pstrs sa}
	\item[brother] \scr{olno} \fw{olno} \phnm{ol\pstrs no}
	\item[sibling] \scr{armi} \fw{armi} \phnm{ar\pstrs mi}
	\item[wife] \scr{botra(mim)} \fw{botra(mim)} \phnm{bot\pstrs ra} or \phnm{bot.ra\pstrs mim}
	\item[husband] \scr{ǩalo(mim)} \fw{ǩalo(mim)} \phnm{xa\pstrs lo} or \phnm{xa.lo\pstrs mim}
	\item[spouse] \scr{sanim} \fw{sanim} \phnm{sa\pstrs nim}
	\item[daughter] \scr{botiken} \fw{botiken} \phnm{bo.ti\pstrs ken}
	\item[son] \scr{ǩatiken} \fw{ǩatiken} \phnm{xa.ti\pstrs ken}
	\item[child] \scr{tatiken} \fw{tatiken} \phnm{ta.ti\pstrs ken}
\end{description}

Relation by marriage is expressed with a suffix \fw{-(m)im}. This suffix can be added to several terms, such as \defn{sister}, \defn{brother}, \defn{daughter}, and \defn{son}.

\begin{description}[leftmargin=!,labelwidth=\widthof{\bfseries daughter-in-law}]
	\item[in-law] \scr{tavomim} \fw{tavomim} \phnm{ta.vo\pstrs mim}
	\item[mother-in-law] \scr{imamim} \fw{imamim} \phnm{i.ma\pstrs mim}
	\item[father-in-law] \scr{avomim} \fw{avomim} \phnm{a.vo\pstrs mim}
	\item[sister-in-law] \scr{persamim} \fw{persamim} \phnm{per.sa\pstrs mim}
	\item[brother-in-law] \scr{olnomim} \fw{olnomim} \phnm{ol.no\pstrs mim}
	\item[daughter-in-law] \scr{botikemmim} \fw{botikemmim} \phnm{bo\sstrs ti.kem\pstrs mim}
	\item[son-in-law] \scr{ǩatikemmim} \fw{ǩatikemmim} \phnm{xa\sstrs ti.kem\pstrs mim}
\end{description}

Terms for one's nieces and nephews are derived from a combination of the terms for \defn{sister} or \defn{brother} and the terms for \defn{daughter} or \defn{son}.

\begin{description}[leftmargin=!,labelwidth=\widthof{\bfseries niece-in-law (brother's daughter-in-law)}]
	\item[niece (sister's daughter)] \scr{perbo} \fw{perbo} \phnm{per\pstrs bo}
	\item[niece (brother's daughter)] \scr{olbo} \fw{olbo} \phnm{ol\pstrs bo}
	\item[niece-in-law (sister's daughter-in-law)] \scr{perbomim} \fw{perbomim} \phnm{per.bo\pstrs mim}
	\item[niece-in-law (brother's daughter-in-law)] \scr{olbomim} \fw{olbomim} \phnm{ol.bo\pstrs mim}
	\item[nephew (sister's son)] \scr{perǩa} \fw{perǩa} \phnm{per\pstrs xa}
	\item[nephew (brother's son)] \scr{olǩa} \fw{olǩa} \phnm{ol\pstrs xa}
	\item[nephew-in-law (sister's son-in-law)] \scr{perǩamim} \fw{perǩamim} \phnm{per.xa\pstrs mim}
	\item[nephew-in-law (brother's son-in-law)] \scr{olǩamim} \fw{olǩamim} \phnm{ol.xa\pstrs mim}
	\item[niefling (gender-neutral)] \scr{turag} \fw{turag} \phnm{tu\pstrs rag}
\end{description}

The child of one's niece or nephew is called \scr{turag} \fw{turag}. Over time, this term became generalized to be used as a classificatory gender-neutral term for all of one's nieces and nephews.

One's grandchildren are distinguished by gender, but not by their parents. In other words, one's daughter's daughter is called the same term as one's son's daughter.

\begin{description}[leftmargin=!,labelwidth=\widthof{\bfseries granddaughter}]
	\item[granddaughter] \scr{keðbotiken} \fw{keðbotiken} \phnm{keð\sstrs bo.ti\pstrs ken}
	\item[grandson] \scr{keþǩatiken} \fw{keþǩatiken} \phnm{keθ\sstrs xa.ti\pstrs ken}
	\item[grandchild] \scr{keþtatiken} \fw{keþtatiken} \phnm{keθ\sstrs ta.ti\pstrs ken}
\end{description}

\langtvk{} distinguishes between parallel and cross aunts and uncles. In other words, one's mother's sister is called differently than one's father's sister. In fact, the term for cross aunts is identical to the term for parallel aunt-in-laws, and the term for cross uncles is identical to the term for parallel uncle-in-laws. Similar to the terms for grandchildren, the terms for parallel aunts and uncles are derived using reduplication.

\exdisplay\noexno
\begin{tabu} {l l l}
	aunt & \fw{nánáatói} & mother's sister\\
	aunt & \fw{náugámói} & father's sister\\
	aunt & \fw{náugámói} & aunt by marriage\\
	uncle & \fw{totoóbói} & father's brother\\
	uncle & \fw{tougámói} & mother's brother\\
	uncle & \fw{tougámói} & uncle by marriage\\
\end{tabu}
\xe

\langtvk{} distinguishes between parallel and cross cousins, but does not distinguish them by gender. Within parallel cousins, different terms are used to distinguish maternal vs. paternal cousins. Cousins' spouses are treated the same as in-laws by adding the \fw{wen-} prefix.

\exdisplay\noexno
\begin{tabu} {l l l}
	cousin & \fw{tírnánáatói} & mother's sister's child\\
	cousin-in-law & \fw{wentírnánáatói} & mother's sister's child's spouse\\
	cousin & \fw{tírtotoóbói} & father's brother's child\\
	cousin-in-law & \fw{wentírtotoóbói} & father's brother's child's spouse\\
	cousin & \fw{tratrabói} & cross cousin\\
	cousin-in-law & \fw{wentratrabói} & cross cousin's spouse\\
\end{tabu}
\xe

The children and grandchildren of one's cousins are not distinguished in any way, even between parallel and cross cousins. In fact, they are all called by the same term as one's cross cousins, \fw{tratrabói}.

Grandparents are distinguished by gender, but there is no distinction made between maternal and paternal grandparents. Similar to the terms for grandchildren, the terms for grandparents are derived using reduplication.

\exdisplay\noexno
\begin{tabu} {l l}
	grandmother & \fw{jújúói}\\
	grandfather & \fw{utuutói}\\
\end{tabu}
\xe

One's grandparents' siblings are called by the same terms as for one's aunts and uncles. In other words, one would call one's maternal grandmother's brother the same term as one's mother would call that person.

\exdisplay\noexno
\begin{tabu} {l l l}
	grand-aunt & \fw{nánáatói} & grandmother's sister\\
	grand-aunt & \fw{náugámói} & grandfather's sister\\
	grand-aunt & \fw{náugámói} & grandparent's sister-in-law\\
	grand-uncle & \fw{totoóbói} & grandfather's brother\\
	grand-uncle & \fw{tougámói} & grandmother's brother\\
	grand-uncle & \fw{tougámói} & grandparent's brother-in-law\\
\end{tabu}
\xe

\begin{sidewaysfigure}[h]\centering
	\caption{Tavonic Kinship Tree}
	\label{fig:tvk-kinship}
	\index{kinship}
	\tiny
	\begin{tikzpicture}[scale=0.5]
	\GraphInit[vstyle=Normal]
	\SetVertexLabelOut
	% Female
	\tikzset{VertexStyle/.append style={shape=circle,minimum size=1em}}
	\Vertex[x=-14,y=8,Lpos=180,L=nánáatói]{MGMS}
	\Vertex[x=-10,y=8,Lpos=270,L=jújúói]{MGM}
	\Vertex[x=-6,y=8,Lpos=275,L=náugámói]{MGFS}
	\Vertex[x=4,y=8,Lpos=180,L=nánáatói]{PGMS}
	\Vertex[x=8,y=8,Lpos=270,L=jújúói]{PGM}
	\Vertex[x=12,y=8,Lpos=275,L=náugámói]{PGFS}
	
	\Vertex[x=-18,y=4,Lpos=90,L=náugámói]{MBW}
	\Vertex[x=-12,y=4,Lpos=0,L=nánáatói]{MS}
	\Vertex[x=-1,y=4,Lpos=270,L=júói]{M}
	\Vertex[x=14,y=4,Lpos=90,L=náugámói]{FBW}
	\Vertex[x=16,y=4,Lpos=265,L=náugámói]{FS}
	
	\Vertex[x=-9,y=0,Lpos=265,L=náatói]{S}
	\Vertex[x=-2,y=0,Lpos=270,L=ólói]{W}
	\Vertex[x=7,y=0,Lpos=265,L=wennáatói]{BW}
	
	\Vertex[x=-11,y=-4,Lpos=270,L=náalár]{SD}
	\Vertex[x=-7,y=-4,Lpos=90,L=wennáalár]{SNW}
	\Vertex[x=-3,y=-4,Lpos=270,L=lárrói]{D}
	\Vertex[x=1,y=-4,Lpos=265,L=wenlárrói]{NW}
	\Vertex[x=5,y=-4,Lpos=270,L=toólár]{BD}
	\Vertex[x=9,y=-4,Lpos=90,L=wentoólár]{BNW}
	
	\Vertex[x=-3,y=-8,Lpos=270,L=lálárrói]{DD}
	\Vertex[x=1,y=-8,Lpos=270,L=lálárrói]{ND}
	
	\Vertex[x=-17,y=-11,L=female]{femalekey}
	% Male
	\tikzset{VertexStyle/.append style={shape=rectangle,minimum size=1em}}
	\Vertex[x=-12,y=8,Lpos=265,L=tougámói]{MGMB}
	\Vertex[x=-8,y=8,Lpos=270,L=utuutói]{MGF}
	\Vertex[x=-4,y=8,Lpos=0,L=totoóbói]{MGFB}
	\Vertex[x=6,y=8,Lpos=265,L=tougámói]{PGMB}
	\Vertex[x=10,y=8,Lpos=270,L=utuutói]{PGF}
	\Vertex[x=14,y=8,Lpos=0,L=totoóbói]{PGFB}
	
	\Vertex[x=-16,y=4,Lpos=275,L=tougámói]{MB}
	\Vertex[x=-14,y=4,Lpos=90,L=tougámói]{MSH}
	\Vertex[x=1,y=4,Lpos=270,L=uutói]{F}
	\Vertex[x=12,y=4,Lpos=180,L=totoóbói]{FB}
	\Vertex[x=18,y=4,Lpos=90,L=tougámói]{FSH}
	
	\Vertex[x=-7,y=0,Lpos=275,L=wentoóbói]{SH}
	\Vertex[x=2,y=0,Lpos=270,L=udói]{H}
	\Vertex[x=9,y=0,Lpos=0,L=toóbói]{B}
	
	\Vertex[x=-9,y=-4,Lpos=275,L=wennáazo]{SDH}
	\Vertex[x=-5,y=-4,Lpos=270,L=náazo]{SN}
	\Vertex[x=-1,y=-4,Lpos=90,L=wenzohiói]{DH}
	\Vertex[x=3,y=-4,Lpos=270,L=zohiói]{N}
	\Vertex[x=7,y=-4,Lpos=275,L=wentoózo]{BDH}
	\Vertex[x=11,y=-4,Lpos=270,L=toózo]{BN}
	
	\Vertex[x=-1,y=-8,Lpos=270,L=zozohiói]{DN}
	\Vertex[x=3,y=-8,Lpos=270,L=zozohiói]{NN}
	
	\Vertex[x=-17,y=-12,L=male]{malekey}
	% Either male or female
	\tikzset{VertexStyle/.append style={shape=diamond,minimum size=0.75em}}
	\Vertex[x=-18,y=0,Lpos=265,L=tratrabói]{MBC}
	\Vertex[x=-16,y=0,Lpos=90,L=wentratrabói]{MBCS}
	\Vertex[x=-14,y=0,Lpos=265,L=tírnánáatói]{MSC}
	\Vertex[x=-12,y=0,Lpos=90,L=wentratrabói]{MSCS}
	\Vertex[x=12,y=0,Lpos=265,L=tírtotoóbói]{FBC}
	\Vertex[x=14,y=0,Lpos=90,L=wentratrabói]{FBCS}
	\Vertex[x=16,y=0,Lpos=265,L=tratrabói]{FSC}
	\Vertex[x=18,y=0,Lpos=90,L=wentratrabói]{FSCS}
	
	\Vertex[x=-17,y=-4,Lpos=265,L=tratrabói]{MBCC}
	\Vertex[x=-13,y=-4,Lpos=265,L=tratrabói]{MSCC}
	\Vertex[x=13,y=-4,Lpos=275,L=tratrabói]{FBCC}
	\Vertex[x=17,y=-4,Lpos=275,L=tratrabói]{FSCC}
	
	\Vertex[x=-17,y=-8,Lpos=270,L=tratrabói]{MBCCC}
	\Vertex[x=-13,y=-8,Lpos=270,L=tratrabói]{MSCCC}
	\Vertex[x=-10,y=-8,Lpos=270,L=tírtrabói]{SDC}
	\Vertex[x=-6,y=-8,Lpos=270,L=tírtrabói]{SNC}
	\Vertex[x=6,y=-8,Lpos=270,L=tírtrabói]{BDC}
	\Vertex[x=10,y=-8,Lpos=270,L=tírtrabói]{BNC}
	\Vertex[x=13,y=-8,Lpos=270,L=tratrabói]{FBCCC}
	\Vertex[x=17,y=-8,Lpos=270,L=tratrabói]{FSCCC}
	
	\Vertex[x=-17,y=-13,L=either female or male]{eitherkey}
	% Ego
	\SetVertexLabelIn
	\tikzset{VertexStyle/.append style={shape=diamond,minimum size=3em}}
	\Vertex[x=0,y=0,Lpos=270,L=mé]{E}
	
	\Edge(MGM)(MGF)
	\Edge(PGM)(PGF)
	\Edge(MS)(MSH)
	\Edge(MBW)(MB)
	\Edge(M)(F)
	\Edge(FBW)(FB)
	\Edge(FS)(FSH)
	\Edge(MBC)(MBCS)
	\Edge(MSC)(MSCS)
	\Edge(S)(SH)
	\Edges(W,E,H)
	\Edge(BW)(B)
	\Edge(FBC)(FBCS)
	\Edge(FSC)(FSCS)
	\Edge(SD)(SDH)
	\Edge(SNW)(SN)
	\Edge(D)(DH)
	\Edge(NW)(N)
	\Edge(BD)(BDH)
	\Edge(BNW)(BN)
	
	\draw (0,4) -- (E);
	\draw (S) -- (-9,2) -- (9,2) -- (B);
	\draw (E) -- (0,-2);
	\draw (D) -- (-3,-2) -- (3,-2) -- (N);
	\draw (-2,-4) -- (-2,-6);
	\draw (DD) -- (-3,-6) -- (-1,-6) -- (DN);
	\draw (2,-4) -- (2,-6);
	\draw (ND) -- (1,-6) -- (3,-6) -- (NN);
	\draw (-8,0) -- (-8,-2);
	\draw (SD) -- (-11,-2) -- (-5,-2) -- (SN);
	\draw (8,0) -- (8,-2);
	\draw (BD) -- (5,-2) -- (11,-2) -- (BN);
	\draw (-10,-4) -- (SDC);
	\draw (-6,-4) -- (SNC);
	\draw (6,-4) -- (BDC);
	\draw (10,-4) -- (BNC);
	\draw (-9,8) -- (-9,6);
	\draw (MB) -- (-16,6) -- (-1,6) -- (M);
	\draw (-12,6) -- (MS);
	\draw (9,8) -- (9,6);
	\draw (F) -- (1,6) -- (16,6) -- (FS);
	\draw (12,6) -- (FB);
	\draw (-17,4) -- (-17,2) -- (-18,2) -- (MBC);
	\draw (-13,4) -- (-13,2) -- (-14,2) -- (MSC);
	\draw (-17,0) -- (MBCC) -- (MBCCC);
	\draw (-13,0) -- (MSCC) -- (MSCCC);
	\draw (13,4) -- (13,2) -- (12,2) -- (FBC);
	\draw (17,4) -- (17,2) -- (16,2) -- (FSC);
	\draw (13,0) -- (FBCC) -- (FBCCC);
	\draw (17,0) -- (FSCC) -- (FSCCC);
	\draw (MGMS) -- (-14,9) -- (-10,9) -- (MGM);
	\draw (-12,9) -- (MGMB);
	\draw (MGF) -- (-8,9) -- (-4,9) -- (MGFB);
	\draw (-6,9) -- (MGFS);
	\draw (PGMS) -- (4,9) -- (8,9) -- (PGM);
	\draw (6,9) -- (PGMB);
	\draw (PGF) -- (10,9) -- (14,9) -- (PGFB);
	\draw (12,9) -- (PGFS);
	\end{tikzpicture}
\end{sidewaysfigure}

\index{kinship|)}

\section{Names}
\label{sec:tvk-names}
\index{names|(}

\subsection{Masculine Names}
\label{subsec:tvk-names-masc}

\begin{itemize}
	\item \scr{Bol} Bol \phnm{\pstrs bol}
	\item \scr{Lerk} Lerk \phnm{\pstrs lerk}
	\item \scr{Mollur} Mollur \phnm{mo\pstrs\gem{l}ur}
	\item \scr{Ote} Ote \phnm{o\pstrs te}
\end{itemize}

\subsection{Feminine Names}
\label{subsec:tvk-names-femi}

\begin{itemize}
	\item \scr{Blimva} Blimva \phnm{blim\pstrs va}
	\item \scr{Goltu} Goltu \phnm{gol\pstrs tu}
	\item \scr{Tlunda} Tlunda \phnm{tlun\pstrs da}
	\item \scr{Zarsa} Zarsa \phnm{zar\pstrs sa}
\end{itemize}

\subsection{Gender-Neutral Names}
\label{subsec:tvk-names-neut}

\begin{itemize}
	\item \scr{Erme} Erme \phnm{er\pstrs me}
	\item \scr{Inki} Inki \phnm{in\pstrs ki}
	\item \scr{Ronne} Ronne \phnm{ron\pstrs ne}
\end{itemize}



\index{names|)}

\excnt=1% Reset the numbering of examples to 1
\chapter{\langtvk{} Reference Grammar}
\label{cha:tvk-reference}

Here is a reference grammar for \langtvk.

\part{Tavonic Family: Alnuric}

\excnt=1% Reset the numbering of examples to 1
\chapter{History and Ethnography}
\label{cha:ank-ethnography}

This chapter will present a brief history of the \langank{} language, followed by a short description of its ethnolinguistic context.

\section{Brief History}
\label{sec:ank-history}

Here will be a brief historical description of the \peopank.

\section{Ethnography}
\label{sec:ank-ethnography}

\subsection{Demonyms and Language Names}
\label{subsec:ank-demonyms}

For hundreds of years, the empire ruled in the southern region of \landn. The \langtvk{} word \fw{unner} \phnm{un\pstrs ner} \defn{empire} evolved into the \langank{} word \fw{alnur} \phnm{al\pstrs nur}. \fw{\nlangank} \phnm{al.nu\pstrs rek} \defn{\langank} takes its name from this word. Meanwhile, the \langrdk{} name for the empire is \fw{nonar} \phnm{no\pstrs nar}, and its name for the \langank{} language is \fw{Nonrik} \phnm{non\pstrs rik}. Similarly, the \langank{} and \langrdk{} names for the \langank{} people are \fw{\npeopank} \phnm{al.nu\pstrs reθ} and \fw{Nonriþ} \phnm{non\pstrs riθ} respectively.

\subsection{Ethnology}
\label{subsec:ank-ethnology}

Here will be a brief ethnological description of the \peopank.

\subsection{Demography}
\label{subsec:ank-demography}

Here will be a brief demographical description of the \peopank.

\excnt=1% Reset the numbering of examples to 1
\chapter{Phonology}

\excnt=1% Reset the numbering of examples to 1
\chapter{Morphological Typology}

\excnt=1% Reset the numbering of examples to 1
\chapter{Grammatical Categories}

\excnt=1% Reset the numbering of examples to 1
\chapter{Syntax}

\excnt=1% Reset the numbering of examples to 1
\chapter{Lexical Operations}

\excnt=1% Reset the numbering of examples to 1
\chapter{Discourse}

\excnt=1% Reset the numbering of examples to 1
\chapter{Sociolinguistic Context}

\excnt=1% Reset the numbering of examples to 1
\chapter{\langank{} Reference Grammar}
\label{cha:ank-reference}

Here is a reference grammar for \langank.

\part{Tavonic Family: Redodhic}

\excnt=1% Reset the numbering of examples to 1
\chapter{History and Ethnography}
\label{cha:rdk-ethnography}

This chapter will present a brief history of the \langrdk{} language, followed by a short description of its ethnolinguistic context.

\section{Brief History}
\label{sec:rdk-history}

Here will be a brief historical description of the \peoprdk.

\section{Ethnography}
\label{sec:rdk-ethnography}

\subsection{Demonyms and Language Names}
\label{subsec:rdk-demonyms}

In the north, the alliance resisted the empire's expansion. The \langtvk{} word \fw{aroltutaþ} \phnm{a\sstrs rol.tu\pstrs taθ} signifies \defn{alliance}, however the alliance instead used the simpler form \fw{arutaþ} \phnm{a.ru\pstrs taθ} \defn{standers} to signify the alliance of those kingdoms standing against the empire. \fw{Arutaþ} evolved into the \langrdk{} word \fw{rejiþ} \phnm{re\pstrs\affr{d}{ʒ}iθ}, and \fw{\nlangrdk} \phnm{re.do\pstrs ðik} \defn{\langrdk} takes its name from this word. The \langank{} name for the alliance is \fw{eradeþ} \phnm{e.ra\pstrs deθ}, and its name for the \langrdk{} language is \fw{Eratþek} \phnm{e.rat\pstrs θek}. Similarly, the \langrdk{} and \langank{} names for the \langrdk{} people are \fw{\npeoprdk} \phnm{re.do\pstrs ðiθ} and \fw{Eratþeþ} \phnm{e.rat\pstrs θeθ} respectively.

\subsection{Ethnology}
\label{subsec:rdk-ethnology}

Here will be a brief ethnological description of the \peoprdk.

\subsection{Demography}
\label{subsec:rdk-demography}

Here will be a brief demographical description of the \peoprdk.

\excnt=1% Reset the numbering of examples to 1
\chapter{Phonology}

\excnt=1% Reset the numbering of examples to 1
\chapter{Morphological Typology}

\excnt=1% Reset the numbering of examples to 1
\chapter{Grammatical Categories}

\excnt=1% Reset the numbering of examples to 1
\chapter{Syntax}

\excnt=1% Reset the numbering of examples to 1
\chapter{Lexical Operations}

\excnt=1% Reset the numbering of examples to 1
\chapter{Discourse}

\excnt=1% Reset the numbering of examples to 1
\chapter{Sociolinguistic Context}

\excnt=1% Reset the numbering of examples to 1
\chapter{\langrdk{} Reference Grammar}
\label{cha:rdk-reference}

Here is a reference grammar for \langrdk.

\part{Kalaakan Family: Kalaakan}

\excnt=1% Reset the numbering of examples to 1
\chapter{History and Ethnography}

\excnt=1% Reset the numbering of examples to 1
\chapter{Phonology}

\excnt=1% Reset the numbering of examples to 1
\chapter{Morphological Typology}

\excnt=1% Reset the numbering of examples to 1
\chapter{Grammatical Categories}

\excnt=1% Reset the numbering of examples to 1
\chapter{Syntax}

\excnt=1% Reset the numbering of examples to 1
\chapter{Lexical Operations}

\excnt=1% Reset the numbering of examples to 1
\chapter{Discourse}

\excnt=1% Reset the numbering of examples to 1
\chapter{Sociolinguistic Context}

\excnt=1% Reset the numbering of examples to 1
\chapter{Kalaakan Reference Grammar}

\part{Kalaakan Family: Elvish}

\excnt=1% Reset the numbering of examples to 1
\chapter{History and Ethnography}

\excnt=1% Reset the numbering of examples to 1
\chapter{Phonology}

\excnt=1% Reset the numbering of examples to 1
\chapter{Morphological Typology}

\excnt=1% Reset the numbering of examples to 1
\chapter{Grammatical Categories}

\excnt=1% Reset the numbering of examples to 1
\chapter{Syntax}

\excnt=1% Reset the numbering of examples to 1
\chapter{Lexical Operations}

\excnt=1% Reset the numbering of examples to 1
\chapter{Discourse}

\excnt=1% Reset the numbering of examples to 1
\chapter{Sociolinguistic Context}

\excnt=1% Reset the numbering of examples to 1
\chapter{Elvish Reference Grammar}

\part{Kalaakan Family: Dwarvish}

\excnt=1% Reset the numbering of examples to 1
\chapter{History and Ethnography}

\excnt=1% Reset the numbering of examples to 1
\chapter{Phonology}

\excnt=1% Reset the numbering of examples to 1
\chapter{Morphological Typology}

\excnt=1% Reset the numbering of examples to 1
\chapter{Grammatical Categories}

\excnt=1% Reset the numbering of examples to 1
\chapter{Syntax}

\excnt=1% Reset the numbering of examples to 1
\chapter{Lexical Operations}

\excnt=1% Reset the numbering of examples to 1
\chapter{Discourse}

\excnt=1% Reset the numbering of examples to 1
\chapter{Sociolinguistic Context}

\excnt=1% Reset the numbering of examples to 1
\chapter{Dwarvish Reference Grammar}

\part{Kalaakan Family: Orcish}

\excnt=1% Reset the numbering of examples to 1
\chapter{History and Ethnography}

\excnt=1% Reset the numbering of examples to 1
\chapter{Phonology}

\excnt=1% Reset the numbering of examples to 1
\chapter{Morphological Typology}

\excnt=1% Reset the numbering of examples to 1
\chapter{Grammatical Categories}

\excnt=1% Reset the numbering of examples to 1
\chapter{Syntax}

\excnt=1% Reset the numbering of examples to 1
\chapter{Lexical Operations}

\excnt=1% Reset the numbering of examples to 1
\chapter{Discourse}

\excnt=1% Reset the numbering of examples to 1
\chapter{Sociolinguistic Context}

\excnt=1% Reset the numbering of examples to 1
\chapter{Orcish Reference Grammar}

\part{Kunmian Family: Kunmian}

\excnt=1% Reset the numbering of examples to 1
\chapter{History and Ethnography}

\excnt=1% Reset the numbering of examples to 1
\chapter{Phonology}

\excnt=1% Reset the numbering of examples to 1
\chapter{Morphological Typology}

\excnt=1% Reset the numbering of examples to 1
\chapter{Grammatical Categories}

\excnt=1% Reset the numbering of examples to 1
\chapter{Syntax}

\excnt=1% Reset the numbering of examples to 1
\chapter{Lexical Operations}

\excnt=1% Reset the numbering of examples to 1
\chapter{Discourse}

\excnt=1% Reset the numbering of examples to 1
\chapter{Sociolinguistic Context}

\excnt=1% Reset the numbering of examples to 1
\chapter{Kunmian Reference Grammar}

\part{Kunmian Family: Gnomish}

\excnt=1% Reset the numbering of examples to 1
\chapter{History and Ethnography}

\excnt=1% Reset the numbering of examples to 1
\chapter{Phonology}

\excnt=1% Reset the numbering of examples to 1
\chapter{Morphological Typology}

\excnt=1% Reset the numbering of examples to 1
\chapter{Grammatical Categories}

\excnt=1% Reset the numbering of examples to 1
\chapter{Syntax}

\excnt=1% Reset the numbering of examples to 1
\chapter{Lexical Operations}

\excnt=1% Reset the numbering of examples to 1
\chapter{Discourse}

\excnt=1% Reset the numbering of examples to 1
\chapter{Sociolinguistic Context}

\excnt=1% Reset the numbering of examples to 1
\chapter{Gnomish Reference Grammar}

% Appendices

\appendix

\part{Appendices}

\excnt=1% Reset the numbering of examples to 1
\chapter{Example Texts}
\label{app:example-texts}

Here are some longer example translations.

% Back matter

\backmatter

% Bibliography

\printbibliography

% Index

\printindex

\end{document}