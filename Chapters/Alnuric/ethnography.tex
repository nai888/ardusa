\chapter{History and Ethnography}
\label{cha:ank-ethnography}

This chapter will present a brief history of the \langank{} language, followed by a short description of its ethnolinguistic context.

\section{Brief History}
\label{sec:ank-history}

Here will be a brief historical description of the \peopank.

\section{Ethnography}
\label{sec:ank-ethnography}

\subsection{Demonyms and Language Names}
\label{subsec:ank-demonyms}

For hundreds of years, the empire ruled in the southern region of \landn. The \langtvk{} word \fw{unner} \phnm{un\pstrs ner} \defn{empire} evolved into the \langank{} word \fw{alnur} \phnm{al\pstrs nur}. \fw{\nlangank} \phnm{al.nu\pstrs rek} \defn{\langank} takes its name from this word. Meanwhile, the \langrdk{} name for the empire is \fw{nonar} \phnm{no\pstrs nar}, and its name for the \langank{} language is \fw{Nonrik} \phnm{non\pstrs rik}. Similarly, the \langank{} and \langrdk{} names for the \langank{} people are \fw{\npeopank} \phnm{al.nu\pstrs reθ} and \fw{Nonriþ} \phnm{non\pstrs riθ} respectively.

\subsection{Ethnology}
\label{subsec:ank-ethnology}

Here will be a brief ethnological description of the \peopank.

\subsection{Demography}
\label{subsec:ank-demography}

Here will be a brief demographical description of the \peopank.