\chapter{Preface}
\label{cha:preface}

This document provides a detailed grammatical description of the languages of \landn, a fictional landmass set in a fictional constructed world. This project serves as a method for linguistic research, as an intellectual exercise, as an outlet for creative and artistic expression, and as a setting for potential future works of fiction. It is intended primarily for my own personal use and entertainment, though others with similar linguistic interests will hopefully find it interesting and entertaining as well. I have chosen to use \LaTeX{} to typeset this grammar because it provides a way to be clear, consistent, and organized. Further, since \LaTeX{} uses plain text files, it allows me to use Git for version control so I can keep track of changes over time.

My goal is to build a series of languages with naturalistic grammars that are linguistically plausible and consistent, yet also original in their content and details. This project consists of three distinct and unrelated language families, each of which contains one or more related languages. Some elements of these languages are influenced by existing languages such as Japanese, Finnish, Navajo, Nahuatl, and Arabic, but they are not meant to simply mimic these, instead drawing this inspiration into new forms along with entirely \fw{a priori} lexicons. \landn{} and the \landadj{} languages is an ongoing project with no fixed endpoint or goal.

This concise grammar is my attempt to document the \landadj{} languages in an official and systematic way, and as comprehensively as possible. It is intended to be the official description of the languages. This is a concise grammar because, admittedly, I am not a professional linguist, nor have I taken any linguistics coursework. My education in linguistics consists solely of self-guided research, which means invariably my knowledge will be limited. It is a concise grammar because, frankly, I don't know enough to go into greater detail. That being said, I'm always eager to learn, and will always accept feedback. Again, learning is one of the reasons for this endeavor.

Since the purpose of writing this grammar is to provide a comprehensive description of the \landadj{} languages, not to teach them to others, it is not intended to serve as a textbook or as a way to learn the languages. I have organized topics thematically, rather than curricularly, and I employ technical terms when they are precise, accurate, and appropriate. I have not conducted a formal analysis of the languages, but I have worked to make it as descriptive as possible.

The discussion is ordered from the smallest elements of the languages to the largest. It begins with a description of each language's place in \landn{} followed by their phonologies, it addresses morphology and the combining of words, it discusses vocabulary and derivation, and it explains syntax and discourse. The final chapter serves as a reference grammar, summarizing all of the previous chapters. There are also several appendices describing the conceptual metaphors that organize much of the lexicons, the naming practices of the fictional speakers of these languages, several translation examples, and lexicons. Other resources include a glossary of linguistic glossing abbreviations, a bibliography, and an index.

This document uses several linguistics conventions to clarify meaning. Any reference to specific orthographic spelling is marked with angled brackets, such as \orth{hin}. Pronunciations are usually given phonemically, in which case they are marked with slashes, such as \phnm{hin}. Phonetic pronunciations are used only when conveying specific details like the difference between allophones, and are marked with square brackets, such as \phnt{çin}. Both phonemic and phonetic pronunciations are given using the International Phonetic Alphabet. Foreign words are always written in italics, such as \fw{lu}. English glosses are surrounded by single quotes, such as \defn{and}. If a morphological gloss is provided in-line, it is surrounded by parentheses, such as \gloss{\Inf}.

Many short examples are provided in one single line.

\ex<ex:prfshortgloss>
	\langtvk: \fw{šek} \phnm{ʃek} \defn{ran} \gloss{run.\Ind.\Pst.\Prf}
\xe

Longer examples are usually provided with a multi-line, or interlinear, gloss. In these examples, the optional first line will indicate which language the example is in, if it is not clear from context. The next line presents the text in that language, followed by the pronunciation. After this, the text is broken into its component morphemes, and the following line provides a morpheme-by-morpheme gloss. The final line provides an English translation of the example phrase or sentence.

\ex<ex:prffullgloss>
	\begingl
		\glpreamble \langtvk\\
		\fw{Oko nan šeðo.}\\
		\phnm{o\pstrs ko nan ʃe\pstrs ðo}//
		\gla oko nan šeðo//
		\glb dog \Pl.\An.\Top{} run.\Ind.\Pst.\Prg//
		\glft \defn{The dogs were running.}//
	\endgl
\xe

As shown in example \getfullref{ex:prffullgloss}, morpheme glosses are labeled with abbreviations in \textsc{small caps}. A full list of all glossing abbreviations is given on page \pageref{cha:glossary}. A hyphen marks a morpheme boundary within a word that is shared between the text and its gloss, while a period marks a boundary present in only one or the other, including when a single word in the text corresponds to multiple words in its gloss. Clitics are marked with an equals sign, reduplication with a tilde, discontinuous affixes (e.g., infixes, circumfixes) with angle brackets, and morphemes that cannot be easily separated out with backslashes.

The \LaTeX{} source code for this grammar and a copy of this PDF are available in a public \href{https://github.com/nai888/ardusa}{\faGithub~GitHub} repository. Undoubtedly, there will be errors in this document. If you notice any, please feel free to open an issue in the GitHub repository with a description and the location of the error.

\begin{flushright}
	\makeatletter
	\textit{\@author}\\
	\textit{Minneapolis, \DTMdisplaydate{2018}{9}{8}}
	\makeatother
\end{flushright}