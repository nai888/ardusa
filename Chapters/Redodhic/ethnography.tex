\chapter{History and Ethnography}
\label{cha:rdk-ethnography}

This chapter will present a brief history of the \langrdk{} language, followed by a short description of its ethnolinguistic context.

\section{Brief History}
\label{sec:rdk-history}

Here will be a brief historical description of the \peoprdk.

\section{Ethnography}
\label{sec:rdk-ethnography}

\subsection{Demonyms and Language Names}
\label{subsec:rdk-demonyms}

In the north, the alliance resisted the empire's expansion. The \langtvk{} word \fw{aroltutaþ} \phnm{a\sstrs rol.tu\pstrs taθ} signifies \defn{alliance}, however the alliance instead used the simpler form \fw{arutaþ} \phnm{a.ru\pstrs taθ} \defn{standers} to signify the alliance of those kingdoms standing against the empire. \fw{Arutaþ} evolved into the \langrdk{} word \fw{rejiþ} \phnm{re\pstrs\affr{d}{ʒ}iθ}, and \fw{\nlangrdk} \phnm{re.do\pstrs ðik} \defn{\langrdk} takes its name from this word. The \langank{} name for the alliance is \fw{eradeþ} \phnm{e.ra\pstrs deθ}, and its name for the \langrdk{} language is \fw{Eratþek} \phnm{e.rat\pstrs θek}. Similarly, the \langrdk{} and \langank{} names for the \langrdk{} people are \fw{\npeoprdk} \phnm{re.do\pstrs ðiθ} and \fw{Eratþeþ} \phnm{e.rat\pstrs θeθ} respectively.

\subsection{Ethnology}
\label{subsec:rdk-ethnology}

Here will be a brief ethnological description of the \peoprdk.

\subsection{Demography}
\label{subsec:rdk-demography}

Here will be a brief demographical description of the \peoprdk.