\chapter{Names}
\label{app:names}
\index{names|(}

\lang{} people usually have several names, and usually take more as they age.

\section{Naming}
\label{sec:naming}

For the first year of their life, babies are not given a personal name. They are given a matronymic, a patronymic, and a family name, which are usually enough to specify them. If there are more than one baby of the same gender under the age of one, ordinal numbers such as \scr{Héiném} \fw{Héiném} \defn{First} or \scr{Kríném} \fw{Kríném} \defn{Second} may be used to specify the order of birth.

\index{names!parental|(}

Matronymics and patronymics are formed by taking the mother's and father's newest given name and prefixing \fw{Lá-} for a daughter or \fw{Zo-} for a son. If a girl's mother is named \fw{Brallýýs} and her father is named \fw{Tadóital}, she will be named \scr{LáBrallýýs LáTadóital} \fw{LáBrallýýs LáTadóital}. If a child does not know the identity of one of their parents, they may not have a name corresponding to that parent. If they do not know the identity of either parent, they usually will take the name \fw{LáFynge} or \fw{ZoFynge} to indicate they are the child of \enquote{none}. If, after a child comes of legal age, they wish not to be associated with one or both of their parents (e.g. they officially disown their parents), they may drop the corresponding name or adopt the name \fw{LáFynge} or \fw{ZoFynge} by choice.

\index{names!parental|)}

\index{names!family|(}

The family name is usually the name of an animal, historically as an homage to that animal's spirit and powers, for example \scr{Guníaa} \fw{Guníaa} \defn{horse} or \scr{Tedkóó} \fw{Tedkóó} \defn{hawk}. If a child is born outside of marriage, the parents will usually default to the mother's family name, or to the primary caretaker's. If the child's parents later marry and choose a new family name, the child's family name will often change to match.

Upon marriage, the couple usually chooses a new family name for their new family. This new family name is added after their respective parental family names, rather than replacing them.

\index{names!family|)}

\index{names!given|(}

At the ages of 1, 11, and 21, children are given new personal names in a naming ceremony led by an elder. Each new personal name is simply added in front of the previous ones, rather than replacing them. Very rarely are any personal names removed. After the age of 21, people can occasionally be given a new name when an elder believes that person has achieved a significant accomplishment. Such a name will also be given by the elder in a naming ceremony.

Personal names are usually words that signify objects in nature, accomplishments, accomplishments, or personality traits (whether real or aspirational). The names of animals, however, are never used, as they are reserved for family names.

Personal names are sometimes simply derived grammatically from verbs or nouns. For example, a nominalizer could be added directly to a verb stem, as with normal derivation, such as \scr{Ðýsói} \fw{Ðýsói} \defn{Seeker}, or the verb could be partially or fully conjugated before adding the nominalizer, such as \scr{Ðýsaói} \fw{Ðýsaói} approximately \defn{He Seeks}. Derived words like this are always gender-neutral.

At other times, special name nominalizer suffixes are used. These suffixes don't affect the underlying meaning of the root word, but they convert the word directly into a proper noun. There are eight of these nominalizers, half of which are gender-neutral, for example \scr{Ðýssaan} \fw{Ðýssaan} approximately \defn{One Who Seeks}. The other half are split between male and female names, such as \scr{Ðýslau} \fw{Ðýslau} \defn{He Who Seeks} and \scr{Ðýsýýs} \fw{Ðýsýýs} \defn{She Who Seeks}.

\index{names!given|)}

\section{Addressing a Person}
\label{sec:addressing-a-person}

People are generally addressed with as few names as possible, starting with their family name. If the family name is insufficient to identify the referent, the person's matronymic or patronymic will be used. If that is still insufficient, a person's most recent given name will be used, adding more until the person's identity is fully clear. Only friends and family members are considered close enough to use a person's given name by default, and when someone outside that circle uses a person's given name, it is considered disrespectful and overly familiar.

In extremely familiar settings, such as a parent addressing a child or someone addressing their spouse, the speaker may shorten a person's given name, especially with longer names, disregarding the grammatical structure and the meaning of the name. For example, someone with the name \fw{Bótrudtavól} may be called \fw{Bóvól} by a parent or spouse. Similarly, someone with the name \fw{Getadsaan} may be called \fw{Gesan}. These nicknames would \emph{never} be used by someone who is not close to the person.

The title \scr{Tótki} \fw{Tótki} can be used after a name to show respect, similar to how an English speaker would use \enquote{Mr.} or \enquote{Mrs.}. This title can be used with any of the person's names. Gender prefixes can optionally be added to the title when used with only a family name in order to help clarify the referent if it would otherwise be confusing.

The title \fw{Tótki} can also be used by itself without any other names, but this usage is rare and often seems stilted and old-fashioned.

When multiple names are used, any number of them can be declined according to the referent's role within the clause, however usually only the first stated name is declined. The title \fw{Tótki} is never declined unless used by itself.

Below is an example of a person's full name.

\exdisplay\noexno
	\scr{Dwénsýýs Fluvól Vlovaói LáBrallýýs LáTadóital Guníaa Tedkóó Tótki}\\
	\fw{Dwénsýýs Fluvól Vlovaói LáBrallýýs LáTadóital Guníaa Tedkóó Tótki}
\xe

This person is a woman, as shown by the feminine version of \fw{Dwénsýýs} and the prefix \fw{Lá-} on the matronymic and patronymic names. The woman is at least 21 years old, since she has three given names. \fw{Dwénsýýs} was given when she was 21, \fw{Fluvól} when she was 11, and \fw{Vlovaói} when she was 1. Her mother's newest given name is \fw{Brallýýs}, her father's newest given name is \fw{Tadóital}, and her parents' chosen family name is \fw{Guníaa}. The woman is married, and has chosen the new family name \fw{Tedkóó}. Finally, she is being address with the respectful title \fw{Tótki}.

\section{Example Given Names}
\label{sec:example-names}
\index{names!given|(}

The ability to derive names from almost any verb or noun yields a huge number of possible given names. The lists of examples below are only a small sampling of possibilities.

\subsection*{Gender-Neutral}

Any name that is derived grammatically will be gender-neutral. The most common ending of this type is \fw{-ói}. Verbs can be converted into gender-neutral names with the endings \fw{-saan}, \fw{-lous}, or \fw{vól}, and nouns can be converted into gender-neutral names with the ending \fw{-mið}. The naming suffixes can even be added after grammatical derivation.

\begin{multicols}{2}\noindent
	\scr{Állasaan} \fw{Állasaan}\\
	\scr{Állóilous} \fw{Állóilous}\\
	\scr{Asúvvól} \fw{Asúvvól}\\
	\scr{Asúvyy} \fw{Asúvyy}\\
	\scr{Bótrudtavól} \fw{Bótrudtavól}\\
	\scr{Brétkáusaan} \fw{Brétkáusaan}\\
	\scr{Dósjaói} \fw{Dósjaói}\\
	\scr{Driíhalous} \fw{Driíhalous}\\
	\scr{Ðýsaói} \fw{Ðýsaói}\\
	\scr{Ellous} \fw{Ellous}\\
	\scr{Fluáumið} \fw{Fluáumið}\\
	\scr{Fluvól} \fw{Fluvól}\\
	\scr{Fóodsaan} \fw{Fóodsaan}\\
	\scr{Fóodyymið} \fw{Fóodyymið}\\
	\scr{Getadsaan} \fw{Getadsaan}\\
	\scr{Gworlous} \fw{Gworlous}\\
	\scr{Héibavól} \fw{Héibavól}\\
	\scr{Honkáumið} \fw{Honkáumið}\\
	\scr{Hyráyymið} \fw{Hyráyymið}\\
	\scr{Jódpaói} \fw{Jódpaói}\\
	\scr{Krurralous} \fw{Krurralous}\\
	\scr{Mittasaan} \fw{Mittasaan}\\
	\scr{Sérralous} \fw{Sérralous}\\
	\scr{Sláákavól} \fw{Sláákavól}\\
	\scr{Sláákigrói} \fw{Sláákigrói}\\
	\scr{Tadvól} \fw{Tadvól}\\
	\scr{Tadóilous} \fw{Tadóilous}\\
	\scr{Taððažíímið} \fw{Taððažíímið}\\
	\scr{Taððimsaan} \fw{Taððimsaan}\\
	\scr{Taððóivól} \fw{Taððóivól}\\
	\scr{Tedkalous} \fw{Tedkalous}\\
	\scr{Tedkoí} \fw{Tedkoí}\\
	\scr{Trudtyymið} \fw{Trudtyymið}\\
	\scr{Úþtadói} \fw{Úþtadói}\\
	\scr{Vlovaói} \fw{Vlovaói}\\
	\scr{Žerráumið} \fw{Žerráumið}
\end{multicols}

\subsection*{Female}

Verbs can be converted into female names with the ending \fw{-ýýs}, and nouns can be converted into female names with the ending \fw{-táí}.

\begin{multicols}{2}\noindent
	\scr{Állýýs} \fw{Állýýs}\\
	\scr{Asúvýýs} \fw{Asúvýýs}\\
	\scr{Asúvyytáí} \fw{Asúvyytáí}\\
	\scr{Bájyytáí} \fw{Bájyytáí}\\
	\scr{Brallýýs} \fw{Brallýýs}\\
	\scr{Brétkáutáí} \fw{Brétkáutáí}\\
	\scr{Dósjyytáí} \fw{Dósjyytáí}\\
	\scr{Ðýsýýs} \fw{Ðýsýýs}\\
	\scr{Guníóitáí} \fw{Guníóitáí}\\
	\scr{Gworýýs} \fw{Gworýýs}\\
	\scr{Tadóiýýs} \fw{Tadóiýýs}\\
	\scr{Vlovýýs} \fw{Vlovýýs}
\end{multicols}

\subsection*{Male}

Verbs can be converted into male names with the ending \fw{-lau}, and nouns can be converted into male names with the ending \fw{-tal}.

\begin{multicols}{2}\noindent
	\scr{Állíítal} \fw{Állíítal}\\
	\scr{Brétkáutal} \fw{Brétkáutal}\\
	\scr{Dwénsalau} \fw{Dwénsalau}\\
	\scr{Ellau} \fw{Ellau}\\
	\scr{Flulau} \fw{Flulau}\\
	\scr{Grianóital} \fw{Grianóital}\\
	\scr{Héibóilau} \fw{Héibóilau}\\
	\scr{Honkáutal} \fw{Honkáutal}\\
	\scr{Krurróilau} \fw{Krurróilau}\\
	\scr{Mittalau} \fw{Mittalau}\\
	\scr{Tadóital} \fw{Tadóital}\\
	\scr{Žóýdlau} \fw{Žóýdlau}
\end{multicols}

\index{names!given|)}

\index{names|)}