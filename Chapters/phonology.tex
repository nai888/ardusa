\chapter{Phonology}
\label{cha:phonology}

This chapter will present the inventory of consonants, vowels, and tones, describe their allophones, and explain their romanization. An observational analysis of \lang{}'s syllable structures and phonotactics will follow. The chapter will close with notes on syllable stress within words and a brief exploration of intonation.

\section{Phoneme Inventory}
\label{sec:phonemes}

\subsection{Consonants}
\label{sec:consonants}
\index{consonants|(}

\afterpage{\clearpage
\begin{sidewaystable}
\scriptsize
\index{consonants!inventory}\index{allophony}\index{consonants!allophones|see {allophony}}
\caption[Consonant Inventory]{Phonetic Consonant Inventory (allophones in parentheses)}
\label{tab:consonants}
\begin{tabu} to \textheight {| r | X[c] X[c] X[c] X[c] X[c] X[c] X[c] X[c] X[c] X[c] X[c] X[c] X[c] X[c] X[c] X[c] X[c] X[c]}
	\toprule
	Consonants
		& \multicolumn{2}{c}{Bilabial}
		& \multicolumn{2}{c}{Labio-dental}
		& \multicolumn{2}{c}{Dental}
		& \multicolumn{2}{c}{Alveolar}
		& \multicolumn{2}{c}{Alveolo-palatal}
		& \multicolumn{2}{c}{Palatal}
		& \multicolumn{2}{c}{Labio-velar}
		& \multicolumn{2}{c}{Velar}
		& \multicolumn{2}{c}{Glottal}
		\\
	\midrule
	Nasal
		&      & m (\gem{m}) % Bilabial
		&      &             % Labiodental
		&      &             % Dental
		&      & n (\gem{n}) % Alveolar
		&      &             % Alveolo-palatal
		&      &             % Palatal
		&      &             % Labio-velar
		&      & (ŋ)         % Velar
		&      &             % Glottal
		\\
	\midrule
	Plosive
		& (\eje{p})   &             % Bilabial
		& \dent{p} (\gem{\dent{p}}) & \dent{b} % Labiodental
		& \dent{t} (\gem{\dent{t}}) (\eje{\dent{t}}) & \dent{d} (\gem{\dent{d}}) % Dental
		&             &             % Alveolar
		&             &             % Alveolo-palatal
		&             &             % Palatal
		&             &             % Labio-velar
		& k (\gem{k}) (\eje{k}) & g % Velar
		&             &             % Glottal
		\\
	\midrule
	Fricative
		&        &                  % Bilabial
		& f (\gem{f}) & v (\gem{v}) % Labiodental
		& θ (\gem{θ}) & ð (\gem{ð}) % Dental
		& s (\gem{s}) & z (\gem{z}) % Alveolar
		& ɕ (\gem{ɕ}) & ʑ (\gem{ʑ}) % Alveolo-palatal
		& (ç)    &                  % Palatal
		&        &                  % Labio-velar
		&        &                  % Velar
		& h      &                  % Glottal
		\\
	\midrule
	Flap/Tap
		&      &      % Bilabial
		&      &      % Labiodental
		&      &      % Dental
		&      & ɾ    % Alveolar
		&      &      % Alveolo-palatal
		&      &      % Palatal
		&      &      % Labio-velar
		&      &      % Velar
		&      &      % Glottal
		\\
	\midrule
	Trill
		&      &           % Bilabial
		&      &           % Labiodental
		&      &           % Dental
		&      & (\gem{r}) % Alveolar
		&      &           % Alveolo-palatal
		&      &           % Palatal
		&      &           % Labio-velar
		&      &           % Velar
		&      &           % Glottal
		\\
	\midrule
	Approximant
		&      &      % Bilabial
		&      &      % Labiodental
		&      &      % Dental
		&      &      % Alveolar
		&      &      % Alveolo-palatal
		&      & j    % Palatal
		&      & w    % Labio-velar
		&      &      % Velar
		&      &      % Glottal
		\\
	\midrule
	Lateral Approximant
		&      &             % Bilabial
		&      &             % Labiodental
		&      &             % Dental
		&      & l (\gem{l}) % Alveolar
		&      &             % Alveolo-palatal
		&      &             % Palatal
		&      &             % Labio-velar
		&      &             % Velar
		&      &             % Glottal
		\\
	\bottomrule
\end{tabu}
\end{sidewaystable}
\clearpage
\index{consonants!romanization}
\begin{longtabu} to \textwidth {c c c X[l]}
	\caption{Consonant Romanization}\label{tab:consromanization}\\
	\toprule
	Phone & Phoneme & Romanization & Notes\\
	\midrule
	\endhead
	\multicolumn{4}{r}{\textit{continued on the next page\ldots}}\\
	\endfoot
	\bottomrule
	\endlastfoot
	\phnt{m} & \phnm{m} & \orth{m} & \\
	\midrule
	\phnt{\gem{m}} & \phnm{\gem{m}} & \orth{mm} & \\
	\midrule
	\phnt{n} & \phnm{n} & \orth{n} & \\
	\midrule
	\phnt{\gem{n}} & \phnm{\gem{n}} & \orth{nn} & \\
	\midrule
	\phnt{ŋ} & \phnm{n} & \orth{\underline{n}g} or \orth{\underline{n}k} & \phnm{n} becomes velarized before a velar consonant\\
	\midrule
	\phnt{\dent{p}} & \phnm{p} & \orth{p} & \\
	\midrule
	\phnt{\gem{\dent{p}}} & \phnm{\gem{p}} & \orth{dp} & the \orth{dp} consonant cluster becomes a geminate \phnm{\gem{p}}\\
	\midrule
	\phnt{\eje{p}} & \phnm{\eje{p}} & \orth{tp} & the \orth{tp} consonant cluster becomes an ejective \phnm{\eje{p}}\\
	\midrule
	\phnt{\dent{b}} & \phnm{b} & \orth{b} & \\
	\midrule
	\phnt{\dent{t}} & \phnm{t} & \orth{t} & \\
	\midrule
	\phnt{\gem{\dent{t}}} & \phnm{\gem{t}} & \orth{dt} & the \orth{dt} consonant cluster becomes a geminate \phnm{\gem{t}}\\
	\midrule
	\phnt{\eje{\dent{t}}} & \phnm{\eje{t}} & \orth{tt} & a doubled \orth{tt} becomes an ejective \phnm{\eje{t}}\\
	\midrule
	\phnt{\dent{d}} & \phnm{d} & \orth{d} & \\
	\midrule
	\phnt{\gem{\dent{d}}} & \phnm{\gem{d}} & \orth{dd} & \\
	\midrule
	\phnt{k} & \phnm{k} & \orth{k} & \\
	\midrule
	\phnt{\gem{k}} & \phnm{\gem{k}} & \orth{dk} & the \orth{dk} consonant cluster becomes a geminate \phnm{\gem{k}}\\
	\midrule
	\phnt{\eje{k}} & \phnm{\eje{k}} & \orth{tk} & the \orth{tk} consonant cluster becomes an ejective \phnm{k}\\
	\midrule
	\phnt{g} & \phnm{g} & \orth{g} & \\
	\midrule
	\phnt{f} & \phnm{f} & \orth{f} & \\
	\midrule
	\phnt{\gem{f}} & \phnm{\gem{f}} & \orth{vf} & the \orth{vf} consonant cluster becomes a geminate \phnm{\gem{f}}\\
	\midrule
	\phnt{v} & \phnm{v} & \orth{v} & \\
	\midrule
	\phnt{\gem{v}} & \phnm{\gem{v}} & \orth{vv} & \\
	\midrule
	\phnt{θ} & \phnm{θ} & \orth{þ} & \\
	\midrule
	\phnt{\gem{θ}} & \phnm{\gem{θ}} & \orth{þþ} & \\
	\midrule
	\phnt{ð} & \phnm{ð} & \orth{ð} & \\
	\midrule
	\phnt{\gem{ð}} & \phnm{\gem{ð}} & \orth{ðð} & \\
	\midrule
	\phnt{s} & \phnm{s} & \orth{s} & \\
	\midrule
	\phnt{\gem{s}} & \phnm{\gem{s}} & \orth{ss} & \\
	\midrule
	\phnt{z} & \phnm{z} & \orth{z} & \\
	\midrule
	\phnt{\gem{z}} & \phnm{\gem{z}} & \orth{zz} & \\
	\midrule
	\phnt{ɕ} & \phnm{ɕ} & \orth{š} & \\
	\midrule
	\phnt{\gem{ɕ}} & \phnm{\gem{ɕ}} & \orth{šš} & \\
	\midrule
	\phnt{ʑ} & \phnm{ʑ} & \orth{ž} & \\
	\midrule
	\phnt{\gem{ʑ}} & \phnm{\gem{ʑ}} & \orth{žž} & \\
	\midrule
	\phnt{h} & \phnm{h} & \orth{h} & \\
	\midrule
	\phnt{ç} & \phnm{h} & \orth{\underline{h}i} or \orth{\underline{h}y} & \phnm{h} becomes palatalized before a high front vowel\\
	\midrule
	\phnt{l} & \phnm{l} & \orth{l} & \\
	\midrule
	\phnt{\gem{l}} & \phnm{\gem{l}} & \orth{ll} & \\
	\midrule
	\phnt{ɾ} & \phnm{r} & \orth{r} & \\
	\midrule
	\phnt{\gem{r}} & \phnm{\gem{r}} & \orth{rr} & \\
	\midrule
	\phnt{j} & \phnm{j} & \orth{j} & \\
	\midrule
	\phnt{w} & \phnm{w} & \orth{w} & \\
\end{longtabu}
\clearpage
}

With approximately 21 consonants, \lang{} has an \enquote{average} inventory.\autocite{wals-1} \lang{}'s consonant inventory tends to emphasize dentals, alveolars, and continuants. \autoref{tab:consonants} shows the full chart of consonant phonemes, along with several allophones enclosed in parentheses. \autoref{tab:consromanization} shows how each consonant in \lang{} is romanized.

Despite its \enquote{average} inventory of consonants, there are many more allophones\index{allophony} that occur in the language. First, any doubled consonant is realized as a geminated\index{consonants!gemination} (elongated) consonant.

\pex<gemcons>
	\a<L> \scr{geállil} \fw{geállil} \phnm{\pstrs geá\gem{l}il} \defn{to blow over}
	\a<Ð> \scr{taððou} \fw{taððou} \phnm{\pstrs ta\gem{ð}o\nsyl{u}} \defn{gathering place}
\xe

Thus, example~\getfullref{gemcons.L} above is realized with a lengthened \phnt{l}, and example~\getfullref{gemcons.Ð} is realized with a lengthened \phnt{ð}. A doubled \orth{r} is similarly geminated, but the pronunciation changes from a flap/tap to a trill.

\ex<gemr>
	\scr{krurril} \fw{krurril} \phnt{\pstrs kɾu\gem{r}il} \defn{to understand}
\xe

Note the difference in pronunciations of \phnm{r} in example~\getfullref{gemr}: the first is a flap/tap \phnt{ɾ} while the second is a geminate trill \phnt{\gem{r}}.

\index{phonological changes|(}

The remaining allophones\index{allophony} occur due to various sound change processes, mostly by assimilation. For example, \phnm{n} becomes velarized when it appears immediately before a velar plosive.

\ex<velarx>
	\scr{honkáu} \fw{honkáu} \phnt{\pstrs hoŋká\nsyl{u}} \defn{smoke}
\xe

Any high front vowel (i.e. \phnm{i} or \phnm{y}) palatalizes a preceding \phnm{h} to become \phnt{ç}.

\pex<palatalh>
	\a<i> \scr{fráúhil} \fw{fráúhil} \phnt{\pstrs fɾɑ́\nsyl{ú}çil} \defn{to afford}
	\a<y> \scr{hyráyy} \fw{hyráyy} \phnt{\pstrs çyɾǽ\gem{y}} \defn{sufficience}
\xe

When a \orth{vf} combination occurs, the \phnm{v} assimilates and geminates\index{consonants!gemination} the \phnm{f}.

\pex<assimv>
	\a<f> \scr{fryýnil} \fw{fryýnil} \phnm{\pstrs fr\gem{y̌}nil} \defn{to admit}
	\a<ff> \scr{Evfryýnan.} \fw{Evfryýnan.} \phnm{ef\pstrs fr\gem{y̌}nan} \defn{May s/he admit it.}
\xe

Similarly, when a \orth{dp}, \orth{dt}, or \orth{dk} combination occurs, the \phnm{d} assimilates and geminates\index{consonants!gemination} the following plosive.

\pex<assimd>
	\a<p> \scr{Jódpe.} \fw{Jódpe.} \phnm{\pstrs jó\gem{p}e} \defn{It is red.}
	\a<t> \scr{trudtil} \fw{trudtil} \phnm{\pstrs tru\gem{t}il} \defn{to thank}
	\a<k> \scr{tedkóó} \fw{tedkóó} \phnm{\pstrs te\gem{k}\gem{ó}} \defn{hawk}
\xe

When a \orth{tp}, \orth{tt}, or \orth{tk} combination occurs, the \phnm{t} assimilates, but the consonant cluster becomes an ejective rather than a geminate.

\pex<assimt>
	\a<p> \scr{šékretpái} \fw{šékretpái} \phnm{\pstrs ɕékre\eje{p}á\nsyl{i}} \defn{disaster}
	\a<t> \scr{mittaa} \fw{mittaa} \phnm{\pstrs mi\eje{t}\gem{a}} \defn{text}
	\a<k> \scr{brétkáu} \fw{brétkáu} \phnm{\pstrs bré\eje{k}á\nsyl{u}} \defn{fire}
\xe

\index{phonological changes|)}
\index{consonants|)}

\subsection{Vowels}
\label{sec:vowels}
\index{vowels|(}
\afterpage{\clearpage
\begin{table}\centering
	\index{vowels!inventory}
	\caption{Vowel Inventory}
	\label{tab:vowels}
	{\large
		\begin{vowel}
			\putcvowel[l]{i}{1}
			\putcvowel[r]{y}{1}
			\putcvowel[l]{e}{2}
			\putcvowel[r]{(ø)}{2}
			\putcvowel{a}{4}
			\putcvowel{(ɑ)}{5}
			\putcvowel{o}{7}
			\putcvowel{u}{8}
			\putcvowel{(æ)}{16}
		\end{vowel}
	}
\end{table}
\begin{table}\centering
	\index{vowels!romanization}
	\caption{Vowel Romanization}
	\label{tab:vowromanization}
	\begin{tabu} to \textwidth {c c c X[l]}
		\toprule
		Phone & Phoneme & Romanization & Notes\\
		\midrule
		\phnt{i} & \phnm{i} & \orth{i} & \\
		\midrule
		\phnt{y} & \phnm{y} & \orth{y} & \\
		\midrule
		\phnt{e} & \phnm{e} & \orth{e} & \\
		\midrule
		\phnt{a} & \phnm{a} & \orth{a} & \\
		\midrule
		\phnt{æ} & \phnm{a} & \orth{\underline{a}y} or \orth{y\underline{a}} & \phnm{a} tends to raise when preceding or following a \phnm{y}\\
		\midrule
		\phnt{ɑ} & \phnm{a} & \orth{\underline{a}u} or \orth{u\underline{a}} & \phnm{a} tends to back when preceding or following a \phnm{u}\\
		\midrule
		\phnt{o} & \phnm{o} & \orth{o} & \\
		\midrule
		\phnt{ø} & \phnm{o} & \orth{\underline{o}y} or \orth{y\underline{o}} & \phnm{o} tends to front when preceding or following a \phnm{y} \\
		\midrule
		\phnt{u} & \phnm{u} & \orth{u} & \\
		\bottomrule
	\end{tabu}
\end{table}
}

\lang{} distinguishes six vowel qualities, as shown in \autoref{tab:vowels}, giving it an \enquote{average} inventory.\autocite{wals-2} This means the consonant--vowel ratio is 21:6 or 3.5, which is \enquote{average}.\autocite{wals-3} Additionally, each of the six vowels can be geminated, and \lang{} allows seven different diphthongs.

Just like with consonants, vowels are geminated when they are doubled.\index{vowels!gemination}

\pex<gemvowels>
	\a<i> \scr{ankiim} \fw{sláákyy} \phnm{\pstrs sl\gem{á}k\gem{y}} \defn{achievement}
	\a<ä> \scr{tráávái} \fw{tráávái} \phnm{\pstrs tr\gem{á}vá\nsyl{i}} \defn{party}
	\a<o> \scr{hoojil} \fw{hoojil} \phnm{\pstrs h\gem{o}jil} \defn{to trade}
\xe

In examples~\getfullref{gemvowels.i}--\getref{gemvowels.o}, the doubled letters are all lexical, that is they occur within the root of the word. Gemination can also occur as a result of morphological changes.

\pex<gemvowmph>
	\a<e> \scr{ávail} \fw{ávail} \phnm{\pstrs áva\nsyl{i}l} \defn{to mail}
	\a<ee> \scr{Ávaan.} \fw{Ávaan.} \phnm{\pstrs áv\gem{a}n} \defn{S/he mailed it.}
\xe

\lang{} allows seven diphthongs\index{vowels!diphthongs}, including \orth{ai}, \orth{au}, \orth{ay}, \orth{ei}, \orth{oi}, \orth{ou}, and \orth{oy}. In all diphthongs, the first vowel is always the syllable nucleus and the second vowel is always non-syllabic.

\pex<diphth>
	\a<au> \scr{laúnil} \fw{laúnil} \phnm{\pstrs la\nsyl{ú}nil} \defn{to buy}
	\a<oy> \scr{žóýdil} \fw{žóýdil} \phnm{\pstrs ʑó\nsyl{ý}dil} \defn{to want}
	\a<ei> \scr{héi} \fw{héi} \phnm{\pstrs hé\nsyl{i}} \defn{one}
\xe

Any combination of consecutive vowels that is not one of the permissible diphthongs is realized as distinct syllables.

\pex<nodiphth>
	\a<ia> \scr{friá} \fw{friá} \phnm{\pstrs fri.á} \defn{three}
	\a<ue> \scr{héávail} \fw{héávail} \phnm{\pstrs hé.á.va\nsyl{i}l} \defn{to re-mail}
\xe

When there is a combination of three or more consecutive vowels, gemination always takes precedence.

\pex<cmpxvow>
	\a<aai> \ljudge{\excl} \scr{állííaðtaððil} \fw{állííaðtaððil} \phnm{\pstrs á\gem{l}\gem{í}\sstrs að.ta\gem{ð}il} \defn{to disperse wind}
	\a<ayoo> \scr{ávaýóonmittil} \fw{ávaýóonmittil} \phnm{\pstrs á.va\nsyl{ý}\sstrs \gem{ô}n.mi.\eje{t}il} \defn{to write mail in calligraphy}
\xe

\index{vowels|)}

\subsection{Tones}
\label{sec:tones}
\index{tones|(}

\lang{} has two tones---high and low---which means it can be classified as a \enquote{simple tone system}.\autocite{wals-13} Vowels have low tone by default, while marked syllables have high tone. Orthographically, high tone is represented with an acute accent.

\pex<tonestraight>
	\a<low> \scr{tadil} \fw{tadil} \phnm{\pstrs tadil} \defn{to speak}
	\a<high> \scr{Brétké.} \fw{Brétké.} \phnm{\pstrs bré\eje{k}é} \defn{They burned.}
\xe

Each syllable within a single word can have a different tone.

\ex<multitones>
	\scr{Úþtedký.} \fw{Úþtedký.} \phnm{\pstrs úθte\gem{k}ý} \defn{You (plural) are watchful.}
\xe

Long vowels\index{vowels} and diphthongs\index{vowels!diphthongs}\index{diphthongs|see {vowels}} often have a single prolonged tone.

\pex<gemtonestraight>
	\a<low> \scr{surmyy} \fw{surmyy} \phnm{\pstrs surm\gem{y}} \defn{behavior}
	\a<high> \scr{állíí} \fw{állíí} \phnm{\pstrs á\gem{l}\gem{í}} \defn{wind}
	\a<diphth> \scr{Fráúhóún.} \fw{Fráúhóún.} \phnm{\pstrs frá\nsyl{ú}hó\nsyl{ú}n} \defn{We were able to afford it.}
\xe

Long vowels\index{vowels} and diphthongs\index{vowels!diphthongs} can also have a rising or falling tone, rather than a single prolonged tone.

\pex<tonescontour>
	\a<diphthrising> \scr{mittoí} \fw{mittoí} \phnm{\pstrs mi\eje{t}o\nsyl{í}} \defn{writer}
	\a<diphthfalling> \scr{mittáu} \fw{mittáu} \phnm{\pstrs mi\eje{t}á\nsyl{u}} \defn{script}
	\a<gemrising> \scr{brétkeé} \fw{brétkeé} \phnm{\pstrs bré\eje{k}\gem{ě}} \defn{furnace}
	\a<gemfalling> \scr{fóodyy} \fw{fóodyy} \phnm{\pstrs f\gem{ô}d\gem{y}} \defn{success}
\xe

\index{tones|)}

\section{Phonotactics}
\label{sec:phonotactics}
\index{phonotactics|see {syllable}}

At the time of this writing, there does not yet exist a sufficient corpus for a meaningful statistical analysis of \lang{}'s phonotactics. Therefore, this section will present only a cursory observational analysis.

\subsection{Syllable Structures}
\label{sec:syllable-structures}
\index{syllable|(}
\index{syllable!types|(}

Syllables in \lang{} must contain a vowel or diphthong to serve as the syllable's nucleus. Syllables may also include any single consonant or one of a limited set of two-consonant clusters as the onset\index{syllable!onset}, and may include one of a limited set of consonants as the coda\index{syllable!coda}.

In other words, the most complex syllable structure allowed in \lang{} is CCVC, with clear restrictions on the allowable consonant clusters at the beginning of the syllable, giving \lang{} a \enquote{moderately complex syllable structure}.\autocite{wals-12}

\subsubsection{V}

The most basic syllable structure is simply a vowel\index{vowels} (V), whether it is short, long\index{vowels!gemination}, or a diphthong\index{vowels!diphthongs}.

\pex<syllV>
	\a<init> \scr{Elu.} \fw{Elu.} \phnm{\pstrs e.lu} \defn{You tried.}
	\a<diphth> \scr{aumyy} \fw{aumyy} \phnm{\pstrs a\nsyl{u}.m\gem{y}} \defn{green}
	\a<noninit> \scr{díríóó} \fw{díríóó} \phnm{\pstrs dí.rí.\gem{óó}} \defn{dog}
\xe

V syllables are most likely to appear at the beginning of a word. Other positions are rarer, but they do occur (as in the final syllable of example \getfullref{syllV.noninit}).

\subsubsection{CV}

The second simplest syllable structure is a single-consonant onset\index{syllable!onset} and a vowel (CV). There is no restriction on which consonants\index{consonants} can appear in the onset\index{syllable!onset}. This is likely the most frequent type of syllable in \lang{}.

\pex<syllCV>
	\a<V> \scr{pováu} \fw{pováu} \phnm{\pstrs po.vá\nsyl{u}} \defn{dignity}
	\a<VV> \scr{foinyy} \fw{foinyy} \phnm{\pstrs fo\nsyl{i}.n\gem{y}} \defn{shortness}
\xe

\subsubsection{VC}

The next simplest syllable structure is a vowel\index{vowels} and a coda\index{syllable!coda} (VC). This structure is more restrictive; most continuants and dentals can serve as a coda\index{syllable!coda}, but no others. This includes the nasals \orth{m} and \orth{n}, the fricatives \orth{v}, \orth{þ}, \orth{ð}, \orth{s}, \orth{z}, \orth{š}, and \orth{ž}, the liquids \orth{r} and \orth{l}, and the dentals \orth{t} and \orth{d}\index{consonants}.

\pex<syllVC>
	\a<init> \scr{Állo.} \fw{Állo.} \phnm{\pstrs á\gem{l}o} \defn{I blew air.}
	\a<noninit> \scr{fluáu} \fw{fluáu} \phnm{\pstrs flu.á\nsyl{u}} \defn{gift}
\xe

\subsubsection{CVC}

A syllable can also include both an onset\index{syllable!onset} and a coda\index{syllable!coda} surrounding the vowel\index{vowels} (CVC). The restriction on consonants\index{consonants} serving as a coda\index{syllable!coda} still applies. This is likely the second-most frequent type of syllable in \lang{}.

\ex<syllCVC>
	\scr{kárdil} \fw{kárdil} \phnm{\pstrs kár.dil} \defn{to use up}
\xe

\subsubsection{CCV}

The second-most complex syllable structure in \lang{} is a two-consonant onset\index{syllable!onset} and a vowel\index{vowels} (CCV). The consonant\index{consonants} cluster\index{consonants!clusters} in the onset\index{syllable!onset} is greatly restricted, allowing only specific combinations, such as:

\begin{itemize}
	\item a dental plosive and \orth{r}, i.e. \orth{tr} or \orth{dr};
	\item \orth{f}, \orth{v}, or any non-dental plosive, combined with any liquid, e.g. \orth{pl}, \orth{gr}, \orth{vr}, or \orth{bl};
	\item any nasal, plosive, or fricative except those that are bilabial or labio-dental, combined with \orth{w}, e.g. \orth{sw}, \orth{žw}, or \orth{kw}; and
	\item any nasal, plosive, or fricative except those that are velar or glottal, combined with \orth{j}, e.g. \orth{pj}, \orth{ðj}, or \orth{nj}.
\end{itemize}

\pex<syllCCV>
	\a<gw> \scr{gworóó} \fw{gworóó} \phnm{\pstrs gwo.r\gem{ó}} \defn{hummingbird}
	\a<gr> \scr{grianil} \fw{grianil} \phnm{\pstrs gri.a.nil} \defn{to break}
\xe

\subsubsection{CCVC}

The final, most complex syllable type includes a two-consonant\index{consonants}\index{consonants!clusters} onset\index{syllable!onset}, a vowel\index{vowels}, and a coda\index{syllable!coda} (CCVC), the same as the previous syllable type but with an added coda\index{syllable!coda}. The same restriction applies to the coda\index{syllable!coda} here as in the VC and CVC syllable types.


\pex<syllCCVC>
	\a<br> \scr{brallóó} \fw{brallóó} \phnm{\pstrs bra\gem{l}\gem{ó}} \defn{bird}
	\a<fl> \scr{tjož} \fw{tjož} \phnm{\pstrs tjoʑ} \defn{sixty}
	\a<tw> \scr{fryýnáitwil} \fw{fryýnáitwil} \phnm{\pstrs fr\gem{y̌}.ná\nsyl{i}.twil} \defn{to interrogate}
\xe

\index{syllable!types|)}

\subsection{Phonological Changes}
\label{sec:phonological-changes}
\index{phonological changes|(}

There are a number of phonological rules that affect \lang{} pronunciation and spelling, mostly caused by assimilation\index{assimilation|see {phonological changes}}.

\subsubsection{Consonants}
\index{consonants|(}
\index{consonants!clusters|(}

Due to the restrictions on the structure of syllables, most of the consonant-affecting sound changes occur only across syllable boundaries.

All nasals assimilate to the place of articulation of any immediately-following obstruent\index{consonants}\index{consonants!clusters} except for \orth{h}.

\begin{figure}[h]\centering
	\index{phonological changes}
	\caption{Phonological Change: \phnm{m} to \phnm{n}}
	\label{fig:m-to-n-change}
	\phonc{\phnm{m}}{\phnm{n}}{\oneof{
			\phold\phonfeat{
				+ obstruent\\
				+ dental}\\
			\phold\phonfeat{
				+ obstruent\\
				+ alveolar}\\
			\phold\phonfeat{
				+ obstruent\\
				+ alveolo-palatal}}}
\end{figure}

As shown in \autoref{fig:m-to-n-change}, \phnm{m} becomes \phnm{n} when immediately followed by an obstruent that is dental, alveolar, or alveolo-palatal. The orthography reflects this change, and \orth{m} becomes \orth{n}.

\pex<nasal-alveolar-assim>
	\a<m-m> \scr{Humállin.} \fw{Humállin.} \phnm{hu\pstrs má\gem{l}in} \defn{I blow across it.}
	\a<m-n> \scr{Hunþronin.} \fw{Hunþronin.} \phnm{hun\pstrs þro.nin} \defn{I walk across it.}
\xe

Notice in example \getfullref{nasal-alveolar-assim.m-n} that the prefix \fw{hum-} changes to \fw{hun-} because it is followed by \orth{þ}.

\begin{figure}[h]\centering
	\index{phonological changes}
	\caption{Phonological Change: \phnm{n} to \phnm{m}}
	\label{fig:n-to-m-change}
	\phonr{\phnm{n}}{\phnm{m}}{\oneof{
			\phold\phonfeat{
				+ obstruent\\
				+ bilabial}\\
			\phold\phonfeat{
				+ obstruent\\
				+ labio-dental}}}
\end{figure}

As shown in \autoref{fig:n-to-m-change}, \phnm{n} becomes \phnm{m} when immediately followed by an obstruent that is bilabial or labio-dental. The orthography reflects this change, and \orth{n} becomes \orth{m}.

\pex<nasal-labial-assim>
	\a<n-n> \scr{Ínhóltor.} \fw{Ínhóltor.} \phnm{ín\pstrs hól.tor} \defn{I would meet up with you.}
	\a<n-m> \scr{Ímvódir.} \fw{Ímvódir.} \phnm{ím\pstrs vó.dir} \defn{I would love you.}
\xe

Notice in example \getfullref{nasal-labial-assim.n-m} that the prefix \fw{ín-} changes to \fw{ím-} because it is followed by \orth{v}.

\begin{figure}[h]\centering
	\index{phonological changes}
	\caption{Phonological Change: nasal to \phnm{ŋ}}
	\label{fig:nasal-to-ŋ-change}
	\phonr{\phonfeat{+ nasal}}{\phnm{ŋ}}{\phonfeat{
			+ obstruent\\
			+ velar}}
\end{figure}

As shown in \autoref{fig:nasal-to-ŋ-change}, any nasal becomes \phnm{ŋ} when immediately followed by a velar obstruent. The orthography reflects this change, and \orth{m} becomes \orth{n}.

\pex<nasal-velar-assim>
	\a<m-ŋ> \scr{Hungeþronon.} \fw{Hungeþronon.} \phnm{huŋ\pstrs geθ.ro.non}
	\a<n-ŋ> \scr{Íngruahir.} \fw{Íngruahir.} \phnm{íŋ\pstrs gru.a.hir} \defn{I would doubt you.}
\xe

Notice in example \getfullref{nasal-velar-assim.m-ŋ} that the prefix \fw{hum-} changes to \fw{hun-}, pronounced \phnt{huŋ}, because it is followed by \orth{g}. Meanwhile, in example \getfullref{nasal-velar-assim.n-ŋ}, the prefix \fw{ín-} is pronounced \phnt{íŋ} because it is followed by \orth{g}.

\begin{figure}[h]\centering
	\index{phonological changes}
	\caption{Phonological Change: \phnm{v} to \phnm{f}}
	\label{fig:v-to-f-change}
	\phonr{\phnm{v}}{\phnm{f}}{\phnm{f}}
\end{figure}

As shown in \autoref{fig:v-to-f-change}, \phnm{v} becomes \phnm{f} when immediately followed by \phnm{f}. The two consonants are then pronounced as a geminate \phnm{\gem{f}}. The orthography does not reflect this change, so the combination is shown as \orth{vf}. This is shown in example \getfullref{assimv.ff}.

\pex[exno=\getref{assimv}]
	\a<f> \scr{fryýnil} \fw{fryýnil} \phnm{\pstrs fr\gem{y̌}nil} \defn{to admit}
	\a<ff> \scr{Evfryýnan.} \fw{Evfryýnan.} \phnm{ef\pstrs fr\gem{y̌}nan} \defn{May s/he admit it.}
\xe

\begin{figure}[h]\centering
	\index{phonological changes}
	\caption{Phonological Change: Ejective Plosives}
	\label{fig:ejective-plosives-change}
	\phon{\phnm{t}\phonfeat{
			+ plosive\\
			+ unvoiced}}{
			∅\phonfeat{
			+ plosive\\
			+ ejective}}
\end{figure}

As shown in \autoref{fig:ejective-plosives-change}, \phnm{t} is dropped and causes a following unvoiced plosive to become ejective. The orthography does not reflect this change. \orth{tp} is pronounced as \phnm{\eje{p}}, \orth{tt} is pronounced as \phnm{\eje{t}}, and \orth{tk} is pronounced as \phnm{\eje{k}}. This is shown in example \getref{assimt}.

\pex[exno=\getref{assimt}]
	\a<p> \scr{šékretpái} \fw{šékretpái} \phnm{\pstrs ɕékre\eje{p}á\nsyl{i}} \defn{disaster}
	\a<t> \scr{mittaa} \fw{mittaa} \phnm{\pstrs mi\eje{t}\gem{a}} \defn{text}
	\a<k> \scr{brétkáu} \fw{brétkáu} \phnm{\pstrs bré\eje{k}á\nsyl{u}} \defn{fire}
\xe

\begin{figure}[h]\centering
	\index{phonological changes}
	\caption{Phonological Change: \phnm{t} to \phnm{d}}
	\label{fig:t-to-d-change}
	\phonr{\phnm{t}}{\phnm{d}}{\phnm{d}}
\end{figure}

As shown in \autoref{fig:t-to-d-change}, \phnm{t} becomes \phnm{d} when it is immediately followed by \phnm{d}. The orthography reflects this change, and \orth{t} becomes \orth{d}.

\pex<t-to-d>
	\a<no> \scr{Kótfízun.} \fw{Kótfízun.} \phnm{kót\pstrs fí.zun} \defn{Don't sell it.}
	\a<yes> \scr{Kóddriíhun.} \fw{Kóddriíhun.} \phnm{kód\pstrs dr\gem{ǐ}.hun} \defn{Don't accept it.}
\xe

Notice in example \getfullref{t-to-d.yes} that the prefix \fw{kót-} changes to \fw{kód-} because it is followed by \orth{d}.

\begin{figure}[h]\centering
	\index{phonological changes}
	\caption{Phonological Change: Geminate Unvoiced Plosives}
	\label{fig:geminate-unvoiced-plosives-change}
	\phon{\phnm{d}\phonfeat{
			+ plosive\\
			+ unvoiced}}{
		∅\phonfeat{
			+ plosive\\
			+ unvoiced\\
			+ geminate}}
\end{figure}

As shown in \autoref{fig:geminate-unvoiced-plosives-change}, \phnm{d} is dropped and geminates a following unvoiced plosive. The orthography does not reflect this change. \orth{dp} is pronounced as \phnm{\gem{p}}, \orth{dt} is pronounced as \phnm{\gem{t}}, and \orth{dk} is pronounced as \phnm{\gem{k}}. This is shown in example \getref{assimd}.

\pex[exno=\getref{assimd}]
	\a<p> \scr{Jódpe.} \fw{Jódpe.} \phnm{\pstrs jó\gem{p}e} \defn{It is red.}
	\a<t> \scr{trudtil} \fw{trudtil} \phnm{\pstrs tru\gem{t}il} \defn{to thank}
	\a<k> \scr{tedkóó} \fw{tedkóó} \phnm{\pstrs te\gem{k}\gem{ó}} \defn{hawk}
\xe

\begin{figure}[h]\centering
	\index{phonological changes}
	\caption{Phonological Change: \phnt{h} to \phnt{ç}}
	\label{fig:h-to-ç-change}
	\phonr{\phnt{h}}{\phnt{ç}}{\phonfeat{
								+ vowel\\
								+ high\\
								+ front}}
\end{figure}

As shown in \autoref{fig:h-to-ç-change}, \phnt{h} becomes \phnt{ç} when it is immediately followed by a high front vowel, i.e. \phnm{i} or \phnm{y}. The orthography does not reflect this change. This is shown in example \getref{palatalh}.

\pex[exno=\getref{palatalh}]
	\a<i> \scr{fráúhil} \fw{fráúhil} \phnt{\pstrs fɾɑ́\nsyl{ú}çil} \defn{to afford}
	\a<y> \scr{hyráyy} \fw{hyráyy} \phnt{\pstrs çyɾǽ\gem{y}} \defn{sufficience}
\xe

\begin{figure}[h]\centering
	\index{phonological changes}
	\caption{Phonological Change: \phnt{ɾɾ} to \phnt{\gem{r}}}
	\label{fig:rflap-to-rtrill-change}
	\phon{\phnt{ɾɾ}}{\phnt{\gem{r}}}
\end{figure}

As shown in \autoref{fig:rflap-to-rtrill-change}, a doubled \phnt{ɾ} is pronounced as a geminate trill \phnt{\gem{r}}. The orthography does not reflect this change, so it is written simply as \orth{rr}. This is shown in example \getref{gemr}.

\ex[exno=\getref{gemr}]
	\scr{krurril} \fw{krurril} \phnt{\pstrs kɾu\gem{r}il} \defn{to understand}
\xe

\autoref{tab:phonochangesconsonants} lists all of the individual changes governed by the rules listed above.

\begin{table}\centering
	\index{phonological changes}
	\caption{Phonological Changes of Consonants}
	\label{tab:phonochangesconsonants}
	\begin{tabu} to \textwidth {l | l}
		\toprule
		Phonological Change & Orthographic Change\\
		\midrule % M
		\phonr{\phnt{m}}{\phnt{n}}{\phnm{t}} & \phonr{\orth{m}}{\orth{n}}{\orth{t}}\\
		\phonr{\phnt{m}}{\phnt{n}}{\phnm{d}} & \phonr{\orth{m}}{\orth{n}}{\orth{d}}\\
		\phonr{\phnt{m}}{\phnt{n}}{\phnm{þ}} & \phonr{\orth{m}}{\orth{n}}{\orth{þ}}\\
		\phonr{\phnt{m}}{\phnt{n}}{\phnm{ð}} & \phonr{\orth{m}}{\orth{n}}{\orth{ð}}\\
		\phonr{\phnt{m}}{\phnt{n}}{\phnm{s}} & \phonr{\orth{m}}{\orth{n}}{\orth{s}}\\
		\phonr{\phnt{m}}{\phnt{n}}{\phnm{z}} & \phonr{\orth{m}}{\orth{n}}{\orth{z}}\\
		\phonr{\phnt{m}}{\phnt{n}}{\phnm{š}} & \phonr{\orth{m}}{\orth{n}}{\orth{š}}\\
		\phonr{\phnt{m}}{\phnt{n}}{\phnm{ž}} & \phonr{\orth{m}}{\orth{n}}{\orth{ž}}\\
		\phonr{\phnt{m}}{\phnt{ŋ}}{\phnm{k}} & \phonr{\orth{m}}{\orth{n}}{\orth{k}}\\
		\phonr{\phnt{m}}{\phnt{ŋ}}{\phnm{g}} & \phonr{\orth{m}}{\orth{n}}{\orth{g}}\\
		\midrule % N
		\phonr{\phnt{n}}{\phnt{m}}{\phnm{p}} & \phonr{\orth{n}}{\orth{m}}{\orth{p}}\\
		\phonr{\phnt{n}}{\phnt{m}}{\phnm{b}} & \phonr{\orth{n}}{\orth{m}}{\orth{b}}\\
		\phonr{\phnt{n}}{\phnt{m}}{\phnm{f}} & \phonr{\orth{n}}{\orth{m}}{\orth{f}}\\
		\phonr{\phnt{n}}{\phnt{m}}{\phnm{v}} & \phonr{\orth{n}}{\orth{m}}{\orth{v}}\\
		\phonr{\phnt{n}}{\phnt{ŋ}}{\phnm{k}} & \\
		\phonr{\phnt{n}}{\phnt{ŋ}}{\phnt{g}} & \\
		\midrule % V
		\phonr{\phnm{v}}{\phnm{f}}{\phnm{f}} & \\
		\midrule % T
		\phon{\phnm{tp}}{\phnm{\eje{p}}} & \\
		\phon{\phnm{tt}}{\phnm{\eje{t}}} & \\
		\phon{\phnm{td}}{\phnm{\gem{d}}} & \phonr{\orth{t}}{\orth{d}}{\orth{d}}\\
		\phon{\phnm{tk}}{\phnm{\eje{k}}} & \\
		\midrule % D
		\phon{\phnm{dp}}{\phnm{\gem{p}}} & \\
		\phon{\phnm{dt}}{\phnm{\gem{t}}} & \\
		\phon{\phnm{dk}}{\phnm{\gem{k}}} & \\
		\midrule % H
		\phonr{\phnt{h}}{\phnt{ç}}{\phnm{i}} & \\
		\phonr{\phnt{h}}{\phnt{ç}}{\phnm{y}} & \\
		\midrule % R
		\phon{\phnt{ɾɾ}}{\phnt{\gem{r}}} & \\
		\bottomrule
	\end{tabu}
\end{table}

\index{consonants!clusters|)}
\index{consonants|)}

\subsubsection{Vowels}
\index{vowels|(}

There are only a handful of phonological rules that affect the pronunciation of \lang{} vowels. None of these rules change the spelling of the affected words.

\begin{figure}[h]\centering
	\index{phonological changes}
	\caption{Phonological Change: \phnt{a} to \phnt{ɑ}}
	\label{fig:a-to-ɑ-change}
	\phonc{\phnt{a}}{\phnt{ɑ}}{\oneof{
			\phold\phnt{u}\\
			\phnt{u}\phold}}
\end{figure}

As shown in \autoref{fig:a-to-ɑ-change}, \phnt{a} becomes \phnt{ɑ} when immediately preceded or followed by \phnm{u}. The orthography does not reflect this change.

\pex<a-to-ɑ>
	\a<none> \scr{taððil} \fw{taððil} \phnt{\pstrs\dent{t}a\gem{ð}il} \defn{to gather}
	\a<after> \scr{aumil} \fw{aumil} \phnt{\pstrs ɑ\nsyl{u}.mil} \defn{to be green}
	\a<before> \scr{ðogruahil} \fw{ðogruahil} \phnt{\pstrs ðo.gru\sstrs ɑ.çil} \defn{to be worthy of doubt}
\xe

\begin{figure}[h]\centering
	\index{phonological changes}
	\caption{Phonological Change: \phnt{a} to \phnt{æ}}
	\label{fig:a-to-æ-change}
	\phonc{\phnt{a}}{\phnt{æ}}{\oneof{
			\phold\phnt{y}\\
			\phnt{y}\phold}}
\end{figure}

As shown in \autoref{fig:a-to-æ-change}, \phnt{a} becomes \phnt{æ} when immediately preceded or followed by \phnm{y}. The orthography does not reflect this change.

\pex<a-to-æ>
	\a<none> \scr{ájégil} \fw{ájégil} \phnt{\pstrs á.jé.gil} \defn{to enjoy}
	\a<after> \scr{Ðoávay.} \fw{Ðoávay.} \phnt{\pstrs ðo.á.væ\nsyl{y}} \defn{You are mailable.}
	\a<before> \ljudge{\excl} \scr{ávyyállil} \fw{ávyyállil} \phnt{\pstrs á.v\gem{y}\sstrs æ\gem{l}il} \defn{to blow mail}
\xe

When an \phnm{a} is surrounded by an \phnm{u} and \phnm{y}, thus creating a conflict as to which of the above rules to follow, neither takes effect and it remains \phnt{a}.

\begin{figure}[h]\centering
	\index{phonological changes}
	\caption{Phonological Change: \phnt{o} to \phnt{ø}}
	\label{fig:o-to-ø-change}
	\phonc{\phnt{o}}{\phnt{ø}}{\oneof{
			\phold\phnt{y}\\
			\phnt{y}\phold}}
\end{figure}

As shown in \autoref{fig:o-to-ø-change}, \phnt{o} becomes \phnt{ø} when immediately preceded or followed by \phnm{y}. The orthography does not reflect this change.

\pex<o-to-ø>
	\a<none> \scr{vlovil} \fw{vlovil} \phnt{\pstrs vlo.vil} \defn{to see}
	\a<after> \scr{ðoýndil} \fw{ðoýndil} \phnt{\pstrs ðø\nsyl{ý}n.dil} \defn{to be possible to do by magic}
	\a<before> \scr{ávaýóonmittil} \fw{ávaýóonmittil} \phnt{\pstrs á.væ\nsyl{ý}\sstrs \gem{ø̂}n.mi.\eje{\dent{t}}il} \defn{to write mail in calligraphy}
\xe

\autoref{tab:phonochangesvowels} lists all of the individual changes governed by the rules listed above.

\begin{table}\centering
	\index{phonological changes}
	\caption{Phonological Changes of Vowels}
	\label{tab:phonochangesvowels}
	\begin{tabu} to \textwidth {l}
		\toprule % A
		\phonc{\phnt{a}}{\phnt{ɑ}}{\oneof{
				\phold\phnt{u}\\
				\phnt{u}\phold}}\\
		\phonc{\phnt{a}}{\phnt{æ}}{\oneof{
				\phold\phnt{y}\\
				\phnt{y}\phold}}\\
		\phonc{\phnt{a}}{\phnt{a}}{\oneof{
				\phnt{u}\phold\phnt{y}\\
				\phnt{y}\phold\phnt{u}}}\\
		\midrule % O
		\phonc{\phnt{o}}{\phnt{ø}}{\oneof{
				\phold\phnt{y}\\
				\phnt{y}\phold}}\\
		\bottomrule
	\end{tabu}
\end{table}

\index{vowels|)}
\index{phonological changes|)}

\subsection{Syllable Parsing}
\label{sec:syllable-parsing}

Words in \lang{} are always produced with the simplest syllable structure possible. For example, if a single consonant appears between two vowels, it will be pronounced as the onset\index{syllable!onset} of the second syllable rather than the coda\index{syllable!coda} of the first.

\ex<pronVCV>
	\scr{koráu} \fw{koráu} \defn{child} is pronounced  \phnm{\pstrs ko.rá\nsyl{u}}, not \phnm{\pstrs kor.á\nsyl{u}}
\xe

Similarly, if a valid onset\index{syllable!onset} consonant\index{consonants} cluster\index{consonants!clusters} begins with a valid coda\index{syllable!coda} consonant and appears mid-word between two vowels, the first consonant will be pronounced as the coda\index{syllable!coda} of the first syllable rather than the beginning of the second syllable's onset\index{syllable!onset}.

\ex<pronVCCV>
	\scr{dósjái} \fw{dósjái} \defn{miracle} \phnm{\pstrs dós.já\nsyl{i}}, not \phnm{\pstrs dó.sjá\nsyl{i}}
\xe

\subsection{Number of Syllables per Word}
\label{sec:syllable-counts}
\index{syllable!number per word|(}

Most root words in \lang{} seem to be two or three syllables in length. In common usage, however, there is perhaps no limit to the length of a word beyond that of pragmatics, and indeed many words are quite long. This is thanks to the combination of a robust morphology (see \autoref{cha:morphology}), a large derivational system (see \autoref{cha:derivation}), and the commonality of incorporation (see \autoref{cha:compounding-incorporation}). In literary usage, many authors purposefully stretch word lengths as long as they are able.

Unsurprisingly, shorter words tend to be simpler, such as roots and function words, while words with greater syllable counts tend to be those that are inflected, compounded, and combined through incorporation.

\pex<wordlength>
	\a<one> \scr{je} \fw{je} \phnm{je} \defn{and}
	\a<two> \begingl
		\glpreamble \scr{Þronmi.}\\
		\fw{Þronmi.}\\
		\phnm{\pstrs θron.mi}//
		\gla þron -m -i//
		\glb walk -\Prg{} -\Pt.\Dir.\Fps//
		\glft \defn{I am walking.}//
	\endgl
	\a<three> \begingl
		\glpreamble \scr{Ittóúðje.}\\
		\fw{Ittóúðje.}\\
		\phnm{\pstrs i\eje{t}ó\nsyl{ú}ð.je}//
		\gla ittóúð -j -e//
		\glb pretend -\Cnt{} -\Pt.\Dir.\Tps//
		\glft \defn{S/he is still pretending.}//
	\endgl
	\a<four> \begingl
		\glpreamble \scr{Gruahyyswé.}\\
		\fw{Gruahyyswé.}\\
		\phnm{\pstrs gru.a\sstrs h\gem{y}s.wé}//
		\gla gruahyysw -é//
		\glb lack.doubt -\Pt.\Dir.\Tpp//
		\glft \defn{They have no doubt.}//
	\endgl
	\a<six> \begingl
		\glpreamble \scr{Kínðoájégigre.}\\
		\fw{Kínðoájégigre.}\\
		\phnm{kín\pstrs ðo.á\sstrs jé.gi.gre}//
		\gla k- ín- ðoájég -igr -e//
		\glb \Neg- \Cond- be.enjoyable -\Rtsp{} -\Pt.\Dir.\Tps//
		\glft \defn{It would not have been enjoyable.}//
	\endgl
	\a<eight> \begingl
		\glpreamble \scr{ávtírgrevóóedkýhimmá}\\
		\fw{ávtírgrevóóedkýhimmá}\\
		\phnm{\pstrs áv.tír\sstrs gre.v\gem{ó}.\sstrs e\gem{k}ý\sstrs hi\gem{m}á}//
		\gla áv- tírgrevóó- edkýh -im -má//
		\glb non- kitten- smell.like -\Grv{} -\Ins//
		\glft \defn{by smelling unlike a kitten}//
	\endgl	
	\a<eleven> \begingl
		\glpreamble \scr{Kuðbotbájoudsýttírdíríóóžaáróú.}\\
		\fw{Kuðbotbájoudsýttírdíríóóžaáróú.}\\
		\phnm{kuð\pstrs bot.bá.joud\sstrs sý.\eje{t}ír\sstrs dí.rí.óó\sstrs ʑ\gem{ǎ}.ró\nsyl{ú}}//
		\gla k- uð- bot- bájou -d- sýt- tírdíríóó- žaár -óú//
		\glb \Neg- \Des- \Aug- church -\Loc- \Dim- puppy- meet -\At.\Dir.\Fpe//
		\glft \defn{I don't want us (exclusive) to meet the little puppy at the big church.}//
	\endgl
\xe

\index{syllable!number per word|)}
\index{syllable|)}

\section{Prosody}
\label{sec:prosody}
\index{prosody|(}

\subsection{Syllable Weight}
\label{sec:syllable-weight}
\index{syllable!weight|(}

Syllables can be divided into three weights: light, heavy, and superheavy. Syllable weight is measured by counting morae\index{syllable!mora}, or units of sound.

\begin{itemize}
	\item Syllable onsets\index{syllable!onset} (whether a single consonant\index{consonants} or a cluster\index{consonants!clusters}) do not represent any morae\index{syllable!mora}.
	\item A short vowel\index{vowels} in a syllable nucleus represents one mora\index{syllable!mora}.
	\item Long vowels\index{vowels!gemination} and diphthongs\index{vowels!diphthongs} in a syllable nucleus represent two morae\index{syllable!mora}.
	\item A syllable coda\index{syllable!coda} represents one mora\index{syllable!mora}.
\end{itemize}

Monomoraic\index{syllable!mora} syllables are light, bimoraic\index{syllable!mora} syllables are heavy, and trimoraic\index{syllable!mora} syllables are superheavy.

\index{syllable!weight|)}

\subsection{Word Stress}
\label{sec:word-stress}
\index{prosody!stress|(}

\subsubsection{Primary Stress}
\label{sec:primary-stress}

Word stress in \lang{} is not distinctive, so two words cannot be distinguished by stress placement alone. Further, the difference between stressed and unstressed syllables is small, and the presence or lack of stress has no affect on phonemic pronunciation.

\lang{} has fixed primary stress, which means the stress location is independent of the weight of the syllables in the word.\autocite{wals-14} Primary stress almost always occurs on the first syllable of a word.

\ex<pstrroot>
	\scr{Vódir.} \fw{Vódir.} \phnm{\pstrs vó.dir} \defn{I love you.}
\xe

The primary stress moves onto a prefix\index{affixes!prefixes}\index{prefixes|see {affixes}} if that prefix is derivational.

\ex<pstrder>
	\scr{Rilvódir.} \fw{Rilvódir.} \phnm{\pstrs ril.vó.dir} \defn{I love you back.}
\xe

However, the primary stress does \emph{not} move onto inflectional prefixes\index{affixes!prefixes}. The first syllable of the root (or derived root) still receives the primary stress.

\pex<pstri>
	\a<nod> \scr{Kavódir.} \fw{Kavódir.} \phnm{ka\pstrs vó.dir} \defn{I don't love you.}
	\a<d> \scr{Karilvódir.} \fw{Karilvódir.} \phnm{ka\pstrs ril.vó.dir} \defn{I don't love you back.}
\xe

If a word has two inflectional prefixes\index{affixes!prefixes}, the first inflectional prefix will receive secondary stress.

\pex<pstrdoublei>
	\a<noi> \scr{Þronwi.} \fw{Þronwi.} \phnm{\pstrs θron.wi} \defn{I resume walking.}
	\a<onei1> \scr{Kaþronwi.} \fw{Kaþronwi.} \phnm{ka\pstrs θron.wi} \defn{I don't resume walking.}
	\a<onei2> \scr{Lusþronwir.} \fw{Lusþronwir.} \phnm{lus\pstrs θron.wir} \defn{I resume walking toward you.}
	\a<twoi> \scr{Kalusþronwir.} \fw{Kalusþronwir.} \phnm{\sstrs ka.lus\pstrs θron.wir} \defn{I don't resume walking toward you.}
\xe

\subsubsection{Secondary Stress}
\label{sec:secondary-stress}

Words with only one or two syllables after the primary stress will not have secondary stress (as shown in examples~\getfullref{pstrroot}--\getfullref{pstrdoublei.onei2}).

On longer words, secondary stress normally falls on every odd syllable following the primary stress, giving \lang{} a trochaic rhythm.\autocite{wals-17} This holds true even when derivational\index{derivation} prefixes\index{affixes!prefixes} are added to a root. In other words, a derived term may place secondary stress in a different location from its root.

\pex<scndstress>
	\a<root> \scr{Bóasúvy.} \fw{Bóasúvy.} \phnm{\pstrs bó.a\sstrs sú.vy} \defn{You are worthy of honor.}
	\a<deriv> \scr{Bókriasúvy.} \fw{Bókriasúvy.} \phnm{\pstrs bó.kri\sstrs a.sú.vy} \defn{You are worthy of dishonor.}
\xe

However, unlike primary stress, secondary stress \emph{is} weight\index{syllable!weight}-sensitive;\autocite{wals-16} if a syllable that would normally receive secondary stress is immediately followed by a heavier syllable, the secondary stress shifts one syllable later.

\pex<scndstressmvmt>
	\a<unmovedsh> \scr{Úþgruaha.} \fw{Úþgruaha.} \phnm{\pstrs úþ.gru\sstrs a.ha} \defn{S/he is a skeptic.}
	\a<reversed> \scr{úþgruahil} \fw{úþgruahil} \phnm{\pstrs úþ.gru.a\sstrs hil} \defn{to be a skeptic}
	\a<unmovedlo> \scr{guðofráúhil} \fw{guðofráúhil} \phnm{\pstrs gu.ðo\sstrs frá\nsyl{ú}.hil} \defn{to be cheap}
	\a<moved> \scr{geguðofráúhil} \fw{geguðofráúhil} \phnm{\pstrs ge.gu.ðo\sstrs frá\nsyl{ú}.hil} \defn{to cheapen}
\xe

If this is the final foot of the word, as in example \getfullref{scndstressmvmt.reversed}, that means the foot simply reverses. If it is not the final foot of the word, as in example \getfullref{scndstressmvmt.moved}, then all following syllables with secondary stress are shifted as well.

Secondary stress can also shift in cases of compounding and noun incorporation if a root has an odd number of syllables. In any compound with more than three total syllables, each lexical item within the compound will receive secondary stress as though it were a standalone word.

\pex<cmpdscndstressmvmt>
	\a<unmoved> \scr{brallóóvloveékwil} \fw{brallóóvloveékwil} \defn{to photograph birds} is pronounced \phnm{\pstrs bra\gem{l}\gem{ó}\sstrs vlo.v\gem{ě}.kwil}, not \phnm{\pstrs bra\gem{l}\gem{ó}.vlo\sstrs v\gem{ě}.kwil}
	\a<moved> \scr{tírgrevóóvloveékwil} \fw{tírgrevóóvloveékwil} \defn{to photograph kittens} is pronounced \phnm{\pstrs tír.gre.v\gem{ó}\sstrs vlo.v\gem{ě}.kwil}, not \phnm{\pstrs tír.gre\sstrs v\gem{ó}.vlo\sstrs v\gem{ě}.kwil}
\xe

In example \getfullref{cmpdscndstressmvmt.unmoved}, the secondary stress does \emph{not} shift to the heavier second syllable of \fw{vloveékwil} as may be expected, because the first syllable of that root is stressed. Meanwhile, in example \getfullref{cmpdscndstressmvmt.moved}, the secondary stress \emph{does} shift to the first syllable of \fw{vloveékwil}, even though it is lighter than the preceding syllable.

\index{prosody!stress|)}

\subsection{Intonation}
\label{sec:intonation}
\index{prosody!intonation|(}

At the time of this writing, there has not been a meaningful analysis on the use of intonation in \lang{}.

\subsubsection{Declarative Statements}

\subsubsection{Yes--No Questions}

\subsubsection{Other Questions}

\subsubsection{Lists}

\subsubsection{Complement and Relative Clauses}

\subsubsection{Contrast}

\index{prosody!intonation|)}
\index{prosody|)}
