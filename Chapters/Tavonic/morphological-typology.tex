\chapter{Morphological Typology}
\label{cha:tvk-morphological-typology}

Now that \langtvk, \langank, and \langrdk's phonologies have been defined in \autoref{cha:tvk-phonology}, this chapter will discuss the next larger unit of language: morphemes. A morpheme is the smallest meaningful unit in a language. A morpheme can be a root, or it can be another element that affects or modifies the meaning of a root. Further, a morpheme may be freestanding, or it may be bound to other morphemes to form a larger word.

The discussion will begin with a general explanation of the \langtvk{} family's morphological typology. Following this will be a brief summary of the various morphological processes that occur in the languages.

\section{Morphological Typology}
\label{sec:tvk-typology}
\index{morphological typology|(}

Traditional research would show that \langtvk{} is typologically partially isolating and partially fusional, meaning that morphemes are often either separated into distinct words or fused together such that a single phonological unit represents several morphemes. However, according to Bickel and Nichols, \blockquote{Recent research has shown that such a scale [ranging from isolating to agglutinative to fusional to introflexive] conflates many different typological variables and incorrectly assumes that these parameters covary universally\autocite{Plank-1999,Bickel-and-Nichols-2005}. Three prominent variables involved in this are phonological fusion, formative exponence, and flexivity (i.e. allomorphy, inflectional classes).\autocite{wals-20}} Therefore, we will examine each of these areas---phonological fusion, formative exponence, and flexivity, as well as the degree of synthesis---separately.

\subsection{Phonological Fusion}
\label{sec:tvk-fusion}
\index{morphological typology!fusion|(}

\langtvk's phonological formatives are partially fusional, being partially \enquote{isolating} and partially \enquote{concatenative}\autocite{wals-20}. The concatenative morphemes are phonologically bound, requiring a \enquote{host word} with which they form one single phonological word, while the isolating morphemes are \enquote{full-fledged phonological words of their own}.

Verbs are almost exclusively concatenative, with tense, aspect, and mood morphemes attached directly to the verb's stem.

\pex<tvk-concat-verbs>
	\a<inf>\begingl
		\glpreamble\fw{ufuli}\\
		\phnm{u\pstrs fu.li}//
		\gla uf -uli//
		\glb sing -\Inf//
		\glft \defn{to sing}//
	\endgl
	\a<imp>\begingl
		\glpreamble\fw{Ufunte!}\\
		\phnm{u\pstrs fun.te}//
		\gla uf -unte//
		\glb sing -\Imp//
		\glft \defn{Sing!}//
	\endgl
	\a<phrase>\begingl
		\glpreamble\fw{Mon ufuk.}\\
		\phnm{\pstrs mon u\pstrs fuk}//
		\gla mon uf -uk//
		\glb \Fps.\Top{} sing -\Ind.\Pst.\Prf//
	\glft \defn{I sang.}//
	\endgl
\xe

Example \getref{tvk-concat-verbs} shows how morphemes are attached to the stem of a verb through suffixes, rather than with separate (isolating) modifying words or nonlinear ablaut or tone modifications.

Example \getfullref{tvk-concat-verbs.phrase} similarly shows how personal pronouns are fusional. Each personal pronoun simultaneously indicates the person, number, animacy in the third person, case, and whether it is the topic.

\pex<ex:tvk-concat-prons>
	\a<fpsa> \fw{mor} \phnm{moɾ} \defn{I} \gloss{\Fps.\Abs}
	\a<sppa> \fw{þeton} \phnm{θe\pstrs ton} \defn{you} \gloss{\Spp.\Acc}
	\a<tpcitd> \fw{ginsek} \phnm{gin\pstrs sek} \defn{to it} \gloss{\Tpc.\In.\Top.\Dat}
\xe

This concatenation appears not only in inflectional morphology, but also in derivational morphology. For example, the word \fw{ablutik} \phnm{a.blu\pstrs tik} \defn{kitten} is formed from the root noun \fw{ablu} \phnm{a\pstrs blu} \defn{cat} with a diminutive suffix attached \gloss{\fw{ablu}-\Dim}. Similarly, the word \fw{akradir} \phnm{ak.ra\pstrs dir} \defn{pen} is formed from the root verb \fw{akrali} \phnm{ak\pstrs ra.li} \defn{to write} with a nominalizing suffix \gloss{\fw{akra}-\Nmz}.

Nouns, on the other hand, are exclusively isolating. All grammatical markings, including number, gender, case, and topicality, are indicated using phonologically separate postpositions.

\pex<tvk-iso-nouns>
	\a<asta>\begingl
		\glpreamble\fw{Akrakon no aruþ.}\\
		\phnm{ak.ra\pstrs kon no a\pstrs ruθ}//
		\gla akrakon no ar -uþ//
		\glb writer \An.\Sg.\Top.\Abs{} stand -\Ind.\Npst.\Prg//
		\glft \defn{The writer is standing.}//
	\endgl
	\a<null>\begingl
		\glpreamble\fw{Esokon elbi moþes šus botra ken draš.}\\
		\phnm{e.so\pstrs kon el\pstrs bi mo\pstrs θes ʃus bot\pstrs ɾa ken \pstrs dɾaʃ}//
		\gla esokon elbi moþes šus botra ken dr -aš//
		\glb farmer egg \In.\Pc.\Top.\Acc{} \Tps.\An.\Gen{} wife \An.\Pl.\Dat{} give -\Ind.\Npst.\Rtsp//
		\glft \defn{The farmer has given the eggs to his wife.}//
	\endgl
\xe

\index{morphological typology!fusion|)}

\index{morphological typology|)}
