\chapter{Morphological Typology}
\label{cha:tvk-morphological-typology}

Now that \langtvk, \langank, and \langrdk's phonologies have been defined in \autoref{cha:tvk-phonology}, this chapter will discuss the next larger unit of language: morphemes. A morpheme is the smallest meaningful unit in a language. A morpheme can be a root, or it can be another element that affects or modifies the meaning of a root. Further, a morpheme may be freestanding, or it may be bound to other morphemes to form a larger word.

The discussion will begin with a general explanation of the \langtvk{} family's morphological typology. Following this will be a brief summary of the various morphological processes that occur in the languages, ending with an explanation of the locus of marking.

\section{Morphological Typology}
\label{sec:tvk-typology}
\index{morphological typology|(}

Traditional research would show that \langtvk{} is typologically partially isolating and partially fusional, meaning that morphemes are often either separated into distinct words or fused together such that a single phonological unit represents several morphemes. However, according to Bickel and Nichols, \blockquote{Recent research has shown that such a scale [ranging from isolating to agglutinative to fusional to introflexive] conflates many different typological variables and incorrectly assumes that these parameters covary universally\autocite{Plank-1999,Bickel-and-Nichols-2005}. Three prominent variables involved in this are phonological fusion, formative exponence, and flexivity (i.e. allomorphy, inflectional classes).\autocite{wals-20}} Therefore, we will examine each of these areas---phonological fusion, formative exponence, and flexivity, as well as the degree of synthesis---separately.

\subsection{Phonological Fusion}
\label{sec:tvk-fusion}
\index{morphological typology!fusion|(}

\langtvk's phonological formatives are partially fusional, being partially \enquote{isolating} and partially \enquote{concatenative}\autocite{wals-20}. The concatenative morphemes are phonologically bound, requiring a \enquote{host word} with which they form one single phonological word, while the isolating morphemes are \enquote{full-fledged phonological words of their own}.

Verbs are almost exclusively concatenative, with tense, aspect, and mood morphemes attached directly to the verb's stem.

\pex<ex:tvk-concat-verbs>
	\a<inf>\begingl
		\glpreamble\fw{ufuli}\\
		\phnm{u\pstrs fu.li}//
		\gla uf-uli//
		\glb sing-\Inf//
		\glft \defn{to sing}//
	\endgl
	\a<imp>\begingl
		\glpreamble\fw{Ufunte!}\\
		\phnm{u\pstrs fun.te}//
		\gla uf-unte//
		\glb sing-\Imp//
		\glft \defn{Sing!}//
	\endgl
	\a<phrase>\begingl
		\glpreamble\fw{Mon ufuk.}\\
		\phnm{\pstrs mon u\pstrs fuk}//
		\gla mon uf-uk//
		\glb \Fps.\Top{} sing-\Ind.\Pst.\Prf//
	\glft \defn{I sang.}//
	\endgl
\xe

Example \getref{ex:tvk-concat-verbs} shows how morphemes are attached to the stem of a verb through suffixes, rather than with separate (isolating) modifying words or nonlinear ablaut or tone modifications.

Example \getfullref{ex:tvk-concat-verbs.phrase} similarly shows how personal pronouns are fusional. Example \getref{ex:tvk-concat-prons} demonstrates further how each personal pronoun simultaneously indicates the person, number, animacy in the third person, case, and whether it is the topic.

\pex<ex:tvk-concat-prons>
	\a<fpsa> \fw{mor} \phnm{moɾ} \defn{I} \gloss{\Fps.\Abs}
	\a<sppa> \fw{þeton} \phnm{θe\pstrs ton} \defn{you} \gloss{\Spp.\Acc}
	\a<tpcitd> \fw{ginsek} \phnm{gin\pstrs sek} \defn{to it} \gloss{\Tpc.\In.\Top.\Dat}
\xe

This concatenation appears not only in inflectional morphology, but also in derivational morphology. For example, the word \fw{ablutik} \phnm{a.blu\pstrs tik} \defn{kitten} is formed from the root noun \fw{ablu} \phnm{a\pstrs blu} \defn{cat} with a diminutive suffix attached \gloss{\fw{ablu}-\Dim}. Similarly, the word \fw{akradir} \phnm{ak.ra\pstrs dir} \defn{pen} is formed from the root verb \fw{akrali} \phnm{ak\pstrs ra.li} \defn{to write} with a nominalizing suffix \gloss{\fw{akra}-\Nmz}.

Nouns, on the other hand, are exclusively isolating. All grammatical markings, including number, gender, case, and topicality, are indicated using phonologically separate postpositions.

\pex<ex:tvk-iso-nouns>
	\a<asta>\begingl
		\glpreamble\fw{Akrakon no aruþ.}\\
		\phnm{ak.ra\pstrs kon no a\pstrs ruθ}//
		\gla akrakon -no ar-uþ//
		\glb writer -\An.\Sg.\Top.\Abs{} stand-\Ind.\Npst.\Prg//
		\glft \defn{The writer is standing.}//
	\endgl
	\a<phrase>\begingl
		\glpreamble\fw{Esokon elbi moþes šus botra ken draš.}\\
		\phnm{e.so\pstrs kon el\pstrs bi mo\sstrs θes \pstrs ʃus bot\pstrs ɾa ken \pstrs dɾaʃ}//
		\gla esokon -∅ elbi -moþes šus botra -ken dr-aš//
		\glb farmer -\An.\Sg.\Abs{} egg -\In.\Pc.\Top.\Acc{} \Tps.\An.\Gen{} wife -\An.\Pl.\Dat{} give-\Ind.\Npst.\Rtsp//
		\glft \defn{The farmer has given the eggs to his wife.}//
	\endgl
\xe

Notice in example \getref{ex:tvk-iso-nouns} how every noun is followed by a postposition that identifies that noun's grammatical role within the sentence.

\index{morphological typology!fusion|)}

\subsection{Formative Exponence}
\label{sec:tvk-exponence}
\index{morphological typology!exponence|(}

\langtvk{} has mostly polyexponential formatives, meaning that, in almost all cases, single morphemes express multiple grammatical categories each\autocite{wals-21}. Derivational morphemes are all monoexponential while inflectional morphemes are almost exclusively polyexponential.

\ex<ex:tvk-exponence>
	\begingl
		\glpreamble\fw{Tavotik nan one vi?}\\
		\phnm{ta.vo\pstrs tik nan o\pstrs ne vi}//
		\gla tavo-tik -nan on-e =vi//
		\glb person-\Dim{} -\An.\Pl.\Top{} play-\Ind.\Npst.\Ipfv{} =\Int//
		\glft \defn{Do children play?}//
	\endgl
\xe

Example \getref{ex:tvk-exponence} includes one derivational morpheme and three inflectional morphemes attached to the roots \fw{tavo} and \fw{oneli}, two of which are polyexponential. The first morpheme, \fw{-tik}, a diminutive that derives the word \defn{child} from the root \defn{person}, is a monoexponential derivational suffix. The postposition \fw{nan} is a polyexponential morpheme that identifies the preceding noun's gender (animate), number (plural), and topicality. The single-letter suffix \fw{-e} attaches to the verb to express the mood (indicative), tense (nonpast), and aspect (imperfective). Finally, the word \fw{vi} is a monoexponential interrogative clitic that turns the sentence into a question.

Noun postpositions can additionally encode case. In example \getref{ex:tvk-exponence}, the noun \fw{tavotik} is inferred to be in the absolutive case despite being unmarked for it. In many other situations, this grammatical case is similarly encoded within the same polyexponential postposition. In example \getfullref{ex:tvk-iso-nouns.phrase}, the word \fw{moþes} indicates that the noun \defn{egg} is inanimate, paucal, the topic, and in the accusative case.

One noun postposition, \fw{nut} has not fully cumulated, with the noun's number being still separated into a distinct segment.

\pex<ex:tvk-noncumulated>
	\a<sg>\fw{nut-∅} \phnm{nut} \gloss{\An.\Top.\Acc-\Sg}
	\a<pc>\fw{nut-os} \phnm{nu\pstrs tos} \gloss{\An.\Top.\Acc-\Pc}
	\a<pl>\fw{nut-on} \phnm{nu\pstrs ton} \gloss{\An.\Top.\Acc-\Pl}
\xe

All other noun postpositions are fully cumulated and cannot be separated into their component morphemes.

\pex<ex:tvk-cumulated>
	\a<ie>Inanimate Ergative
	\beginsubsub
		\b{i.}\fw{ða} \phnm{ða} \gloss{\In.\Sg.\Erg}
		\b{ii.}\fw{ðes} \phnm{ðes} \gloss{\In.\Pc.\Erg}
		\b{iii.}\fw{dun} \phnm{dun} \gloss{\In.\Pl.\Erg}
	\endsubsub
	\a<itd>Inanimate Topic Dative
	\beginsubsub
		\b{i.}\fw{moǩ} \phnm{mox} \gloss{\In.\Sg.\Top.\Dat}
		\b{ii.}\fw{mekos} \phnm{me\pstrs kos} \gloss{\In.\Pc.\Top.\Dat}
		\b{iii.}\fw{nikun} \phnm{ni \pstrs kun} \gloss{\In.\Pl.\Top.\Dat}
	\endsubsub
\xe

\index{morphological typology!exponence|)}

\subsection{Flexivity}
\label{sec:flexivity}
\index{morphological typology!flexivity|(}

\langtvk{} nouns, adjectives, and verbs display flexivity, which means that these words are divided into separate classes that receive distinct inflectional allomorphs. On such allomorphs, otherwise identical morphemes take distinct phonological shapes.

Nouns are divided into animate and inanimate genders. These two genders determine which postpositions are used to provide the grammatical context of the noun.

\pex<ex:tvk-flex-nouns>
	\a<animate>\begingl
		\glpreamble\fw{bilt ri}\\
		\phnm{\pstrs bilt ɾi}//
		\gla bilt -ri//
		\glb breath -\An.\Pc.\Abs//
		\glft \defn{breaths}//
	\endgl
	\a<inanimate>\begingl
		\glpreamble\fw{eðer -le}\\
		\phnm{e\pstrs ðer le}//
		\gla eðer le//
		\glb pen -\In.\Pc.\Abs//
		\glft \defn{pens}//
	\endgl
\xe

In example \getref{ex:tvk-flex-nouns}, both \fw{bilt} and \fw{eðer} are marked for the paucal number and the absolutive case, but because \fw{bilt} is animate and \fw{eðer} is inanimate, the shape of the postpositions are entirely different.

Although they are distinct, the shapes are often more closely related than in example \getref{ex:tvk-flex-nouns}. Example \getref{ex:tvk-flex-nouns-related} shows the animate and inanimate forms of the plural ergative postposition; the relation between the two forms is much clearer, as only the vowel changes.

\pex<ex:tvk-flex-nouns-related>
	\a<animate>\begingl
		\glpreamble\fw{bilt din}\\
		\phnm{\pstrs bilt din}//
		\gla bilt -din//
		\glb breath -\An.\Pl.\Erg//
		\glft \defn{breaths}//
	\endgl
	\a<inanimate>\begingl
		\glpreamble\fw{eðer dun}\\
		\phnm{e\pstrs ðer dun}//
		\gla eðer -dun//
		\glb pen -\In.\Pl.\Erg//
		\glft \defn{pens}//
	\endgl
\xe

Nouns do not show possessive flexivity, as there is no possessive classification\autocite{wals-59}. There is only one method of forming a possessive relationship: using the genitive case.

Adjectives also show flexivity since they decline to match the gender of the noun they modify. Each adjective has a distinct animate and inanimate form, with animate adjectives ending in \fw{-a}, \fw{-i}, or \fw{-u} and inanimate adjectives ending in \fw{-e} or \fw{-o}.

\pex<ex:tvk-flex-adjectives>
	\a<animate>\begingl
		\glpreamble\fw{frandi bilt su}\\
		\phnm{fran\pstrs di \pstrs bilt su}//
		\gla frandi bilt -su//
		\glb visible.\An{} breath -\An.\Sg.\Gen//
		\glft \defn{of the visible breath}//
	\endgl
	\a<inanimate>\begingl
		\glpreamble\fw{frando eðer šo}\\
		\phnm{fran\pstrs do e\pstrs ðer ʃo}//
		\gla frando eðer -šo//
		\glb visible.\In{} pen -\In.\Sg.\Gen//
		\glft \defn{of the visible pen}//
	\endgl
\xe

In example \getref{ex:tvk-flex-adjectives}, the form of \fw{frandi} changes depending on whether it is modifying an animate noun like \fw{bilt} or an inanimate noun like \fw{eðer}.

Verbs are divided into three distinct conjugation classes, each identified by the infinitive form. Class I verb infinitives end in \fw{-ali}, class II verb infinitives end in \fw{-eli}, and class III verb infinitives end in \fw{-uli}.

\pex<ex:tvk-flex-verbs>
	\a<cl1>Class I: \fw{bruþat-ali} \phnm{bru.θa\pstrs ta.li} \defn{to handle} \gloss{handle-\Inf}
	\a<cl2>Class II: \fw{š-eli} \phnm{\pstrs ʃe.li} \defn{to run} \gloss{run-\Inf}
	\a<cl3>Class III: \fw{teg-uli} \phnm{te\pstrs gu.li} \defn{to worry} \gloss{worry-\Inf}
\xe

Beyond just the form of the infinitive, the verb's class determines the entire conjugation paradigm for that verb.

\pex<ex:tvk-flex-verbs>
	\a<cl1>Class I: \fw{bruþat-abe} \phnm{bru.θa\pstrs ta.be} \defn{handling} \gloss{handle-\Act.\Ptcp}
	\a<cl2>Class II: \fw{š-iba} \phnm{\pstrs ʃi.ba} \defn{running} \gloss{run-\Act.\Ptcp}
	\a<cl3>Class III: \fw{teg-ube} \phnm{te\pstrs gu.be} \defn{worrying} \gloss{worry-\Act.\Ptcp}
\xe

As shown in example \getref{ex:tvk-flex-verbs}, the same inflection takes a different form when attached to a verb of a different class. To form the active participle, \fw{bruþatali} becomes \fw{bruþatabe} and \fw{teguli} becomes \fw{tegube}. Following this pattern, one might expect \fw{šeli} to become \ungr\fw{šebe}, but instead it becomes \fw{šiba}.

\index{morphological typology!flexivity|)}

\subsection{Synthesis}
\label{sec:tvk-synthesis}
\index{morphological typology!synthesis|(}

As discussed in \autoref{sec:tvk-fusion}, derivation and verb inflection occurs by attaching affixes to a stem or root, forming singular phonological words. Meanwhile, noun declension occurs using postpositions that mark the grammatical information for the noun. These postpositions are separate phonological words from the nouns themselves.

In all cases, however, inflected forms constitute singular \emph{syntactic} words because the inflections cannot be separated or reordered at all. This means that \langtvk{} morphology is synthetic\autocite{wals-22}.

\langtvk{} verbs normally inflect to show mood, tense, and aspect, a total of three morpheme categories per word. The maximally inflected form adds negation, a particle that is a separate phonological word but remains a part of the syntactic word of the verb, bringing \langtvk's category-per-word ratio up to 4\autocite{wals-22}.

\ex<ex:tvk-verb-cpw>
	\begingl
		\glpreamble\fw{Šun onek bo.}\\
		\phnm{\pstrs ʃun o\pstrs nek bo}//
		\gla šun on-ek -bo//
		\glb \Tps.\An.\Top{} play-\Ind.\Pst.\Pfv{} -\Neg//
		\glft \defn{S/he did not play.}//
	\endgl
\xe

\index{morphological typology!synthesis|)}

\index{morphological typology|)}

\section{Morphological Processes}
\label{sec:tvk-processes}
\index{morphological typology!processes|(}

\index{morphological typology!processes|)}

\section{Locus of Marking}
\label{sec:tvk-locus}
\index{morphological processes!locus of marking|(}

\index{morphological processes!locus of marking|)}
