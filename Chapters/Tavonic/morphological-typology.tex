\chapter{Morphological Typology}
\label{cha:tvk-morphological-typology}

Now that \langtvk, \langank, and \langrdk's phonologies have been defined in \autoref{cha:tvk-phonology}, this chapter will discuss the next larger unit of language: morphemes. A morpheme is the smallest meaningful unit in a language. A morpheme can be a root, or it can be another element that affects or modifies the meaning of a root. Further, a morpheme may be freestanding, or it may be bound to other morphemes to form a larger word.

The discussion will begin with a general explanation of the \langtvk{} family's morphological typology. Following this will be a brief summary of the various morphological processes that occur in the languages, ending with an explanation of the locus of marking.

\section{Morphological Typology}
\label{sec:tvk-typology}
\index{morphological typology|(}

Traditional research would show that \langtvk{} is typologically partially isolating and partially fusional, meaning that morphemes are often either separated into distinct words or fused together such that a single phonological unit represents several morphemes. However, according to Bickel and Nichols, \blockquote{Recent research has shown that such a scale [ranging from isolating to agglutinative to fusional to introflexive] conflates many different typological variables and incorrectly assumes that these parameters covary universally\autocite{Plank-1999,Bickel-and-Nichols-2005}. Three prominent variables involved in this are phonological fusion, formative exponence, and flexivity (i.e. allomorphy, inflectional classes).\autocite{wals-20}} Therefore, we will examine each of these areas---phonological fusion, formative exponence, and flexivity, as well as the degree of synthesis---separately.

\subsection{Phonological Fusion}
\label{subsec:tvk-fusion}
\index{morphological typology!fusion|(}

\langtvk's phonological formatives are partially fusional, being partially \enquote{isolating} and partially \enquote{concatenative}\autocite{wals-20}. The concatenative morphemes are phonologically bound, requiring a \enquote{host word} with which they form one single phonological word, while the isolating morphemes are \enquote{full-fledged phonological words of their own}.

Verbs are almost exclusively concatenative, with tense, aspect, and mood morphemes attached directly to the verb's stem.

\pex<ex:tvk-concat-verbs>
	\a<inf>\begingl
		\glpreamble\fw{ufuli}\\
		\phnm{u\pstrs fu.li}//
		\gla uf-uli//
		\glb sing-\Inf//
		\glft \defn{to sing}//
	\endgl
	\a<imp>\begingl
		\glpreamble\fw{Ufunte!}\\
		\phnm{u\pstrs fun.te}//
		\gla uf-unte//
		\glb sing-\Imp//
		\glft \defn{Sing!}//
	\endgl
	\a<phrase>\begingl
		\glpreamble\fw{Mon ufuk.}\\
		\phnm{\pstrs mon u\pstrs fuk}//
		\gla mon uf-uk//
		\glb \Fps.\Top{} sing-\Ind.\Pst.\Pfv//
	\glft \defn{I sang.}//
	\endgl
\xe

Example~\getref{ex:tvk-concat-verbs} shows how morphemes are attached to the stem of a verb through suffixes, rather than with separate (isolating) modifying words or nonlinear ablaut or tone modifications.

Example~\getfullref{ex:tvk-concat-verbs.phrase} similarly shows how personal pronouns are fusional. Example~\getref{ex:tvk-concat-prons} demonstrates further how each personal pronoun simultaneously indicates the person, number, animacy in the third person, case, and whether it is the topic.

\pex<ex:tvk-concat-prons>
	\a<fpsa> \fw{mor} \phnm{mor} \defn{I} \gloss{\Fps.\Abs}
	\a<sppa> \fw{þeton} \phnm{θe\pstrs ton} \defn{you} \gloss{\Spp.\Acc}
	\a<tpcitd> \fw{ginsek} \phnm{gin\pstrs sek} \defn{to it} \gloss{\Tpc.\In.\Top.\Dat}
\xe

This concatenation appears not only in inflectional morphology, but also in derivational morphology. For example, the word \fw{ablutik} \phnm{a.blu\pstrs tik} \defn{kitten} is formed from the root noun \fw{ablu} \phnm{a\pstrs blu} \defn{cat} with a diminutive suffix attached \gloss{\fw{ablu}-\Dim}. Similarly, the word \fw{akradir} \phnm{ak.ra\pstrs dir} \defn{pen} is formed from the root verb \fw{akrali} \phnm{ak\pstrs ra.li} \defn{to write} with a nominalizing suffix \gloss{\fw{akra}-\Nmz}.

Nouns, on the other hand, are exclusively isolating. All grammatical markings, including number, gender, case, and topicality, are indicated using phonologically separate prepositions.

\pex<ex:tvk-iso-nouns>
	\a<asta>\begingl
		\glpreamble\fw{No akrakon aruþ.}\\
		\phnm{no ak.ra\pstrs kon a\pstrs ruθ}//
		\gla no= akrakon ar-uþ//
		\glb \An.\Sg.\Top.\Abs= writer stand-\Ind.\Npst.\Prg//
		\glft \defn{The writer is standing.}//
	\endgl
	\a<phrase>\begingl
		\glpreamble\fw{Esokon moþes elbi šus ken botra draš.}\\
		\phnm{e.so\pstrs kon mo\sstrs θes el\pstrs bi \pstrs ʃus ken bot\pstrs ra \pstrs draʃ}//
		\gla ∅= esokon moþes= elbi šus ken= botra dr-aš//
		\glb \An.\Sg.\Abs= farmer \In.\Pc.\Top.\Acc= egg \Tps.\An.\Gen{} \An.\Pl.\Dat= wife give-\Ind.\Npst.\Rtsp//
		\glft \defn{The farmer has given the eggs to his wife.}//
	\endgl
\xe

Notice in example~\getref{ex:tvk-iso-nouns} how every noun is preceded by a preposition that identifies that noun's grammatical role within the sentence.

\index{morphological typology!fusion|)}

\subsection{Formative Exponence}
\label{subsec:tvk-exponence}
\index{morphological typology!exponence|(}

\langtvk{} has mostly polyexponential formatives, meaning that, in almost all cases, single morphemes express multiple grammatical categories each\autocite{wals-21}. Derivational morphemes are all monoexponential while inflectional morphemes are almost exclusively polyexponential.

\ex<ex:tvk-exponence>
	\begingl
		\glpreamble\fw{Nan tavotik one vi?}\\
		\phnm{nan ta.vo\pstrs tik o\pstrs ne vi}//
		\gla nan= tavo-tik on-e =vi//
		\glb \An.\Pl.\Top= person-\Dim{} play-\Ind.\Npst.\Ipfv{} =\Int//
		\glft \defn{Do children play?}//
	\endgl
\xe

Example~\getref{ex:tvk-exponence} includes one derivational morpheme and three inflectional morphemes attached to the roots \fw{tavo} and \fw{oneli}, two of which are polyexponential. The preposition \fw{nan} is a polyexponential morpheme that identifies the preceding noun's gender (animate), number (plural), and topicality. The affix \fw{-tik}, a diminutive that derives the word \defn{child} from the root \defn{person}, is a monoexponential derivational suffix. The single-letter suffix \fw{-e} attaches to the verb to express the mood (indicative), tense (nonpast), and aspect (imperfective). Finally, the word \fw{vi} is a monoexponential interrogative clitic that turns the sentence into a question.

Noun prepositions can additionally encode case. In example~\getref{ex:tvk-exponence}, the noun \fw{tavotik} is inferred to be in the absolutive case despite being unmarked for it. In many other situations, this grammatical case is additionally encoded within the same polyexponential preposition. In example~\getfullref{ex:tvk-iso-nouns.phrase}, the word \fw{moþes} indicates that the noun \defn{egg} is inanimate, paucal, the topic, and in the accusative case.

One noun preposition, \fw{nut} has not fully cumulated, with the noun's number being still separated into a distinct segment.

\pex<ex:tvk-noncumulated>
	\a<sg>\fw{nut-∅} \phnm{nut} \gloss{\An.\Top.\Acc-\Sg}
	\a<pc>\fw{nut-os} \phnm{nu\pstrs tos} \gloss{\An.\Top.\Acc-\Pc}
	\a<pl>\fw{nut-on} \phnm{nu\pstrs ton} \gloss{\An.\Top.\Acc-\Pl}
\xe

All other noun prepositions are fully cumulated and cannot be separated into their component morphemes.

\pex<ex:tvk-cumulated>
	\a<ie>Inanimate Ergative
	\beginsubsub
		\b{i.}\fw{ða} \phnm{ða} \gloss{\In.\Sg.\Erg}
		\b{ii.}\fw{ðes} \phnm{ðes} \gloss{\In.\Pc.\Erg}
		\b{iii.}\fw{dun} \phnm{dun} \gloss{\In.\Pl.\Erg}
	\endsubsub
	\a<itd>Inanimate Topic Dative
	\beginsubsub
		\b{i.}\fw{moǩ} \phnm{mox} \gloss{\In.\Sg.\Top.\Dat}
		\b{ii.}\fw{mekos} \phnm{me\pstrs kos} \gloss{\In.\Pc.\Top.\Dat}
		\b{iii.}\fw{nikun} \phnm{ni \pstrs kun} \gloss{\In.\Pl.\Top.\Dat}
	\endsubsub
\xe

\index{morphological typology!exponence|)}

\subsection{Flexivity}
\label{subsec:tvk-flexivity}
\index{morphological typology!flexivity|(}

\langtvk{} nouns, adjectives, and verbs display flexivity, which means that these words are divided into separate classes that receive distinct inflectional allomorphs. On such allomorphs, otherwise identical morphemes take distinct phonological shapes.

Nouns are divided into animate and inanimate genders. These two genders determine which prepositions are used to provide the grammatical context of the noun.

\pex<ex:tvk-flex-nouns>
	\a<animate>\begingl
		\glpreamble\fw{ri bilt}\\
		\phnm{ri \pstrs bilt}//
		\gla ri= bilt//
		\glb \An.\Pc.\Abs= breath//
		\glft \defn{breaths}//
	\endgl
	\a<inanimate>\begingl
		\glpreamble\fw{l'eðer}\\
		\phnm{le\pstrs ðer}//
		\gla le=eðer//
		\glb \In.\Pc.\Abs=pen//
		\glft \defn{pens}//
	\endgl
\xe

In example~\getref{ex:tvk-flex-nouns}, both \fw{bilt} and \fw{eðer} are marked for the paucal number and the absolutive case, but because \fw{bilt} is animate and \fw{eðer} is inanimate, the shape of the prepositions are entirely different.

Although they are distinct, the shapes are often more closely related than in example~\getref{ex:tvk-flex-nouns}. Example~\getref{ex:tvk-flex-nouns-related} shows the animate and inanimate forms of the plural ergative preposition; the relation between the two forms is much clearer, as only the vowel changes.

\pex<ex:tvk-flex-nouns-related>
	\a<animate>\begingl
		\glpreamble\fw{din bilt}\\
		\phnm{din \pstrs bilt}//
		\gla din= bilt//
		\glb \An.\Pl.\Erg= breath//
		\glft \defn{breaths}//
	\endgl
	\a<inanimate>\begingl
		\glpreamble\fw{dun eðer}\\
		\phnm{dun e\pstrs ðer}//
		\gla dun= eðer//
		\glb \In.\Pl.\Erg= pen//
		\glft \defn{pens}//
	\endgl
\xe

Nouns do not show possessive flexivity, as there is no possessive classification\autocite{wals-59}. There is only one method of forming a possessive relationship: using the genitive case.

Adjectives also show flexivity since they decline to match the gender of the noun they modify. Each adjective has a distinct animate and inanimate form, with animate adjectives ending in \fw{-a}, \fw{-i}, or \fw{-u} and inanimate adjectives ending in \fw{-e} or \fw{-o}.

\pex<ex:tvk-flex-adjectives>
	\a<animate>\begingl
		\glpreamble\fw{su frandi bilt}\\
		\phnm{su fran\pstrs di \pstrs bilt}//
		\gla su= frandi bilt//
		\glb \An.\Sg.\Gen= visible.\An{} breath//
		\glft \defn{of the visible breath}//
	\endgl
	\a<inanimate>\begingl
		\glpreamble\fw{šo frando eðer}\\
		\phnm{ʃo fran\pstrs do e\pstrs ðer}//
		\gla šo= frando eðer//
		\glb \In.\Sg.\Gen= visible.\In{} pen//
		\glft \defn{of the visible pen}//
	\endgl
\xe

In example~\getref{ex:tvk-flex-adjectives}, the form of \fw{frandi} changes depending on whether it is modifying an animate noun like \fw{bilt} or an inanimate noun like \fw{eðer}.

Verbs are divided into three distinct conjugation classes, each identified by the infinitive form. Class I verb infinitives end in \fw{-ali}, class II verb infinitives end in \fw{-eli}, and class III verb infinitives end in \fw{-uli}.

\pex<ex:tvk-flex-verbs>
	\a<cl1>Class I: \fw{bruþat-ali} \phnm{bru.θa\pstrs ta.li} \defn{to handle} \gloss{handle-\Inf}
	\a<cl2>Class II: \fw{š-eli} \phnm{\pstrs ʃe.li} \defn{to run} \gloss{run-\Inf}
	\a<cl3>Class III: \fw{teg-uli} \phnm{te\pstrs gu.li} \defn{to worry} \gloss{worry-\Inf}
\xe

Beyond just the form of the infinitive, the verb's class determines the entire conjugation paradigm for that verb.

\pex<ex:tvk-flex-verbs>
	\a<cl1>Class I: \fw{bruþat-abe} \phnm{bru.θa\pstrs ta.be} \defn{handling} \gloss{handle-\Act.\Ptcp}
	\a<cl2>Class II: \fw{š-iba} \phnm{\pstrs ʃi.ba} \defn{running} \gloss{run-\Act.\Ptcp}
	\a<cl3>Class III: \fw{teg-ube} \phnm{te\pstrs gu.be} \defn{worrying} \gloss{worry-\Act.\Ptcp}
\xe

As shown in example~\getref{ex:tvk-flex-verbs}, the same inflection takes a different form when attached to a verb of a different class. To form the active participle, \fw{bruþatali} becomes \fw{bruþatabe} and \fw{teguli} becomes \fw{tegube}. Following this pattern, one might expect \fw{šeli} to become \ungr\fw{šebe}, but instead it becomes \fw{šiba}.

\index{morphological typology!flexivity|)}

\subsection{Synthesis}
\label{subsec:tvk-synthesis}
\index{morphological typology!synthesis|(}

As discussed in \autoref{subsec:tvk-fusion}, derivation and verb inflection occurs by attaching affixes to a stem or root, forming singular phonological words. Meanwhile, noun declension occurs using prepositions that mark the grammatical information for the noun. These prepositions are separate phonological words from the nouns themselves.

In all cases, however, inflected forms constitute singular \emph{syntactic} words because the inflections cannot be separated or reordered at all. This means that \langtvk{} morphology is synthetic\autocite{wals-22}.

\langtvk{} verbs normally inflect to show mood, tense, and aspect, a total of three morpheme categories per word. The maximally inflected form adds negation, a particle that is a separate phonological word but remains a part of the syntactic word of the verb, bringing \langtvk's category-per-word ratio up to 4\autocite{wals-22}.

\ex<ex:tvk-verb-cpw>
	\begingl
		\glpreamble\fw{Šun onek bo.}\\
		\phnm{\pstrs ʃun o\pstrs nek bo}//
		\gla šun on-ek -bo//
		\glb \Tps.\An.\Top{} play-\Ind.\Pst.\Pfv{} -\Neg//
		\glft \defn{S/he did not play.}//
	\endgl
\xe

\index{morphological typology!synthesis|)}

\index{morphological typology|)}

\section{Morphological Processes}
\label{sec:tvk-processes}
\index{morphological typology!processes|(}

\langtvk{} primarily makes use of suffixes and clitics to derive and inflect words. The language does not employ infixation, stem modification, or suprafixation, no prefixation has yet been identified, and reduplication only appears in wordplay and child-directed speech.

\subsection{Suffixation}
\label{subsec:tvk-suffixation}
\index{morphological typology!processes!suffixation|(}

Suffixes in \langtvk{} apply mainly to verbs. All verbal inflections occur via the addition of suffixes, whether phonologically bound or not. This is illustrated in example~\getref{ex:tvk-verbs-sfxs}.

\pex<ex:tvk-verbs-sfxs>
	\a<i:ipr>\begingl
		\glpreamble\fw{Šona git akraǧ.}\\
		\phnm{ʃo\pstrs na git ak\pstrs raɣ}//
		\gla šona git akr-aǧ//
		\glb \Tpp.\An.\Top{} \Tps.\In.\Acc{} write-\Ind.\Pst.\Rtsp//
		\glft \defn{They had written it.}//
	\endgl
	\a<sbjv>\begingl
		\glpreamble\fw{Monsa ufut oþ nikis.}\\
		\phnm{mon\pstrs sa u\pstrs fut oθ ni\pstrs kis}//
		\gla monsa uf-ut oþ nik-is//
		\glb \Fpc.\Top{} sing-\Ind.\Npst.\Pfv{} if be.able-\Sbjv.\Npst.\Ipfv//
		\glft \defn{We will sing if we are able.}//
	\endgl
	\a<iii:pptcp>\begingl
		\glpreamble\fw{usombe akrapis}\\
		\phnm{u\pstrs som.be ak.ra\pstrs pis}//
		\gla us-ombe akrapis//
		\glb hold-\Pass.\Ptcp.\In{} letter//
		\glft \defn{held letter}//
	\endgl
	\a<i:imp>\begingl
		\glpreamble\fw{Mi þro akrorganta.}\\
		\phnm{mi \pstrs θro ak.ror\pstrs gan.ta}//
		\gla mi þro akrorg-anta//
		\glb \In.\Sg.\Top{} that.\Med{} erase-\Imp//
		\glft \defn{Erase that.}//
	\endgl
	\a<neg>\begingl
		\glpreamble\fw{Mana kantenta bo.}\\
		\phnm{ma\pstrs na kan\pstrs ten.ta bo}//
		\gla mana kant-enta -bo//
		\glb \Fpp.\Top{} thank-\Imp{} -\Neg//
		\glft \defn{Don't thank us.}//
	\endgl
\xe

As discussed in \autoref{subsec:tvk-synthesis}, although the particle \fw{bo} is a separate phonological word, it functions syntactically as a suffix. This is shown in example~\getfullref{ex:tvk-verbs-sfxs.neg} where it attaches to the verb \fw{kantenta} to negate it.

Suffixes are also present on adjectives, though only minimally. Adjectives take one of two vowel endings to mark the gender of its referent, with animate adjectives ending in \fw{-i}, \fw{-a}, or \fw{u} and inanimate adjectives ending in \fw{-e} or \fw{-o}.

\pex<ex:tvk-adjective-sfxs>
	\a<ae>\fw{ablunga} \phnm{ab.lun\pstrs ga} \gloss{\An} vs. \fw{ablunge} \phnm{ab.lun\pstrs ge} \gloss{\In} \defn{catlike}
	\a<io>\fw{akrandi} \phnm{ak.ran\pstrs di} \gloss{\An} vs. \fw{akrando} \phnm{ak.ran\pstrs do} \gloss{\In} \defn{writable}
	\a<ao>\fw{bruþatla} \phnm{bru.θat\pstrs la} \gloss{\An} vs. \fw{bruþatlo} \phnm{bru.θat\pstrs lo} \gloss{\In} \defn{manual}
	\a<uo>\fw{fraþru} \phnm{fraθ\pstrs ru} \gloss{\An} vs. \fw{fraþro} \phnm{fraθ\pstrs ro} \gloss{\In} \defn{observant}
\xe

Suffixation also occurs regularly in derivational inflection. In fact, several derivational suffixes can be strung together to derive yet more words. Example~\getref{ex:tvk-derivation-sfxs} shows this process.

\pex<ex:tvk-derivation-sfxs>
	\a<root1>\fw{frali} \phnm{\pstrs fra.li} \defn{to see}
	\a<der11>\fw{fravem} \phnm{fra\pstrs vem} \defn{sight}
	\a<der12>\fw{fravemitla} \fw{-o} \phnm{fra.vem.it\pstrs la} \defn{visual}
	\a<root2>\fw{onaš} \phnm{o\pstrs naʃ} \defn{rug}
	\a<der21>\fw{onašuli} \phnm{o.na\pstrs ʃu.li} \defn{to place}
	\a<der22>\fw{onašinsuli} \phnm{o.na.ʃin\pstrs su.li} \defn{to re-place}
\xe

In example~\getfullref{ex:tvk-derivation-sfxs.der22}, the \fw{-ins} affix may not immediately appear to be a suffix, however it should be noted that it is being attached to the end of the \emph{stem} of the word, which is \fw{onaš-}, prior to the verb's infinitive ending \fw{-uli}, which is an \emph{inflectional} suffix.

\index{morphological typology!processes!suffixation|)}

\subsection{Cliticization}
\label{subsec:tvk-cliticization}
\index{morphological typology!processes!cliticization|(}

Clitics can be difficult to define in a formal way, and it is therefore worthwhile to explain how certain morphemes in \langtvk{} can be classified as such.

A \enquote*{clitic} is often characterized as \enquote{a \enquote{small}, prosodically weak, or non-prominent word which fails to respect normal principles of syntactic distribution because it requires a host to which it can attach phonologically}\autocite{spencer-luis-2012}. Clitics are different from affixes in that they will typically \enquote{cliticize \enquote{promiscuously} to a word of any old category, including uninflectable words which otherwise fail to take any affixes whatever}\autocite{spencer-luis-2012}, whereas affixes are limited to only specific parts of speech to which they can connect\autocite{Zwicky-and-pullum-1983}. Yet, they are different from function words in that they are bound, that is they do not have the free ordering afforded to words\autocite{Zwicky-1985}.

The primary example of clitics in \langtvk{} is the noun prepositions. These particles cannot appear alone, conveying solely grammatical, not lexical, information. They are not affixes because they attach to the beginning of the entire noun phrase, no matter what word comes after, rather than attaching directly to the head noun.

\pex<ex:tvk-noun-clitics>
	\a<simple>\begingl
		\glpreamble\fw{Mod nas oko fra.}\\
		\phnm{\pstrs mod nas o\pstrs ko \pstrs fra}//
		\gla mod nas= oko fr-a//
		\glb \Fps.\Erg{} \An.\Pc.\Top{} dog see-\Ind.\Npst.\Ipfv//
		\glft \defn{I see the dogs.}//
	\endgl
	\a<adjective>\begingl
		\glpreamble\fw{Mod nas urda oko fra.}\\
		\phnm{\pstrs mod nas ur\pstrs da o\pstrs ko \pstrs fra}//
		\gla mod nas= urd-a oko fr-a//
		\glb \Fps.\Erg{} \An.\Pc.\Top{} protected-\An{} dog see-\Ind.\Npst.\Ipfv//
		\glft \defn{I see the protected dogs.}//
	\endgl
	\a<possessedpro>\begingl
		\glpreamble\fw{Mod nas tesar urda oko fra.}\\
		\phnm{\pstrs mod nas te\pstrs sar ur\pstrs da o\pstrs ko \pstrs fra}//
		\gla mod nas= tesar urd-a oko fr-a//
		\glb \Fps.\Erg{} \An.\Pc.\Top{} \Spc.\Gen{} protected-\An{} dog see-\Ind.\Npst.\Ipfv//
		\glft \defn{I see your protected dogs.}//
	\endgl
	\a<possessednoun>\begingl
		\glpreamble\fw{Mod nas su esokon urda oko fra.}\\
		\phnm{\pstrs mod nas su e.so\pstrs kon ur\pstrs da o\pstrs ko \pstrs fra}//
		\gla mod nas= su= esokon urd-a oko fr-a//
		\glb \Fps.\Erg{} \An.\Pc.\Top{} \An.\Sg.\Gen{} farmer protected-\An{} dog see-\Ind.\Npst.\Ipfv//
		\glft \defn{I see the farmer's protected dogs.}//
	\endgl
\xe

Notice in example~\getref{ex:tvk-noun-clitics} how the particle \fw{nas} directly precedes the entire noun phrase, even when separated from the head noun by an adjective (\getfullref{ex:tvk-noun-clitics.adjective}), a pronoun (\getfullref{ex:tvk-noun-clitics.possessedpro}), and even another modifying noun and its preposition (\getfullref{ex:tvk-noun-clitics.possessednoun}).

In some cases, the noun prepositions reduce phonologically and attach to the following word. Any time a noun preposition ends with the same vowel with which the following word begins, that vowel is dropped and the preposition is attached orthographically to the following word with an apostrophe.

\pex<ex:tvk-clitic-reduction>
	\a<le>\fw{le eðer} → \fw{l'eðer} \phnm{le\pstrs ðer} \defn{pens} \gloss{\In.\Pc.\Abs-pen}
	\a<mati>\fw{mati inam} → \fw{mat'inam} \phnm{ma.ti\pstrs nam} \defn{location} \gloss{\In.\Sg.\Top.\Acc-location}
	\a<no>\fw{no oko} → \fw{n'oko} \phnm{no\pstrs ko} \defn{dog} \gloss{\An.\Sg.\Top-pen}
	\a<su>\fw{su urda ablu} → \fw{s'urda ablu} \phnm{sur\pstrs da ab\pstrs lu} \defn{of the protected cat} \gloss{\An.\Sg.\Gen-protected-\An{} cat}
\xe

The other main example of cliticization is the particle \fw{vi}. It is used to ask questions and is most often added at the end of a sentence after the verb, as shown in example~\getref{ex:tvk-clitic-question}.

\ex<ex:tvk-clitic-question>
	\begingl
		\glpreamble\fw{No šekon tu fraþru oko usu vi?}\\
		\phnm{no ʃe\pstrs kon tu fraθ\pstrs ru o\pstrs ko u\pstrs su vi}//
		\gla no= šekon tu= fraþr-u oko us-u =vi//
		\glb \An.\Sg.\Top{}= runner \An.\Sg.\Acc{}= observant-\An{} dog have-\Ind.\Npst.\Ipfv{} =\Int//
		\glft \defn{Does the runner have an observant dog?}//
	\endgl
\xe

A speaker can, however, move the interrogative particle earlier in the sentence to focus the question on some specific element.

\pex<ex:tvk-clitic-question-focus>
	\a<erg>\begingl
		\glpreamble\fw{No šekon vi tu fraþru oko usu?}\\
		\phnm{no ʃe\pstrs kon vi tu fraθ\pstrs ru o\pstrs ko u\pstrs su}//
		\gla no= šekon =vi tu= fraþr-u oko us-u//
		\glb \An.\Sg.\Top{}= runner =\Int{} \An.\Sg.\Acc{}= observant-\An{} dog have-\Ind.\Npst.\Ipfv//
		\glft \defn{Is it the runner who has an observant dog?}//
	\endgl
	\a<acc>\begingl
		\glpreamble\fw{No šekon tu fraþru vi oko usu?}\\
		\phnm{no ʃe\pstrs kon tu fraθ\pstrs ru vi o\pstrs ko u\pstrs su}//
		\gla no= šekon tu= fraþr-u =vi oko us-u//
		\glb \An.\Sg.\Top{}= runner \An.\Sg.\Acc{}= observant-\An{} =\Int{} dog have-\Ind.\Npst.\Ipfv//
		\glft \defn{Is it an \emph{observant} dog the runner has?}//
	\endgl
	\a<adj>\begingl
		\glpreamble\fw{No šekon tu fraþru oko vi usu?}\\
		\phnm{no ʃe\pstrs kon tu fraθ\pstrs ru o\pstrs ko vi u\pstrs su}//
		\gla no= šekon tu= fraþr-u oko =vi us-u//
		\glb \An.\Sg.\Top{}= runner \An.\Sg.\Acc{}= observant-\An{} dog =\Int{} have-\Ind.\Npst.\Ipfv//
		\glft \defn{Is it an observant \emph{dog} the runner has?}//
	\endgl
\xe

\index{morphological typology!processes!cliticization|)}
\index{morphological typology!processes|)}

\section{Locus of Marking}
\label{sec:tvk-locus}
\index{morphological typology!locus of marking|(}

\langtvk{} is almost exclusively dependent marking.

\index{morphological typology!locus of marking|)}
