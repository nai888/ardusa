\chapter{History and Ethnography}
\label{cha:tvk-ethnography}

This chapter will present a brief history of the \langtvk{} language family, followed by a short description of its ethnolinguistic context.

\section{Brief History}
\label{sec:tvk-history}

The \peoptvk{} (the \langtvk{} people) migrated to \landn{} hundreds of years ago in what they termed Year 1 of the \landadj{} Era (\acrnm{ae}). \landn{} is far from any other landmasses and is isolated from the influence of other lands and other peoples. The \peoptvk{} landed in the warm southeastern regions of \landn{} where they first established their new home, naming this new realm \scr{Urdeso} \fw{Urdeso}, a compound word meaning \defn{Safe Land}. Over the following centuries, the \peoptvk{} spread westward and northward throughout the whole of \landn.

As the \peoptvk{} spread, they formed several individual territories, each of which eventually developed into small kingdoms. These kingdoms constantly battled one another for power, and borders were continually shifting. Those who fled the fighting fled northward, furthering the \langtvk{} expansion throughout \landn. As the \peoptvk{} spread farther apart and splintered, their language diverged. Two main dialects emerged, one in the north and one in the south.

After a few hundred years, one kingdom in the south emerged as dominant, conquering or allying with more and more kingdoms until, by 327~\acrnm{ae}, the entire south of \landn{} was united under one empire. This empire enforced the usage of the language that had emerged in the south, thus forming the \langank{} language. The empire continued to push northward until it spread too thin and reached a stalemate with the allied kingdoms in the north around 371~\acrnm{ae}. Finally, in 582~\acrnm{ae} after a couple hundred years of relatively stable rule, the empire declined and divided again into individual territories, leaving behind six sovereign kingdoms.

While the empire was emerging in the south, the kingdoms in the north formed a loose alliance to resist its spread. The alliance managed to reach a stalemate with the empire, stopping its spread northward. The allied kingdoms together maintained the language that emerged in the north, thus forming the \langrdk{} language. Eventually, as the empire split in 582~\acrnm{ae} and the northern alliance was no longer needed, the north also split into individual territories, leaving behind four sovereign kingdoms.

\section{Ethnography}
\label{sec:tvk-ethnography}

\subsection{Demonyms and Language Names}
\label{subsec:tvk-demonyms}

The \peoptvk{} were a tribe that migrated to \landn{} together, fleeing their previous home. The \langtvk{} word \scr{tavo} \fw{tavo} \phnm{ta\pstrs vo} means \defn{person}, and so the derived word \scr{\npeoptvk} \fw{\npeoptvk} \phnm{ta.vo\pstrs taθ} means \defn{people} or \defn{tribe}. In other words, the \peoptvk{} referred to themselves as the People, with \scr{\nlangtvk} \fw{\nlangtvk} being the Language of the People. The \langank- and \langrdk-derived words, \scr{Tevodeþ} \fw{Tevodeþ} \phnm{te.vo\pstrs deθ} and \scr{Tovujiþ} \fw{Tovujiþ} \phnm{to.vu\pstrs \affr{d}{ʒ}iθ} respectively, refer to all people who descended from the original \peoptvk{} tribe. Both \langank{} and \langrdk{} are \peoptvk{} languages and part of the \langtvk{} language family.

\subsection{Ethnology}
\label{subsec:tvk-ethnology}

Here will be a brief ethnological description of the \peoptvk.

\subsection{Demography}
\label{subsec:tvk-demography}

Here will be a brief demographical description of the \peoptvk.