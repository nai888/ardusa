\chapter{Grammatical Categories}
\label{cha:tvk-grammatical-categories}

\langtvk{} words can be divided into several different categories, or parts of speech. While the previous chapter dealt with the general mechanisms of marking words, this chapter will examine each of the various parts of speech in order to define their morphology more closely. The discussion will begin with an examination of nouns, pronouns, and verbs. Following this will be a discussion of the remaining parts of speech, including adverbs, numerals, and conjunctions.

\section{Nouns}
\label{sec:tvk-nouns}

Nouns in \langtvk{} decline to express number and gender (animacy) and are marked for case to indicate their grammatical role within the clause. As discussed in \autoref{cha:tvk-morphological-typology}, this inflection takes place not directly on the noun itself but on prepositional clitics that convey this grammatical meaning\autocite{wals-51}. For a full illustration of the declension paradigms, compare \autoref{tab:tvk-an-vowel-decl} and \autoref{tab:tvk-in-vowel-decl}. As shown in these tables, \langtvk{} noun inflections are never syncretic\autocite{wals-28}.

\afterpage{\clearpage
	\begin{table}\centering
		\caption[\langtvk{} Animate Noun Declension Paradigm]{\langtvk{} Animate Noun Declension Paradigm for the word \fw{bruþa} \defn{hand} or \defn{tool}}
		\label{tab:tvk-an-vowel-decl}
		\begin{tabu}{| l | l l l |}
			\toprule
			\rowfont[c]\bfseries & \Sg & \Pc & \Pl\\
			\midrule
			\textbf{\Abs} & \fw{bruþa} & \fw{ri bruþa} & \fw{ran bruþa}\\
			\textbf{\Erg} & \fw{do bruþa} & \fw{das bruþa} & \fw{din bruþa}\\
			\textbf{\Acc} & \fw{tu bruþa} & \fw{tos bruþa} & \fw{ton bruþa}\\
			\textbf{\Dat} & \fw{ke bruþa} & \fw{kas bruþa} & \fw{ken bruþa}\\
			\textbf{\Gen} & \fw{su bruþa} & \fw{sar bruþa} & \fw{san bruþa}\\
			\midrule
			\textbf{\Top} & \fw{no bruþa} & \fw{nas bruþa} & \fw{nan bruþa}\\
			\textbf{\Top.\Acc} & \fw{nut bruþa} & \fw{nutos bruþa} & \fw{nuton bruþa}\\
			\textbf{\Top.\Dat} & \fw{nek bruþa} & \fw{nekas bruþa} & \fw{naken bruþa}\\
			\textbf{\Top.\Gen} & \fw{nus bruþa} & \fw{nosar bruþa} & \fw{nosan bruþa}\\
			\bottomrule
		\end{tabu}
	\end{table}
	\begin{table}\centering
		\caption[\langtvk{} Inanimate Noun Declension Paradigm]{\langtvk{} Inanimate Noun Declension Paradigm for the word \fw{šem} \defn{busyness}}
		\label{tab:tvk-in-vowel-decl}
		\begin{tabu}{| l | l l l |}
			\toprule
			\rowfont[c]\bfseries & \Sg & \Pc & \Pl\\
			\midrule
			\textbf{\Abs} & \fw{šem} & \fw{le šem} & \fw{ren šem}\\
			\textbf{\Erg} & \fw{ða šem} & \fw{ðes šem} & \fw{dun šem}\\
			\textbf{\Acc} & \fw{ti šem} & \fw{þis šem} & \fw{ten šem}\\
			\textbf{\Dat} & \fw{ǩo šem} & \fw{kos šem} & \fw{ǩun šem}\\
			\textbf{\Gen} & \fw{šo šem} & \fw{se šem} & \fw{šen šem}\\
			\midrule
			\textbf{\Top} & \fw{mi šem} & \fw{mes šem} & \fw{nun šem}\\
			\textbf{\Top.\Acc} & \fw{mati šem} & \fw{moþes šem} & \fw{noten šem}\\
			\textbf{\Top.\Dat} & \fw{moǩ šem} & \fw{mekos šem} & \fw{nikun šem}\\
			\textbf{\Top.\Gen} & \fw{miš šem} & \fw{mise šem} & \fw{nušen šem}\\
			\bottomrule
		\end{tabu}
	\end{table}
}

\subsection{Gender}
\label{subsec:tvk-nouns-gender}

Grammatical gender in \langtvk{} consists of two\autocite{wals-30} non-sex-based\autocite{wals-31} classes based primarily on semantic ontological properties\autocite{wals-32}. The animate gender refers primarily to entities that are considered alive or are associated with life, movement, change, or dynamism. The inanimate gender refers primarily to entities that are not alive and are generally stationary or abstract. Grammatical gender in \langtvk{} can also be referred to as \enquote{animacy} since that is what the genders denote. Examples of nouns in each gender can be seen in example~\getref{ex:tvk-noun-genders}.

\pex<ex:tvk-noun-genders>
	\a<an>Animate nouns:\\
		\fw{botra} \defn{woman}, \fw{ǩalo} \defn{man}, \fw{eson} \defn{farmer}, \fw{okotik} \defn{puppy}, \fw{urdatil} \defn{ward}, \fw{bilt} \defn{breath}
	\a<in>Inanimate nouns:\\
		\fw{esotik} \defn{country}, \fw{dedu} \defn{sky}, \fw{elbi} \defn{egg}, \fw{usudir} \defn{basket}, \fw{akrapis} \defn{letter}, \fw{fradir} \defn{glasses}
\xe

Since the nouns themselves are not directly inflected, with grammatical information instead shown on prepositional particles, it is impossible to tell what gender a noun is based solely on its word form.

Some nouns are able to change category in certain circumstances. For example, plants and animals switch from the animate gender to the inanimate gender when they serve as food. Further, there exist some duplicates with otherwise identical words declining to opposite genders.

\subsection{Number}
\label{subsec:tvk-nouns-number}

Grammatical number in \langtvk{} consists of three numbers, all of which are coded on the noun prepositions\autocite{wals-33}. The singular is always used when there is only one of the referent noun, the paucal is used when there are two to five of the referent noun, and the plural is used when there are more than five of the referent noun.

\pex<ex:tvk-noun-numbers>
	\a<sg> \fw{su ima} \phnm{su i\pstrs ma} \defn{of mother} \gloss{\Sg.\An.\Gen= mother}
	\a<pc> \fw{sar ima} \phnm{sar i\pstrs ma} \defn{of (some) mothers} \gloss{\Pc.\An.\Gen= mother}
	\a<pl> \fw{san ima} \phnm{san i\pstrs ma} \defn{of (several) mothers} \gloss{\Pl.\An.\Gen= mother}
\xe

When a numeral is used to identify the number of a referent noun, the singular is used instead of the paucal or plural, even if without the numeral the other forms would be used\autocite{wals-34}.

\pex<ex:tvk-noun-numbers-numerals>
	\a<sg> \fw{k'eþ ima} \phnm{keθ i\pstrs ma} \defn{to one mother} \gloss{\Sg.\An.\Dat=one mother}
	\a<pc> \fw{ke arsi ima} \phnm{ke ar\pstrs si i\pstrs ma} \defn{to three mothers} \gloss{\Sg.\An.\Dat= three mother}, not \ungr{\fw{kas arsi ima}}
	\a<pl> \fw{ke bruð abom ima} \phnm{ke bruð a\pstrs bom i\pstrs ma} \defn{to seven mothers} \gloss{\Sg.\An.\Dat= five two mother}, not \ungr{\fw{ken bruð abom ima}}
\xe

Most nouns that represent concrete entities are countable, including some words that in English are uncountable like corn, and by default they are used in the singular form unlike English words like pants or glasses. However, many entities that are not easily split into discreet parts like liquids, grains, and certain abstract concepts are uncountable, such as \fw{elto} \phnm{el\pstrs to} \defn{water}. Occasionally, when a word's semantics cover multiple concepts, a word can be variably countable or uncountable; when \fw{dedu} \phnm{de\pstrs du} is used to mean \defn{sky} or \defn{heaven}, it is uncountable, but when it is used to mean \defn{ceiling}, it is countable and can be made paucal or plural.

People's names can also be declined to the paucal or plural number to indicate the associative plural\autocite{wals-36}. This form is used to refer to a person and the other people associated with that person. For example, \fw{ri Bol} \phnm{ri bol} \gloss{\Pc.\An.\Abs{} Bol} refers to Bol and two to five other people associated with him. Similarly, \fw{ran Ote} \phnm{ran o\pstrs te} \gloss{\Pl.\An.\Abs{} Ote} refers to Ote and the group he is with.

\subsection{Case}
\label{subsec:tvk-nouns-case}

As shown in Tables \ref{tab:tvk-an-vowel-decl} and \ref{tab:tvk-in-vowel-decl}, \langtvk{} noun phrases decline to five different grammatical cases\autocite{wals-49} in order to show their role in the sentence. These cases are governed by the phrase's verb or assigned to adjuncts depending on their purpose or meaning. As shown in the same declension tables, any of these grammatical cases can be replaced by or combined with topic markers. See \autoref{subsec:tvk-nouns-topicality} for more information on topicality.

\subsubsection{Absolutive and Intransitive}
\label{subsubsec:tvk-nouns-absolutive}

The intransitive case marks a noun or noun phrase that serves as the subject of an intransitive verb like \fw{šeli} \defn{to run} or a transitive verb used intransitively like \fw{ufuli} \defn{to sing} (without naming the object, what is being sung). This means that when a verb has only a single argument, that argument will by default be in the intransitive case. That is true whether the subject is serving like an agent as in words like \fw{šeli} \defn{to run} or \fw{ufuli} \defn{to sing} or when the subject is serving more like a patient as in words like \fw{orðali} \defn{to fall}.

\pex<ex:tvk-noun-abs>
	\a<a1>\begingl
		\glpreamble\fw{Mollur šeþ.}\\
		\phnm{mo\pstrs\gem{l}ur \pstrs ʃeθ}//
		\gla ∅= Mollur š-eþ//
		\glb \An.\Sg.\Intr= Mollur run-\Ind.\Npst.\Prg//
		\glft\defn{Mollur is running.}//
	\endgl
	\a<a2>\begingl
		\glpreamble\fw{R'ima ufu.}\\
		\phnm{ri\pstrs ma u\pstrs fu}//
		\gla ri=ima uf-u//
		\glb \An.\Pc.\Intr=mother sing-\Ind.\Npst.\Ipfv//
		\glft\defn{The mothers sing.}//
	\endgl
	\a<p>\begingl
		\glpreamble\fw{Ren fild orðak.}\\
		\phnm{ren \pstrs fild or\pstrs ðak}//
		\gla ren= fild orð-ak//
		\glb \In.\Pl.\Intr= doll fall-\Ind.\Pst.\Pfv//
		\glft\defn{The dolls fell.}//
	\endgl
\xe

Note that the singular intransitive case is entirely unmarked by any preposition. This is true whether the noun is animate or inanimate.

\pex<ex:tvk-noun-abs-sg>
	\a<an>\begingl
		\glpreamble\fw{Alum uldeteš.}\\
		\phnm{a\pstrs lum ul.de\pstrs teʃ}//
		\gla ∅= alum uldet-eš//
		\glb \An.\Sg.\Intr= cloud change-\Ind.\Npst.\Rtsp//
		\glft\defn{The cloud has changed.}//
	\endgl
	\a<in>\begingl
		\glpreamble\fw{Almaþ uldeteš.}\\
		\phnm{al\pstrs maθ ul.de\pstrs teʃ}//
		\gla ∅= almaþ uldet-eš//
		\glb \In.\Sg.\Intr= village change-\Ind.\Npst.\Rtsp//
		\glft\defn{The village has changed.}//
	\endgl
\xe

However, the subject of certain transitive verbs will also take the intransitive case if the semantic meaning of the verb is stative. See \autoref{subsubsec:tvk-nouns-dative} Dative for more information on this. Since it is used in these situations, and since the intransitive is the citation form, the case is normally referred to as the absolutive case, even when used intransitively. These terms are interchangeable.

\ex<ex:tvk-noun-abs-trans>
	\begingl
		\glpreamble\fw{Ter ke arb fra vi?}\\
		\phnm{ter ke arb \pstrs fra vi}//
		\gla ter ke= arb fr-a =vi//
		\glb \Sps.\Abs{} \An.\Sg.\Dat= bird see-\Ind.\Npst.\Ipfv{} =\Int//
		\glft\defn{Do you see a bird?}//
	\endgl
\xe

The absolutive case is frequently used with postpositions to indicate a location where or through which an action is taken, for example being placed at, on, or in something.

\pex<ex:tvk-noun-abs-pp>
	\a<on>\begingl
		\glpreamble\fw{Ablu onaš e onek.}\\
		\phnm{ab\pstrs lu o\pstrs naʃ e o\pstrs nek}//
		\gla ∅= ablu ∅= onaš e on-ek//
		\glb \An.\Sg.\Abs= cat \In.\Sg.\Abs= rug on play-\Ind.\Pst.\Pfv//
		\glft\defn{The cat played on the rug.}//
	\endgl
	\a<over>\begingl
		\glpreamble\fw{Mod ti ennis l'elbi arku ǧirak.}\\
		\phnm{mod ti e\pstrs\gem{n}is lel\pstrs bi ar\pstrs ku ɣi\pstrs rak}//
		\gla mod ti= ennis le=elbi arku ǧir-ak//
		\glb \Fps.\Erg{} \In.\Sg.\Acc= ball \In.\Pc.\Abs=egg above throw-\Ind.\Pst.\Pfv//
		\glft\defn{I threw the ball over the eggs.}//
	\endgl
\xe

When an action is done \defn{with} or \defn{without} a noun, the absolutive case will be used.

\ex<ex:tvk-noun-abs-with>
	\begingl
		\glpreamble\fw{Oko ablu mo oneþ.}\\
		\phnm{o\pstrs ko ab\pstrs lu mo o\pstrs neθ}//
		\gla ∅= oko ∅= ablu mo on-eþ//
		\glb \An.\Sg.\Abs= dog \An.\Sg.\Abs= cat with play-\Ind.\Npst.\Prg//
		\glft\defn{The dog is playing with the cat.}//
	\endgl
\xe

The absolutive case is also used when directly addressing someone in a vocative function. The noun functioning in this way is often placed at the beginning or end of the sentence separated by a pause in speech or a comma in writing.

\pex<ex:tvk-noun-abs-voc>
	\a<imp>\begingl
		\glpreamble\fw{Lerk, šebanta.}\\
		\phnm{\pstrs lerk, ʃe\pstrs ban.ta}//
		\gla ∅= Lerk šeb-anta//
		\glb \An.\Sg.\Abs= Lerk run-\Imp//
		\glft\defn{Run, Lerk.}//
	\endgl
	\a<nimp>\begingl
		\glpreamble\fw{Sud tu tavotik urdateþ, Erme.}\\
		\phnm{sud tu ta.vo\pstrs tik ur.da\pstrs teθ er\pstrs me}//
		\gla sud tu= tavotik urdat-eþ ∅= Erme//
		\glb \Tps.\An.\Erg{} \An.\Sg.\Acc= child guard-\Ind.\Npst.\Prg{} \An.\Sg.\Abs= Erme//
		\glft\defn{He is guarding the child, Erme.}//
	\endgl
\xe

\subsubsection{Ergative}
\label{subsubsec:tvk-nouns-ergative}

The ergative case marks a noun or noun phrase that serves as the subject of an active transitive verb or any ditransitive verb. This means that when a verb has multiple arguments and the semantic meaning of the verb is active, the subject argument will by default by in the ergative case.

\pex<ex:tvk-noun-erg>
	\a<eðerali>\begingl
		\glpreamble\fw{Do Tlunda ti akrapis eðeraš.}\\
		\phnm{do tlun\pstrs da ti ak.ra\pstrs pis e.ðe\pstrs raʃ}//
		\gla do= Tlunda ti= akrapis eðer-aš//
		\glb \An.\Sg.\Erg= Tlunda \In.\Sg.\Acc= letter pen-\Ind.\Npst.\Rtsp//
		\glft\defn{Tlunda has penned a letter.}//
	\endgl
	\a<okotali>\begingl
		\glpreamble\fw{Das oko tu ablu okotam.}\\
		\phnm{das o\pstrs ko tu ab\pstrs lu o.ko\pstrs tam}//
		\gla das= oko tu= ablu okot-am//
		\glb \An.\Pc.\Erg= dog \An.\Sg.\Acc= cat chase-\Ind.\Pst.\Ipfv//
		\glft\defn{The dogs chased the cat.}//
	\endgl
	\a<visali>\begingl
		\glpreamble\fw{Din avo ten usudir visaǧ.}\\
		\phnm{din a\pstrs vo ten u.su\pstrs dir vi\pstrs saɣ}//
		\gla din= avo ten= usudir vis-aǧ//
		\glb \An.\Pl.\Erg= father \In.\Pl.\Acc= basket take.away-\Ind.\Pst.\Rtsp//
		\glft\defn{The father and his associates had taken away the baskets.}//
	\endgl
\xe

\subsubsection{Accusative}
\label{subsubsec:tvk-nouns-accusative}

The accusative case marks a noun or noun phrase that serves as the direct object of an active transitive verb or any ditransitive verb.

\pex<ex:tvk-noun-acc>
	\a<onašuli>\begingl
		\glpreamble\fw{Do akrakon þis eðerik alma e onašuk.}\\
		\phnm{do ak.ra\pstrs kon θis e.ðe\pstrs rik al\pstrs ma e o.na\pstrs ʃuk}//
		\gla do= akrakon þis= eðerik alma e onaš-uk//
		\glb \An.\Sg.\Erg= writer \In.\Pc.\Acc= pencil house in place-\Ind.\Pst.\Pfv//
		\glft\defn{The writer placed the pencils in the house.}//
	\endgl
	\a<uldetuli>\begingl
		\glpreamble\fw{Do šus botra ti šus akrapis uldetuk.}\\
		\phnm{do ʃus bot\pstrs ra ti ʃus ak.ra\pstrs pis ul.de\pstrs tuk}//
		\gla do= šus botra ti= šus akrapis uldet-uk//
		\glb \An.\Sg.\Erg= \Tpp.\An.\Gen{} wife \In.\Sg.\Acc= \Tpp.\An.\Gen{} letter change-\Ind.\Pst.\Pfv//
		\glft\defn{His wife changed his letter.}//
	\endgl
\xe

\subsubsection{Dative}
\label{subsubsec:tvk-nouns-dative}

The dative case marks a noun or noun phrase that serves as the indirect object of a ditransitive verb, a recipient of an action, or the entity for whose benefit or detriment the action is taken.

\ex<ex:tvk-noun-dat>
	\begingl
		\glpreamble\fw{Do eson tu ennis ke oko draš.}\\
		\phnm{do e\pstrs son tu e\pstrs\gem{n}is ke o\pstrs ko \pstrs draʃ}//
		\gla do= eson tu= ennis ke= oko dr-aš//
		\glb \An.\Sg.\Erg= farmer \In.\Sg.\Acc= ball \An.\Sg.\Dat= dog give-\Ind.\Npst.\Rtsp//
		\glft\defn{The farmer has given the dog a ball.}//
	\endgl
\xe

Certain monotransitive verbs are used with the absolutive and dative cases instead of the ergative and accusative cases. These tend to be stative verbs in which the object of the verb is unaffected by the action or there is little volition on the part of the subject.

\pex<ex:tvk-noun-dat-trans>
	\a<teguli>\begingl
		\glpreamble\fw{Mor tek tegu.}\\
		\phnm{mor tek te\pstrs gu}//
		\gla mor tek teg-u//
		\glb \Fps.\Abs{} \Sps.\Dat{} worry-\Ind.\Npst.\Ipfv//
		\glft\defn{I worry for you.}//
	\endgl
	\a<keðali>\begingl
		\glpreamble\fw{Ran urdaton ken ufukon keðam.}\\
		\phnm{ran ur.da\pstrs ton ken u.fu\pstrs kon ke\pstrs ðam}//
		\gla ran= urdaton ton= ufukon keð-am//
		\glb \An.\Pl.\Abs= guard \An.\Pl.\Dat= singer admire-\Ind.\Pst.\Ipfv//
		\glft\defn{The guards admired the singers.}//
	\endgl
\xe

When a non-ditransitive verb is done on behalf of or for someone or something, the beneficiary of that action will be declined to the dative and followed by the postposition \fw{li} \phnm{li} \defn{for}.

\subsubsection{Genitive}
\label{subsubsec:tvk-nouns-genitive}

\subsection{Topicality}
\label{subsec:tvk-nouns-topicality}

How does topicality work?

\section{Pronouns}
\label{sec:tvk-pronouns}

\langtvk{} pronouns are symmetrical to other noun phrases\autocite{wals-50}.
