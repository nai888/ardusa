\chapter{Grammatical Categories}
\label{cha:tvk-grammatical-categories}

\langtvk{} words can be divided into several different categories, or parts of speech. While the previous chapter dealt with the general mechanisms of marking words, this chapter will examine each of the various parts of speech in order to define their morphology more closely. The discussion will begin with an examination of nouns, pronouns, and verbs. Following this will be a discussion of the remaining parts of speech, including adverbs, numerals, and conjunctions.

\section{Nouns}
\label{sec:tvk-nouns}

Nouns in \langtvk{} decline to express number and gender (animacy) and are marked for case to indicate their grammatical role within the clause. As discussed in \autoref{cha:tvk-morphological-typology}, this inflection takes place not directly on the noun itself but on prepositional clitics that convey this grammatical meaning. For a full illustration of the declension paradigms, compare.

\afterpage{\clearpage\begin{table}
	\caption{\langtvk{} Animate Noun Declension Paradigm}
	\label{tab:tvk-an-vowel-decl}
	\begin{tabu}{| r | c c c |}
		\toprule
		\textbf{Case} & \textbf{\Sg} & \textbf{\Pc} & \textbf{\Pl}\\
		\midrule
		\textbf{\Abs} & - & ri & ran\\
		\bottomrule
	\end{tabu}\end{table}
}
