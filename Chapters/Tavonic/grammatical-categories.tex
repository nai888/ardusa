\chapter{Grammatical Categories}
\label{cha:tvk-grammatical-categories}

\langtvk{} words can be divided into several different categories, or parts of speech. While the previous chapter dealt with the general mechanisms of marking words, this chapter will examine each of the various parts of speech in order to define their morphology more closely. The discussion will begin with an examination of nouns, pronouns, and verbs. Following this will be a discussion of the remaining parts of speech, including adverbs, numerals, and conjunctions.

\section{Nouns}
\label{sec:tvk-nouns}

Nouns in \langtvk{} decline to express number and gender (animacy) and are marked for case to indicate their grammatical role within the clause. As discussed in \autoref{cha:tvk-morphological-typology}, this inflection takes place not directly on the noun itself but on prepositional clitics that convey this grammatical meaning. For a full illustration of the declension paradigms, compare \autoref{tab:tvk-an-vowel-decl} and \autoref{tab:tvk-in-vowel-decl}. As shown in these tables, \langtvk{} noun inflections are never syncretic\autocite{wals-28}.

\afterpage{\clearpage
	\begin{table}\centering
		\caption[\langtvk{} Animate Noun Declension Paradigm]{\langtvk{} Animate Noun Declension Paradigm for the word \fw{ima} \defn{mother}}
		\label{tab:tvk-an-vowel-decl}
		\begin{tabu}{| l | l l l |}
			\toprule
			\rowfont[c]\bfseries & \Sg & \Pc & \Pl\\
			\midrule
			\textbf{\Abs} & \fw{ima} & \fw{r'ima} & \fw{ran ima}\\
			\textbf{\Erg} & \fw{do ima} & \fw{das ima} & \fw{din ima}\\
			\textbf{\Acc} & \fw{tu ima} & \fw{tos ima} & \fw{ton ima}\\
			\textbf{\Dat} & \fw{ke ima} & \fw{kas ima} & \fw{ken ima}\\
			\textbf{\Gen} & \fw{su ima} & \fw{sar ima} & \fw{san ima}\\
			\midrule
			\textbf{\Top} & \fw{no ima} & \fw{nas ima} & \fw{nan ima}\\
			\textbf{\Top.\Acc} & \fw{nut ima} & \fw{nutos ima} & \fw{nuton ima}\\
			\textbf{\Top.\Dat} & \fw{nek ima} & \fw{nekas ima} & \fw{naken ima}\\
			\textbf{\Top.\Gen} & \fw{nus ima} & \fw{nosar ima} & \fw{nosan ima}\\
			\bottomrule
		\end{tabu}
	\end{table}
	\begin{table}\centering
		\caption[\langtvk{} Inanimate Noun Declension Paradigm]{\langtvk{} Inanimate Noun Declension Paradigm for the word \fw{ongo} \defn{pan}}
		\label{tab:tvk-in-vowel-decl}
		\begin{tabu}{| l | l l l |}
			\toprule
			\rowfont[c]\bfseries & \Sg & \Pc & \Pl\\
			\midrule
			\textbf{\Abs} & \fw{ongo} & \fw{le ongo} & \fw{ren ongo}\\
			\textbf{\Erg} & \fw{ða ongo} & \fw{ðes ongo} & \fw{dun ongo}\\
			\textbf{\Acc} & \fw{ti ongo} & \fw{þis ongo} & \fw{ten ongo}\\
			\textbf{\Dat} & \fw{ǩ'ongo} & \fw{kos ongo} & \fw{ǩun ongo}\\
			\textbf{\Gen} & \fw{š'ongo} & \fw{se ongo} & \fw{šen ongo}\\
			\midrule
			\textbf{\Top} & \fw{mi ongo} & \fw{mes ongo} & \fw{nun ongo}\\
			\textbf{\Top.\Acc} & \fw{mati ongo} & \fw{moþes ongo} & \fw{noten ongo}\\
			\textbf{\Top.\Dat} & \fw{moǩ ongo} & \fw{mekos ongo} & \fw{nikun ongo}\\
			\textbf{\Top.\Gen} & \fw{miš ongo} & \fw{mise ongo} & \fw{nušen ongo}\\
			\bottomrule
		\end{tabu}
	\end{table}
}

\subsection{Gender}
\label{subsec:tvk-nouns-gender}

Grammatical gender in \langtvk{} consists of two\autocite{wals-30} non-sex-based\autocite{wals-31} classes based primarily on semantic ontological properties\autocite{wals-32}. The animate gender refers primarily to entities that are considered alive or are associated with life, movement, change, or dynamism. The inanimate gender refers primarily to entities that are not alive and are generally stationary or abstract. Grammatical gender in \langtvk{} can also be referred to as \enquote{animacy} since that is what the genders denote. Examples of nouns in each gender can be seen in example~\getref{ex:tvk-noun-genders}.

\pex<ex:tvk-noun-genders>
	\a<an>Animate nouns:\\
		\fw{botra} \defn{woman}, \fw{ǩalo} \defn{man}, \fw{esokon} \defn{farmer}, \fw{okotik} \defn{puppy}, \fw{urdatil} \defn{ward}, \fw{bilt} \defn{breath}
	\a<in>Inanimate nouns:\\
		\fw{esotik} \defn{country}, \fw{ševem} \defn{busyness}, \fw{elbi} \defn{egg}, \fw{usudir} \defn{basket}, \fw{akrapis} \defn{letter}, \fw{fradir} \defn{glasses}
\xe

Since the nouns themselves are not directly inflected, with grammatical information instead shown on prepositional particles, it is impossible to tell what gender a noun is based solely on its word form.

Some nouns are able to change category in certain circumstances. For example, plants and animals switch from the animate gender to the inanimate gender when they serve as food. Further, there exist some duplicates with otherwise identical words declining to opposite genders.
