\chapter{Grammatical Categories}
\label{cha:tvk-grammatical-categories}

\langtvk{} words can be divided into several different categories, or parts of speech. While the previous chapter dealt with the general mechanisms of marking words, this chapter will examine each of the various parts of speech in order to define their morphology more closely. The discussion will begin with an examination of nouns, pronouns, and verbs. Following this will be a discussion of the remaining parts of speech, including adverbs, numerals, and conjunctions.

\section{Nouns}
\label{sec:tvk-nouns}

Nouns in \langtvk{} decline to express number and gender (animacy) and are marked for case to indicate their grammatical role within the clause. As discussed in \autoref{cha:tvk-morphological-typology}, this inflection takes place not directly on the noun itself but on prepositional clitics that convey this grammatical meaning\autocite{wals-51}. For a full illustration of the declension paradigms, compare \autoref{tab:tvk-an-noun-decl} and \autoref{tab:tvk-in-noun-decl}. As shown in these tables, \langtvk{} noun inflections are never syncretic\autocite{wals-28}.

\afterpage{\clearpage
	\begin{table}\centering
		\caption[\langtvk{} Animate Noun Declension Paradigm]{\langtvk{} Animate Noun Declension Paradigm for the word \scr{bruþa} \fw{bruþa} \defn{hand} or \defn{tool}}
		\label{tab:tvk-an-noun-decl}
		\begin{tabu}{| l | l l l |}
			\toprule
			\rowfont[c]\bfseries Case & \Sg & \Pc & \Pl\\
			\midrule
			\textbf{\Abs} & \fw{bruþa} & \fw{ri bruþa} & \fw{ran bruþa}\\
			\textbf{\Erg} & \fw{do bruþa} & \fw{das bruþa} & \fw{din bruþa}\\
			\textbf{\Acc} & \fw{tu bruþa} & \fw{tos bruþa} & \fw{ton bruþa}\\
			\textbf{\Dat} & \fw{ke bruþa} & \fw{kas bruþa} & \fw{ken bruþa}\\
			\textbf{\Gen} & \fw{su bruþa} & \fw{sar bruþa} & \fw{san bruþa}\\
			\midrule
			\textbf{\Top} & \fw{no bruþa} & \fw{nas bruþa} & \fw{nan bruþa}\\
			\textbf{\Top.\Acc} & \fw{nut bruþa} & \fw{nutos bruþa} & \fw{nuton bruþa}\\
			\textbf{\Top.\Dat} & \fw{nek bruþa} & \fw{nekas bruþa} & \fw{naken bruþa}\\
			\textbf{\Top.\Gen} & \fw{nus bruþa} & \fw{nosar bruþa} & \fw{nosan bruþa}\\
			\bottomrule
		\end{tabu}
	\end{table}
	\begin{table}\centering
		\caption[\langtvk{} Inanimate Noun Declension Paradigm]{\langtvk{} Inanimate Noun Declension Paradigm for the word \scr{šem} \fw{šem} \defn{busyness}}
		\label{tab:tvk-in-noun-decl}
		\begin{tabu}{| l | l l l |}
			\toprule
			\rowfont[c]\bfseries Case & \Sg & \Pc & \Pl\\
			\midrule
			\textbf{\Abs} & \fw{šem} & \fw{le šem} & \fw{ren šem}\\
			\textbf{\Erg} & \fw{ða šem} & \fw{ðes šem} & \fw{dun šem}\\
			\textbf{\Acc} & \fw{ti šem} & \fw{þis šem} & \fw{ten šem}\\
			\textbf{\Dat} & \fw{ǩo šem} & \fw{kos šem} & \fw{ǩun šem}\\
			\textbf{\Gen} & \fw{šo šem} & \fw{se šem} & \fw{šen šem}\\
			\midrule
			\textbf{\Top} & \fw{mi šem} & \fw{mes šem} & \fw{nun šem}\\
			\textbf{\Top.\Acc} & \fw{mati šem} & \fw{moþes šem} & \fw{noten šem}\\
			\textbf{\Top.\Dat} & \fw{moǩ šem} & \fw{mekos šem} & \fw{nikun šem}\\
			\textbf{\Top.\Gen} & \fw{miš šem} & \fw{mise šem} & \fw{nušen šem}\\
			\bottomrule
		\end{tabu}
	\end{table}
}

\subsection{Gender}
\label{subsec:tvk-nouns-gender}

Grammatical gender in \langtvk{} consists of two\autocite{wals-30} non-sex-based\autocite{wals-31} classes based primarily on semantic ontological properties\autocite{wals-32}. The animate gender refers primarily to entities that are considered alive or are associated with life, movement, change, or dynamism. The inanimate gender refers primarily to entities that are not alive and are generally stationary or abstract. Grammatical gender in \langtvk{} can also be referred to as \enquote{animacy} since that is what the genders denote. Examples of nouns in each gender can be seen in example~\getref{ex:tvk-noun-genders}.

\pex<ex:tvk-noun-genders>
	\a<an>Animate nouns:\\
		\scr{botra} \fw{botra} \defn{woman}; \scr{ǩalo} \fw{ǩalo} \defn{man}; \scr{eson} \fw{eson} \defn{farmer}; \scr{okotik} \fw{okotik} \defn{puppy}; \scr{urdatil} \fw{urdatil} \defn{ward}; \scr{bilt} \fw{bilt} \defn{breath}
	\a<in>Inanimate nouns:\\
		\scr{esotik} \fw{esotik} \defn{country}; \scr{dedu} \fw{dedu} \defn{sky}; \scr{elbi} \fw{elbi} \defn{egg}; \scr{usudir} \fw{usudir} \defn{basket}; \scr{akrapis} \fw{akrapis} \defn{letter}; \scr{fradir} \fw{fradir} \defn{glasses}
\xe

Since the nouns themselves are not directly inflected, with grammatical information instead shown on prepositional particles, it is impossible to tell what gender a noun is based solely on its word form.

Some nouns are able to change category in certain circumstances. For example, plants and animals switch from the animate gender to the inanimate gender when they serve as food. Further, there exist some duplicates with otherwise identical words declining to opposite genders.

\subsection{Number}
\label{subsec:tvk-nouns-number}

Grammatical number in \langtvk{} consists of three numbers, all of which are coded on the noun prepositions\autocite{wals-33}. The singular is always used when there is only one of the referent noun, the paucal is used when there are two to five of the referent noun, and the plural is used when there are more than five of the referent noun.

\pex<ex:tvk-noun-numbers>
	\a<sg> \scr{su ima} \fw{su ima} \phnm{su i\pstrs ma} \defn{of mother} \gloss{\Sg.\An.\Gen= mother}
	\a<pc> \scr{sar ima} \fw{sar ima} \phnm{sar i\pstrs ma} \defn{of (some) mothers} \gloss{\Pc.\An.\Gen= mother}
	\a<pl> \scr{san ima} \fw{san ima} \phnm{san i\pstrs ma} \defn{of (several) mothers} \gloss{\Pl.\An.\Gen= mother}
\xe

When a numeral is used to identify the number of a referent noun, the singular is used instead of the paucal or plural, even if without the numeral the other forms would be used\autocite{wals-34}.

\pex<ex:tvk-noun-numbers-numerals>
	\a<sg> \scr{k’eþ ima} \fw{k'eþ ima} \phnm{keθ i\pstrs ma} \defn{to one mother} \gloss{\Sg.\An.\Dat=one mother}
	\a<pc> \scr{ke arsi ima} \fw{ke arsi ima} \phnm{ke ar\pstrs si i\pstrs ma} \defn{to three mothers} \gloss{\Sg.\An.\Dat= three mother}, not \ungr{\scr{kas arsi ima}} \ungr{\fw{kas arsi ima}}
	\a<pl> \scr{ke bruð abom ima} \fw{ke bruð abom ima} \phnm{ke bruð a\pstrs bom i\pstrs ma} \defn{to seven mothers} \gloss{\Sg.\An.\Dat= five two mother}, not \ungr{\scr{ken bruð abom ima}} \ungr{\fw{ken bruð abom ima}}
\xe

Most nouns that represent concrete entities are countable, including some words that in English are uncountable like corn, and by default they are used in the singular form unlike English words like pants or glasses. However, many entities that are not easily split into discreet parts like liquids, grains, and certain abstract concepts are uncountable, such as \scr{elto} \fw{elto} \phnm{el\pstrs to} \defn{water}. Occasionally, when a word's semantics cover multiple concepts, a word can be variably countable or uncountable; when \scr{dedu} \fw{dedu} \phnm{de\pstrs du} is used to mean \defn{sky} or \defn{heaven}, it is uncountable, but when it is used to mean \defn{ceiling}, it is countable and can be made paucal or plural.

People's names can also be declined to the paucal or plural number to indicate the associative plural\autocite{wals-36}. This form is used to refer to a person and the other people associated with that person. For example, \scr{ri Bol} \fw{ri Bol} \phnm{ri bol} \gloss{\Pc.\An.\Abs{} Bol} refers to Bol and two to five other people associated with him. Similarly, \scr{ran Ote} \fw{ran Ote} \phnm{ran o\pstrs te} \gloss{\Pl.\An.\Abs{} Ote} refers to Ote and the group he is with.

\subsection{Case}
\label{subsec:tvk-nouns-case}

As shown in Tables \ref{tab:tvk-an-noun-decl} and \ref{tab:tvk-in-noun-decl}, \langtvk{} noun phrases decline to five different grammatical cases\autocite{wals-49} in order to show their role in the sentence. These cases are governed by the phrase's verb or assigned to adjuncts depending on their purpose or meaning. As shown in the same declension tables, any of these grammatical cases can be replaced by or combined with topic markers. See \autoref{subsec:tvk-nouns-topicality} for more information on topicality.

\subsubsection{Absolutive and Intransitive}
\label{subsubsec:tvk-nouns-absolutive}

The intransitive case marks a noun or noun phrase that serves as the subject of an intransitive verb like \scr{šeli} \fw{šeli} \defn{to run} or a transitive verb used intransitively like \scr{ufuli} \fw{ufuli} \defn{to sing} (without naming the object, what is being sung). This means that when a verb has only a single argument, that argument will by default be in the intransitive case. That is true whether the subject is serving like an agent as in words like \scr{šeli} \fw{šeli} \defn{to run} or \scr{ufuli} \fw{ufuli} \defn{to sing} or when the subject is serving more like a patient as in words like \scr{orðali} \fw{orðali} \defn{to fall}.

\pex<ex:tvk-noun-abs>
	\a<a1>\begingl
		\glpreamble\scr{Mollur šeþ.}\\
		\fw{Mollur šeþ.}\\
		\phnm{mo\pstrs\gem{l}ur \pstrs ʃeθ}//
		\gla ∅= Mollur š-eþ//
		\glb \An.\Sg.\Intr= Mollur run-\Ind.\Npst.\Prg//
		\glft\defn{Mollur is running.}//
	\endgl
	\a<a2>\begingl
		\glpreamble\scr{R’ima ufu.}\\
		\fw{R'ima ufu.}\\
		\phnm{ri\pstrs ma u\pstrs fu}//
		\gla ri=ima uf-u//
		\glb \An.\Pc.\Intr=mother sing-\Ind.\Npst.\Ipfv//
		\glft\defn{The mothers sing.}//
	\endgl
	\a<p>\begingl
		\glpreamble\scr{Ren fild orðak.}\\
		\fw{Ren fild orðak.}\\
		\phnm{ren \pstrs fild or\pstrs ðak}//
		\gla ren= fild orð-ak//
		\glb \In.\Pl.\Intr= doll fall-\Ind.\Pst.\Pfv//
		\glft\defn{The dolls fell.}//
	\endgl
\xe

Note that the singular intransitive case is entirely unmarked by any preposition. This is true whether the noun is animate or inanimate.

\pex<ex:tvk-noun-abs-sg>
	\a<an>\begingl
		\glpreamble\scr{Alum uldeteš.}\\
		\fw{Alum uldeteš.}\\
		\phnm{a\pstrs lum ul.de\pstrs teʃ}//
		\gla ∅= alum uldet-eš//
		\glb \An.\Sg.\Intr= cloud change-\Ind.\Npst.\Rtsp//
		\glft\defn{The cloud has changed.}//
	\endgl
	\a<in>\begingl
		\glpreamble\scr{Almaþ uldeteš.}\\
		\fw{Almaþ uldeteš.}\\
		\phnm{al\pstrs maθ ul.de\pstrs teʃ}//
		\gla ∅= almaþ uldet-eš//
		\glb \In.\Sg.\Intr= village change-\Ind.\Npst.\Rtsp//
		\glft\defn{The village has changed.}//
	\endgl
\xe

However, the subject of certain transitive verbs will also take the intransitive case if the semantic meaning of the verb is stative. See \autoref{subsubsec:tvk-nouns-dative} Dative for more information on this. Since it is used in these situations, and since the intransitive is the citation form, the case is normally referred to as the absolutive case, even when used intransitively. These terms are interchangeable.

\ex<ex:tvk-noun-abs-trans>
	\begingl
		\glpreamble\scr{Ter ke arb fra vi?}\\
		\fw{Ter ke arb fra vi?}\\
		\phnm{ter ke arb \pstrs fra vi}//
		\gla ter ke= arb fr-a =vi//
		\glb \Sps.\Abs{} \An.\Sg.\Dat= bird see-\Ind.\Npst.\Ipfv{} =\Q//
		\glft\defn{Do you see a bird?}//
	\endgl
\xe

The absolutive case is frequently used with postpositions to indicate a location where or through which an action is taken, for example being placed at, on, or in something.

\pex<ex:tvk-noun-abs-pp>
	\a<on>\begingl
		\glpreamble\scr{Ablu onaš e onek.}\\
		\fw{Ablu onaš e onek.}\\
		\phnm{ab\pstrs lu o\pstrs naʃ e o\pstrs nek}//
		\gla ∅= ablu ∅= onaš e on-ek//
		\glb \An.\Sg.\Abs= cat \In.\Sg.\Abs= rug on play-\Ind.\Pst.\Pfv//
		\glft\defn{The cat played on the rug.}//
	\endgl
	\a<over>\begingl
		\glpreamble\scr{Mod ti ennis l’elbi arku ǧirak.}\\
		\fw{Mod ti ennis l'elbi arku ǧirak.}\\
		\phnm{mod ti e\pstrs\gem{n}is lel\pstrs bi ar\pstrs ku ɣi\pstrs rak}//
		\gla mod ti= ennis le=elbi arku ǧir-ak//
		\glb \Fps.\Erg{} \In.\Sg.\Acc= ball \In.\Pc.\Abs=egg above throw-\Ind.\Pst.\Pfv//
		\glft\defn{I threw the ball over the eggs.}//
	\endgl
\xe

When an action is done \defn{with} or \defn{without} a noun, the absolutive case will be used.

\ex<ex:tvk-noun-abs-with>
	\begingl
		\glpreamble\scr{Oko ablu mo oneþ.}\\
		\fw{Oko ablu mo oneþ.}\\
		\phnm{o\pstrs ko ab\pstrs lu mo o\pstrs neθ}//
		\gla ∅= oko ∅= ablu mo on-eþ//
		\glb \An.\Sg.\Abs= dog \An.\Sg.\Abs= cat with play-\Ind.\Npst.\Prg//
		\glft\defn{The dog is playing with the cat.}//
	\endgl
\xe

The absolutive case is also used when directly addressing someone in a vocative function. The noun functioning in this way is often placed at the beginning or end of the sentence separated by a pause in speech or a comma in writing.

\pex<ex:tvk-noun-abs-voc>
	\a<imp>\begingl
		\glpreamble\scr{Lerk, šebanta.}\\
		\fw{Lerk, šebanta.}\\
		\phnm{\pstrs lerk, ʃe\pstrs ban.ta}//
		\gla ∅= Lerk šeb-anta//
		\glb \An.\Sg.\Abs= Lerk run-\Imp//
		\glft\defn{Run, Lerk.}//
	\endgl
	\a<nimp>\begingl
		\glpreamble\scr{Sud tu tavotik urdateþ, Erme.}\\
		\fw{Sud tu tavotik urdateþ, Erme.}\\
		\phnm{sud tu ta.vo\pstrs tik ur.da\pstrs teθ er\pstrs me}//
		\gla sud tu= tavotik urdat-eþ ∅= Erme//
		\glb \Tps.\An.\Erg{} \An.\Sg.\Acc= child guard-\Ind.\Npst.\Prg{} \An.\Sg.\Abs= Erme//
		\glft\defn{He is guarding the child, Erme.}//
	\endgl
\xe

\subsubsection{Ergative}
\label{subsubsec:tvk-nouns-ergative}

The ergative case marks a noun or noun phrase that serves as the subject of an active transitive verb or any ditransitive verb. This means that when a verb has multiple arguments and the semantic meaning of the verb is active, the subject argument will by default by in the ergative case.

\pex<ex:tvk-noun-erg>
	\a<eðerali>\begingl
		\glpreamble\scr{Do Tlunda ti akrapis eðeraš.}\\
		\fw{Do Tlunda ti akrapis eðeraš.}\\
		\phnm{do tlun\pstrs da ti ak.ra\pstrs pis e.ðe\pstrs raʃ}//
		\gla do= Tlunda ti= akrapis eðer-aš//
		\glb \An.\Sg.\Erg= Tlunda \In.\Sg.\Acc= letter pen-\Ind.\Npst.\Rtsp//
		\glft\defn{Tlunda has penned a letter.}//
	\endgl
	\a<okotali>\begingl
		\glpreamble\scr{Das oko tu ablu okotam.}\\
		\fw{Das oko tu ablu okotam.}\\
		\phnm{das o\pstrs ko tu ab\pstrs lu o.ko\pstrs tam}//
		\gla das= oko tu= ablu okot-am//
		\glb \An.\Pc.\Erg= dog \An.\Sg.\Acc= cat chase-\Ind.\Pst.\Ipfv//
		\glft\defn{The dogs chased the cat.}//
	\endgl
	\a<visali>\begingl
		\glpreamble\scr{Din avo ten usudir visaǧ.}\\
		\fw{Din avo ten usudir visaǧ.}\\
		\phnm{din a\pstrs vo ten u.su\pstrs dir vi\pstrs saɣ}//
		\gla din= avo ten= usudir vis-aǧ//
		\glb \An.\Pl.\Erg= father \In.\Pl.\Acc= basket take.away-\Ind.\Pst.\Rtsp//
		\glft\defn{The father and his associates had taken away the baskets.}//
	\endgl
\xe

\subsubsection{Accusative}
\label{subsubsec:tvk-nouns-accusative}

The accusative case marks a noun or noun phrase that serves as the direct object of an active transitive verb or any ditransitive verb.

\pex<ex:tvk-noun-acc>
	\a<onašuli>\begingl
		\glpreamble\scr{Do akrakon þis eðerik alma e onašuk.}\\
		\fw{Do akrakon þis eðerik alma e onašuk.}\\
		\phnm{do ak.ra\pstrs kon θis e.ðe\pstrs rik al\pstrs ma e o.na\pstrs ʃuk}//
		\gla do= akrakon þis= eðerik alma e onaš-uk//
		\glb \An.\Sg.\Erg= writer \In.\Pc.\Acc= pencil house in place-\Ind.\Pst.\Pfv//
		\glft\defn{The writer placed the pencils in the house.}//
	\endgl
	\a<uldetuli>\begingl
		\glpreamble\scr{Do šus botra ti šus akrapis uldetuk.}\\
		\fw{Do šus botra ti šus akrapis uldetuk.}\\
		\phnm{do ʃus bot\pstrs ra ti ʃus ak.ra\pstrs pis ul.de\pstrs tuk}//
		\gla do= šus botra ti= šus akrapis uldet-uk//
		\glb \An.\Sg.\Erg= \Tpp.\An.\Gen{} wife \In.\Sg.\Acc= \Tpp.\An.\Gen{} letter change-\Ind.\Pst.\Pfv//
		\glft\defn{His wife changed his letter.}//
	\endgl
\xe

\subsubsection{Dative}
\label{subsubsec:tvk-nouns-dative}

The dative case marks a noun or noun phrase that serves as the indirect object of a ditransitive verb, a recipient of an action, or the entity for whose benefit or detriment the action is taken.

\ex<ex:tvk-noun-dat>
	\begingl
		\glpreamble\scr{Do eson ti ennis ke oko draš.}\\
		\fw{Do eson ti ennis ke oko draš.}\\
		\phnm{do e\pstrs son ti e\pstrs\gem{n}is ke o\pstrs ko \pstrs draʃ}//
		\gla do= eson ti= ennis ke= oko dr-aš//
		\glb \An.\Sg.\Erg= farmer \In.\Sg.\Acc= ball \An.\Sg.\Dat= dog give-\Ind.\Npst.\Rtsp//
		\glft\defn{The farmer has given the dog a ball.}//
	\endgl
\xe

Certain monotransitive verbs are used with the absolutive and dative cases instead of the ergative and accusative cases. These tend to be stative verbs in which the object of the verb is unaffected by the action or there is little volition on the part of the subject.

\pex<ex:tvk-noun-dat-trans>
	\a<teguli>\begingl
		\glpreamble\scr{Mor tek tegu.}\\
		\fw{Mor tek tegu.}\\
		\phnm{mor tek te\pstrs gu}//
		\gla mor tek teg-u//
		\glb \Fps.\Abs{} \Sps.\Dat{} worry-\Ind.\Npst.\Ipfv//
		\glft\defn{I worry for you.}//
	\endgl
	\a<keðali>\begingl
		\glpreamble\scr{Ran urdaton ken ufukon keðam.}\\
		\fw{Ran urdaton ken ufukon keðam.}\\
		\phnm{ran ur.da\pstrs ton ken u.fu\pstrs kon ke\pstrs ðam}//
		\gla ran= urdaton ken= ufukon keð-am//
		\glb \An.\Pl.\Abs= guard \An.\Pl.\Dat= singer admire-\Ind.\Pst.\Ipfv//
		\glft\defn{The guards admired the singers.}//
	\endgl
\xe

When a verb is done on behalf of or for someone or something, the beneficiary of that action will be declined to the dative and followed by the postposition \scr{li} \fw{li} \phnm{li} \defn{for}.

\pex<ex:tvk-noun-dat-beneficiary>
	\a<oveli>\begingl
		\glpreamble\scr{Sur kas šus botrašut li ove.}\\
		\fw{Sur kas šus botrašut li ove.}\\
		\phnm{sur kas ʃus bot.ra\pstrs ʃut li o\pstrs ve}//
		\gla sur kas= šus botrašut li ov-e//
		\glb \Tps.\An.\Abs{} \An.\Pc.\Dat= \Tps.\An.\Gen{} fiancée for cook-\Ind.\Npst.\Ipfv//
		\glft\defn{He cooks for his fiancée and her friends.}//
	\endgl
	\a<urdateli>\begingl
		\glpreamble\scr{Do Blimva tu okotik ke šus avo li urdateþ.}\\
		\fw{Do Blimva tu okotik ke šus avo li urdateþ.}\\
		\phnm{do blim\pstrs va tu o.ko\pstrs tik ke ʃus a\pstrs vo li ur.da\pstrs teθ}//
		\gla do= Blimva tu= okotik ke= šus avo li urdat-eþ//
		\glb \An.\Sg.\Erg= Blimva \An.\Sg.\Acc= puppy \An.\Sg.\Dat= \Tps.\An.\Gen{} father for protect-\Ind.\Npst.\Prg//
		\glft\defn{Blimva is protecting the puppy for her father.}//
	\endgl
\xe

The dative case can also be used in an allative sense to express movement to or toward.

\ex<ex:tvk-noun-dat-movement>
	\begingl
		\glpreamble\scr{Mor ǩo alma bi šeba.}\\
		\fw{Mor ǩo alma bi šeba.}\\
		\phnm{mor xo al\pstrs ma bi ʃe\pstrs ba}//
		\gla mor ǩo= alma to šeb-a//
		\glb \Fps.\An{} \In.\Sg.\Dat= house to run-\Ind.\Npst.\Ipfv//
		\glft\defn{I run to the house.}//
	\endgl
\xe

This can result in subtle changes in meaning when used with ditransitive verbs.

\pex<ex:tvk-noun-dat-movement-ditr>
	\a<to>\begingl
		\glpreamble\scr{Mod þis ennis tek ǧira.}\\
		\fw{Mod þis ennis tek ǧira.}\\
		\phnm{mod θis e\pstrs\gem{n}is tek ɣi\pstrs ra}//
		\gla mod þis= ennis tek ǧir-a//
		\glb \Fps.\Erg{} \In.\Pc.\Acc= ball \Sps.\Dat{}  throw-\Ind.\Npst.\Ipfv//
		\glft\defn{I throw the balls to you.}//
	\endgl
	\a<at>\begingl
		\glpreamble\scr{Mod þis ennis tek bi ǧira.}\\
		\fw{Mod þis ennis tek bi ǧira.}\\
		\phnm{mod θis e\pstrs\gem{n}is tek bi ɣi\pstrs ra}//
		\gla mod þis= ennis tek bi ǧir-a//
		\glb \Fps.\Erg{} \In.\Pc.\Acc= ball \Sps.\Dat{} at  throw-\Ind.\Npst.\Ipfv//
		\glft\defn{I throw the balls at you.}//
	\endgl
\xe

Notice in example~\getfullref{ex:tvk-noun-dat-movement-ditr.to} that \scr{tek} \fw{tek} is the recipient of the action while in example~\getfullref{ex:tvk-noun-dat-movement-ditr.at} \scr{tek} \fw{tek} is the target of the action.

\subsubsection{Genitive}
\label{subsubsec:tvk-nouns-genitive}

The genitive case is used to mark the possessor of a noun or noun phrase.

\ex<ex:tvk-noun-gen-roles>
	\begingl
		\glpreamble\scr{Su Goltu botra mok fra.}\\
		\fw{Su Goltu botra mok fra.}\\
		\phnm{su gol\pstrs tu bot\pstrs ra mok \pstrs fra}//
		\gla ∅= su= Goltu botra mok fr-a//
		\glb \An.\Sg.\Abs= \An.\Sg.\Gen= Goltu wife \Fps.\Dat{} see-\Ind.\Npst.\Ipfv//
		\glft\defn{Goltu's wife sees me.}//
	\endgl
\xe

Just like other attributives, the genitive phrase will occur between the possessee and its declension clitic.

\pex<ex:tvk-noun-gen>
	\a<ergacc>\begingl
		\glpreamble\scr{Do su Zarsa oko tu mos ablu okotaða!}\\
		\fw{Do su Zarsa oko tu mos ablu okotaða!}\\
		\phnm{do su zar\pstrs sa o\pstrs ko tu mos ab\pstrs lu o.ko\pstrs ta.ða}//
		\gla do= su= Zarsa oko tu= mos ablu okot-aða//
		\glb \An.\Sg.\Erg= \An.\Sg.\Gen= Zarsa dog \An.\Sg.\Acc= \Fps.\Gen{} cat chase-\Ind.\Pst.\Prg//
		\glft\defn{Zarsa's dog was chasing my cat!}//
	\endgl
	\a<dat>\begingl
		\glpreamble\scr{Mod ti ennis ke su Inki oko ǧira.}\\
		\fw{Mod ti ennis ke su Inki oko ǧira.}\\
		\phnm{mod ti e\pstrs\gem{n}is ke su in\pstrs ki o\pstrs ko ɣi\pstrs ra}//
		\gla mod þis= ennis ke= su= Inki oko ǧir-a//
		\glb \Fps.\Erg{} \In.\Sg.\Acc= ball \An.\Sg.\Dat= \An.\Sg.\Gen= Inki dog throw-\Ind.\Npst.\Ipfv//
		\glft\defn{I throw the ball to Inki's dog.}//
	\endgl
\xe

When a verb is done because of or due to someone or something, the cause of that action will be declined to the genitive and followed by the postposition \scr{li} \fw{li} \phnm{li} \defn{because of}.

\pex<ex:tvk-noun-gen-cause>
	\a<oveli>\begingl
		\glpreamble\scr{Sur su šus botrašut li puzaða bas ovek.}\\
		\fw{Sur su šus botrašut li puzaða bas ovek.}\\
		\phnm{sur su ʃus bot.ra\pstrs ʃut li pu\pstrs za.ða bas o\pstrs vek}//
		\gla sur su= šus botrašut li puz-aða bas ov-ek//
		\glb \Tps.\An.\Abs{} \An.\Sg.\Gen= \Tps.\An.\Gen{} fiancée because.of cry-\Ind.\Pst.\Prg{} \Rel.\Nrtrv{} cook-\Ind.\Pst.\Pfv//
		\glft\defn{He cooked because his fiancée was crying.}//
	\endgl
	\a<urdateli>\begingl
		\glpreamble\scr{Do Blimva tu okotik su šus avo li urdateþ.}\\
		\fw{Do Blimva tu okotik su šus avo li urdateþ.}\\
		\phnm{do blim\pstrs va tu o.ko\pstrs tik su ʃus a\pstrs vo li ur.da\pstrs teθ}//
		\gla do= Blimva tu= okotik su= šus avo li urdat-eþ//
		\glb \An.\Sg.\Erg= Blimva \An.\Sg.\Acc= puppy \An.\Sg.\Gen= \Tps.\An.\Gen{} father because.of protect-\Ind.\Npst.\Prg//
		\glft\defn{Blimva is protecting the puppy from her father.}//
	\endgl
\xe

The genitive can also be used in an ablative sense to express movement from or away.

\ex<ex:tvk-noun-gen-movement>
	\begingl
		\glpreamble\scr{Mor šo alma gu šeba.}\\
		\fw{Mor šo alma gu šeba.}\\
		\phnm{mor ʃo al\pstrs ma gu ʃe\pstrs ba}//
		\gla mor šo= alma to šeb-a//
		\glb \Fps.\An{} \In.\Sg.\Gen= house from run-\Ind.\Npst.\Ipfv//
		\glft\defn{I run from the house.}//
	\endgl
\xe

\subsection{Topicality}
\label{subsec:tvk-nouns-topicality}

Several noun cases have variants that mark a noun as the topic of a discourse. The topic is the entity most closely associated with the higher-level theme of the paragraph.

The case preposition that encodes \emph{only} topicality completely replaces the case marking for a noun that is in the absolutive or the ergative.

\pex<ex:tvk-noun-top-subj>
	\a<abs>\begingl
		\glpreamble\scr{No Mollur šeþ.}\\
		\fw{No Mollur šeþ.}\\
		\phnm{no mo\pstrs\gem{l}ur \pstrs ʃeθ}//
		\gla no= Mollur š-eþ//
		\glb \An.\Sg.\Top= Mollur run-\Ind.\Npst.\Prg//
		\glft\defn{Mollur is running.}//
	\endgl
	\a<abs-tr>\begingl
		\glpreamble\scr{Þan ke arb fra vi?}\\
		\fw{Þan ke arb fra vi?}\\
		\phnm{θan ke arb \pstrs fra vi}//
		\gla þan ke= arb fr-a =vi//
		\glb \Sps.\Top{} \An.\Sg.\Dat= bird see-\Ind.\Npst.\Ipfv{} =\Q//
		\glft\defn{Do you see a bird?}//
	\endgl
	\a<erg>\begingl
		\glpreamble\scr{Nas oko tu ablu okotam.}\\
		\fw{Nas oko tu ablu okotam.}\\
		\phnm{nas o\pstrs ko tu ab\pstrs lu o.ko\pstrs tam}//
		\gla nas= oko tu= ablu okot-am//
		\glb \An.\Pc.\Top= dog \An.\Sg.\Acc= cat chase-\Ind.\Pst.\Ipfv//
		\glft\defn{The dogs chased the cat.}//
	\endgl
\xe

This case preposition also completely replaces the accusative and dative cases, but only in certain situations when the intended case is inferable. In other words, it replaces the accusative case only when the ergative is present within the sentence, it replaces the dative in a monotransitive sentence only when the absolutive case is present, and it replaces the dative in a ditransitive sentence only when both the ergative and the accusative are present.

\pex<ex:tvk-noun-top-nonsubj>
	\a<acc>\begingl
		\glpreamble\scr{Do šus botra mi šus akrapis uldetuk.}\\
		\fw{Do šus botra mi šus akrapis uldetuk.}\\
		\phnm{do ʃus bot\pstrs ra mi ʃus ak.ra\pstrs pis ul.de\pstrs tuk}//
		\gla do= šus botra mi= šus akrapis uldet-uk//
		\glb \An.\Sg.\Erg= \Tpp.\An.\Gen{} wife \In.\Sg.\Top= \Tpp.\An.\Gen{} letter change-\Ind.\Pst.\Pfv//
		\glft\defn{His wife changed his letter.}//
	\endgl
	\a<dat-st>\begingl
		\glpreamble\scr{Ran urdaton nan ufukon keðam.}\\
		\fw{Ran urdaton nan ufukon keðam.}\\
		\phnm{ran ur.da\pstrs ton nan u.fu\pstrs kon ke\pstrs ðam}//
		\gla ran= urdaton nan= ufukon keð-am//
		\glb \An.\Pl.\Abs= guard \An.\Pl.\Top= singer admire-\Ind.\Pst.\Ipfv//
		\glft\defn{The guards admired the singers.}//
	\endgl
	\a<dat-ac>\begingl
		\glpreamble\scr{Do eson ti ennis no oko draš.}\\
		\fw{Do eson ti ennis no oko draš.}\\
		\phnm{do e\pstrs son ti e\pstrs\gem{n}is no o\pstrs ko \pstrs draʃ}//
		\gla do= eson ti= ennis no= oko dr-aš//
		\glb \An.\Sg.\Erg= farmer \In.\Sg.\Acc= ball \An.\Sg.\Top= dog give-\Ind.\Npst.\Rtsp//
		\glft\defn{The farmer has given the dog a ball.}//
	\endgl
\xe

For other situations, there exist combined forms to mark a noun as the topic when it is in the accusative, dative, or genitive case.

\pex<ex:tvk-noun-top-combined>
	\a<acc>\begingl
		\glpreamble\scr{Nut ablu okotam.}\\
		\fw{Nut ablu okotam.}\\
		\phnm{nut ab\pstrs lu o.ko\pstrs tam}//
		\gla nut= ablu okot-am//
		\glb \An.\Sg.\Acc.\Top= cat chase-\Ind.\Pst.\Ipfv//
		\glft\defn{The cats were chased.}//
	\endgl
	\a<dat>\begingl
		\glpreamble\scr{Naken ufukon keðam.}\\
		\fw{Naken ufukon keðam.}\\
		\phnm{na\pstrs ken u.fu\pstrs kon ke\pstrs ðam}//
		\gla naken= ufukon keð-am//
		\glb \An.\Pl.\Dat.\Top= singer admire-\Ind.\Pst.\Ipfv//
		\glft\defn{The singers were admired.}//
	\endgl
	\a<gen>\begingl
		\glpreamble\scr{Mod ti ennis ke þansu oko ǧira.}\\
		\fw{Mod ti ennis ke þansu oko ǧira.}\\
		\phnm{mod ti e\pstrs\gem{n}is ke θan\pstrs su o\pstrs ko ɣi\pstrs ra}//
		\gla mod þis= ennis ke= þansu oko ǧir-a//
		\glb \Fps.\Erg{} \In.\Sg.\Acc= ball \An.\Sg.\Dat= \Sps.\Gen.\Top{} dog throw-\Ind.\Npst.\Ipfv//
		\glft\defn{I throw the ball to your dog.}//
	\endgl
\xe

See \autoref{sec:tvk-discourse-topic} for a greater explanation of how the topic is used within discourse.

\section{Pronouns and Determiners}
\label{sec:tvk-pronouns-determiners}
\langtvk{} has several types of pronouns and determiners that serve as anaphora, including personal pronouns, demonstrative pronouns, interrogative pronouns, relative pronouns, and other indefinite pronouns.

\subsection{Personal Pronouns}
\label{subsec:tvk-personal-pronouns}

\afterpage{\clearpage
	\begin{table}\centering
		\caption[\langtvk{} Personal Pronouns]{\langtvk{} Personal Pronouns}
		\label{tab:tvk-prs-pn}
		\begin{tabu}{| l | l l l l l l l l l |}
			\toprule
			\rowfont[c]\bfseries Person & \Abs{} & \Erg{} & \Acc{} & \Dat{} & \Gen{} & \Top{} & \Top.\Acc{} & \Top.\Dat{} & \Top.\Gen{}\\
			\midrule
			\Fps{} & \fw{mor} & \fw{mod} & \fw{mot} & \fw{mok} & \fw{mos} & \fw{mon} & \fw{montu} & \fw{monke} & \fw{monsu}\\
			\Fpc{} & \fw{morsa} & \fw{modas} & \fw{motos} & \fw{mokas} & \fw{mosar} & \fw{monsa} & \fw{monsut} & \fw{monsek} & \fw{monsus}\\
			\Fpp{} & \fw{morna} & \fw{modin} & \fw{moton} & \fw{moken} & \fw{mosan} & \fw{mana} & \fw{manut} & \fw{manek} & \fw{manus}\\
			\midrule
			\Sps{} & \fw{ter} & \fw{ted} & \fw{þet} & \fw{tek} & \fw{tes} & \fw{þan} & \fw{þantu} & \fw{þanke} & \fw{þansu}\\
			\Spc{} & \fw{tersa} & \fw{tedas} & \fw{þetos} & \fw{tekas} & \fw{tesar} & \fw{tensa} & \fw{tensut} & \fw{tensek} & \fw{tensus}\\
			\Spp{} & \fw{terna} & \fw{tedin} & \fw{þeton} & \fw{token} & \fw{tesan} & \fw{tana} & \fw{tanut} & \fw{tanek} & \fw{tanus}\\
			\midrule
			\Tps.\An{} & \fw{sur} & \fw{sud} & \fw{sut} & \fw{suk} & \fw{šus} & \fw{šun} & \fw{šuntu} & \fw{šunke} & \fw{šunsu}\\
			\Tpc.\An{} & \fw{suša} & \fw{sudas} & \fw{sutos} & \fw{sukas} & \fw{šusar} & \fw{sunas} & \fw{šunsut} & \fw{šunsek} & \fw{šunsus}\\
			\Tpp.\An{} & \fw{surna} & \fw{sudin} & \fw{suton} & \fw{suken} & \fw{šusan} & \fw{šona} & \fw{šonut} & \fw{šonek} & \fw{šonus}\\
			\midrule
			\Tps.\In{} & \fw{gir} & \fw{gid} & \fw{git} & \fw{gake} & \fw{gis} & \fw{gin} & \fw{gintu} & \fw{ginke} & \fw{ginsu}\\
			\Tpc.\In{} & \fw{girsa} & \fw{gidas} & \fw{gitos} & \fw{gokas} & \fw{gisar} & \fw{ginsa} & \fw{ginsut} & \fw{ginsek} & \fw{ginsus}\\
			\Tpp.\In{} & \fw{girna} & \fw{gidun} & \fw{giton} & \fw{goken} & \fw{gisan} & \fw{gana} & \fw{ganut} & \fw{ganek} & \fw{ganus}\\
			\bottomrule
		\end{tabu}
	\end{table}
}

As shown in \autoref{tab:tvk-prs-pn}, \langtvk{} contains several personal pronouns. These pronouns are symmetrical to other nouns and noun phrases\autocite{wals-50}, declining to show gender, number, case, and topicality just like nouns while adding person.

Historically, all pronouns were regular formations with the case-marking preposition and a person-marking pronoun, but over time, these words combined and fused as grammaticalization progressed. The forms are now completely fused.

\pex<ex:tvk-pers-pronoun>
	\a<none>\begingl
		\glpreamble\scr{Do Tlunda ti ennis ke su Lerk oko ǧirak.}\\
		\fw{Do Tlunda ti ennis ke su Lerk oko ǧirak.}\\
		\phnm{do tlun\pstrs da ti e\pstrs\gem{n}is ke su \pstrs lerk o\pstrs ko ɣi\pstrs rak}//
		\gla do= Tlunda ti= ennis ke= su= Lerk oko ǧir-ak//
		\glb \An.\Sg.\Erg= Tlunda \In.\Sg.\Acc= ball \An.\Sg.\Dat= \An.\Sg.\Gen= Lerk dog throw-\Ind.\Pst.\Pfv//
		\glft\defn{Tlunda threw the ball to Lerk's dog.}//
	\endgl
	\a<erg>\begingl
		\glpreamble\scr{Sud ti ennis ke su Lerk oko ǧirak.}\\
		\fw{Sud ti ennis ke su Lerk oko ǧirak.}\\
		\phnm{\pstrs sud ti e\pstrs\gem{n}is ke su \pstrs lerk o\pstrs ko ɣi\pstrs rak}//
		\gla Sud ti= ennis ke= su= Lerk oko ǧir-ak//
		\glb \Tps.\An.\Erg{} \In.\Sg.\Acc= ball \An.\Sg.\Dat= \An.\Sg.\Gen= Lerk dog throw-\Ind.\Pst.\Pfv//
		\glft\defn{She threw the ball to Lerk's dog.}//
	\endgl
	\a<acc>\begingl
		\glpreamble\scr{Do Tlunda git ke su Lerk oko ǧirak.}\\
		\fw{Do Tlunda git ke su Lerk oko ǧirak.}\\
		\phnm{do tlun\pstrs da \pstrs git ke su \pstrs lerk o\pstrs ko ɣi\pstrs rak}//
		\gla do= Tlunda git ke= su= Lerk oko ǧir-ak//
		\glb \An.\Sg.\Erg= Tlunda \Tps.\In.\Acc{} \An.\Sg.\Dat= \An.\Sg.\Gen= Lerk dog throw-\Ind.\Pst.\Pfv//
		\glft\defn{Tlunda threw it to Lerk's dog.}//
	\endgl
	\a<gen>\begingl
		\glpreamble\scr{Do Tlunda ti ennis ke šus oko ǧirak.}\\
		\fw{Do Tlunda ti ennis ke šus oko ǧirak.}\\
		\phnm{do tlun\pstrs da ti e\pstrs\gem{n}is ke \pstrs ʃus o\pstrs ko ɣi\pstrs rak}//
		\gla do= Tlunda ti= ennis ke= šus oko ǧir-ak//
		\glb \An.\Sg.\Erg= Tlunda \In.\Sg.\Acc= ball \An.\Sg.\Dat= \Tps.\An.\Gen{} dog throw-\Ind.\Pst.\Pfv//
		\glft\defn{Tlunda threw the ball to his dog.}//
	\endgl
	\a<dat>\begingl
		\glpreamble\scr{Do Tlunda ti ennis suk ǧirak.}\\
		\fw{Do Tlunda ti ennis suk ǧirak.}\\
		\phnm{do tlun\pstrs da ti e\pstrs\gem{n}is \pstrs suk ɣi\pstrs rak}//
		\gla do= Tlunda ti= ennis suk ǧir-ak//
		\glb \An.\Sg.\Erg= Tlunda \In.\Sg.\Acc= ball \An.\Sg.\Dat= \An.\Sg.\Gen= Lerk dog throw-\Ind.\Pst.\Pfv//
		\glft\defn{Tlunda threw the ball to him.}//
	\endgl
\xe

Personal pronouns are used the same way their full noun phrase counterparts are, in both core and non-core cases, and replace the full noun phrase for which they are serving as anaphor. Example~\getfullref{ex:tvk-pers-pronoun.none} shows a full sentence without any pronouns; examples~\getfullref{ex:tvk-pers-pronoun.erg}–\getref{ex:tvk-pers-pronoun.dat} then show variations on this sentence with different noun phrases replaced with pronouns. Notice that the pronoun replaces the full noun phrase, for example in \getfullref{ex:tvk-pers-pronoun.gen} where \scr{šus} \fw{šus} replaces only \scr{su Lerk} \fw{su Lerk}, the noun in the genitive, whereas in \getfullref{ex:tvk-pers-pronoun.dat}, \scr{suk} \fw{suk} replaces \scr{ke su Lerk oko} \fw{ke su Lerk oko}, the full dative noun phrase. Similarly, when a noun phrase contains an adjective, the whole noun phrase is replaced, including the adjective, as in example~\getref{ex:tvk-pers-pronoun-adj}.

\pex<ex:tvk-pers-pronoun-adj>
	\a<none>\begingl
		\glpreamble\scr{Bol no fraþru botra kantek.}\\
		\fw{Bol no fraþru botra kantek.}\\
		\phnm{\pstrs bol no fraθ\pstrs ru bot\pstrs ra kan\pstrs tek}//
		\gla ∅= Bol no= fraþru botra kant-ek//
		\glb \An.\Sg.\Abs= Bol \An.\Sg.\Top= observant woman thank-\Ind.\Pst.\Ipfv//
		\glft\defn{Bol thanked the observant woman.}//
	\endgl
	\a<bad>\begingl
		\glpreamble\ljudge{\ungr}\scr{Bol fraþru šun kantek.}\\
		\ljudge{\ungr}\fw{Bol fraþru šun kantek.}\\
		\phnm{\pstrs bol fraθ\pstrs ru \pstrs ʃun kan\pstrs tek}//
		\gla ∅= Bol fraþru šun kant-ek//
		\glb \An.\Sg.\Abs= Bol observant \Tps.\An.\Top{} thank-\Ind.\Pst.\Ipfv//
		\glft\ljudge{\ungr}\defn{Bol thanked the observant her.}//
	\endgl
	\a<good>\begingl
		\glpreamble\scr{Bol šun kantek.}\\
		\fw{Bol šun kantek.}\\
		\phnm{\pstrs bol \pstrs ʃun kan\pstrs tek}//
		\gla ∅= Bol šun kant-ek//
		\glb \An.\Sg.\Abs= Bol \Tps.\An.\Top{} thank-\Ind.\Pst.\Ipfv//
		\glft\defn{Bol thanked her.}//
	\endgl
\xe

\subsection{Demonstrative Pronouns and Determiners}
\label{subsec:tvk-demonstrative-pronouns-determiners}

There exist three demonstratives in \langtvk, including \scr{ðle} \fw{ðle} \phnm{ðle} \defn{this} (proximal), \scr{þro} \fw{þro} \phnm{θro} \defn{that} (medial), and \scr{lerǩo} \fw{lerǩo} \phnm{ler\pstrs xo} \defn{that} (distal). Just like the personal pronouns, these demonstratives replace the whole noun phrase for which they serve as anaphor. However, unlike pronouns, they do not have fused declensional forms; instead, they decline the same way nouns do.

\pex<ex:tvk-dem-pronoun>
	\a<prox>\begingl
		\glpreamble\scr{Mor ǩo ðle usu.}\\
		\fw{Mor ǩo ðle usu.}\\
		\phnm{\pstrs mor xo ðle u\pstrs su}//
		\gla mor ǩo= ðle us-u//
		\glb \Fps.\An.\Abs{} \In.\Sg.\Dat= \Dem.\Prox{} have-\Ind.\Npst.\Ipfv//
		\glft\defn{I have this.}//
	\endgl
	\a<medi>\begingl
		\glpreamble\scr{Mor ǩo þro usu.}\\
		\fw{Mor ǩo þro usu.}\\
		\phnm{\pstrs mor xo θro u\pstrs su}//
		\gla mor ǩo= þro us-u//
		\glb \Fps.\An.\Abs{} \In.\Sg.\Dat= \Dem.\Med{} have-\Ind.\Npst.\Ipfv//
		\glft\defn{I have that.}//
	\endgl
	\a<dist>\begingl
		\glpreamble\scr{Mor ǩo lerǩo usu.}\\
		\fw{Mor ǩo lerǩo usu.}\\
		\phnm{\pstrs mor xo ler\pstrs xo u\pstrs su}//
		\gla mor ǩo= lerǩo us-u//
		\glb \Fps.\An.\Abs{} \In.\Sg.\Dat= \Dem.\Dist{} have-\Ind.\Npst.\Ipfv//
		\glft\defn{I have that.}//
	\endgl
\xe

The proximal demonstrative \scr{ðle} \fw{ðle} refers to an object close to the speaker. The medial demonstrative \scr{þro} \fw{þro} refers to an object close to the addressee. The distal demonstrative \scr{lerǩo} \fw{lerǩo} refers to an object far from both the speaker and the addressee.

The demonstrative pronouns also inflect to show number, just like nouns. Example~\getfullref{ex:tvk-dem-pronoun-pl.pc} shows the proximal demonstrative \scr{ðle} \fw{ðle} used in the paucal number, while \getfullref{ex:tvk-dem-pronoun-pl.pl} shows the same in the plural.

\pex<ex:tvk-dem-pronoun-pl>
	\a<pc>\begingl
		\glpreamble\scr{Mor kos ðle usu.}\\
		\fw{Mor kos ðle usu.}\\
		\phnm{\pstrs mor xo ðle u\pstrs su}//
		\gla mor ǩo= ðle us-u//
		\glb \Fps.\An.\Abs{} \In.\Pc.\Dat= \Dem.\Prox{} have-\Ind.\Npst.\Ipfv//
		\glft\defn{I have these.}//
	\endgl
	\a<pl>\begingl
		\glpreamble\scr{Mor ǩun ðle usu.}\\
		\fw{Mor ǩun ðle usu.}\\
		\phnm{\pstrs mor xo ðle u\pstrs su}//
		\gla mor ǩo= ðle us-u//
		\glb \Fps.\An.\Abs{} \In.\Pl.\Dat= \Dem.\Prox{} have-\Ind.\Npst.\Ipfv//
		\glft\defn{I have these.}//
	\endgl
\xe

The demonstratives can also be used as determiners by pairing them with a noun. These determiners lack flexivity and do not inflect to match the gender of the referent noun like adjectives do. Determiners are placed \emph{after} the noun they modify.

\pex<ex:tvk-dem-det>
	\a<prox>\begingl
		\glpreamble\scr{D’oko nas ablu ðle okotak.}\\
		\fw{D'oko nas ablu ðle okotak.}\\
		\phnm{do\pstrs ko nas ab\pstrs lu ðle o.ko\pstrs tak}//
		\gla do=oko nas= ablu ðle okot-ak//
		\glb \An.\Sg.\Erg=dog \An.\Pc.\Top= cat \Dem.\Det.\Prox{} chase-\Ind.\Pst.\Pfv//
		\glft\defn{The dog chased these cats.}//
	\endgl
	\a<medi>\begingl
		\glpreamble\scr{D’oko no ablu þro okotak.}\\
		\fw{D'oko no ablu þro okotak.}\\
		\phnm{do\pstrs ko no ab\pstrs lu θro o.ko\pstrs tak}//
		\gla do=oko no= ablu þro okot-ak//
		\glb \An.\Sg.\Erg=dog \An.\Sg.\Top= cat \Dem.\Det.\Med{} chase-\Ind.\Pst.\Pfv//
		\glft\defn{The dog chased that cat.}//
	\endgl
	\a<dist>\begingl
		\glpreamble\scr{D’oko nan ablu lerǩo okotak.}\\
		\fw{D'oko nan ablu lerǩo okotak.}\\
		\phnm{do\pstrs ko nan ab\pstrs lu ler\pstrs xo o.ko\pstrs tak}//
		\gla do=oko nan= ablu lerǩo okot-ak//
		\glb \An.\Sg.\Erg=dog \An.\Pl.\Top= cat \Dem.\Det.\Dist{} chase-\Ind.\Pst.\Pfv//
		\glft\defn{The dog chased those cats.}//
	\endgl
\xe

\subsection{Interrogative Pronouns and Determiners}
\label{subsec:tvk-interrogative-pronouns-determiners}

\langtvk{} contains only one interrogative, \scr{arke} \fw{arke} \phnm{ar\pstrs ke}. By default, \scr{arke} \fw{arke} means \defn{who} or \defn{what}, depending on how it is declined.

\pex<ex:tvk-int-pronoun>
	\a<obj>\begingl
		\glpreamble\scr{Ter ǩo arke frak?}\\
		\fw{Ter ǩo arke frak?}\\
		\phnm{\pstrs ter xo ar\pstrs ke \pstrs frak}//
		\gla ter ǩo= arke fr-ak//
		\glb \Sps.\Abs{} \In.\Sg.\Dat= \Int{} see-\Ind.\Pst.\Pfv//
		\glft\defn{What did you see?}//
	\endgl
	\a<subj>\begingl
		\glpreamble\scr{Arke gin frak?}\\
		\fw{Arke gin frak?}\\
		\phnm{ar\pstrs ke gin \pstrs frak}//
		\gla ∅= arke gin fr-ak//
		\glb \Sg.\Abs= \Int{} \Tps.\In.\Top{} see-\Ind.\Pst.\Pfv//
		\glft\defn{Who saw it?}//
	\endgl
\xe

As shown in example~\getfullref{ex:tvk-int-pronoun.obj}, the interrogative pronoun is placed \scr{in situ} \fw{in situ}. In other words, the question word stays in place rather than being fronted to the beginning of the sentence like in English.

Notice also in example~\getref{ex:tvk-int-pronoun} that the particle \scr{vi} \fw{vi} is not used. Any sentence that contains the interrogative \scr{arke} \fw{arke} can be seen to be a question, obviating the need for \scr{vi} \fw{vi}. However, \scr{vi} \fw{vi} can be added back in to emphasize or, conceivably in rare instances, clarify the question.

\scr{Arke} \fw{Arke} can be paired with certain nouns or postpositions to form other interrogatives such as \defn{where}, \defn{when}, and \defn{how}.

\pex<ex:tvk-int-pronoun-others>
	\a<where>\begingl
		\glpreamble\scr{Ter gin inam arke e frak?}\\
		\fw{Ter gin inam arke e frak?}\\
		\phnm{\pstrs ter gin i\pstrs nam ar\pstrs ke e \pstrs frak}//
		\gla ter gin inam arke e fr-ak//
		\glb \Sps.\Abs{} \Tps.\In.\Top{} place \Int{} at see-\Ind.\Pst.\Pfv//
		\glft\defn{Where did you see it?}//
	\endgl
	\a<when>\begingl
		\glpreamble\scr{Ter gin etri arke e frak?}\\
		\fw{Ter gin etri arke e frak?}\\
		\phnm{\pstrs ter gin et\pstrs ri ar\pstrs ke e \pstrs frak}//
		\gla ter gin etri arke e fr-ak//
		\glb \Sps.\Abs{} \Tps.\In.\Top{} time \Int{} at see-\Ind.\Pst.\Pfv//
		\glft\defn{When did you see it?}//
	\endgl
	\a<how-tool>\begingl
		\glpreamble\scr{Ter gin arke mo frak?}\\
		\fw{Ter gin arke mo frak?}\\
		\phnm{\pstrs ter gin ar\pstrs ke mo \pstrs frak}//
		\gla ter gin arke mo fr-ak//
		\glb \Sps.\Abs{} \Tps.\In.\Top{} \Int{} with see-\Ind.\Pst.\Pfv//
		\glft\defn{How (with what) did you see it?}//
	\endgl
	\a<how-method>\begingl
		\glpreamble\scr{Ter gin pul arke frak?}\\
		\fw{Ter gin pul arke frak?}\\
		\phnm{\pstrs ter gin pul ar\pstrs ke \pstrs frak}//
		\gla ter gin pul arke fr-ak//
		\glb \Sps.\Abs{} \Tps.\In.\Top{} way \Int{} see-\Ind.\Pst.\Pfv//
		\glft\defn{How (what way) did you see it?}//
	\endgl
\xe

\scr{Arke} \fw{Arke} can also be paired with other nouns as a determiner to narrow the scope of the question, as in example~\getref{ex:tvk-int-det}.

\ex<ex:tvk-int-det>
	\begingl
		\glpreamble\scr{Ter ke oko arke frak?}\\
		\fw{Ter ke oko arke frak?}\\
		\phnm{\pstrs ter ke o\pstrs ko ar\pstrs ke \pstrs frak}//
		\gla ter ke= oko arke fr-ak//
		\glb \Sps.\Abs{} \An.\Sg.\Dat= dog \Int{} see-\Ind.\Pst.\Pfv//
		\glft\defn{What dog did you see?}//
	\endgl
\xe

\subsection{Relative Pronouns}
\label{subsec:tvk-relative-pronouns}

Relative pronouns

\subsection{Indefinite Pronouns and Determiners}
\label{subsec:tvk-indefinite-pronouns-determiners}

Indefinite pronouns and determiners

\section{Adjectives}
\label{sec:tvk-adjectives}

Adjectives

\section{Adpositions}
\label{sec:tvk-adpositions}

Adpositions

\section{Verbs}
\label{sec:tvk-verbs}

Verbs

\section{Adverbs}
\label{sec:tvk-adverbs}

Adverbs

\section{Numerals}
\label{sec:tvk-numerals}

Numerals

\section{Quantifiers and Intensifiers}
\label{sec:tvk-quant-intens}

Quantifiers and Intensifiers

\section{Conjunctions}
\label{sec:tvk-conjunctions}

Conjunctions
