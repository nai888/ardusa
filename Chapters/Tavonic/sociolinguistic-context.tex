\chapter{Sociolinguistic Context}
\label{app:tvk-sociolinguistic-context}

\section{Conceptual Metaphors}
\label{sec:tvk-conceptual-metaphors}

What metaphors do the vocabulary convey?

For example: language is a tool. I speak \textit{with} or \textit{using} \langtvk, rather than just speaking \langtvk.

\section{Kinship Terms}
\label{sec:tvk-kinship-terms}
\index{kinship|(}

\afterpage{\clearpage
	\begin{landscape}
		\begin{figure}[h]\centering
			\caption{\langtvk{} Kinship Tree}
			\label{fig:tvk-kinship}
			\index{kinship}
			\tiny
			\begin{tikzpicture}[scale=0.55]
			\GraphInit[vstyle=Normal]
			\SetVertexLabelOut
			% Female
			\tikzset{VertexStyle/.append style={shape=circle,minimum size=1em}}
			\Vertex[x=-14,y=8,Lpos=270,L=impersa]{MGMZ}
			\Vertex[x=-10,y=8,Lpos=270,L=keðima]{MGM}
			\Vertex[x=-6,y=8,Lpos=270,L=aversa]{MGFZ}
			\Vertex[x=4,y=8,Lpos=270,L=impersa]{PGMZ}
			\Vertex[x=8,y=8,Lpos=270,L=keðima]{PGM}
			\Vertex[x=12,y=8,Lpos=270,L=aversa]{PGFZ}
			
			\Vertex[x=-18,y=4,Lpos=90,L=impersamim]{MBW}
			\Vertex[x=-12,y=4,Lpos=270,L=impersa]{MZ}
			\Vertex[x=-1,y=4,Lpos=270,L=ima]{M}
			\Vertex[x=14,y=4,Lpos=90,L=aversamim]{FBW}
			\Vertex[x=16,y=4,Lpos=270,L=aversa]{FZ}
			
			\Vertex[x=-9,y=0,Lpos=270,L=persa]{Z}
			\Vertex[x=-2,y=0,Lpos=270,L=botra]{W}
			\Vertex[x=7,y=0,Lpos=90,L=persamim]{BW}
			
			\Vertex[x=-11,y=-4,Lpos=270,L=perbo]{ZD}
			\Vertex[x=-7,y=-4,Lpos=265,L=perbomim]{ZSW}
			\Vertex[x=-3,y=-4,Lpos=270,L=botiken]{D}
			\Vertex[x=1,y=-4,Lpos=90,L=botikemmim]{SW}
			\Vertex[x=5,y=-4,Lpos=270,L=olbo]{BD}
			\Vertex[x=9,y=-4,Lpos=90,L=olbomim]{BSW}
			
			\Vertex[x=-3,y=-8,Lpos=175,L=keðotiken]{DD}
			\Vertex[x=1,y=-8,Lpos=275,L=keðotiken]{SD}
			
			\Vertex[x=-17,y=-11,L=female]{femalekey}
			% Male
			\tikzset{VertexStyle/.append style={shape=rectangle,minimum size=1em}}
			\Vertex[x=-12,y=8,Lpos=270,L=imol]{MGMB}
			\Vertex[x=-8,y=8,Lpos=270,L=keðavo]{MGF}
			\Vertex[x=-4,y=8,Lpos=270,L=avolon]{MGFB}
			\Vertex[x=6,y=8,Lpos=270,L=imol]{PGMB}
			\Vertex[x=10,y=8,Lpos=270,L=keðavo]{PGF}
			\Vertex[x=14,y=8,Lpos=270,L=avolon]{PGFB}
			
			\Vertex[x=-16,y=4,Lpos=270,L=imol]{MB}
			\Vertex[x=-14,y=4,Lpos=90,L=imolim]{MZH}
			\Vertex[x=1,y=4,Lpos=270,L=avo]{F}
			\Vertex[x=12,y=4,Lpos=270,L=avolon]{FB}
			\Vertex[x=18,y=4,Lpos=90,L=avolonim]{FZH}
			
			\Vertex[x=-7,y=0,Lpos=90,L=olnomim]{ZH}
			\Vertex[x=2,y=0,Lpos=270,L=ǩalo]{H}
			\Vertex[x=9,y=0,Lpos=270,L=olno]{B}
			
			\Vertex[x=-9,y=-4,Lpos=90,L=perǩamim]{ZDH}
			\Vertex[x=-5,y=-4,Lpos=270,L=perǩa]{ZS}
			\Vertex[x=-1,y=-4,Lpos=275,L=ǩatikemmim]{DH}
			\Vertex[x=3,y=-4,Lpos=270,L=ǩatiken]{S}
			\Vertex[x=7,y=-4,Lpos=90,L=olǩamim]{BDH}
			\Vertex[x=11,y=-4,Lpos=270,L=olǩa]{BS}
			
			\Vertex[x=-1,y=-8,Lpos=265,L=keþaǩiken]{DS}
			\Vertex[x=3,y=-8,Lpos=5,L=keþaǩiken]{SS}
			
			\Vertex[x=-17,y=-12,L=male]{malekey}
			% Either male or female
			\tikzset{VertexStyle/.append style={shape=diamond,minimum size=0.75em}}
			\Vertex[x=-18,y=0,Lpos=265,L=konnis]{MBC}
			\Vertex[x=-16,y=0,Lpos=90,L=konnisim]{MBCSP}
			\Vertex[x=-14,y=0,Lpos=265,L=orže]{MZC}
			\Vertex[x=-12,y=0,Lpos=90,L=oržemim]{MZCSP}
			\Vertex[x=12,y=0,Lpos=265,L=komuš]{FBC}
			\Vertex[x=14,y=0,Lpos=90,L=komušim]{FBCSP}
			\Vertex[x=16,y=0,Lpos=265,L=konnis]{FZC}
			\Vertex[x=18,y=0,Lpos=90,L=konnisim]{FZCSP}
			
			\Vertex[x=-17,y=-4,Lpos=265,L=konnis]{MBCC}
			\Vertex[x=-13,y=-4,Lpos=265,L=konnis]{MZCC}
			\Vertex[x=13,y=-4,Lpos=275,L=konnis]{FBCC}
			\Vertex[x=17,y=-4,Lpos=275,L=konnis]{FZCC}
			
			\Vertex[x=-17,y=-8,Lpos=270,L=konnis]{MBCCC}
			\Vertex[x=-13,y=-8,Lpos=270,L=konnis]{MZCCC}
			\Vertex[x=-10,y=-8,Lpos=270,L=turag]{ZDC}
			\Vertex[x=-6,y=-8,Lpos=270,L=turag]{ZSC}
			\Vertex[x=6,y=-8,Lpos=270,L=turag]{BDC}
			\Vertex[x=10,y=-8,Lpos=270,L=turag]{BSC}
			\Vertex[x=13,y=-8,Lpos=270,L=konnis]{FBCCC}
			\Vertex[x=17,y=-8,Lpos=270,L=konnis]{FZCCC}
			
			\Vertex[x=-17,y=-13,L=either female or male]{eitherkey}
			% Ego
			\SetVertexLabelIn
			\tikzset{VertexStyle/.append style={shape=diamond,minimum size=3em}}
			\Vertex[x=0,y=0,Lpos=270,L=mor]{E}
			
			\Edge(MGM)(MGF)
			\Edge(PGM)(PGF)
			\Edge(MZ)(MZH)
			\Edge(MBW)(MB)
			\Edge(M)(F)
			\Edge(FBW)(FB)
			\Edge(FZ)(FZH)
			\Edge(MBC)(MBCSP)
			\Edge(MZC)(MZCSP)
			\Edge(Z)(ZH)
			\Edges(W,E,H)
			\Edge(BW)(B)
			\Edge(FBC)(FBCSP)
			\Edge(FZC)(FZCSP)
			\Edge(ZD)(ZDH)
			\Edge(ZSW)(ZS)
			\Edge(D)(DH)
			\Edge(SW)(S)
			\Edge(BD)(BDH)
			\Edge(BSW)(BS)
			
			\draw (0,4) -- (E);
			\draw (Z) -- (-9,2) -- (9,2) -- (B);
			\draw (E) -- (0,-2);
			\draw (D) -- (-3,-2) -- (3,-2) -- (S);
			\draw (-2,-4) -- (-2,-6);
			\draw (DD) -- (-3,-6) -- (-1,-6) -- (DS);
			\draw (2,-4) -- (2,-6);
			\draw (SD) -- (1,-6) -- (3,-6) -- (SS);
			\draw (-8,0) -- (-8,-2);
			\draw (ZD) -- (-11,-2) -- (-5,-2) -- (ZS);
			\draw (8,0) -- (8,-2);
			\draw (BD) -- (5,-2) -- (11,-2) -- (BS);
			\draw (-10,-4) -- (ZDC);
			\draw (-6,-4) -- (ZSC);
			\draw (6,-4) -- (BDC);
			\draw (10,-4) -- (BSC);
			\draw (-9,8) -- (-9,6);
			\draw (MB) -- (-16,6) -- (-1,6) -- (M);
			\draw (-12,6) -- (MZ);
			\draw (9,8) -- (9,6);
			\draw (F) -- (1,6) -- (16,6) -- (FZ);
			\draw (12,6) -- (FB);
			\draw (-17,4) -- (-17,2) -- (-18,2) -- (MBC);
			\draw (-13,4) -- (-13,2) -- (-14,2) -- (MZC);
			\draw (-17,0) -- (MBCC) -- (MBCCC);
			\draw (-13,0) -- (MZCC) -- (MZCCC);
			\draw (13,4) -- (13,2) -- (12,2) -- (FBC);
			\draw (17,4) -- (17,2) -- (16,2) -- (FZC);
			\draw (13,0) -- (FBCC) -- (FBCCC);
			\draw (17,0) -- (FZCC) -- (FZCCC);
			\draw (MGMZ) -- (-14,9) -- (-10,9) -- (MGM);
			\draw (-12,9) -- (MGMB);
			\draw (MGF) -- (-8,9) -- (-4,9) -- (MGFB);
			\draw (-6,9) -- (MGFZ);
			\draw (PGMZ) -- (4,9) -- (8,9) -- (PGM);
			\draw (6,9) -- (PGMB);
			\draw (PGF) -- (10,9) -- (14,9) -- (PGFB);
			\draw (12,9) -- (PGFZ);
			\end{tikzpicture}
		\end{figure}
	\end{landscape}
}

The \langtvk{} kinship system is similar to Lewis Henry Morgan's Sudanese kinship pattern, being largely descriptive with only a few classificatory terms. Siblings are distinguished from cousins, and parallel cousins are distinguished from cross cousins. Siblings and parallel cousins are identified by gender, while cross cousins are not. Parallel aunts and uncles are distinguished from cross aunts and uncles. Grandparents are identified by gender, but are otherwise undistinguished. Children and grandchildren are similarly identified by gender but otherwise undistinguished. See \autoref{fig:tvk-kinship} for a full kinship tree.

All of the kinship terms within a nuclear family have distinct names distinguishing gender and generation.

\begin{description}[leftmargin=!,labelwidth=\widthof{\bfseries daughter}]
	\item[mother] \scr{ima} \fw{ima} \phnm{i\pstrs ma}
	\item[father] \scr{avo} \fw{avo} \phnm{a\pstrs vo}
	\item[parent] \scr{vota} \fw{vota} \phnm{vo\pstrs ta}
	\item[sister] \scr{persa} \fw{persa} \phnm{per\pstrs sa}
	\item[brother] \scr{olno} \fw{olno} \phnm{ol\pstrs no}
	\item[sibling] \scr{armi} \fw{armi} \phnm{ar\pstrs mi}
	\item[wife] \scr{botra(mim)} \fw{botra(mim)} \phnm{bot\pstrs ra} or \phnm{bot.ra\pstrs mim}
	\item[husband] \scr{ǩalo(mim)} \fw{ǩalo(mim)} \phnm{xa\pstrs lo} or \phnm{xa.lo\pstrs mim}
	\item[spouse] \scr{sanim} \fw{sanim} \phnm{sa\pstrs nim}
	\item[daughter] \scr{botiken} \fw{botiken} \phnm{bo.ti\pstrs ken}
	\item[son] \scr{ǩatiken} \fw{ǩatiken} \phnm{xa.ti\pstrs ken}
	\item[child] \scr{tatiken} \fw{tatiken} \phnm{ta.ti\pstrs ken}
\end{description}

Relation by marriage is expressed with a suffix \fw{-(m)im}. This suffix can be added to several terms, such as \defn{sister}, \defn{brother}, \defn{daughter}, and \defn{son}.

\begin{description}[leftmargin=!,labelwidth=\widthof{\bfseries daughter-in-law}]
	\item[in-law] \scr{tavomim} \fw{tavomim} \phnm{ta.vo\pstrs mim}
	\item[mother-in-law] \scr{imamim} \fw{imamim} \phnm{i.ma\pstrs mim}
	\item[father-in-law] \scr{avomim} \fw{avomim} \phnm{a.vo\pstrs mim}
	\item[sister-in-law] \scr{persamim} \fw{persamim} \phnm{per.sa\pstrs mim}
	\item[brother-in-law] \scr{olnomim} \fw{olnomim} \phnm{ol.no\pstrs mim}
	\item[daughter-in-law] \scr{botikemmim} \fw{botikemmim} \phnm{bo\sstrs ti.kem\pstrs mim}
	\item[son-in-law] \scr{ǩatikemmim} \fw{ǩatikemmim} \phnm{xa\sstrs ti.kem\pstrs mim}
\end{description}

Terms for one's nieces and nephews are derived from a combination of the terms for \defn{sister} or \defn{brother} and the terms for \defn{daughter} or \defn{son}.

\begin{description}[leftmargin=!,labelwidth=\widthof{\bfseries niece-in-law (brother's daughter-in-law)}]
	\item[niece (sister's daughter)] \scr{perbo} \fw{perbo} \phnm{per\pstrs bo}
	\item[niece (brother's daughter)] \scr{olbo} \fw{olbo} \phnm{ol\pstrs bo}
	\item[niece-in-law (sister's daughter-in-law)] \scr{perbomim} \fw{perbomim} \phnm{per.bo\pstrs mim}
	\item[niece-in-law (brother's daughter-in-law)] \scr{olbomim} \fw{olbomim} \phnm{ol.bo\pstrs mim}
	\item[nephew (sister's son)] \scr{perǩa} \fw{perǩa} \phnm{per\pstrs xa}
	\item[nephew (brother's son)] \scr{olǩa} \fw{olǩa} \phnm{ol\pstrs xa}
	\item[nephew-in-law (sister's son-in-law)] \scr{perǩamim} \fw{perǩamim} \phnm{per.xa\pstrs mim}
	\item[nephew-in-law (brother's son-in-law)] \scr{olǩamim} \fw{olǩamim} \phnm{ol.xa\pstrs mim}
	\item[niefling (gender-neutral)] \scr{turag} \fw{turag} \phnm{tu\pstrs rag}
\end{description}

The child of one's niece or nephew is called \scr{turag} \fw{turag}, regardless of gender. Over time, this term became generalized to be used as a classificatory gender-neutral term for all of one's nieces and nephews along with their descendants.

One's grandchildren are distinguished by gender, but not by their parents. In other words, one's daughter's daughter is called the same term as one's son's daughter. The terms for grandchildren are formed as a compound with the word \scr{keðali} \fw{keðali} \defn{to watch}.

\begin{description}[leftmargin=!,labelwidth=\widthof{\bfseries granddaughter}]
	\item[granddaughter] \scr{keðotiken} \fw{keðotiken} \phnm{ke\sstrs ðo.ti\pstrs ken}
	\item[grandson] \scr{keþaǩiken} \fw{keþaǩiken} \phnm{ke\sstrs θa.xi\pstrs ken}
	\item[grandchild] \scr{keþantiken} \fw{keþantiken} \phnm{ke\sstrs θan.ti\pstrs ken}
\end{description}

\langtvk{} distinguishes between parallel and cross aunts and uncles. In other words, one's mother's sister is called differently than one's father's sister. These terms are further distinguished for the in-law variants with the \fw{-(m)im} suffix.

\begin{description}[leftmargin=!,labelwidth=\widthof{\bfseries uncle (mother's brother-in-law)}]
	\item[aunt (mother's sister)] \scr{impersa} \fw{impersa} \phnm{im.per\pstrs sa}
	\item[aunt (father's sister)] \scr{aversa} \fw{aversa} \phnm{a.ver\pstrs sa}
	\item[uncle (mother's brother)] \scr{imol} \fw{imol} \phnm{i\pstrs mol}
	\item[uncle (father's brother)] \scr{avolon} \fw{avolon} \phnm{a.vo\pstrs lon}
	\item[aunt (mother's sister-in-law)] \scr{impersamim} \fw{impersamim} \phnm{im\sstrs per.sa\pstrs mim}
	\item[aunt (father's sister-in-law)] \scr{aversamim} \fw{aversamim} \phnm{a\sstrs ver.sa\pstrs mim}
	\item[uncle (mother's brother-in-law)] \scr{imolim} \fw{imolim} \phnm{i.mo\pstrs lim}
	\item[uncle (father's brother-in-law)] \scr{avolonim} \fw{avolonim} \phnm{a\sstrs vo.lo\pstrs nim}
\end{description}

\langtvk{} distinguishes between parallel and cross cousins, but does not distinguish them by gender. Within parallel cousins, different terms are used to distinguish maternal vs. paternal cousins, while all cross cousins are labeled the same. Cousins' spouses are treated the same as in-laws by adding the \fw{-(m)im} suffix.

\begin{description}[leftmargin=!,labelwidth=\widthof{\bfseries cousin (father's brother's child-in-law)}]
	\item[cousin (mother's sister's child)] \scr{orže} \fw{orže} \phnm{or\pstrs ʒe}
	\item[cousin (father's brother's child)] \scr{komuš} \fw{komuš} \phnm{ko\pstrs muʃ}
	\item[cousin (cross cousin)] \scr{konnis} \fw{konnis} \phnm{kon\pstrs nis}
	\item[cousin (mother's sister's child-in-law)] \scr{oržemim} \fw{oržemim} \phnm{or.ʒe\pstrs mim}
	\item[cousin (father's brother's child-in-law)] \scr{komušim} \fw{komušim} \phnm{ko.mu\pstrs ʃim}
	\item[cousin (cross cousin-in-law)] \scr{konnisim} \fw{konnisim} \phnm{kon.ni\pstrs sim}
\end{description}

The descendants of one's cousins are not distinguished in any way, even between parallel and cross cousins. Further, they are all called by the same term as one's cross cousins: \fw{konnis}.

Grandparents are distinguished by gender, but there is no distinction made between maternal and paternal grandparents. Similar to the terms for grandchildren, the terms for grandparents are formed as a compound with the word \scr{keðali} \fw{keðali} \defn{to watch}.

\begin{description}[leftmargin=!,labelwidth=\widthof{\bfseries grandmother}]
	\item[grandmother] \scr{keðima} \fw{keðima} \phnm{ke.ði\pstrs ma}
	\item[grandfather] \scr{keðavo} \fw{keðavo} \phnm{ke.ða\pstrs vo}
	\item[grandparent] \scr{keðotav} \fw{keðotav} \phnm{ke.ðo\pstrs tav}
\end{description}

One's grandparents' siblings are called by the same terms as for one's aunts and uncles. In other words, one would call one's maternal grandmother's brother the same term as one's mother would call that person, or as one would call one's own mother's brother.

\index{kinship|)}

\section{Names}
\label{sec:tvk-names}
\index{names|(}

\subsection{Masculine Names}
\label{subsec:tvk-names-masc}

\begin{itemize}
	\item \scr{Bol} Bol \phnm{\pstrs bol}
	\item \scr{Lerk} Lerk \phnm{\pstrs lerk}
	\item \scr{Mollur} Mollur \phnm{mo\pstrs\gem{l}ur}
	\item \scr{Ote} Ote \phnm{o\pstrs te}
\end{itemize}

\subsection{Feminine Names}
\label{subsec:tvk-names-femi}

\begin{itemize}
	\item \scr{Blimva} Blimva \phnm{blim\pstrs va}
	\item \scr{Goltu} Goltu \phnm{gol\pstrs tu}
	\item \scr{Tlunda} Tlunda \phnm{tlun\pstrs da}
	\item \scr{Zarsa} Zarsa \phnm{zar\pstrs sa}
\end{itemize}

\subsection{Gender-Neutral Names}
\label{subsec:tvk-names-neut}

\begin{itemize}
	\item \scr{Erme} Erme \phnm{er\pstrs me}
	\item \scr{Inki} Inki \phnm{in\pstrs ki}
	\item \scr{Ronne} Ronne \phnm{ron\pstrs ne}
\end{itemize}

\index{names|)}