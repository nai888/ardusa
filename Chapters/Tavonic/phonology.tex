\chapter{Phonology}
\label{cha:tvk-phonology}

This chapter will present the inventory of consonants and vowels. An observational analysis of the \langtvk{} languages' syllable structures and phonotactics will follow. The chapter will close with notes on syllable stress within words and a brief exploration of intonation.

\section{\langtvk{} Phoneme Inventory}
\label{sec:tvk-phone-inventory}

\subsection{Consonants}
\label{subsec:tvk-consonants}

\afterpage{\clearpage
	\begin{sidewaystable}
		\scriptsize
		\index{consonants!inventory}\index{allophony}\index{consonants!allophones|see {allophony}}
		\caption[\langtvk{} Consonant Inventory]{\langtvk{} Phonetic Consonant Inventory (allophones in parentheses)}
		\label{tab:tvk-consonants}
		\begin{tabu} to \textheight {| r | X[c] X[c] X[c] X[c] X[c] X[c] X[c] X[c] X[c] X[c] X[c] X[c] X[c] X[c] X[c]}
			\toprule
			Consonants
			& \multicolumn{2}{c}{Bilabial}
			& \multicolumn{2}{c}{Labio-dental}
			& \multicolumn{2}{c}{Dental}
			& \multicolumn{2}{c}{Alveolar}
			& \multicolumn{2}{c}{Post-alveolar}
			& \multicolumn{2}{c}{Velar}
			\\
			\midrule
			Nasal
			&      & m    % Bilabial
			&      &      % Labiodental
			&      &      % Dental
			&      & n    % Alveolar
			&      &      % Post-alveolar
			&      & (ŋ)  % Velar
			\\
			\midrule
			Plosive
			&      &      % Bilabial
			& p    & b    % Labiodental
			& t    & d    % Dental
			&      &      % Alveolar
			&      &      % Post-alveolar
			& k    & g    % Velar
			\\
			\midrule
			Fricative
			&      &      % Bilabial
			& f    & v    % Labiodental
			& θ    & ð    % Dental
			& s    & z    % Alveolar
			& ʃ    & ʒ    % Post-alveolar
			& x    & ɣ    % Velar
			\\
			\midrule
			Flap/Tap
			&      &      % Bilabial
			&      &      % Labiodental
			&      &      % Dental
			&      & ɾ    % Alveolar
			&      &      % Post-alveolar
			&      &      % Velar
			\\
			\midrule
			Trill
			&      &      % Bilabial
			&      &      % Labiodental
			&      &      % Dental
			&      & (r)  % Alveolar
			&      &      % Post-alveolar
			&      &      % Velar
			\\
			\midrule
			Approximant
			&      &      % Bilabial
			&      &      % Labiodental
			&      &      % Dental
			&      & (ɹ)  % Alveolar
			&      &      % Post-alveolar
			&      &      % Velar
			\\
			\midrule
			Lateral
			&      &      % Bilabial
			&      &      % Labiodental
			&      &      % Dental
			&      & l    % Alveolar
			&      &      % Post-alveolar
			&      &      % Velar
			\\
			\bottomrule
		\end{tabu}
	\end{sidewaystable}
	\clearpage
	\index{consonants!romanization}
	\begin{longtabu} to \textwidth {c c c c X[l]}
		\caption{\langtvk{} Consonant Romanization}\label{tab:tvk-consromanization}\\
		\toprule
		Phone & Phoneme & Romanization & English & Notes\\
		\midrule
		\endhead
		\multicolumn{4}{r}{\textit{continued on the next page\ldots}}\\
		\endfoot
		\bottomrule
		\endlastfoot
		\phnt{m} & \phnm{m} & \orth{m} & \orth{m} & \\
		\midrule
		\phnt{n} & \phnm{n} & \orth{n} & \orth{n} & \\
		\midrule
		\phnt{ŋ} & \phnm{n} & \orth{n} & \orth{n} & \phnm{n} becomes velarized before a velar consonant\\
		\midrule
		\phnt{p} & \phnm{p} & \orth{p} & \orth{p} & \\
		\midrule
		\phnt{b} & \phnm{b} & \orth{b} & \orth{b} & \\
		\midrule
		\phnt{t} & \phnm{t} & \orth{t} & \orth{t} & \\
		\midrule
		\phnt{d} & \phnm{d} & \orth{d} & \orth{d} & \\
		\midrule
		\phnt{k} & \phnm{k} & \orth{k} & \orth{k} & \\
		\midrule
		\phnt{g} & \phnm{g} & \orth{g} & \orth{g} & \\
		\midrule
		\phnt{f} & \phnm{f} & \orth{f} & \orth{f} & \\
		\midrule
		\phnt{v} & \phnm{v} & \orth{v} & \orth{v} & \\
		\midrule
		\phnt{θ} & \phnm{θ} & \orth{þ} & \orth{th} & \\
		\midrule
		\phnt{ð} & \phnm{ð} & \orth{ð} & \orth{dh} & \\
		\midrule
		\phnt{s} & \phnm{s} & \orth{s} & \orth{s} & \\
		\midrule
		\phnt{z} & \phnm{z} & \orth{z} & \orth{z} & \\
		\midrule
		\phnt{ʃ} & \phnm{ʃ} & \orth{š} & \orth{sh} & \\
		\midrule
		\phnt{ʒ} & \phnm{ʒ} & \orth{ž} & \orth{zh} & \\
		\midrule
		\phnt{x} & \phnm{x} & \orth{ǩ} & \orth{kh} & \\
		\midrule
		\phnt{ɣ} & \phnm{ɣ} & \orth{ǧ} & \orth{gh} & \\
		\midrule
		\phnt{ɾ} & \phnm{r} & \orth{r} & \orth{r} & \\
		\midrule
		\phnt{r} & \phnm{r} & \orth{rr} & \orth{rr} & \orth{r} is trilled when doubled \\
		\midrule
		\phnt{ɹ} & \phnm{r} & \orth{r} & \orth{r} & \orth{r} is occasionally pronounced as an approximant when a part of a consonant cluster \\
		\midrule
		\phnt{l} & \phnm{l} & \orth{l} & \orth{l} & \\
	\end{longtabu}
	\clearpage
}

With approximately 20 consonants, \langtvk{} has an \enquote{average} inventory.\autocite{wals-1} \autoref{tab:tvk-consonants} shows the full chart of consonant phonemes, along with several allophones enclosed in parentheses. \autoref{tab:tvk-consromanization} shows how each consonant in \langtvk{} is romanized.

Despite its \enquote{average} inventory of consonants, there are many more allophones\index{allophony} that occur in the language. First, any doubled consonant is realized as a geminated\index{consonants!gemination} (elongated) consonant.

\pex<gemcons>
	\fw{unner} \phnm{u\gem{n}er} \defn{empire}
\xe

Thus, example~\getfullref{gemcons} above is realized with a lengthened \phnt{n}. A doubled \orth{r} is similarly geminated, but the pronunciation changes from a flap/tap to a trill.

The remaining allophones\index{allophony} occur due to various sound change processes, mostly by assimilation. For example, \phnm{n} becomes velarized when it appears immediately before a velar consonant.

\ex<velarn>
	\fw{tavonga} \phnt{ta.voŋ\pstrs ga} \defn{humanlike}
\xe

As discussed above, \orth{r} can be pronounced as both a tap/flap \phnt{ɾ} and as a trill \phnt{r}. Additionally, when part of certain consonant clusters, it can be pronounced as an approximant \phnt{ɹ}. This primarily occurs when the \orth{r} leads into a cluster or immediately follows a nasal.

\ex<velarn>
	\fw{frorgali} \phnt{fɾoɹ.\pstrs ga.li} \defn{to un-see}
\xe

\subsection{Vowels}
\label{subsec:tvk-vowels}

\afterpage{\clearpage
	\begin{table}\centering
		\index{vowels!inventory}
		\caption{\langtvk{} Vowel Inventory}
		\label{tab:tvk-vowels}
		{\large
			\begin{vowel}
				\putcvowel{i}{1}
				\putcvowel{e}{2}
				\putcvowel{a}{4}
				\putcvowel{o}{7}
				\putcvowel{u}{8}
			\end{vowel}
		}
	\end{table}
}

\langtvk{} distinguishes five vowel qualities, as shown in \autoref{tab:tvk-vowels}, giving it an \enquote{average} inventory.\autocite{wals-2} This means the consonant--vowel ratio is 20:5 or 4.0, which is \enquote{average}.\autocite{wals-3} \langtvk{} does not distinguish long and short vowels and does not allow any diphthongs.

Note that all \langtvk{} vowels have a very rigid acceptable pronunciation with very little variance.

\pex<gemvowels>
	\a<i> \fw{akrinsali} \defn{to rewrite} is pronounced \phnm{ak.rin\pstrs sa.li}. \orth{i} is not pronounced with a lax \phnt{ɪ} in closed syllables (i.e., \phnm{ak.rɪn\pstrs sa.li})
	\a<e> \fw{tloþevem} \defn{permission} is pronounced \phnm{tlo.θe\pstrs vem}. \orth{e} is not pronounced with a central \phnt{ə} in unaccented syllables or an open \phnt{ɛ} in closed syllables (i.e., \phnm{tlo.θə\pstrs vɛm}), nor is it diphthongized to \phnt{e\nsyl{ɪ}} (i.e., \phnm{tlo.θe\pstrs ve\nsyl{ɪ}m})
	\a<a> \fw{ǩalo} \defn{man} is pronounced \phnm{xa\pstrs lo}. \orth{a} is not pronounced with a raised \phnt{æ} (i.e., \phnm{xæ\pstrs lo}), a backed \phnt{ɑ} (i.e., \phnm{xɑ\pstrs lo}), or a centralized \phnt{ɜ} (i.e., \phnm{xɜ\pstrs lo})
	\a<o> \fw{esondi} \defn{arable} is pronounced \phnm{e.son\pstrs di}. \orth{o} is not pronounced with an open \phnt{ɔ} (i.e., \phnt{e.sɔn\pstrs di}), nor is it diphthongized to \phnt{o\nsyl{u}} (i.e., \phnm{e.so\nsyl{u}n\pstrs di})
	\a<u> \fw{frumbali} \defn{to misunderstand} is pronounced \phnm{frum\pstrs ba.li}. \orth{u} is not pronounced with an open \phnt{ʌ} (i.e., \phnm{frʌm\pstrs ba.li}) or a centralized \phnt{ʊ} (i.e., \phnm{frʊm\pstrs ba.li})
\xe

\section{\langtvk{} Phonotactics}
\label{sec:tvk-phonotactics}

At the time of writing, there does not yet exist a sufficient corpus for a meaningful statistical analysis of \langtvk's phonotactics. Therefore, this section will present only a cursory observational analysis.

\subsection{Syllable Structures}
\label{subsec:tvk-syll-struc}

Placeholder

\subsection{Phonological Changes}
\label{subsec:tvk-phone-changes}

Placeholder

\subsection{Syllable Parsing}
\label{subsec:tvk-syll-parse}

Placeholder

\subsection{Number of Syllables per Word}
\label{subsec:tvk-num-syll}

Placeholder

\section{\langtvk{} Prosody}
\label{sec:tvk-prosody}

Placeholder

\subsection{Syllable Weight}
\label{subsec:tvk-syll-weight}

Placeholder

\subsection{Word Stress}
\label{subsec:tvk-word-stress}

Placeholder

\subsection{Intonation}
\label{subsec:tvk-intonation}

Placeholder

\section{\langank{} Phoneme Inventory}
\label{sec:ank-phone-inventory}

Placeholder

\subsection{Consonants}
\label{subsec:ank-consonants}

Placeholder

\subsection{Vowels}
\label{subsec:ank-vowels}

Placeholder

\section{\langank{} Phonotactics}
\label{sec:ank-phonotactics}

Placeholder

\subsection{Syllable Structures}
\label{subsec:ank-syll-struc}

Placeholder

\subsection{Phonological Changes}
\label{subsec:ank-phone-changes}

Placeholder

\subsection{Syllable Parsing}
\label{subsec:ank-syll-parse}

Placeholder

\subsection{Number of Syllables per Word}
\label{subsec:ank-num-syll}

Placeholder

\section{\langank{} Prosody}
\label{sec:ank-prosody}

Placeholder

\subsection{Syllable Weight}
\label{subsec:ank-syll-weight}

Placeholder

\subsection{Word Stress}
\label{subsec:ank-word-stress}

Placeholder

\subsection{Intonation}
\label{subsec:ank-intonation}

Placeholder

\section{\langrdk{} Phoneme Inventory}
\label{sec:rdk-phone-inventory}

Placeholder

\subsection{Consonants}
\label{subsec:rdk-consonants}

Placeholder

\subsection{Vowels}
\label{subsec:rdk-vowels}

Placeholder

\section{\langrdk{} Phonotactics}
\label{sec:rdk-phonotactics}

Placeholder

\subsection{Syllable Structures}
\label{subsec:rdk-syll-struc}

Placeholder

\subsection{Phonological Changes}
\label{subsec:rdk-phone-changes}

Placeholder

\subsection{Syllable Parsing}
\label{subsec:rdk-syll-parse}

Placeholder

\subsection{Number of Syllables per Word}
\label{subsec:rdk-num-syll}

Placeholder

\section{\langrdk{} Prosody}
\label{sec:rdk-prosody}

Placeholder

\subsection{Syllable Weight}
\label{subsec:rdk-syll-weight}

Placeholder

\subsection{Word Stress}
\label{subsec:rdk-word-stress}

Placeholder

\subsection{Intonation}
\label{subsec:rdk-intonation}

Placeholder