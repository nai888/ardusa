\chapter{Phonology and Orthography}
\label{cha:tvk-phonology}

This chapter will present the phonological inventory of consonants and vowels and the orthography used to write them. An observational analysis of the \langtvk{} languages' syllable structures and phonotactics will follow. The chapter will close with notes on syllable stress within words and a brief exploration of intonation.

\section{Phoneme Inventory}
\label{sec:tvk-phone-inventory}

\subsection{Consonants}
\label{subsec:tvk-consonants}
\index{consonants|(}

\afterpage{\clearpage
	\begin{sidewaystable}
		\scriptsize
		\index{consonants!inventory}\index{allophony}\index{consonants!allophones|see {allophony}}
		\caption[\langtvk{} Consonant Inventory]{\langtvk{} Phonetic Consonant Inventory (allophones in parentheses)}
		\label{tab:tvk-consonants}
		\begin{tabu} to \textheight {| r | X[c] X[c] X[c] X[c] X[c] X[c] X[c] X[c] X[c] X[c] X[c] X[c] X[c] X[c] X[c]}
			\toprule
			Consonants
			& \multicolumn{2}{c}{Bilabial}
			& \multicolumn{2}{c}{Labio-dental}
			& \multicolumn{2}{c}{Dental}
			& \multicolumn{2}{c}{Alveolar}
			& \multicolumn{2}{c}{Post-alveolar}
			& \multicolumn{2}{c}{Velar}
			\\
			\midrule
			Nasal
			&      & m    % Bilabial
			&      &      % Labiodental
			&      &      % Dental
			&      & n    % Alveolar
			&      &      % Post-alveolar
			&      & (ŋ)  % Velar
			\\
			\midrule
			Plosive
			&      &      % Bilabial
			& p    & b    % Labiodental
			& t    & d    % Dental
			&      &      % Alveolar
			&      &      % Post-alveolar
			& k    & g    % Velar
			\\
			\midrule
			Fricative
			&      &      % Bilabial
			& f    & v    % Labiodental
			& θ    & ð    % Dental
			& s    & z    % Alveolar
			& ʃ    & ʒ    % Post-alveolar
			& x    & ɣ    % Velar
			\\
			\midrule
			Flap/Tap
			&      &      % Bilabial
			&      &      % Labiodental
			&      &      % Dental
			&      & ɾ    % Alveolar
			&      &      % Post-alveolar
			&      &      % Velar
			\\
			\midrule
			Trill
			&      &      % Bilabial
			&      &      % Labiodental
			&      &      % Dental
			&      & (r)  % Alveolar
			&      &      % Post-alveolar
			&      &      % Velar
			\\
			\midrule
			Approximant
			&      &      % Bilabial
			&      &      % Labiodental
			&      &      % Dental
			&      & (ɹ)  % Alveolar
			&      &      % Post-alveolar
			&      &      % Velar
			\\
			\midrule
			Lateral
			&      &      % Bilabial
			&      &      % Labiodental
			&      &      % Dental
			&      & l    % Alveolar
			&      &      % Post-alveolar
			&      &      % Velar
			\\
			\bottomrule
		\end{tabu}
	\end{sidewaystable}
	\clearpage
	\index{consonants!romanization}
	\begin{longtabu} to \textwidth {c c c c c X[l]}
		\caption{\langtvk{} Consonant Romanization}\label{tab:tvk-consromanization}\\
		\toprule
		\textbf{Phone} & \textbf{Phoneme} & \textbf{Script} & \textbf{Romanization} & \textbf{English} & \textbf{Notes}\\
		\midrule
		\endhead
		\multicolumn{5}{r}{\textit{continued on the next page\ldots}}\\
		\endfoot
		\bottomrule
		\endlastfoot
		\phnt{m} & \phnm{m} & \orth{\scr{m}} & \orth{m} & \orth{m} & \\
		\midrule
		\phnt{n} & \phnm{n} & \orth{\scr{n}} & \orth{n} & \orth{n} & \\
		\midrule
		\phnt{ŋ} & \phnm{n} & \orth{\scr{n}} & \orth{n} & \orth{n} & \phnm{n} becomes velarized before a velar consonant\\
		\midrule
		\phnt{p} & \phnm{p} & \orth{\scr{p}} & \orth{p} & \orth{p} & \\
		\midrule
		\phnt{b} & \phnm{b} & \orth{\scr{b}} & \orth{b} & \orth{b} & \\
		\midrule
		\phnt{t} & \phnm{t} & \orth{\scr{t}} & \orth{t} & \orth{t} & \\
		\midrule
		\phnt{d} & \phnm{d} & \orth{\scr{d}} & \orth{d} & \orth{d} & \\
		\midrule
		\phnt{k} & \phnm{k} & \orth{\scr{k}} & \orth{k} & \orth{k} & \\
		\midrule
		\phnt{g} & \phnm{g} & \orth{\scr{g}} & \orth{g} & \orth{g} & \\
		\midrule
		\phnt{f} & \phnm{f} & \orth{\scr{f}} & \orth{f} & \orth{f} & \\
		\midrule
		\phnt{v} & \phnm{v} & \orth{\scr{v}} & \orth{v} & \orth{v} & \\
		\midrule
		\phnt{θ} & \phnm{θ} & \orth{\scr{þ}} & \orth{þ} & \orth{th} & \\
		\midrule
		\phnt{ð} & \phnm{ð} & \orth{\scr{ð}} & \orth{ð} & \orth{dh} & \\
		\midrule
		\phnt{s} & \phnm{s} & \orth{\scr{s}} & \orth{s} & \orth{s} & \\
		\midrule
		\phnt{z} & \phnm{z} & \orth{\scr{z}} & \orth{z} & \orth{z} & \\
		\midrule
		\phnt{ʃ} & \phnm{ʃ} & \orth{\scr{š}} & \orth{š} & \orth{sh} & \\
		\midrule
		\phnt{ʒ} & \phnm{ʒ} & \orth{\scr{ž}} & \orth{ž} & \orth{zh} & \\
		\midrule
		\phnt{x} & \phnm{x} & \orth{\scr{ǩ}} & \orth{ǩ} & \orth{kh} & \\
		\midrule
		\phnt{ɣ} & \phnm{ɣ} & \orth{\scr{ǧ}} & \orth{ǧ} & \orth{gh} & \\
		\midrule
		\phnt{ɾ} & \phnm{r} & \orth{\scr{r}} & \orth{r} & \orth{r} & \\
		\midrule
		\phnt{r} & \phnm{r} & \orth{\scr{rr}} & \orth{rr} & \orth{rr} & \orth{r} is trilled when doubled \\
		\midrule
		\phnt{ɹ} & \phnm{r} & \orth{\scr{r}} & \orth{r} & \orth{r} & \orth{r} is occasionally pronounced as an approximant when a part of a consonant cluster \\
		\midrule
		\phnt{l} & \phnm{l} & \orth{\scr{l}} & \orth{l} & \orth{l} & \\
	\end{longtabu}
	\clearpage
}

With approximately 20 consonants, \langtvk{} has an \enquote{average} inventory.\autocite{wals-1} \autoref{tab:tvk-consonants} shows the full chart of consonant phonemes, along with several allophones enclosed in parentheses. \autoref{tab:tvk-consromanization} shows how each consonant in \langtvk{} is romanized.

Despite its \enquote{average} inventory of consonants, there are many more allophones\index{allophony} that occur in the language. First, any doubled consonant is realized as a geminated\index{consonants!gemination} (elongated) consonant.

\pex<gemcons>
	\scr{unner} \fw{unner} \phnm{u\pstrs\gem{n}er} \defn{empire}
\xe

Thus, example~\getfullref{gemcons} above is realized with a lengthened \phnt{n}. A doubled \orth{r} is similarly geminated, but the pronunciation changes from a flap/tap to a trill.

The remaining allophones\index{allophony} occur due to various sound change processes, mostly by assimilation. For example, \phnm{n} becomes velarized\index{consonants!velarization} when it appears immediately before a velar consonant.

\ex<velarn>
	\scr{tavonga} \fw{tavonga} \phnt{ta.voŋ\pstrs ga} \defn{humanlike}
\xe

As discussed above, \orth{r} can be pronounced as both a tap/flap \phnt{ɾ} and as a trill \phnt{r}. Additionally, when part of certain consonant clusters, it can be pronounced as an approximant \phnt{ɹ}. This primarily occurs when the \orth{r} leads into a cluster or immediately follows a nasal.

\ex<velarn>
	\scr{frorgali} \fw{frorgali} \phnt{fɾoɹ.\pstrs ga.li} \defn{to un-see}
\xe

\index{consonants|)}
\subsection{Vowels}
\label{subsec:tvk-vowels}

\afterpage{\clearpage
	\begin{table}\centering
		\index{vowels!inventory}
		\caption{\langtvk{} Vowel Inventory}
		\label{tab:tvk-vowels}
		{\large
			\begin{vowel}
				\putcvowel{i \orth{\scr{i}}}{1}
				\putcvowel{e \orth{\scr{e}}}{2}
				\putcvowel{a \orth{\scr{a}}}{4}
				\putcvowel{o \orth{\scr{o}}}{7}
				\putcvowel{u \orth{\scr{u}}}{8}
			\end{vowel}
		}
	\end{table}
}

\langtvk{} distinguishes five vowel qualities, as shown in \autoref{tab:tvk-vowels}, giving it an \enquote{average} inventory.\autocite{wals-2} This means the consonant--vowel ratio is 20:5 or 4.0, which is \enquote{average}.\autocite{wals-3} \langtvk{} does not distinguish long and short vowels and does not allow any diphthongs.

Note that all \langtvk{} vowels have a very rigid acceptable pronunciation with very little variance.

\pex<tvk-vowels>
	\a<i> \scr{akrinsali} \fw{akrinsali} \defn{to rewrite} is pronounced \phnm{ak.rin\pstrs sa.li}. \orth{i} is not pronounced with a lax \phnt{ɪ} in closed syllables (i.e., \phnm{ak.rɪn\pstrs sa.li})
	\a<e> \scr{eðerik} \fw{eðerik} \defn{pencil} is pronounced \phnm{e.ðe\pstrs rik}. \orth{e} is not pronounced with an open \phnt{ɛ} in closed syllables or syllables with secondary stress or with a central \phnt{ə} in unaccented syllables (i.e., \phnm{ɛ.ðə\pstrs rik}), nor is it diphthongized to \phnt{e\nsyl{ɪ}} (i.e., \phnm{e\nsyl{ɪ}.ðe\pstrs rik})
	\a<a> \scr{ǩalo} \fw{ǩalo} \defn{man} is pronounced \phnm{xa\pstrs lo}. \orth{a} is not pronounced with a raised \phnt{æ} (i.e., \phnm{xæ\pstrs lo}), a backed \phnt{ɑ} (i.e., \phnm{xɑ\pstrs lo}), or a centralized \phnt{ɜ} (i.e., \phnm{xɜ\pstrs lo})
	\a<o> \scr{esondi} \fw{esondi} \defn{arable} is pronounced \phnm{e.son\pstrs di}. \orth{o} is not pronounced with an open \phnt{ɔ} (i.e., \phnt{e.sɔn\pstrs di}), nor is it diphthongized to \phnt{o\nsyl{u}} (i.e., \phnm{e.so\nsyl{u}n\pstrs di})
	\a<u> \scr{frumbali} \fw{frumbali} \defn{to misunderstand} is pronounced \phnm{frum\pstrs ba.li}. \orth{u} is not pronounced with an open \phnt{ʌ} (i.e., \phnm{frʌm\pstrs ba.li}) or a centralized \phnt{ʊ} (i.e., \phnm{frʊm\pstrs ba.li})
\xe

\section{Phonotactics}
\label{sec:tvk-phonotactics}

At the time of writing, there does not yet exist a sufficient corpus for a meaningful statistical analysis of \langtvk's phonotactics. Therefore, this section will present only a cursory observational analysis.

\subsection{Syllable Structures}
\label{subsec:tvk-syll-struc}

Syllables in \langtvk{} must contain a vowel to serve as the syllable's nucleus. Each syllable will only have at most one vowel. Syllables may also include any single consonant or one of a limited set of two-consonant clusters as the onset, coda, or both.

In other words, the most complex syllable structure allowed in \langtvk{} is CCVCC, with restrictions on the allowable consonant clusters, giving \langtvk{} a \enquote{moderately complex syllable structure}.\autocite{wals-12}

\subsubsection{V}

Since vowels are required to form a syllable nucleus, the most basic syllable structure is simply a vowel (V). Any syllable that starts with a vowel will occur exclusively at the beginning of a word.

\pex<tvk-syll-V>
	\a<e1> \scr{e} \fw{e} \phnm{e} \defn{in} or \defn{on}
	\a<e2> \scr{eðer} \fw{eðer} \phnm{e\pstrs ðer} \defn{pen}
	\a<a> \scr{abom} \fw{abom} \phnm{a\pstrs bom} \defn{two}
	\a<o> \scr{oko} \fw{oko} \phnm{o\pstrs ko} \defn{dog}
	\a<u> \scr{usukon} \fw{usukon} \phnm{u.su\pstrs kon} \defn{possessor}
\xe

\subsubsection{C}

A syllable can contain a single-consonant onset or coda. There is no restriction on which consonants may appear in the onset or coda with just one consonant. CV is likely the most frequent type of syllable in \langtvk, with CVC probably being the second-most-frequent syllable type.

\pex<tvk-syll-CV>
	\a<ga> \scr{ga} \fw{ga} \phnm{ga} \defn{but}
	\a<lu> \scr{lu} \fw{lu} \phnm{lu} \defn{and}
	\a<mo> \scr{mo} \fw{mo} \phnm{mo} \defn{with}
	\a<ǩalo> \scr{ǩalo} \fw{ǩalo} \phnm{xa\pstrs lo} \defn{man}
	\a<šeðo> \scr{šeðo} \fw{šeðo} \phnm{\pstrs ʃe.ðo} \gloss{run.\Pst.\Ind.\Prg} \defn{was running}
	\a<ab> \scr{ablu} \fw{ablu} \phnm{ab\pstrs lu} \defn{cat}
	\a<ur> \scr{urda} \fw{urda} \phnm{ur\pstrs da} \defn{safe}
	\a<dir> \scr{akradir} \fw{akradir} \phnm{ak.ra\pstrs dir} \defn{writing implement}
	\a<nak> \scr{esonak} \fw{esonak} \phnm{e.so\pstrs nak} \defn{citizen}
\xe

Across two syllables within a word, there are restrictions on the combination of consonants that are possible. At such syllable boundaries, a plosive\footnote{\label{ftn:tvk-plosives}i.e., \orth{p}, \orth{t}, \orth{k}, \orth{b}, \orth{d}, or \orth{g}} or a fricative\footnote{\label{ftn:tvk-fricatives}i.e., \orth{f}, \orth{þ}, \orth{s}, \orth{š}, \orth{ǩ}, \orth{v}, \orth{ð}, \orth{z}, \orth{ž}, or \orth{ǧ}} can be followed by a liquid\footnote{\label{ftn:tvk-liquids}i.e., \orth{l} or \orth{r}}; a liquid may be followed by a plosive, fricative, nasal\footnote{\label{ftn:tvk-nasals}i.e., \orth{m} or \orth{n}}, or a different liquid; or a nasal can be followed by any other consonant.

\pex<tvk-syll-VCCV>
	\a<el> \scr{elbi} \fw{elbi} \phnm{el\pstrs bi} \defn{egg}
	\a<on> \scr{ongo} \fw{ongo} \phnm{on\pstrs go} \defn{pan}
	\a<ǩalven> \scr{ǩalven} \fw{ǩalven} \phnm{xal\pstrs ven} \defn{400}
	\a<lun> \scr{ablunga} \fw{ablunga} \phnm{ab.lun\pstrs ga} \defn{catlike}
\xe

\subsubsection{CC}

Syllables may contain onsets or codas with two consonants, but these shapes are less common and there are restrictions on the possible combinations. Syllable onsets with two consonants may only occur at the beginning of a word and may only contain a plosive or fricative followed by a liquid. Syllable codas with two consonants may only occur at the end of a word and may only contain a liquid followed by a plosive.

\pex<tvk-syll-CC>
	\a<pr> \scr{pral} \fw{pral} \phnm{pral} \defn{some}
	\a<tl> \scr{tloþendi} \fw{tloþendi} \phnm{tlo.θen\pstrs di} \defn{permittable}
	\a<fr> \scr{frandi} \fw{frandi} \phnm{fran\pstrs di} \defn{visible}
	\a<lk> \scr{šolk} \fw{šolk} \phnm{ʃolk} \defn{yet}
	\a<lš> \scr{delš} \fw{delš} \phnm{delʃ} \defn{zero}
\xe

\subsection{Phonological Changes}
\label{subsec:tvk-phone-changes}

Placeholder

\subsection{Syllable Parsing}
\label{subsec:tvk-syll-parse}

Placeholder

\subsection{Number of Syllables per Word}
\label{subsec:tvk-num-syll}

Placeholder

\section{Prosody}
\label{sec:tvk-prosody}

Placeholder

\subsection{Syllable Weight}
\label{subsec:tvk-syll-weight}

Placeholder

\subsection{Word Stress}
\label{subsec:tvk-word-stress}

Placeholder

\subsection{Intonation}
\label{subsec:tvk-intonation}

Placeholder

\section{Orthography}
\label{sec:tvk-orthography}

Placeholder