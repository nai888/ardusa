\chapter{Kinship Terminology}
\label{app:kinship}
\index{kinship|(}

The \lang{} kinship system is large descriptive, with only a few classificatory terms. Siblings are distinguished from cousins, and parallel cousins are distinguished from cross cousins. Siblings and parallel cousins are identified by gender, while cross cousins are not. Parallel aunts and uncles are distinguished from cross aunts and uncles. Grandparents are identified by gender, but are otherwise undistinguished. Children and grandchildren are similarly identified by gender but otherwise undistinguished. See \autoref{fig:kinship} for a full kinship tree.

All of the kinship terms within a nuclear family have distinct names distinguishing gender and generation.

\exdisplay\noexno
	\begin{tabu} {l l}
		mother & \fw{júói}\\
		father & \fw{uutói}\\
		sister & \fw{náatói}\\
		brother & \fw{toóbói}\\
		wife & \fw{ólói}\\
		husband & \fw{udói}\\
		daughter & \fw{lárrói}\\
		son & \fw{zohiói}\\
	\end{tabu}
\xe

Relation by marriage is expressed with a prefix \fw{wen-}. This prefix can be added to several terms, such as \defn{sister}, \defn{brother}, \defn{daughter}, and \defn{son}.

\exdisplay\noexno
	\begin{tabu} {l l}
		mother-in-law & \fw{wenjúói}\\
		father-in-law & \fw{wenuutói}\\
		sister-in-law & \fw{wennáatói}\\
		brother-in-law & \fw{wentoóbói}\\
		daughter-in-law & \fw{wenlárrói}\\
		son-in-law & \fw{wenzohiói}\\
	\end{tabu}
\xe

Terms for one's nieces and nephews are derived from a combination of the terms for \defn{sister} or \defn{brother} and the terms for \defn{daughter} or \defn{son}.

\exdisplay\noexno
	\begin{tabu} {l l l}
		niece & \fw{náalár} & sister's daughter\\
		niece & \fw{toólár} & brother's daughter\\
		niece-in-law & \fw{wennáalár} & sister's daughter-in-law\\
		niece-in-law & \fw{wentoólár} & brother's daughter-in-law\\
		nephew & \fw{náazo} & sister's son\\
		nephew & \fw{toózo} & brother's son\\
		nephew-in-law & \fw{wennáazo} & sister's son-in-law\\
		nephew-in-law & \fw{wentoózo} & brother's son-in-law\\
		niefling & \fw{tírtrabói} & gender-neutral term for niece or nephew\\
	\end{tabu}
\xe

One's grandchildren are distinguished by gender, but not by their parents. In other words, one's daughter's daughter is called the same term as one's son's daughter. As discussed in \autoref{sec:morphological-processes}, these terms are derived using reduplication\index{reduplication}.

\exdisplay\noexno
	\begin{tabu} {l l}
		granddaughter & \fw{lálárrói}\\
		grandson & \fw{zozohiói}\\
	\end{tabu}
\xe

The children of one's nieces and nephews are all called \fw{tírtrabói}, regardless of their gender. This term is identical to the gender-neutral term for one's nieces and nephews.

\lang{} distinguishes between parallel and cross aunts and uncles. In other words, one's mother's sister is called differently than one's father's sister. In fact, the term for cross aunts is identical to the term for parallel aunt-in-laws, and the term for cross uncles is identical to the term for parallel uncle-in-laws. Similar to the terms for grandchildren, the terms for parallel aunts and uncles are derived using reduplication.

\exdisplay\noexno
	\begin{tabu} {l l l}
		aunt & \fw{nánáatói} & mother's sister\\
		aunt & \fw{náugámói} & father's sister\\
		aunt & \fw{náugámói} & aunt by marriage\\
		uncle & \fw{totoóbói} & father's brother\\
		uncle & \fw{tougámói} & mother's brother\\
		uncle & \fw{tougámói} & uncle by marriage\\
	\end{tabu}
\xe

\lang{} distinguishes between parallel and cross cousins, but does not distinguish them by gender. Within parallel cousins, different terms are used to distinguish maternal vs. paternal cousins. Cousins' spouses are treated the same as in-laws by adding the \fw{wen-} prefix.

\exdisplay\noexno
	\begin{tabu} {l l l}
		cousin & \fw{tírnánáatói} & mother's sister's child\\
		cousin-in-law & \fw{wentírnánáatói} & mother's sister's child's spouse\\
		cousin & \fw{tírtotoóbói} & father's brother's child\\
		cousin-in-law & \fw{wentírtotoóbói} & father's brother's child's spouse\\
		cousin & \fw{tratrabói} & cross cousin\\
		cousin-in-law & \fw{wentratrabói} & cross cousin's spouse\\
	\end{tabu}
\xe

The children and grandchildren of one's cousins are not distinguished in any way, even between parallel and cross cousins. In fact, they are all called by the same term as one's cross cousins, \fw{tratrabói}.

Grandparents are distinguished by gender, but there is no distinction made between maternal and paternal grandparents. Similar to the terms for grandchildren, the terms for grandparents are derived using reduplication.

\exdisplay\noexno
	\begin{tabu} {l l}
		grandmother & \fw{jújúói}\\
		grandfather & \fw{utuutói}\\
	\end{tabu}
\xe

One's grandparents' siblings are called by the same terms as for one's aunts and uncles. In other words, one would call one's maternal grandmother's brother the same term as one's mother would call that person.

\exdisplay\noexno
	\begin{tabu} {l l l}
		grand-aunt & \fw{nánáatói} & grandmother's sister\\
		grand-aunt & \fw{náugámói} & grandfather's sister\\
		grand-aunt & \fw{náugámói} & grandparent's sister-in-law\\
		grand-uncle & \fw{totoóbói} & grandfather's brother\\
		grand-uncle & \fw{tougámói} & grandmother's brother\\
		grand-uncle & \fw{tougámói} & grandparent's brother-in-law\\
	\end{tabu}
\xe

\begin{sidewaysfigure}[h]\centering
	\caption{Kinship Tree}
	\label{fig:kinship}
	\index{kinship}
	\tiny
	\begin{tikzpicture}[scale=0.5]
	\GraphInit[vstyle=Normal]
	\SetVertexLabelOut
	% Female
	\tikzset{VertexStyle/.append style={shape=circle,minimum size=1em}}
	\Vertex[x=-14,y=8,Lpos=180,L=nánáatói]{MGMS}
	\Vertex[x=-10,y=8,Lpos=270,L=jújúói]{MGM}
	\Vertex[x=-6,y=8,Lpos=275,L=náugámói]{MGFS}
	\Vertex[x=4,y=8,Lpos=180,L=nánáatói]{PGMS}
	\Vertex[x=8,y=8,Lpos=270,L=jújúói]{PGM}
	\Vertex[x=12,y=8,Lpos=275,L=náugámói]{PGFS}
	
	\Vertex[x=-18,y=4,Lpos=90,L=náugámói]{MBW}
	\Vertex[x=-12,y=4,Lpos=0,L=nánáatói]{MS}
	\Vertex[x=-1,y=4,Lpos=270,L=júói]{M}
	\Vertex[x=14,y=4,Lpos=90,L=náugámói]{FBW}
	\Vertex[x=16,y=4,Lpos=265,L=náugámói]{FS}
	
	\Vertex[x=-9,y=0,Lpos=265,L=náatói]{S}
	\Vertex[x=-2,y=0,Lpos=270,L=ólói]{W}
	\Vertex[x=7,y=0,Lpos=265,L=wennáatói]{BW}
	
	\Vertex[x=-11,y=-4,Lpos=270,L=náalár]{SD}
	\Vertex[x=-7,y=-4,Lpos=90,L=wennáalár]{SNW}
	\Vertex[x=-3,y=-4,Lpos=270,L=lárrói]{D}
	\Vertex[x=1,y=-4,Lpos=265,L=wenlárrói]{NW}
	\Vertex[x=5,y=-4,Lpos=270,L=toólár]{BD}
	\Vertex[x=9,y=-4,Lpos=90,L=wentoólár]{BNW}
	
	\Vertex[x=-3,y=-8,Lpos=270,L=lálárrói]{DD}
	\Vertex[x=1,y=-8,Lpos=270,L=lálárrói]{ND}
	
	\Vertex[x=-17,y=-11,L=female]{femalekey}
	% Male
	\tikzset{VertexStyle/.append style={shape=rectangle,minimum size=1em}}
	\Vertex[x=-12,y=8,Lpos=265,L=tougámói]{MGMB}
	\Vertex[x=-8,y=8,Lpos=270,L=utuutói]{MGF}
	\Vertex[x=-4,y=8,Lpos=0,L=totoóbói]{MGFB}
	\Vertex[x=6,y=8,Lpos=265,L=tougámói]{PGMB}
	\Vertex[x=10,y=8,Lpos=270,L=utuutói]{PGF}
	\Vertex[x=14,y=8,Lpos=0,L=totoóbói]{PGFB}
	
	\Vertex[x=-16,y=4,Lpos=275,L=tougámói]{MB}
	\Vertex[x=-14,y=4,Lpos=90,L=tougámói]{MSH}
	\Vertex[x=1,y=4,Lpos=270,L=uutói]{F}
	\Vertex[x=12,y=4,Lpos=180,L=totoóbói]{FB}
	\Vertex[x=18,y=4,Lpos=90,L=tougámói]{FSH}
	
	\Vertex[x=-7,y=0,Lpos=275,L=wentoóbói]{SH}
	\Vertex[x=2,y=0,Lpos=270,L=udói]{H}
	\Vertex[x=9,y=0,Lpos=0,L=toóbói]{B}
	
	\Vertex[x=-9,y=-4,Lpos=275,L=wennáazo]{SDH}
	\Vertex[x=-5,y=-4,Lpos=270,L=náazo]{SN}
	\Vertex[x=-1,y=-4,Lpos=90,L=wenzohiói]{DH}
	\Vertex[x=3,y=-4,Lpos=270,L=zohiói]{N}
	\Vertex[x=7,y=-4,Lpos=275,L=wentoózo]{BDH}
	\Vertex[x=11,y=-4,Lpos=270,L=toózo]{BN}
	
	\Vertex[x=-1,y=-8,Lpos=270,L=zozohiói]{DN}
	\Vertex[x=3,y=-8,Lpos=270,L=zozohiói]{NN}
	
	\Vertex[x=-17,y=-12,L=male]{malekey}
	% Either male or female
	\tikzset{VertexStyle/.append style={shape=diamond,minimum size=0.75em}}
	\Vertex[x=-18,y=0,Lpos=265,L=tratrabói]{MBC}
	\Vertex[x=-16,y=0,Lpos=90,L=wentratrabói]{MBCS}
	\Vertex[x=-14,y=0,Lpos=265,L=tírnánáatói]{MSC}
	\Vertex[x=-12,y=0,Lpos=90,L=wentratrabói]{MSCS}
	\Vertex[x=12,y=0,Lpos=265,L=tírtotoóbói]{FBC}
	\Vertex[x=14,y=0,Lpos=90,L=wentratrabói]{FBCS}
	\Vertex[x=16,y=0,Lpos=265,L=tratrabói]{FSC}
	\Vertex[x=18,y=0,Lpos=90,L=wentratrabói]{FSCS}
	
	\Vertex[x=-17,y=-4,Lpos=265,L=tratrabói]{MBCC}
	\Vertex[x=-13,y=-4,Lpos=265,L=tratrabói]{MSCC}
	\Vertex[x=13,y=-4,Lpos=275,L=tratrabói]{FBCC}
	\Vertex[x=17,y=-4,Lpos=275,L=tratrabói]{FSCC}
	
	\Vertex[x=-17,y=-8,Lpos=270,L=tratrabói]{MBCCC}
	\Vertex[x=-13,y=-8,Lpos=270,L=tratrabói]{MSCCC}
	\Vertex[x=-10,y=-8,Lpos=270,L=tírtrabói]{SDC}
	\Vertex[x=-6,y=-8,Lpos=270,L=tírtrabói]{SNC}
	\Vertex[x=6,y=-8,Lpos=270,L=tírtrabói]{BDC}
	\Vertex[x=10,y=-8,Lpos=270,L=tírtrabói]{BNC}
	\Vertex[x=13,y=-8,Lpos=270,L=tratrabói]{FBCCC}
	\Vertex[x=17,y=-8,Lpos=270,L=tratrabói]{FSCCC}
	
	\Vertex[x=-17,y=-13,L=either female or male]{eitherkey}
	% Ego
	\SetVertexLabelIn
	\tikzset{VertexStyle/.append style={shape=diamond,minimum size=3em}}
	\Vertex[x=0,y=0,Lpos=270,L=mé]{E}
	
	\Edge(MGM)(MGF)
	\Edge(PGM)(PGF)
	\Edge(MS)(MSH)
	\Edge(MBW)(MB)
	\Edge(M)(F)
	\Edge(FBW)(FB)
	\Edge(FS)(FSH)
	\Edge(MBC)(MBCS)
	\Edge(MSC)(MSCS)
	\Edge(S)(SH)
	\Edges(W,E,H)
	\Edge(BW)(B)
	\Edge(FBC)(FBCS)
	\Edge(FSC)(FSCS)
	\Edge(SD)(SDH)
	\Edge(SNW)(SN)
	\Edge(D)(DH)
	\Edge(NW)(N)
	\Edge(BD)(BDH)
	\Edge(BNW)(BN)
	
	\draw (0,4) -- (E);
	\draw (S) -- (-9,2) -- (9,2) -- (B);
	\draw (E) -- (0,-2);
	\draw (D) -- (-3,-2) -- (3,-2) -- (N);
	\draw (-2,-4) -- (-2,-6);
	\draw (DD) -- (-3,-6) -- (-1,-6) -- (DN);
	\draw (2,-4) -- (2,-6);
	\draw (ND) -- (1,-6) -- (3,-6) -- (NN);
	\draw (-8,0) -- (-8,-2);
	\draw (SD) -- (-11,-2) -- (-5,-2) -- (SN);
	\draw (8,0) -- (8,-2);
	\draw (BD) -- (5,-2) -- (11,-2) -- (BN);
	\draw (-10,-4) -- (SDC);
	\draw (-6,-4) -- (SNC);
	\draw (6,-4) -- (BDC);
	\draw (10,-4) -- (BNC);
	\draw (-9,8) -- (-9,6);
	\draw (MB) -- (-16,6) -- (-1,6) -- (M);
	\draw (-12,6) -- (MS);
	\draw (9,8) -- (9,6);
	\draw (F) -- (1,6) -- (16,6) -- (FS);
	\draw (12,6) -- (FB);
	\draw (-17,4) -- (-17,2) -- (-18,2) -- (MBC);
	\draw (-13,4) -- (-13,2) -- (-14,2) -- (MSC);
	\draw (-17,0) -- (MBCC) -- (MBCCC);
	\draw (-13,0) -- (MSCC) -- (MSCCC);
	\draw (13,4) -- (13,2) -- (12,2) -- (FBC);
	\draw (17,4) -- (17,2) -- (16,2) -- (FSC);
	\draw (13,0) -- (FBCC) -- (FBCCC);
	\draw (17,0) -- (FSCC) -- (FSCCC);
	\draw (MGMS) -- (-14,9) -- (-10,9) -- (MGM);
	\draw (-12,9) -- (MGMB);
	\draw (MGF) -- (-8,9) -- (-4,9) -- (MGFB);
	\draw (-6,9) -- (MGFS);
	\draw (PGMS) -- (4,9) -- (8,9) -- (PGM);
	\draw (6,9) -- (PGMB);
	\draw (PGF) -- (10,9) -- (14,9) -- (PGFB);
	\draw (12,9) -- (PGFS);
	\end{tikzpicture}
\end{sidewaysfigure}

\index{kinship|)}